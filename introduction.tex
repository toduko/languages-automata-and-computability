\chapter{Въведение}

Основното нещо, което се прави в този курс, е да се класифицират езици.
За да можем да класифицираме един език, първо трябва знаем какво точно представлява един език.

\section{Основни понятия}

\begin{definition}
    \textbf{Азбука} ще наричаме всяко крайно множество.
    Елементите на азбуката ще наричаме \textbf{букви}.
\end{definition}

Обикновено ще си бележим азбуквата със $\Sigma$.
Също така, ако никъде не е споменато друго, $\Sigma = \{a, b \}$.
Тук буквите ще бъдат $a$ и $b$.

\begin{definition}
    \textbf{Дума} над азбуката $\Sigma$ ще наричаме всяка крайна редица от букви от $\Sigma$.
    Дължината на дума $\alpha$ над $\Sigma$ ще бележим с $|\alpha|$.
\end{definition}

При азбуката $\Sigma = \{0, 1\}$, примерна дума ще бъде $\alpha = 000101$.
Ясно е, че $|\alpha| = 6$.

\begin{definition}
    \textbf{Празната дума} ще наричаме единствената дума с дължина 0.
    Бележим я с $\varepsilon$.
\end{definition}

\begin{warning}
    Важно е да се отбележи, че празното множество и празната дума са различни неща.
    Възможно е да се вземе такава дефиниция за редица, в която те да съвпадат, но това не е съществено.
    За нас думите ще бъдат едни неща, а множествата други.
    \textbf{TLDR:} $\varepsilon \neq \varnothing$
\end{warning}

\begin{definition}
    Със $\Sigma^*$ ще бележим множеството от всички думи над $\Sigma$.
    $L$ ще наричаме \textbf{език} над $\Sigma$, ако $L \subseteq \Sigma^*$.
\end{definition}

Тук нащият универсум ще бъде $\Sigma^*$.
Така че за $L \subseteq \Sigma^*$, под $\overline{L}$ ще имаме предвид $\Sigma^* \setminus L$.

\section{Операции върху думи и езици}

\begin{definition}
    Ще дефинираме \textbf{конкатенацията} (слепването) на две думи $\alpha$ и $\beta$
    и ще го бележим с $\alpha \cdot \beta$
    \begin{itemize}
        \item $\alpha \cdot \varepsilon = \alpha$ (Базов случай)
        \item $\alpha \cdot (\beta x) = (\alpha \cdot \beta)x$ (Свеждане до по-малка ``задача'')
    \end{itemize}
\end{definition}

На пръв поглед такава дефиниция изглежда безсмислена,
но това далеч не е така.
После тя ще се използва постоянно в доказателства. \\

Нека разгледаме един пример за конкатенация:
\begin{align*}
    aaa \cdot bbb & = (aaa \cdot bb)b = ((aaa \cdot b)b)b = (((aaa \cdot \varepsilon)b)b)b = \\
                  & = (((aaa)b)b)b = ((aaab)b)b = (aaabb)b = aaabbb
\end{align*}

Вече можем да дефинираме $\alpha^n$ ($n$ на брой пъти да конкатенираме думата $\alpha$) индуктивно:
\begin{itemize}
    \item $\alpha^0 = \varepsilon$
    \item $\alpha^{n + 1} = \alpha^n \cdot \alpha$
\end{itemize}

За пример можем да вземем $(ab)^3$.
\begin{align*}
    (ab)^3 & = (ab)^2 \cdot ab = ((ab^1) \cdot ab) \cdot ab = (((ab^0) \cdot ab) \cdot ab) \cdot ab = \\
           & = ((\varepsilon \cdot ab) \cdot ab) \cdot ab = ab \cdot ab \cdot ab
\end{align*}

\begin{remark}
    Тук използваме наготово, че $\varepsilon \cdot \alpha = \alpha$ (от Задача \ref{epsilon-neutral-element}).
\end{remark}

Имайки конкатенация на думи, можем да дефинираме и конкатенацията на езици.
Най-естествено е да направим следното:

\begin{definition}
    Нека $L_1, L_2 \subseteq \Sigma^*$.
    Тогава \textbf{конкатенацията} на езиците $L_1$ и $L_2$ ще наричаме множеството:
    \begin{center}
        $L_1 \cdot L_2 = \{ \alpha \cdot \beta \: | \: \alpha \in L_1 \: \& \: \beta \in L_2 \}$ \\
    \end{center}
\end{definition}

Вече можем да дефинираме $L^n$ ($n$ на брой пъти да конкатенираме езика $L$) индуктивно:
\begin{itemize}
    \item $L^0 = \{ \varepsilon \}$
    \item $L^{n + 1} = L^n \cdot L$
\end{itemize}

\begin{definition}[Звезда на Клини]
    Нека $L \subseteq \Sigma^*$. Тогава:
    \begin{itemize}
        \item $L^* = \bigcup\limits_{n \in \mathbb{N}} L^n = L^0 \cup L^1 \cup L^2 \cup \dots $
        \item $L^+ = \bigcup\limits_{\substack{n \in \mathbb{N} \\ n \neq 0}} L^n = L^1 \cup L^2 \cup L^3 \cup \dots $
    \end{itemize}
\end{definition}

Ясно е, че в тази дефиниция $\Sigma^*$ е същото нещо като в другата.

\section{Допълнителни дефиниции}

Тук ще сложим няколко дефиниции, които са стандартни, и ще има задачи, свързани с тях.

\begin{definition}
    Ще дефинираме \textbf{обръщането} на дума и на език.
    Обръщането на дума $\alpha$, което бележим с $\alpha^{rev}$, става индуктивно:
    \begin{itemize}
        \item $\varepsilon^{rev} = \varepsilon$
        \item $(\alpha x)^{rev} = x(\alpha^{rev})$
    \end{itemize}
    Обръщането на език $L$ бележим с $L^{rev} = \{\alpha^{rev} \: | \: \alpha \in L \}$.
\end{definition}

\begin{definition}
    Ще дефинираме кога една дума $\alpha$ е префикс, инфикс или суфикс на друга дума $\beta$:
    \begin{itemize}
        \item $\alpha \preceq_{pref} \beta$, ако $(\exists \gamma \in \Sigma^*)(\alpha \cdot \gamma = \beta)$
        \item $\alpha \preceq_{suff} \beta$, ако $(\exists \gamma \in \Sigma^*)(\gamma \cdot \alpha = \beta)$
        \item $\alpha \preceq_{inf} \beta$, ако $(\exists \gamma_1 \in \Sigma^*)(\exists \gamma_2 \in \Sigma^*)(\gamma_1 \cdot \alpha \cdot \gamma_2 = \beta)$
    \end{itemize}
    Нека $L \subseteq \Sigma^*$. Тогава:
    \begin{itemize}
        \item $\operatorname{Pref}(L) = \{ \alpha \in \Sigma^* \: | \: (\exists \beta \in L)(\alpha \preceq_{pref} \beta) \}$
        \item $\operatorname{Suff}(L) = \{ \alpha \in \Sigma^* \: | \: (\exists \beta \in L)(\alpha \preceq_{suff} \beta) \}$
        \item $\operatorname{Infix}(L) = \{ \alpha \in \Sigma^* \: | \: (\exists \beta \in L)(\alpha \preceq_{inf} \beta) \}$
    \end{itemize}
\end{definition}

\section{Задачи за упражнение}

\begin{problem}[асоциативност]
Да се докаже, че $\alpha \cdot (\beta \cdot \gamma) = (\alpha \cdot \beta) \cdot \gamma$

Упътване: да се направи индукция по $\gamma$
\end{problem}

\begin{problem}[неутрален елемент]\label{epsilon-neutral-element}
Да се докаже, че $\varepsilon \cdot \alpha = \alpha$

Упътване: да се направи индукция по $\alpha$
\end{problem}

\begin{problem}
Да се докаже, че $\alpha^n \cdot \alpha^m = \alpha^{n + m}$

Упътване: да се направи индукция по $m$
\end{problem}


\begin{problem}
Да се докаже, че $(\alpha^n)^m = \alpha^{nm}$

Упътване: да се направи индукция по $m$
\end{problem}

\begin{problem}
Да се докаже, че $L^n \cdot L^m = L^{n + m}$

Упътване: да се направи индукция по $m$
\end{problem}


\begin{problem}
Да се докаже, че $(L^n)^m = L^{nm}$

Упътване: да се направи индукция по $m$
\end{problem}

\begin{problem}[дистрибутивност]
Да се докаже, че:
\begin{itemize}
    \item $(L_1 \cup L_2) \cdot L_3 = (L_1 \cdot L_3) \cup (L_2 \cdot L_3)$
    \item $(L_1 \cap L_2) \cdot L_3 = (L_1 \cdot L_3) \cap (L_2 \cdot L_3)$
\end{itemize}
\end{problem}

\begin{problem}
Да се докаже, че $\Sigma^+ = \Sigma \cdot \Sigma^*$

Упътване: да се използват предните резултати
\end{problem}

\begin{problem}[свойства на reverse]\thlabel{reverse-props}
Да се докаже, че:
\begin{itemize}
    \item $(\alpha \cdot \beta)^{rev} = \beta^{rev} \cdot \alpha^{rev}$
    \item $(L_1 \cdot L_2)^{rev} = L_2^{rev} \cdot L_1^{rev}$
\end{itemize}

Упътване: за първото да се направи индукция по $\beta$
\end{problem}

\begin{problem}
Да се докаже, че:
\begin{itemize}
    \item $(\alpha^{rev})^{rev} = \alpha$
    \item $(L^{rev})^{rev} = L$
\end{itemize}

Упътване: за първото да се направи индукция по $\alpha$
\end{problem}
\begin{problem}[свойства на Pref, Suff, Infix]\thlabel{prefix-suffix-infix-props}
Да се докаже, че:
\begin{itemize}
    \item $\operatorname{Pref}(L) = \operatorname{Suff}(L^{rev})^{rev}$
    \item $\operatorname{Suff}(L) = \operatorname{Pref}(L^{rev})^{rev}$
    \item $\operatorname{Pref}(L_1 \cdot L_2) = \operatorname{Pref}(L_1) \cup (L_1 \cdot \operatorname{Pref}(L_2))$
    \item $\operatorname{Suff}(L_1 \cdot L_2) = \operatorname{Suff}(L_2) \cup (\operatorname{Suff}(L_1) \cdot L_2)$
    \item $\operatorname{Infix}(L_1 \cdot L_2) = \operatorname{Infix}(L_1) \cup \operatorname{Infix}(L_2) \cup (\operatorname{Suff}(L_1) \cdot \operatorname{Pref}(L_2))$
    \item $\operatorname{Infix}(L) = \operatorname{Pref}(\operatorname{Suff}(L)) = \operatorname{Suff}(\operatorname{Pref}(L))$
\end{itemize}

Упътване: да се разсъждава на ниво думи (конкатенация на езици се дефинира с конкатенация на думи)
\end{problem}