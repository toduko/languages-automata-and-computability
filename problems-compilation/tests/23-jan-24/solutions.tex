\documentclass{article}
\usepackage[utf8]{inputenc}
\usepackage[T2A]{fontenc}
\usepackage[english, bulgarian]{babel}
\usepackage{amssymb}
\usepackage{hyperref, fancyhdr, lastpage, fancyvrb, tcolorbox, titlesec}
\usepackage{array, tabularx, colortbl}
\usepackage{tikz}
\usepackage{venndiagram}
\usepackage{amsthm, bm}
\usepackage{relsize}
\usepackage{amsmath,physics}
\usepackage{mathtools}
\usepackage{subcaption}
\usepackage{theoremref}
\usepackage{circuitikz}
\usepackage[a4paper, left=0.50in, right=0.50in, top=1.0in, bottom=1.0in]{geometry}
\usepackage{stmaryrd}
\usepackage{forest}
\usepackage{cancel}
\usepackage{stackrel}
\usetikzlibrary{automata, arrows, positioning, shapes}
\useforestlibrary{linguistics}

\ExplSyntaxOn
\NewDocumentCommand{\opair}{m}
{
    \langle\mspace{2mu}
    \clist_set:Nn \l_tmpa_clist { #1 }
    \clist_use:Nn \l_tmpa_clist {,\mspace{3mu plus 1mu minus 1mu}\allowbreak}
    \mspace{2mu}\rangle
}
\ExplSyntaxOff

\hypersetup{
    colorlinks=true,
    linktoc=all,
    linkcolor=blue
}

\setlength\parindent{0pt}

\newtheorem{problem}{Задача}
\newtheorem*{claim}{Твърдение}
\theoremstyle{definition}
\newtheorem*{solution}{Решение}

\begin{document}
\title{Решения на задачите от писмен изпит по ЕАИ на специалност ``Информатика'', проведено на 23 януари 2024 г.}
\date{}

\maketitle

\begin{problem} {\bf (1 т)} Нека $G$ е граматиката с начална променлива $S$, зададена чрез следните правила:

$$S \rightarrow aT \mid aM \mid aX$$
$$T \rightarrow bS \mid \varepsilon$$
$$M \rightarrow bM \mid bX \mid \varepsilon$$
$$X \rightarrow aS$$
Да се построи краен детерминиран автомат $\mathcal{A}$, за който е изпълнено $\mathcal{L}(\mathcal{A})=\mathcal{L}(G)$, като се използват изучавани конструкции или се докаже коректността на автомата.
\end{problem}

\begin{solution}
    Първо строим недетерминиран автомат $\mathcal{N}$ и после го детерминизираме, за да получим $\mathcal{A}$:

    \begin{figure*}[h]
        \begin{subfigure}{0.5\linewidth}
            \centering
            \begin{tabular}{|c|c|c|}
                \hline
                $\mathcal{N}$   & $a$       & $b$    \\
                \hline
                $\rightarrow s$ & $t, m, x$ &        \\
                \hline
                $*t$            &           & $s$    \\
                \hline
                $*m$            &           & $m, x$ \\
                \hline
                $x$             & $s$       &        \\
                \hline
            \end{tabular}
            \caption*{НКА $\mathcal{N}$ за $\mathcal{L}(G)$}
        \end{subfigure}
        \begin{subfigure}{0.5\linewidth}
            \centering
            \begin{tabular}{|c|c|c|}
                \hline
                $\mathcal{A}$   & $a$           & $b$           \\
                \hline
                $\rightarrow s$ & $t, m, x$     & $\varnothing$ \\
                \hline
                $*t, m, x$      & $s$           & $s, m, x$     \\
                \hline
                $*s, m, x$      & $s, t, m, x$  & $m, x$        \\
                \hline
                $*s, t, m, x$   & $s, t, m, x$  & $s, m, x$     \\
                \hline
                $*m, x$         & $s$           & $m, x$        \\
                \hline
                $\varnothing$   & $\varnothing$ & $\varnothing$ \\
                \hline
            \end{tabular}
            \caption*{ДКА $\mathcal{A}$ за $\mathcal{L(N)} = \mathcal{L}(G)$}
        \end{subfigure}
    \end{figure*}

    През цялото време сме използвали изучавани конструкции, така че няма какво да доказваме.
\end{solution}

\begin{problem} {\bf (2 т)}
\begin{itemize}

    \item {\bf (1 т)} Докажете, че следният език е регулярен:
          $$L_1 = \{a^n \# a^{m \cdot (n \; mod \; 2)} b^{m \cdot ((n+1) \; mod \; 2)} \mid n, m \in \mathbb{N}\}$$
    \item {\bf (1 т)} Докажете, че следният език не е регулярен:
          $$L_2 = \{a^n \# a^{m \cdot (n \; mod \; 2)} b^{m \cdot ((n+1) \; mod \; 3)} \mid n, m \in \mathbb{N}\}$$

\end{itemize}
\end{problem}

\begin{solution}
    За $L_1$ получаваме следното изразяване:
    \begin{align*}
        L_1 & = \{ a^n \# a^{m \cdot{(n \; mod \; 2))}} \underbrace{b^{m \cdot (n + 1 \; mod \; 2)}}_{\substack{\varepsilon \text{, понеже }                                                                     \\ (n + 1 \; mod \; 2) = 0}} \mid n, m \in \mathbb{N} \: \& \: n \text{ е нечетно} \} \cup \{ a^n \# \underbrace{a^{m \cdot{(n \; mod \; 2))}}}_{\substack{\varepsilon \text{, понеже }                                                                     \\ (n \; mod \; 2) = 0}} b^{m \cdot (n + 1 \; mod \; 2)} \mid n, m \in \mathbb{N} \: \& \: n \text{ е четно} \} \\
            & = \{ a^n \# a^m \mid n, m \in \mathbb{N} \: \& \: n \text{ е нечетно} \} \cup \{ a^n \# b^m \mid n, m \in \mathbb{N} \: \& \: n \text{ е четно} \}                                                 \\
            & = (\{ a^n \mid n \in \mathbb{N} \: \& \: n \text{ е нечетно} \} \cdot \{ \# \} \cdot  \{ a \}^*) \cup (\{ a^n \mid n \in \mathbb{N} \: \& \: n \text{ е четно} \} \cdot \{ \# \} \cdot  \{ b \}^*) \\
            & = (\{ a^{2n + 1} \mid n \in \mathbb{N} \} \cdot \{ \# \} \cdot  \{ a \}^*) \cup (\{ a^{2n} \mid n \in \mathbb{N} \} \cdot \{ \# \} \cdot  \{ b \}^*)                                               \\
            & = (\{ a \} \cdot \{ a^{2n} \mid n \in \mathbb{N} \} \cdot \{ \# \} \cdot  \{ a \}^*) \cup (\{ a^{2n} \mid n \in \mathbb{N} \} \cdot \{ \# \} \cdot  \{ b \}^*)                                     \\
            & = (\{ a \} \cdot \{ (aa)^n \mid n \in \mathbb{N} \} \cdot \{ \# \} \cdot  \{ a \}^*) \cup (\{ (aa)^n \mid n \in \mathbb{N} \} \cdot \{ \# \} \cdot  \{ b \}^*)                                     \\
            & = (\{ a \} \cdot \{ aa \}^* \cdot \{ \# \} \cdot \{ a \}^*) \cup (\{ aa \}^* \cdot \{ \# \} \cdot \{ b \}^*)
    \end{align*}

    Получихме $L_1$ чрез прилагане на регулярните операции, започвайки от регулярни езици.
    Така $L_1$ е регулярен.

    \pagebreak

    За $L_2$ нека разгледаме класовете от вида $[a \# a^n]_{\approx_{L_2}}$.
    Нека $n, k \in \mathbb{N}$ и $n \neq k$.
    Тогава:
    \begin{itemize}
        \item $a \# a^{n (1 \; mod \; 2)} b^{n (1 + 1 \; mod \; 3)} = a \# a^n b^{2n} \in L$
        \item $a \# a^{k (1 \; mod \; 2)} b^{n (1 + 1 \; mod \; 3)} = a \# a^k b^{2n} \notin L$, понеже $n \neq k$
    \end{itemize}
    От тук можем да заключим, че $a \# a^n \not\approx_{L_2} a \# a^k$ за $n \neq k$.
    Така намерихме безброй много елементи на множеството $\{ [\alpha]_{\approx_{L_2}} \mid \alpha \in \Sigma^* \}$, откъдето $L_2$ не е регулярен.
\end{solution}

\begin{problem} {\bf (1 т)}
Нека $L = \{\omega.\omega^{rev} \mid \omega \in \{a,b\}^{\ast}\}$. Докажете, че следният език е контекстно-свободен:
$$L_3 = \{ a^n b^n \# \beta \mid n \in \mathbb{N} \: \& \: \beta \in L^{(n \; mod \; 3)} \}$$
\end{problem}

\begin{solution}
    За $L_3$ получаваме следното изразяване:
    \begin{align*}
        L_3 & = \bigcup\limits_{r \in \{0, 1, 2\}} (\{ a^n b^n \mid n \in \mathbb{N} \: \& \: (n \; mod \; 3) = r \} \cdot L^r)          \\
            & = \bigcup\limits_{r \in \{0, 1, 2\}} (\{ a^r \} \cdot \{ a^{3n} b^{3n} \mid n \in \mathbb{N} \} \cdot \{ b^r \} \cdot L^r)
    \end{align*}

    Достатъчно е да направим граматики за $\{ a^{3n} b^{3n} \mid n \in \mathbb{N} \}$ и $L$:
    $$S \rightarrow aaa S bbb \mid \varepsilon \text{ - граматика за } \{ a^{3n} b^{3n} \mid n \in \mathbb{N} \}$$
    $$S_L \rightarrow a S_L a \mid b S_L b \mid \varepsilon \text{ - граматика за } L$$

    Коректността на тези или подобни граматики е доказвана на упражнение, но на изпита се очаква от студента да направи доказателство.
\end{solution}

\end{document}