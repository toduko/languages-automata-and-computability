\section{Нормална форма на Чомски}

Тук ще покажем, че всеки безконтекстен език с точност до липсата на $\varepsilon$ може да се представи чрез граматика във ``хубав'' вид.
За целта постъпково ще си ``опростяваме'' нашата граматика т.е. ще гледаме да променим свойствата на граматика запазвайки езика (с точност до липсата на $\varepsilon$).

\subsection*{Премахване на дългите правила}

Под дълго правило имаме предвид $X \rightarrow X_1 \dots X_k$, където $X_i \in \Sigma \cup V$ и $k \geq 3$.
За да премахнем едно такова правило просто добававяме променливите $Y_1, \dots, Y_{k - 2}$ и заменяме горното правило с:

\begin{center}
    $X \rightarrow X_1 Y_1, \: Y_1 \rightarrow X_2 Y_2, \: \dots, \: Y_{k - 2} \rightarrow X_{k - 1} X_k$

\end{center}
Прилагайки тази конструкция за всички правила получаваме граматика със същия език, която има само правила от вида $A \rightarrow \beta$, където $|\beta| \leq 2$.

\subsection*{Премахване не $\varepsilon$ правилата}

Целта ни ще бъде да премахнем всички правила от вида $X \rightarrow \varepsilon$ като запазим езика (без $\varepsilon$).
Ще направим рекурсия:

\begin{itemize}
    \item $\operatorname{Eps}(0) = \varnothing$
    \item $\operatorname{Eps}(n+1) = \{ A \in V \mid (\exists \beta \in \operatorname{Eps}(n)^*) \: (A \rightarrow \beta) \}$

\end{itemize}

Първо в $\operatorname{Eps}(1)$ ще бъдат променливите, които генерират $\varepsilon$, после в $\operatorname{Eps}(2)$ ще бъдат променливите, които генерират тях и т.н.
Лесно може да се покаже, че:

\begin{center}
    $\operatorname{Eps}(n) = \{ A \in V \mid A \tri{\leq n} \varepsilon \}$
\end{center}

Тъй като $\operatorname{Eps}(n) \subseteq \operatorname{Eps}(n+1) \subseteq V$, то от някъде нататък ще получаваме едно и също множество.
Нека това множество бележим с $\operatorname{Eps}$.
Нека сега променим правилата.
Ако има правило $X \rightarrow X_1 \dots X_k$, при условието че $X_1 \dots X_k \neq \varepsilon$, добавяме всички правила от вида $X \rightarrow Y_1 \dots Y_k$, където:

\begin{itemize}
    \item ако $X_i \notin \operatorname{Eps}$, то $Y_i = X_i$
    \item ако $X_i \in \operatorname{Eps}$, то $Y_i = X_i$ или $Y_i = \varepsilon$
\end{itemize}

Очевидно е, че в новата граматика няма да има $\varepsilon$ правила.

\subsection*{Премахване на преименуващите правила}

Целта ни тук ще бъде да премахнем правила от вида $A \rightarrow B$.
Обаче тогава трябва да видим какви думи генерира $B$ и по някакъв начин да ги добавим към тези, които генерира $A$.
Отново правим рекурсия:

\begin{itemize}
    \item $\operatorname{Rename}(0) = V \cross V$
    \item $\operatorname{Rename}(n + 1) = \operatorname{Rename}(n) \cup \{ \opair{A, C} \mid (\exists B \in V) \: (A \rightarrow B \: \& \: \opair{B, C} \in \operatorname{Rename}(n)) \}$
\end{itemize}

Цялата идея е да се види колко далече може да стигне една променлива с преименуване.
Лесно може да се съобрази, че:

\begin{center}
    $\operatorname{Rename}(n) = \{ \opair{A, B} \mid A \tri{\leq n} B \}$
\end{center}

Тъй като $\operatorname{Rename}(n) \subseteq \operatorname{Rename}(n+1) \subseteq V \cross V$, то от някъде нататък ще получаваме едно и също множество.
Нека това множество бележим с $\operatorname{Rename}$.
Вече можем да кажем кои ще са новите правила в граматиката:

\begin{center}
    $R_{no rename} = \{ \opair{A, \beta} \in V \cross (\Sigma \cup V)^* \mid (\exists B \in V) \: (\underbrace{\opair{A, B} \in \operatorname{Rename}}_{A \text{ може да се замени с } B} \& \underbrace{\opair{B, \alpha} \in R \setminus (V \cross V)}_{B \rightarrow \alpha \text{ не е преименуващо правило}}) \}$
\end{center}

\subsection*{Премахване на правилата, които генерират повече от 1 буква}

Ако имаме правило от вида $X \rightarrow x_1 x_2$, където $x_1, x_2 \in \Sigma \cup V$.
За всяко $x_i \in \Sigma$ можем да добавим нова променлива $X_i$ с правилото $X_i \rightarrow x_i$ и да заменим в предното правило $x_i$ със $X_i$.
Например за правилото $A \rightarrow bc$ можем да добавим нови променливи $B$ и $C$, заменяйки старото правило с правилата:

\begin{center}
    $A \rightarrow BC, \: B \rightarrow b, \: C \rightarrow c$
\end{center}

\begin{definition}
    Една безконтекстна граматика $G = \opair{\Sigma, V, S, R}$ е в \textbf{нормална форма на Чомски} (накратко НФЧ), ако всичките и правила са от вида:

    \begin{itemize}
        \item $A \rightarrow BC$ за някои $A, B, C \in V$
        \item $A \rightarrow a$ за някои $A \in V, \: a \in \Sigma$
    \end{itemize}
\end{definition}

Прилагайки тези алгоритми в тази последователност, получаваме при вход безконтекстна граматика $G$ получаваме граматика $G'$ в НФЧ с $\mathcal{L}(G') = \mathcal{L}(G) \setminus \{ \varepsilon \}$.

\begin{remark}
    Ако искаме да добавим $\varepsilon$ в езика можем да го направим много лесно.
    Добавяме нова променлива $S_0$ и правилата $S_0 \rightarrow \varepsilon$ заедно с $S_0 \rightarrow \alpha$ за всяко правило $S \rightarrow \alpha$.
    Тогава правилата ще бъдат от вида:

    \begin{itemize}
        \item $S \rightarrow \varepsilon$
        \item $A \rightarrow BC$ за някои $A, B, C \in V$ като $B, C \neq S$
        \item $A \rightarrow a$ за някои $A \in V, \: a \in \Sigma$
    \end{itemize}

    За простота няма да се занимаваме с този вид граматика.
    Ще приемем, че случаите, в които участва $\varepsilon$ са тривиални са разглеждане, и няма да се занимваме с тях.
\end{remark}

\begin{claim}
    За всеки безконтекстен език $L$, езикът $sub(L)$ (от \thref{sub-lang-def}) също е безконтекстен.
\end{claim}

\begin{proof}
    Само ще покажем конструкцията и откъде идва удобството от НФЧ при конструкции, като пълното доказателство остава за читателя.

    Ще е хубаво да имаме предвид някои хубави свойства на поддумите:

    \begin{itemize}
        \item ако $\alpha$ е поддума на $\beta$ можем да си мислим как ``задраскваме'' някои букви от $\beta$ и получаваме $\alpha$
        \item $sub(L_1 \cdot L_2) = sub(L_1) \cdot sub(L_2)$
    \end{itemize}

    А имайки граматика в НФЧ можем да разсъждаваме така:

    \begin{itemize}
        \item ако си мислим изолирано за правилата от вида $A \rightarrow a$ можем много лесно да направим ``задраскването''
        \item за правилата от вида $A \rightarrow BC$ можем да си мислим, че към $\mathcal{L}_G(A)$ се добавя $\mathcal{L}_G(B) \cdot \mathcal{L}_G(C)$
    \end{itemize}

    Сега нека опишем вече конструкцията.
    Нека $G = \opair{\Sigma, V, S, R}$ е граматика в НФЧ за $L$.
    В новата граматика $G_{sub}$, която строим ще имаме същите променливи, същата начална променлива, като правилата са следните:

    \begin{itemize}
        \item имаме същите правила от $G$
        \item ако $A \rightarrow a$ е правило в $G$, то добавяме правилото $A \rightarrow \varepsilon$ (тук става ``задраскването'')
    \end{itemize}

    Сега да опишем на картинка какво ще направим.
    Да кажем, че генерираме думата $a_1 a_2 a_3 a_4$ в оригиналния език.
    Искаме в новия език да можем да генерираме думата $a_1 a_4$:

    \begin{center}
        \begin{forest}
            [$S$ [$A$ [$C$ [$a_1$]] [$D$ [$\cancel{a_2}$ \\ $\varepsilon$, align=center]]] [$B$ [$E$ [$\cancel{a_3}$ \\ $\varepsilon$, align=center]] [$F$ [$a_4$]]]]
        \end{forest}
    \end{center}

    Понеже сме имали правилото $D \rightarrow a_2$ сме добавили правилото $D \rightarrow \varepsilon$, същото за $E$ и $a_3$.
    Това ни позволи да премахнем тези букви.

    Ако искаме да сме подробни, можем да покажем, че за всички $A \in V, \: \alpha \in \Sigma$ и $n \in \mathbb{N}$:

    \begin{itemize}
        \item ако $A \tri{n}_G \alpha$, то $A \tri{n}_{G_{sub}} \beta$ за всяка поддума $\beta$ на $\alpha$
        \item ако $A \tri{n}_{G_{sub}} \alpha$, то има $\beta$ такава, че $A \tri{n}_G \beta$ и $\alpha$ е поддума на $\beta$.
    \end{itemize}
\end{proof}

\begin{claim}
    За всеки безконтекстен език $L$ езикът $L^{rev}$ е безконтекстен.
\end{claim}

\begin{proof}
    Нека $G = \opair{\Sigma, V, S, R}$ е граматика в НФЧ за $L$.
    В новата граматика $G_{rev}$ за $L^{rev}$ имаме същите променливи и начална променлива.
    Новите правила са следните:

    \begin{itemize}
        \item ако $A \rightarrow_G a$, то тогава $A \rightarrow_{G_{rev}} a$
        \item ако $A \rightarrow_G BC$, то тогава $A \rightarrow_{G_{rev}} CB$
    \end{itemize}

    Използвайки, че $(\beta_1 \beta_2)^{rev} = \beta_2^{rev} \beta_1^{rev}$, лесно се показва, че за всички $A \in V, \: \alpha \in \Sigma$ и $n \in \mathbb{N}$

    \begin{center}
        $A \tri{n}_G \alpha \iff A \tri{n}_{G_{rev}} \alpha^{rev}$
    \end{center}

    И двете посоки са аналогични, за това ще направим само едната:

    \begin{itemize}
        \item ако $A \tri{0}_G \alpha$, то $\alpha = A \notin \Sigma^*$ \checkmark
        \item ако $A \tri{n + 1}_G \alpha$, то сме приложили правило:
              \begin{itemize}
                  \item[1 сл.] приложили сме правило от вида $A \rightarrow_G a$: \\
                      Тогава $\alpha = a, \: n = 0$ и $A \rightarrow_{G_rev} a$, откъдето $A \tri{1}_{G_{rev}} a = \alpha$
                  \item[2 сл.] приложили сме правило от вида $A \rightarrow_G BC$: \\
                      Тогава $B \tri{n_1}_G \alpha_1, \: C \tri{n_2}_G \alpha_2$, като $\alpha = \alpha_1 \alpha_2$ и $n = \max \{ n_1, n_2 \}$.
                      Тогава по ИП, $B \tri{n_1}_{G_{rev}} \alpha_1^{rev}$ и $C \tri{n_2}_{G_{rev}} \alpha_2^{rev}$.
                      Също така има правилото $A \rightarrow_{G_{rev}} CB$, откъдето $A \tri{n + 1} \alpha_2^{rev} \alpha_1^{rev} = (\alpha_1 \alpha_2)^{rev}$.
              \end{itemize}
    \end{itemize}

    Показвайки твърдението получаваме, че $\mathcal{L}_{G_{rev}}(A) = \mathcal{L}_G(A)^{rev}$ за всяка променлива $A$, в частност:

    \begin{center}
        $\mathcal{L}(G_{rev}) = \mathcal{L}_{G_{rev}}(S) = \mathcal{L}_G(S)^{rev} = \mathcal{L}(G) = L^{rev}$
    \end{center}
\end{proof}

Хубавото на такива конструкции е, че промените, които правим не са големи.
Тук дърветата на извод имат същата ``форма''.
Разбира се, при сложни примери ще трябва малко повече да си поиграем,
но много помага да си мислим за това което бихме искали да направим,
как би станало върху дървото на извод и дали тази информация може хубаво да се кодира в правилата.
