\section{Няколко интересни задачи}

Тук ще разгледаме няколко хубави задачи за безконтекстни граматики.
Първата ще има малко по-нестандартно решение.

\begin{claim}\thlabel{equal-a-b-cfg}
    Езикът $L = \{ \alpha \in \Sigma^* \mid |\alpha|_a = |\alpha|_b \}$е безконтекстен.
\end{claim}

\begin{proof}
    Граматиката за $L$ е следната:

    \begin{center}
        $S \rightarrow aSb \mid bSa \mid SS \mid \varepsilon$
    \end{center}

    Сега нека покажем, че $\mathcal{L}(G) = L$:

    \begin{itemize}
        \item[$(\subseteq)$] За тази посока ще покажем с индукция по $n \in \mathbb{N}$, че ако $S \tri{n} \alpha$ и $\alpha \in \Sigma^*$, то $\alpha \in L$
            \begin{itemize}
                \item ако $S \tri{0} \alpha$, то $\alpha = S \notin \Sigma^*$ \checkmark
                \item ако $S \tri{n + 1} \alpha$, тогава сме приложили някое правило
                      \begin{itemize}
                          \item[1 сл.] Приложили сме правилото $S \rightarrow \varepsilon$.
                              Тогава $\alpha = \varepsilon \in L$
                          \item[2 сл.] Приложили сме правилото $S \rightarrow aSb$.
                              Тогава $S \tri{n} \beta$, като $\alpha = a \beta b$.
                              Лесно се вижда, че имаме равенствата $|\alpha|_a = 1 + |\beta|_a \stackrel{\text{ИП}}{=} 1 + |\beta|_b = |\alpha|_b$, откъдето $\alpha \in L$
                          \item[3 сл.] Приложили сме правилото $S \rightarrow bSa$.
                              Този случай е аналогичен на 2 сл.
                          \item[4 сл.] Приложили сме правилото $S \rightarrow SS$.
                              Тогава $S \tri{n_1} \beta_1$ и $S \tri{n_2} \beta_2$, като $n = \max \{ n_1, n_2 \}$ и $\alpha = \beta_1 \beta_2$.
                              Така $|\alpha|_a = |\beta_1|_a + |\beta_2|_a \stackrel{\text{ИП}}{=} |\beta_1|_b + |\beta_2|_b = |\alpha|_b$, откъдето $\alpha \in L$
                      \end{itemize}
            \end{itemize}
        \item[$(\supseteq)$]  За тази посока ще покажем с индукция по $|\alpha|$, че ако $\alpha \in L$, то $S \tri{*} \alpha$
            \begin{itemize}
                \item $S \rightarrow \varepsilon$ е правило в $G$, и $S \tri{0} S$, откъдето $S \tri{1} \varepsilon$ \checkmark
                \item Нека $|\alpha| = 2n + 2$ (дължината очевидно трябва да е четна).
                      Ако $\alpha = a \beta b$ за някое $\beta \in \Sigma^*$, то понеже $|\beta|_a = |\alpha|_a - 1 = |\alpha|_b - 1 = |\beta|_b$, по ИП $S \tri{*} \beta$.
                      Но в $G$ има правилото $S \rightarrow aSb$, откъдето $S \tri{*} \alpha$.
                      Случаят, в който $\alpha = b \beta a$ е аналогичен.
                      Интересните случаи са, когато $\alpha = x \beta x$ за $x \in \Sigma$.
                      И двата са симетрични, така че ще разгледаме само когато $\alpha = a \beta a$.
                      Нека $\alpha = \alpha_1 \dots \alpha_{2n + 2}$, като $\alpha_i \in \Sigma$.
                      Нека $f_\alpha(i) = |\alpha_1 \dots \alpha_i|_a - |\alpha_1 \dots \alpha_i|_b$.
                      Например ако $\beta = aabb$, то $f_\beta(0) = 0, f_\beta(1) = 1, f_\beta(2) = 2, f_\beta(3) = 1$ и $f_\beta(4) = 0$.
                      Ясно е, че понеже $\alpha \in L$, $f_\alpha(|\alpha|) = 0$. Понеже $\alpha_1 = a$, $f_\alpha(1) = 1$, и понеже $\alpha_{2n + 2} = a$, $f_\alpha(2n + 1) = -1$.
                      Това, което можем да направим, е да продължим $f_\alpha$ до непрекъсната $g_\alpha : [0, 2n + 2] \rightarrow \mathbb{R}$ по следния начин:
                      \begin{itemize}
                          \item $g_\alpha(i) = f_\alpha(i)$ за всяко $i \in \{ 0, \dots, 2n + 2 \}$
                          \item $g_\alpha(i + \varepsilon) = (1 - \varepsilon) f_\alpha(i) + \varepsilon f_\alpha(i + 1)$ за всяко $i \in { 0, \dots, 2n + 1}$ и $\varepsilon \in (0, 1)$
                      \end{itemize}
                      Очевидно $g_\alpha$ е непрекъсната в реалните (и не целите) си точки.
                      Нека проверим, че е така и в целите:
                      \begin{itemize}
                          \item $\lim\limits_{\varepsilon \rightarrow 0} g_\alpha(i + \varepsilon) = \lim\limits_{\varepsilon \rightarrow 0} [(1 - \varepsilon) f_\alpha(i) + \varepsilon f_\alpha(i + 1)] = f_\alpha(i) = g_\alpha(i)$
                          \item $\lim\limits_{\varepsilon \rightarrow 1} g_\alpha(i + \varepsilon) = \lim\limits_{\varepsilon \rightarrow 1} [(1 - \varepsilon) f_\alpha(i) + \varepsilon f_\alpha(i + 1)] = f_\alpha(i + 1) = g_\alpha(i + 1)$
                      \end{itemize}
                      Имаме непрекъсната функция $g_\alpha$, за която знаем, че $g_\alpha(1) = 1$ и $g_\alpha(2n + 1) = -1$.
                      Тогава от Теоремата на Болцано, има $i \in (1, 2n + 1)$ такъв, че $g_\alpha(i) = 0$.
                      Нещо повече, конструирали сме $g$ така, че да връща цели стойности само когато и подаваме цели стойности.
                      За това $i \in \{ 2, \dots, 2n \}$.
                      Това означава, че $g_\alpha(i) = f_\alpha(i) = |\alpha_1 \dots \alpha_i|_a - |\alpha_1 \dots \alpha_i|_b = 0$, откъдето $\beta_1 = \alpha_1 \dots \alpha_i \in L$.
                      Е тогава очевидно също и $\beta_2 = \alpha_{i + 1} \dots \alpha_{2n + 1} \in L$.
                      По ИП $S \tri{*} \beta_1$ и $S \tri{*} \beta_2$.
                      Така понеже имаме правилото $S \rightarrow SS$, $S \tri{*} \beta_1 \beta_2 = \alpha$.
            \end{itemize}
    \end{itemize}
\end{proof}

\begin{definition}\thlabel{bal-par-def}
    Ще наричаме една дума $\alpha$ \textbf{добре скобуван относно $[$ и $]$}, ако за всяко $\beta \preceq_{pref} \alpha$, $|\beta|_[ \geq |\beta|_]$ и $|\alpha|_[ = |\alpha|_]$
    Ще бележим този факт с $bal(\alpha, [, ])$.
\end{definition}

Пример за добра скобуван израз относно $($ и $)$ би бил думата (((23+4)$\times$(21-12))+6)/(73-23).
Илюстрирано по-добре:

\begin{center}
    $\underset{1}{\textcolor{red}{(}}$
    $\underset{2}{\textcolor{blue}{(}}$
    $\underset{3}{\textcolor{yellow}{(}}$
    23
    +
    4
    $\underset{2}{\textcolor{yellow}{)}}$
    $\times$
    $\underset{3}{\textcolor{yellow}{(}}$
    21
    -
    12
    $\underset{2}{\textcolor{yellow}{)}}$
    $\underset{1}{\textcolor{blue}{)}}$
    +
    6
    $\underset{0}{\textcolor{red}{)}}$
    /
    $\underset{1}{\textcolor{red}{(}}$
    73
    -
    23
    $\underset{0}{\textcolor{red}{)}}$
\end{center}

Тук много лесно се вижда, че никога броят на $($ не надвишава броя на $)$.

\begin{problem}
Да се построи граматика за езика $L = \{ \alpha \in \{ [, ] \} \mid bal(a, [, ]) \}$

Упътване: много подобно на \thref{equal-a-b-cfg}
\end{problem}

\pagebreak

\begin{claim}
    Ако $L$ е безконтекстен, тогава $\operatorname{Pref}(L)$ също е безконтекстен.
\end{claim}

\begin{proof}
    Нека $G = \opair{\Sigma, V, S, R}$ е граматика в НФЧ без безполезни променливи за $L$.
    Под безполезни променливи се има предвид такива, които не генерират нищо от $\Sigma^*$.

    Строим граматика $G_{pref} = \opair{\Sigma, V_{pref}, \overleftarrow{S}, R_{pref}}$ за $\operatorname{Pref}(L)$:

    \begin{itemize}
        \item $V_{pref} = V \cup \overleftarrow{V}$, където $\overleftarrow{V} = \{ \overleftarrow{A} \mid A \in V \}$ и $V \cap \overleftarrow{V} = \varnothing$
        \item правилата от старата граматика се запазват
        \item ако $A \rightarrow_G BC$, то $\overleftarrow{A} \rightarrow_{G_{pref}} B \overleftarrow{C} \mid \overleftarrow{B}$
        \item ако $A \rightarrow_G a$, то $\overleftarrow{A} \rightarrow_{G_{pref}} a \mid \varepsilon$
    \end{itemize}

    Цялата идея на конструкцията е, че $\operatorname{Pref}(L_1 \cdot L_2) = \operatorname{Pref}(L_1) \cup (L_1 \cdot \operatorname{Pref}(L_2))$ (от \nameref{prefix-suffix-infix-props}).

    За да покажем, че $\operatorname{Pref}(L) \subseteq \mathcal{L}(G_{pref})$, ще докажем, че за всяко $A \in V, \: \alpha \in \Sigma^*$:

    \begin{center}
        ако $A \tri{*}_G \alpha$, то $\overrightarrow{A} \tri{*}_{G_{pref}} \beta$ за всяко $\beta \preceq_{pref} \alpha$ т.е. $\operatorname{Pref}(\mathcal{L}_G(A)) \subseteq \mathcal{L}_{G_{pref}}(\overrightarrow{A})$
    \end{center}

    Доказваме с индукция по големината на извода:

    \begin{itemize}
        \item ако $A \tri{0} \alpha$, то $\alpha = A \notin \Sigma^*$ \checkmark
        \item ако $A \tri{n + 1} \alpha$, то сме приложили някое правило
              \begin{itemize}
                  \item[1 сл.] Приложили сме правилото $A \rightarrow_G BC$.
                      Тогава $B \tri{n_1} \beta_1$ и $C \tri{n_1} \beta_2$, като $n = \max \{ n_1, n_2 \}$ и $\alpha = \beta_1 \beta_2$.
                      Нека $\gamma \preceq_{pref} \alpha$.
                      Тогава има два варианта (от \nameref{prefix-suffix-infix-props}):
                      \begin{itemize}
                          \item $\gamma \preceq_{pref} \beta_1$ - тогава по ИП $\overrightarrow{B} \tri{*} \gamma$, и понеже имаме правилото $\overrightarrow{A} \rightarrow \overrightarrow{B}$, $\overrightarrow{A} \tri{*} \gamma$.
                          \item $\gamma = \beta_1 \lambda$ и $\lambda \preceq_{pref} \beta_2$ - тогава по ИП $\overrightarrow{C} \tri{*} \lambda$, и понеже имаме правилото $\overrightarrow{A} \rightarrow B \overrightarrow{C}$, $\overrightarrow{A} \tri{*} \beta_1 \lambda = \gamma$.
                      \end{itemize}
                  \item[2 сл.] Приложили сме правилото $A \rightarrow_G a$.
                      Тогава $\alpha = a$ и всичките префикси на $\alpha$ са $a$ и $\varepsilon$.
                      Тъй като $A \rightarrow_G a$, $\overrightarrow{A} \rightarrow_{G_{pref}} a \mid \varepsilon$, откъдето $\overrightarrow{A} \tri{*} a$ и $\overrightarrow{A} \tri{*} \varepsilon$
              \end{itemize}
    \end{itemize}

    За да покажем, че $\mathcal{L}(G_{pref}) \subseteq \operatorname{Pref}(L)$, ще докажем, че за всяко $A \in V, \: \alpha \in \Sigma^*$:

    \begin{center}
        ако $\overrightarrow{A} \tri{*}_{G_{pref}} \alpha$, то има $\beta \in \Sigma^*$, че $A \tri{*}_G \alpha \beta$ т.е. $\mathcal{L}_{G_{pref}}(\overrightarrow{A}) \subseteq \operatorname{Pref}(\mathcal{L}_G(A))$
    \end{center}

    Доказваме с индукция по големината на извода:

    \begin{itemize}
        \item ако $\overrightarrow{A} \tri{0} \alpha$, то $\alpha = \overrightarrow{A} \notin \Sigma^*$ \checkmark
        \item ако $\overrightarrow{A} \tri{n + 1} \alpha$, то сме приложили някое правило
              \begin{itemize}
                  \item[1 сл.] Приложили сме правилото $\overrightarrow{A} \rightarrow_{G_{pref}} B \overrightarrow{C}$.
                      Тогава $B \tri{n_1}_{G_{pref}} \beta_1$ $\overrightarrow{C}_{G_{pref}} \tri{n_2} \beta_2$ като $n = \max \{ n_1, n_2 \}$ и $\alpha = \beta_1 \beta_2$.
                      По ИП има $\lambda \in \Sigma^*$ такова, че $C \tri{*}_G \beta_2 \lambda$.
                      От правилото, което сме приложили знаем, че $A \rightarrow_G BC$, откъдето $A \tri{*} \beta_1 \beta_2 \lambda = \alpha \lambda$.
                  \item[2 сл.] Приложили сме правилото $\overrightarrow{A} \rightarrow_G \overrightarrow{B}$.
                      Тогава $\overrightarrow{B} \tri{n}_{G_{pref}} \alpha$ и по ИП има $\beta_1 \in \Sigma^*$ такова, че $B \tri{*}_G \alpha \beta_1$.
                      От правилото, което сме приложили знаем, че $A \rightarrow BC$.
                      Тъй като $C$ не е безполезна променлива, има $\beta_2 \in \Sigma^*$ такова, че $C \tri{*}_G \beta_2$.
                      Тогава $A \tri{*}_G \alpha \beta_1 \beta_2$.
                  \item[3 сл.] Приложили сме правилото $\overrightarrow{A} \rightarrow_{G_{pref}} a$.
                      Тогава $A \rightarrow_G a$, откъдето $A \tri{*}_G a = a \varepsilon$.
                  \item[4 сл.] Приложили сме правилото $\overrightarrow{A} \rightarrow_{G_{pref}} \varepsilon$.
                      Тогава $A \rightarrow_G a$, откъдето $A \tri{*}_G a = \varepsilon a$.
              \end{itemize}
    \end{itemize}

    Накрая завършваме с:

    \begin{center}
        $\operatorname{Pref}(L) = \operatorname{Pref}(\mathcal{L}(G)) = \operatorname{Pref}(\mathcal{L}_G(S)) = \mathcal{L}_{G_{pref}}(\overrightarrow{S}) = \mathcal{L}(G_{pref})$
    \end{center}
\end{proof}

\begin{problem}
Да се покаже, че ако $L$ е безконтекстен, то тогава $\operatorname{Suff}(L)$ и $\operatorname{Infix}(L)$ също са безконтекстни.

Упътване: конструкция за $\operatorname{Suff}$ е много подобна, а тази за $\operatorname{Infix}$ е комбинация от двете; може и да се използват \nameref{prefix-suffix-infix-props}
\end{problem}