\section{Небезконтекстни езици}

Сега ще покажем съществуването на небезконтекстни езици.
За целта ще разгледаме безконтекстната версия на \nameref{pumping-lemma}:

\begin{lemma}[Лема за покачването]\thlabel{pumping-lemma-cf}
    Нека $L$ е безконтекстен език. Тогава:
    \begin{align*}
         & (\exists p \geq 1)                                                                       \\
         & (\forall \alpha \in L, \: |\alpha| \geq p)                                               \\
         & (\exists x, y, u, v, w \in \Sigma^*, \: xyuvw = \alpha, \: |yuv| \leq p, \: |yv| \geq 1) \\
         & (\forall i \in \mathbb{N}) [xy^iuv^iw \in L]
    \end{align*}
\end{lemma}

\begin{proof}
    Нека $L$ е безконтекстен език.
    Нека $G = \opair{\Sigma, V, S, R}$ е безконтекстна граматика за $L$.

    Полагаме $b = \max \{ |\gamma| \mid A \rightarrow_G \gamma \text{ е правило в } G \}$ и $p = b^{|V| + 1}$.
    За $b$ можем да си мислим, че това е най-голямата разклоненост на дърво на извод.
    Няма как да има връх в дърво на извод с повече от $b$ брой преки наследници.
    Тогава едно дърво на извод в $G$ с височина $h$ не може да има дума с дължина повече от $b^h$.
    Така ако $\alpha \in L$ и $|\alpha| \geq p$, в което и да е дърво на извод $T$ такова, че $\w{T} = \alpha$, имаме $\h{T} \geq |V|$.
    Тогава в пътя на едно такова дърво $T$ от корена до някои листа ще има повторение на променливи.
    Нека $T$ е дърво на извод за $\alpha$ с минимален брой елементи.
    Знаем, че има повторение на някаква променлива $A \in V$.
    Нека това повторение е възможно най-надолу.
    Със сигурност това ще бъде измежду последните $|V|$ върха в някои път между корена и листата.
    На картинка имаме следното:
    \begin{center}
        \begin{forest}
            [$S$ [[\wraphspace{$x$}{1em},roof]] [$A$ [[\wraphspace{$y$}{1em}, roof]] [$A$ [[\wraphspace{$u$}{1em}, roof]]] [[\wraphspace{$v$}{1em}, roof]]] [[\wraphspace{$w$}{1em},roof]]]
            \node[text width=3.0cm] at (0, 0) {$T:$};
        \end{forest}
    \end{center}
    Понеже срещанията на а се случват достатъчно надолу, $|yuv| \leq p$.
    Понеже дървото $T$ е с минимален брой върхове $|yv| \geq 1$.
    Ако това не беше вярно, $yv = \varepsilon$ и можехме да премахнем първото срещане на $A$, с което строим по-малко дърво на извод.
    Остава само да проверим, че $xy^iuv^iw \in L$ за всяко $i \in \mathbb{N} \: (\star)$.
    От \thref{tri-der-equiv}, $A \tri{*} yAv$.
    Ще покажем с индукция по $i$ че $A \tri{*} y^iAv^i$ за всяко $i \in \mathbb{N}$.
    \begin{itemize}
        \item $A \tri{0} A = y^0Av^0$ \checkmark
        \item Нека $A \tri{*} y^iAv^i$, тогава понеже $A \tri{*} yAv$, $A \tri{*} y^iyAvv^i = y^{i + 1}Av^{i + 1}$ (от \thref{tri-prop})
    \end{itemize}
    От \thref{tri-der-equiv}, $A \tri{*} u$, откъдето използвайки $(\star)$ получаваме, че $A \tri{*} y^iuv^i$ за всяко $i \in \mathbb{N}$.
    Също така $S \tri{*} xAw$, откъдето $S \tri{*} xy^iuv^i$ за всяко $i \in \mathbb{N}$, с което сме готови.
\end{proof}

Разбира се, ние ще използваме по често следствието от лемата.

\begin{corollary}[Лема за покачването (контрапозиция)]\thlabel{non-cf}
    Ако е изпълнено, че:
    \begin{align*}
         & (\forall p \geq 1)                                                                       \\
         & (\exists \alpha \in L, \: |\alpha| \geq p)                                               \\
         & (\forall x, y, u, v, w \in \Sigma^*, \: xyuvw = \alpha, \: |yuv| \leq p, \: |yv| \geq 1) \\
         & (\exists i \in \mathbb{N}) [xy^iuv^iw \notin L]
    \end{align*}
    то тогава $L$ не е безконтекстен.
\end{corollary}

Започваме с каноничния пример за небезконтекстен език:
\begin{claim}\thlabel{canonical-non-cf}
    Езикът $L = \{ a^nb^nc^n \mid n \in \mathbb{N} \}$ не е безконтекстен.
\end{claim}

\begin{proof}
    Нека $p \geq 1$.
    Нека $\alpha = a^pb^pc^p$.
    Ясно е, че $\alpha \in L$ и $|\alpha| \geq p$.
    Нека $x, y, u, v, w \in \Sigma^*$ са такива, че $\: xyuvw = \alpha, \: |yuv| \leq p, \: |yv| \geq 1$.
    Очевидно няма как $yv = a^nc^m$ за $n, m > 0$.
    Тогава имаме няколко варианта за $yv$:
    \begin{itemize}
        \item[1 сл.] $yv = a^t$ за някое $1 \leq t \leq p$.
            Тогава $xy^0uv^0w = xuw = a^{p - t}b^pc^p \notin L$ ($p - t < p$)
        \item[2 сл.] $yv = b^t$ за някое $1 \leq t \leq p$ - аналогично на 1 сл.
        \item[3 сл.] $yv = c^t$ за някое $1 \leq t \leq p$ - аналогично на 1 сл.
        \item[4 сл.] $yv = a^{t_1}b^{t_2}$ за някое $1 \leq t_1, t_2 \leq p, \: t_1 + t_2 \leq p$.
            Тогава $xy^0uv^0w = xuw = a^{p - t}b^{p - t}c^p \notin L$ ($p - t < p$)
        \item[5 сл.] $yv = b^{t_1}c^{t_2}$ за някое $1 \leq t_1, t_2 \leq p, \: t_1 + t_2 \leq p$ - аналогично на 4 сл.
    \end{itemize}
    От \nameref{non-cf} следва, че $L$ не е безконтекстен.
\end{proof}

\begin{problem}\thlabel{pl-attention-example}
Да се докаже, че следните езици не са безконтекстна:
\begin{itemize}
    \item $L_1 = \{ a^nb^{2n}c^n \mid n \in \mathbb{N} \}$
    \item $L_2 = \{ a^nb^{3n}c^{5n} \mid n \in \mathbb{N} \}$
    \item $L_3 = \{ a^nb^mc^n \mid n \leq m \}$
    \item $L_4 = \{ a^nb^mc^k \mid n \leq m \leq k \}$
\end{itemize}
Упътване: за $L_1$ и $L_2$ да се адаптира решението от \thref{canonical-non-cf}, а за $L_3$ и $L_4$ тряба да се използват думи от същия вид.
\end{problem}

\begin{warning}
    За тази версия на лемата, е много по-важен избора на дума.
    Тук трябва да се стремим думата да е възможно най-близко до това да излезне от езика.
    Да вземем за пример $L_3$ от \thref{pl-attention-example}.
    Ако вземем дума от вида $a^pb^{p+t}c^p$, за $p, t \geq 1$, то ако $yv = b$, думата няма да излезе от $L_3$ каквото и да правим.
    Подобни разсъждения могат да се направят и за $L_4$.
    Когато става въпрос за небезконтекстност, трябва да сме по-нежни и по-деликатни.
\end{warning}

\begin{problem}
Да се докаже, че следните езици не са безконтекстни:
\begin{itemize}
    \item $L_1 = \{ \alpha \alpha \mid \alpha \in \Sigma^* \}$
    \item $L_2 = \{ \alpha \alpha \alpha \mid \alpha \in \Sigma^* \}$
    \item $L_3 = \{ \alpha \alpha^{rev} \alpha \mid \alpha \in \Sigma^* \}$
\end{itemize}
\end{problem}

\begin{claim}
    Операцията сечение и допълнение не запазват безкотекстност
\end{claim}

\begin{proof}
    За сечението елементарен контрапример:

    \begin{center}
        $\underbrace{\{ a^nb^nc^k \mid n, k \in \mathbb{N} \}}_{L_1 - \text{безконтекстен}} \cap \underbrace{\{ a^nb^kc^k \mid n, k \in \mathbb{N} \}}_{L_2 - \text{безконтекстен}} = \underbrace{\{ a^nb^nc^n \mid n \in \mathbb{N} \}}_{L - \text{небезконтекстен}}$
    \end{center}

    Ако допуснем, че безконтекстните езици са затворени отностно допълнение, то тогава понеже $L_1$ и $L_2$ са безконтекстни, $\overline{L_1} \cup \overline{L_2}$ също е безконтекстен (обединението запазва безконтекстност).
    Можем да приложим допълнение още веднъж и ще получим отново безконтекстен език, откъдето $\overline{\overline{L_1} \cup \overline{L_2}} = L_1 \cap L_2 = L$ е безконтекстен, което е противоречие.
\end{proof}

Тук се вижда как въпреки че разпознаваме видове езици, губим някои свойства.
Освен, че губим операции като допълнение и сечение, ние също така губим константната памет, която имахме при автоматите.

\begin{warning}
    Въпреки, че в общия случай не можем да приемем, че допълнението на безконтекстен език е също е безконтекстен език, понякога това е вярно.
    Очевидно е вярно за регулярните езици, но не само и аз тях.
    Има безконтекстни езици, които не са регулярни, чието допълнение е безконтекстен език:
\end{warning}

\begin{claim}
    Езикът $L =\Sigma^* \setminus \{ a^nb^n \mid n \in \mathbb{N} \}$ е безконтекстен.
\end{claim}

\begin{proof}
    Само ще покажем представяне на езика, което очевидно е безконтекстно.
    Подробностите оставяме на читателя.
    Нека помислим какво би означавало $\alpha \in L$.
    Има два варианта:
    \begin{itemize}
        \item[1 сл.] $\alpha = a^nb^k$ и $n \neq k$.
            Това би означавало, че $\alpha \in \underbrace{\{ a^nb^k \mid n > k \}}_{L_1} \cup \underbrace{\{ a^nb^k \mid n < k \}}_{L_2}$.
        \item[2 сл.] $\alpha$ изобщо не е от $\{ a \}^* \{ b \}^*$.
            За да не е вярно това някъде в думата има буква $b$ преди буква $a$, откъдето $\alpha \in \underbrace{\Sigma^* \cdot \{ b \} \cdot \Sigma^* \cdot \{ a \} \cdot \Sigma^*}_{L_3}$.
    \end{itemize}
    Така можем да представим $L$ като обединение на безконтекстни езици:
    \begin{center}
        $L = \underbrace{(\{ a \}^+ \cdot \{ a^nb^n \mid n \in \mathbb{N} \})}_{L_1 - \text{безконтекстен}} \cup \underbrace{(\{ a^nb^n \mid n \in \mathbb{N} \} \cdot \{ b \}^+)}_{L_2 - \text{безконтекстен}} \cup \underbrace{(\Sigma^* \cdot \{ b \} \cdot \Sigma^* \cdot \{ a \} \cdot \Sigma^*)}_{L_3 - \text{безконтекстен}}$
    \end{center}
\end{proof}