\section{Задачи за упражнение}

\begin{problem}
Да се докаже, че сечението с регулярен език запазва безконтекстност изцяло с автомати.

Упътване: конструкцията е много подобна на тази с детерминираните автомати
\end{problem}

\begin{problem}\thlabel{homomorphism-cf}
Да се докаже, че за произволен хомоморфизъм (\thref{homomorphism-def}) $h$ и безконтекстен език $L$, езикът $h[L]$ е безконтекстен.

Упътване: да се вземе граматика в НФЧ и да се помисли как трябва да се променят дърветата на извод
\end{problem}

\begin{problem}
Използвайки, че езикът $\{ a^nb^n \mid n \in \mathbb{N} \}$ е безконтекстен, да се докаже, че следните езици са безконтекстни:

\begin{itemize}
    \item $L_1 = \{ b^na^n \mid n \in \mathbb{N} \}$
    \item $L_2 = \{ b^{2n}a^n \mid n \in \mathbb{N} \}$
\end{itemize}

Упътване: да се представят езиците като хомоморфизми върху $L$
\end{problem}

\begin{problem}
Нека $L$ е регулярен.
Използвайки, че е безконтекстен (\thref{move-operation}) езикът

\begin{center}
    $\operatorname{Move}(L) = \{ a^n b^m \mid (\exists \alpha \in L) \: (|\alpha|_a = n \: \& \: |\alpha|_b = m) \}$
\end{center}

да се докаже, че следните езици са безконтекстни:

\begin{itemize}
    \item $L' = \{ \alpha_1 \dots \alpha_n \beta_1 \dots \beta_m \mid (\exists \alpha \in L) \: (|\alpha|_a = n \: \& \: |\alpha|_b = m) \: \& \: \alpha_i \in L_1 \: \& \: \beta_i \in L_2 \}$
    \item $L'' = \{ \alpha_1 \dots \alpha_{2n} \beta_1 \dots \beta_{3m} \mid (\exists \alpha \in L) \: (|\alpha|_a = n \: \& \: |\alpha|_b = m) \: \& \: \alpha_i \in L_1 \: \& \: \beta_i \in L_2 \}$
    \item $L''' = \{ \beta_1 \dots \beta_m \alpha_1 \dots \alpha_n \mid (\exists \alpha \in L) \: (|\alpha|_a = n \: \& \: |\alpha|_b = m) \: \& \: \alpha_i \in L_1 \: \& \: \beta_i \in L_2 \}$
\end{itemize}

Упътване: да се представят езиците като хомоморфизми или композиция от хомоморфизми върху $\operatorname{Move}(L)$
\end{problem}

\begin{definition}\thlabel{cf-homomorphism-def}
    Функция $h : \Sigma_1^* \rightarrow \mathcal{P}(\Sigma_2^*)$ ще наричаме \textbf{безконтекстен хомоморфизъм}, ако:
    \begin{itemize}
        \item $h(x)$ е безконтекстен за всяко $x \in \Sigma_1$
        \item $h(\alpha \cdot \beta) = h(\alpha) \cdot h(\beta)$ за всички $\alpha, \beta \in \Sigma_1^*$
    \end{itemize}
\end{definition}

\begin{problem}
Да се докаже, че за произволен безконтекстен хомоморфизъм $h$ е вярно, че $h(\varepsilon) = \{ \varepsilon \}$.
\end{problem}

\begin{problem}\thlabel{cf-homomorphism}
Да се докаже, че за произволен безконтекстен хомоморфизъм $h$ и безконтекстен език $L$, езикът $\bigcup h[L]$ е безконтекстен.

Упътване: същата конструкцията като \thref{homomorphism-cf}
\end{problem}

\begin{problem}\thlabel{a^nb^ka^nb^k}
Да се покаже, че $L = \{ a^n b^k c^n d^k \mid n, k \in \mathbb{N} \}$ не е безконтекстен.
\end{problem}

\begin{problem}
Да се покаже, че операцията $\operatorname{Cyc}_{rev}(L) = \{ \alpha \beta^{rev} \in \Sigma^* \mid \beta \alpha \in L \}$ не запазва безконтекстност.

Упътване: за използвайте \thref{non-cf-intersect-with-regular} като пресичате $\{ x_1 \}^* \{ x_2 \}^* \{ x_3 \}^* \{ x_4 \}^*$ с някой безконтекстен език, подобен на езика от \thref{a^nb^ka^nb^k}.
\end{problem}