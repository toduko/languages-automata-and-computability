\section{Регулярни граматики и еквивалентност със автоматите}

Тук ще направим нещо, което на пръв поглед може да изглежда безсмислено.
Ще покажем частен случай на безконтекстните граматики, които са еквивалентни на автоматите като изразителна мощ.

\begin{definition}
    Една безконтекстна граматика $G = \opair{\Sigma, V, S, R}$ ще наричаме \textbf{регулярна}, ако единствените правила в $G$ са от вида:
    \begin{itemize}
        \item $A \rightarrow aB$ за някои $A, B \in V, \: a \in \Sigma$
        \item $A \rightarrow \varepsilon$ за някое $A \in V$
    \end{itemize}
\end{definition}

Това изглежда много подобно на дефиницията на автомат.
Тук обаче вместо от състояние със буква да отидем в друго състояние, просто заместваме старото състояние с буквата и новото състояние.
В единия случай четем символи, а в другия ги генерираме.
Генерирането на $\varepsilon$ е аналога на това да спрем да четем думата (почти).

\begin{claim}\thlabel{dfa-to-reg-grammar}
    За всеки ДКА $\mathcal{A} = \opair{\Sigma, Q, s, \delta, F}$ има регулярна граматика $G = \opair{\Sigma, V, S, R}$ с $\mathcal{L}(G) = \mathcal{L(A)}$
\end{claim}

\begin{proof}
    Нека $\mathcal{A} = \opair{\Sigma, Q, s, \delta, F}$ е ДКА.
    Нека $Q = \{ q_1, \dots, q_n \}$ като $q_1 = s$.

    Строим граматика $G = \opair{\Sigma, V, S, R}$ за $\mathcal{L(A)}$:
    \begin{itemize}
        \item $V = \{ P_1, \dots, P_n \}$ (на всяко състояние $q_i$ съответства променлива $P_i$)
        \item $S = P_1$ ($q_1 = s$)
        \item Ако $\delta(q_i, x) = q_j$, то $P_i \rightarrow x P_j$
        \item Ако $q_i \in F$, то $P_i \rightarrow \varepsilon$
    \end{itemize}

    Твърдението, което трябва да покажем е, че за всяко $\alpha \in \Sigma^*, \: \delta^*(q_i, \alpha) = q_j \iff P_i \tri{|\alpha|} \alpha P_j$.
    И в двете посоки се прави индукция по дължината на $|\alpha|$, като в посоката $(\Rightarrow)$ се кара по дефиниция на $\delta^*$, а в посоката $(\Leftarrow)$ - на $\tri{*}$.
    Това остава за читателя.
    Остават само довършителни разсъждения:
    \begin{align*}
        \alpha \in \mathcal{L(A)}  \iff       & (\exists i \in \{ 1, \dots, n \}) \: (\delta^*(q_1, \alpha) = q_i \: \& \: q_i \in F)              \\
        \iff                                  & (\exists i \in \{ 1, \dots, n \}) \: (P_1 \tri{*} \alpha P_i \: \& \: P_i \rightarrow \varepsilon) \\
        \iff                                  & (\exists i \in \{ 1, \dots, n \}) \: (P_1 \tri{*} \alpha P_i \: \& \: P_i \tri{1} \varepsilon)     \\
        \stackrel{\textcolor{red}{!!!}}{\iff} & P_1 \tri{*} \alpha                                                                                 \\
        \iff                                  & \alpha \in \mathcal{L}(G)
    \end{align*}
    \begin{warning}[\textcolor{red}{!!!}]
        По принцип разсъждения, подобни на $(\Leftarrow)$ не са верни в общият случай.
        Тук можем да си го позволим сега поради природата на нашата граматика.
        Единственото нещо, което може да правят променливите $P_i$ е да генерират думи от вида $\alpha P_j$.
        Разсъждения, подобни на $(\Rightarrow)$ обаче, ще бъдат верни, поради \thref{tri-prop}.
    \end{warning}
\end{proof}

\begin{claim}
    За всяка регулярна граматика $G = \opair{\Sigma, V, S, R}$ има НКА $\mathcal{N} = \opair{\Sigma, Q, S, \Delta, F}$ с $\mathcal{L(N)} = \mathcal{L}(G)$
\end{claim}

\begin{proof}
    Нека $G = \opair{\Sigma, V, S, R}$ е регулярна граматика.
    Нека $V = \{ A_1, \dots, A_n \}$, като $A_1 = S$.

    Строим НКА $\mathcal{N} = \opair{\Sigma, Q, I, \Delta, F}$ за $\mathcal{L}(G)$:
    \begin{itemize}
        \item $Q = \{ q_1, \dots, q_n \}$
        \item $I = \{ q_ 1 \} $
        \item $\Delta(q_i, x) = \{ q_j \in Q \mid A_i \rightarrow x A_j \}$
        \item $F = \{ q_i \in Q \mid A_i \rightarrow \varepsilon \}$
    \end{itemize}

    Ще формулираме само помощното твърдение и ще оставим всичко останало на читателя:
    \begin{center}
        $q_j \in \Delta^*(q_i, \alpha) \iff A_i \tri{|\alpha|} \alpha A_j$
    \end{center}
\end{proof}

С тези две твърдения не направихме много.
Показахме, че има някакъв вид безконтекстни граматики със изразителната мощ на автоматите.
Това по принцип не е лошо нещо, но самата граматика е доста подобна на автомата, така че не е и като да имаме напълно нов начин за изразяване на регулярни езици.
По ценното тук е да се види самата ``линейност'' на извода на регулярната граматика.
Можем тази техника да я разширим и да я адаптираме за по-сложни задачи.

Нека да си помислим, какво ще стане ако приложим конструкцията от \thref{dfa-to-reg-grammar}, но вместо $P_i \rightarrow x P_j$ имаме $P_j \rightarrow P_j x$.
Ще генерираме думите на $\mathcal{L(A)}$, но в обратен ред т.е. $\mathcal{L}(G) = \mathcal{L(A)}^{rev}$.

\begin{problem}
Да се опише подробно конструкцията за $\mathcal{L(A)}^{rev}$.
\end{problem}

Какво пък ще стане, ако решим да сложим $P_i \rightarrow x P_j x$ вместо $P_i \rightarrow x P_j$.
Ще генерираме от ляво дума от $\mathcal{L(A)}$, а от дясно същата дума но в обратен ред.
Тогава $\mathcal{L}(G) = \{ \alpha \alpha^{rev} \mid \alpha \in \mathcal{L(A)} \}$

\begin{problem}
Да се опише подробно конструкцията за $\{ \alpha \alpha^{rev} \mid \alpha \in \mathcal{L(A)} \}$.
\end{problem}

Можем и да станем по креативни:
\begin{claim}\thlabel{move-operation}
    Нека $L$ е регулярен.
    Тогава езикът $\operatorname{Move}(L) = \{ a^nb^m \mid (\exists \alpha \in L) \: (|\alpha|_a = n \: \& \: |\alpha|_b = m) \}$ е безконтекстен.
\end{claim}

\begin{proof}
    Ще покажем само конструкцията и ще формулираме помощно твърдение; другото е за читателя.
    Нека $\mathcal{A} = \opair{\Sigma, Q, s, \delta, F}$ е ДКА.
    Нека $Q = \{ q_1, \dots, q_n \}$ като $q_1 = s$.

    Строим граматика $G = \opair{\Sigma, V, S, R}$ за $\operatorname{Move}(L)$:
    \begin{itemize}
        \item $V = \{ P_1, \dots, P_n \}$ (на всяко състояние $q_i$ съответства променлива $P_i$)
        \item $S = P_1$ ($q_1 = s$)
        \item Ако $\delta(q_i, a) = q_j$, то $P_i \rightarrow a P_j$ (слагаме буквите $a$ от ляво)
        \item Ако $\delta(q_i, b) = q_j$, то $P_i \rightarrow P_j b$ (слагаме буквите $b$ от дясно)
        \item Ако $q_i \in F$, то $P_i \rightarrow \varepsilon$
    \end{itemize}

    Трябва да се покаже, че:
    \begin{center}
        $\delta^*(q_i, \alpha) = q_j \iff P_i \tri{n + m} a^{n} P_j b^{m}$, където $n = |\alpha|_a$ и $m = |\alpha|_b$
    \end{center}
\end{proof}

\begin{problem}
Нека $L$ е регулярен.
Да се покаже, че е безконтекстен езикът:
\begin{center}
    $\operatorname{AltMove}(L) = \{ a^nb^m \mid (\exists \alpha \in L) \: (|\alpha|_a = m \: \& \: |\alpha|_b = n) \}$
\end{center}
Упътване: да се направи дребна модификация на конструкцията от \thref{move-operation}
\end{problem}

\begin{problem}
Нека $L$ е регулярен и нека $L_1, L_2$ са безконтекстни.
Да се докаже, че са безконтекстни следните езици:
\begin{itemize}
    \item $L' = \{ \alpha_1 \dots \alpha_n \beta_1 \dots \beta_m \mid (\exists \alpha \in L) \: (|\alpha|_a = n \: \& \: |\alpha|_b = m) \: \& \: \alpha_i \in L_1 \: \& \: \beta_i \in L_2 \}$
    \item $L'' = \{ \alpha_1 \dots \alpha_{2n} \beta_1 \dots \beta_{3m} \mid (\exists \alpha \in L) \: (|\alpha|_a = n \: \& \: |\alpha|_b = m) \: \& \: \alpha_i \in L_1 \: \& \: \beta_i \in L_2 \}$
    \item $L''' = \{ \beta_1 \dots \beta_m \alpha_1 \dots \alpha_n \mid (\exists \alpha \in L) \: (|\alpha|_a = n \: \& \: |\alpha|_b = m) \: \& \: \alpha_i \in L_1 \: \& \: \beta_i \in L_2 \}$
\end{itemize}
Упътване: да се адаптират техники от предишни задачи
\end{problem}