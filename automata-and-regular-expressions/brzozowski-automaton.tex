\section{Автомат на Brzozowski}

Автоматите, които създаваме, не са единствените за конкретния език.
Най-малкото можем да добавим недостижими състояния.
Но това е глупаво, ние по-скоро искаме да направим автомат с възможно най-малко състояния.
Тогава ако решим да направим програма, тя ще заема по-малко памет.

\begin{definition}
    За език $L \subseteq \Sigma^*$ дефинираме $Q_L = \{ \alpha^{-1}(L) \mid \alpha \in \Sigma^* \}$.
\end{definition}

\begin{claim}
    Ако $Q_L$ е крайно, то $L$ е регулярен.
\end{claim}

\begin{proof}
    Нека $Q_L$ е крайно.

    Строим автомат $\mathcal{B}_L = \opair{\Sigma, Q_L, L, \delta, F}$ за $L$, който ще наричаме \textbf{автомат на Brzozowski} за $L$:

    \begin{itemize}
        \item $\delta(M, x) = x^{-1}(M)$ за $M \in Q_L, \: x \in \Sigma$
        \item $F = \{ M \in Q_L \mid \varepsilon \in M \}$
    \end{itemize}

    \begin{claim}
        За всяко $M \in Q_L \: : \: \delta^*(M, \alpha) = \alpha^{-1}(M)$
    \end{claim}

    \begin{proof}
        С индукция по $|\alpha|$.

        \begin{itemize}
            \item $\delta^*(M, \varepsilon) = M = \varepsilon^{-1}(M)$ \checkmark
            \item $\delta^*(M, \beta x) = \delta(\delta^*(M, \beta), x) \stackrel{\text{ИП}}{=} x^{-1}(\beta^{-1}(M)) = \{ \gamma \in \Sigma^* \mid x \gamma \in \beta^{-1}(M) \} = \{ \gamma \in \Sigma^* \mid \beta x \gamma \in M \} = (\beta x)^{-1}(M)$
        \end{itemize}
    \end{proof}

    Имайки това:
    \begin{align*}
        \alpha \in L & \iff \varepsilon \in \alpha^{-1}(L)      \\
                     & \iff \varepsilon \in \delta^*(L, \alpha) \\
                     & \iff \delta^*(L, \alpha) \in F           \\
                     & \iff \alpha \in \mathcal{L(B)}
    \end{align*}
\end{proof}

Вече сме сигурни, че \nameref{nerode-nonregular} е напълно точен, за разлика от \nameref{pumping-lemma}:

\begin{corollary}
    $L$ е регулярен $\iff Q_L$ е крайно
\end{corollary}

Освен че за всеки регулярен език $L$ може да се построи такъв автомат, той се оказва и минимален.

\begin{claim}[Автоматът на Brzozowski е минимален]\thlabel{brzozowski-minimal-automaton}
    Нека $L$ е регулярен с автомат $\mathcal{A} = \opair{\Sigma, Q, s, \delta, F}$.
    Тогава $|Q_L| \leq |Q|$.
\end{claim}

\begin{proof}
    Б.О.О. всички състояния в $\mathcal{A}$ са достижими.

    Нека $Q = \{ 1, \dots, n \}$.
    Нека $\alpha_i \in \Sigma^*$ е такава дума, че $\delta^*(s, \alpha) = q_i$.
    Дефинираме функцията $f : Q \rightarrow Q_L$:
    \begin{center}
        $f(q_i) = \alpha^{-1}(L)$
    \end{center}
    Трябва само да покажем, че $f$ е сюрективна.

    Нека $M \in Q_L$.
    Тогава има $\alpha \in \Sigma^* \: : \: M = \alpha^{-1}(L)$.
    Знаем, че $\delta^*(s, \alpha) = q_i$ за някое $i \in \{ 1, \dots, n \}$.
    \begin{align*}
        \beta \in M \iff \beta \in \alpha^{-1}(L) & \iff \alpha \beta \in L                                                                                               \\
                                                  & \iff \delta^*(s, \alpha \beta) \in F \iff \delta^*(\delta^*(s, \alpha), \beta) \in F  \iff \delta^*(q_i, \beta) \in F \\
                                                  & \iff \delta^*(\delta^*(s, \alpha_i), \beta) \in F \iff \delta^*(s, \alpha_i \beta) \in F \iff \alpha_i \beta \in L    \\
                                                  & \iff \beta \in \alpha_i^{-1}(L) \iff \beta \in f(q_i)
    \end{align*}
    Така $f(q_i) = M$.
    Накрая понеже $f$ е сюрективна, $\operatorname{Dom}(f) = Q, \: \operatorname{Rng}(f) = Q_L, \: |Q_L| \leq |Q|$.
\end{proof}

\begin{problem}
Да се построи минимален автомат за $\regexlang{a^*b^*}$
\end{problem}

За такива задачи е хубаво да се имат на предвид следните свойства на регулярните изрази:
\begin{itemize}
    \item $\regexlang{r^+} = \regexlang{r \cdot r^*}$
    \item $\regexlang{r^*} = \regexlang{r^+ + \varepsilon}$
    \item $\regexlang{r_1 + r_2} = \regexlang{r_2 + r_1}$
    \item $\regexlang{r(r_1 + r_2)} = \regexlang{r \cdot r_1 + r \cdot r_2}$
\end{itemize}

Започваме да строим автомат на Brzozowski за $\regexlang{a^*b^*}$:
\begin{itemize}
    \item начално състояние $L_0 = \regexlang{a^*b^*} = \regexlang{a^+b^* + b^*} = \regexlang{a \cdot a^*b^* + b^*} = \regexlang{a \cdot a^*b^* + b^+ + \varepsilon} = \regexlang{a \cdot a^*b^* + b \cdot b^* + \varepsilon}$
    \item функцията на преходите е следната:
          \begin{itemize}
              \item $\delta(L_0, a) = a^{-1}(\regexlang{a \cdot a^*b^* + b \cdot b^* + \varepsilon}) = \regexlang{a^*b^*} = L_0$
              \item $\delta(L_0, b) = b^{-1}(\regexlang{a \cdot a^*b^* + b \cdot b^* + \varepsilon}) = \regexlang{b^*} = \regexlang{b \cdot b^* + \varepsilon}= L_1 \neq L_0$, понеже $a \in \regexlang{a^*b^*}, \: a \notin \regexlang{b^*}$
              \item $\delta(L_1, a) = a^{-1}(\regexlang{b \cdot b^* + \varepsilon}) = \varnothing = L_2 \neq L_0, L_1$
              \item $\delta(L_1, b) = b^{-1}(\regexlang{b \cdot b^* + \varepsilon}) = \regexlang{b^*} = L_1$
              \item $\delta(L_2, a) = a^{-1}(\varnothing) = \varnothing = b^{-1}(\varnothing) = \delta(L_2, a)$
          \end{itemize}
    \item $\varepsilon \in L_0, L_1$, следователно те стават финални
\end{itemize}

Ето как ще изглежда автомата на картинка:
\begin{figure}[h]
    \centering
    \begin{tikzpicture}[node distance=2cm, thick]
        \node[accepting, initial, state, initial text=] (1) {$L_0$};
        \node[accepting, state, initial text=, right of=1] (2) {$L_1$};
        \node[state, initial text=, right of=2] (3) {$L_2$};
        \path[->] (1) edge [loop above] node[above] {$a$} (1);
        \path[->] (1) edge [] node[above] {$b$} (2);
        \path[->] (2) edge [loop above] node[above] {$b$} (2);
        \path[->] (2) edge [] node[above] {$a$} (3);
        \path[->] (3) edge [loop above] node[above] {$a, b$} (3);
    \end{tikzpicture}
    \caption*{минимален автомат за $\regexlang{a^*b^*}$}
\end{figure}

Как бихме процедирали ако трябва да се построи минимален автомат за $\Sigma^* \setminus \regexlang{a^*b^*}$?

\begin{warning}
    За всеки регулярен език $L$, ако има автомат за $L$ с $n$ на брой състояния, то тогава има автомат за $\overline{L}$ с $n$ на брой състояния.
\end{warning}

Това го знаем още от \thref{complement-regular}.
В конструкцията там ние използваме същото множество от състояния, същото начално състояния и същите преходи.
Единственото, което променяме, е кои са финалните състояния.
Така можем да заключим, че за да получим минималният автомат за $\Sigma^* \setminus \regexlang{a^*b^*}$, трябва просто да приложим конструкцията от \thref{complement-regular} върху минималният автомат за $\regexlang{a^*b^*}$.
Това е много по-кратко от преминаването към регулярен израз за $\Sigma^* \setminus \regexlang{a^*b^*}$ и прилагането на алгоритъма на Brzozowski.