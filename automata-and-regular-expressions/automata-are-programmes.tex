\section{Друг начин да си мислим за автоматите}

Освен като абстрактни машини, можем да си мислим за детерминираните автомати като за програми.
Тези ``програми'' имат няколко много хубави свойства:
\begin{itemize}
    \item винаги завършват работа
    \item работят с константна памет
    \item работят за линейно време спрямо дължината на дума
\end{itemize}

Нека вземем за пример следният автомат:

\begin{center}
    \begin{tikzpicture}[shorten >=1pt,node distance=2.5cm,>=stealth',thick]
        \node[accepting, initial, state, initial text=] (1) {$0$};
        \node[state] [right of=1] (2) {$1$};
        \path[->] (1) edge [loop above] node[above] {$b$} (1);
        \path[->] (1) edge [bend left] node[above] {$a$} (2);
        \path[->] (2) edge [loop above] node[above] {$b$} (2);
        \path[->] (2) edge [bend left] node[below] {$a$} (1);
    \end{tikzpicture}
\end{center}

Ясно е, че той разпознава думи с четен брой букви $a$.
Нека видим как бихме направили програма на C++, която приема низ и връща дали този низ съдържа четен брой $a$:

\begin{minted}[linenos]{C++}
typedef char state_index;

const state_index table[2][2] = {{1, 0}, {0, 1}};   // таблица на преходите
const bool is_accepting_state[2] = { true, false }; // финални състояния
const state_index INITIAL_STATE = 0;                // начално състояние

// функция на преходите
state_index delta(state_index state, char letter) { return table[state][letter - 'a']; }

bool accepts_word(const std::string &word)
{
  // предполагаме валиден вход
  // т.е. низът съдържа само буквите 'a' и 'b'
  state_index state = INITIAL_STATE; // започваме в началното състояние

  for (size_t i = 0; i < word.size(); ++i)
  {
    state = delta(state, word[i]); // правим преход с настоящото състояние и буква
  }

  return is_accepting_state[state]; // накрая искаме да сме във финално състояние
}
\end{minted}

Лесно можем да видим как тази конструкция може да се адаптира за произволен автомат $\mathcal{A} = \opair{\Sigma, Q, q_0, \delta, F}$:
\begin{itemize}
    \item Нека $Q = \{ q_0, \dots q_{n - 1} \}$.
          За състоянията се разбираме, че на $q_i$ съответства числото $i$.
          За типа на \mintinline{C++}|state_index| ще ни трябва променлива, която ще побере числата от $0$ до $n - 1$.
    \item Нека $\Sigma = \{ \sigma_0, \dots, \sigma_{k - 1} \}$.
          Нужна ни е функция, която да е биекция между $\Sigma$ и $\{ 0, \dots, k - 1 \}$.
          Нека сигнатурата и да бъде \mintinline{C++}|state_index letter_to_index(char letter)|.
          Също така ще искаме да работи в константно време, което е лесно, защото азбуката е с фиксиран размер.
    \item Правим \mintinline{C++}|state_index table[n][k]| спрямо нашия оригинален автомат и кодировката от \mintinline{C++}|letter_to_index|:
          \begin{center}
              \mintinline{C++}|table[i][letter_to_index(c)] == j| е истина $\iff \delta^*(q_i, c) = q_j$
          \end{center}
    \item Правим \mintinline{C++}|bool is_accepting_state[n]| спрямо нашия оригинален автомат и кодировката от \mintinline{C++}|letter_to_index|:
          \begin{center}
              \mintinline{C++}|table[i]| е истина $\iff q_i \in F$
          \end{center}
    \item Понеже $q_0$ е началното състояние:
          \begin{minted}{C++}
const state_index INITIAL_STATE = 0;
          \end{minted}
    \item Правим програмна реализация на $\delta$ чрез \mintinline{C++}|table|:
          \begin{minted}{C++}
state_index delta(state_index state, char letter)
{
    return table[state][letter_to_index(letter)];
}
          \end{minted}
    \item Функцията \mintinline{C++}|accepts_word| няма нужда от модификация.
          В зависимост от това дали има нужда може да се сложи валидация на входните данни.
\end{itemize}

\begin{remark}
    Разбира се това не е единственият възможен начин да се направи програмна реализация на автомат.
    За по-малко на брой състояния също би било удобно да се използват оператори \mintinline{C++}|if| или \mintinline{C++}|switch| със \mintinline{C++}|case|.
    Обаче да имаме функцията $\delta$ записана по някакъв начин в таблица ни прави програмата по-бърза при повече състояния.
    Много по-лесно е да намериш елемент номер $10000$ в масив отколкото да оцениш $10000$ пъти оператор \mintinline{C++}|if|.
\end{remark}

\begin{problem}
Да се направи реализация използвайки \mintinline{C++}|if| или \mintinline{C++}|switch| със \mintinline{C++}|case| вместо таблица за $\delta$ на следния автомат:
\begin{center}
    \begin{tikzpicture}[shorten >=1pt,node distance=2.5cm,>=stealth',thick]
        \node[accepting, initial, state, initial text=] (1) {$0$};
        \node[state] [right of=1] (2) {$1$};
        \path[->] (1) edge [loop above] node[above] {$b$} (1);
        \path[->] (1) edge  node[above] {$a$} (2);
        \path[->] (2) edge [loop above] node[above] {$a, b$} (2);
    \end{tikzpicture}
\end{center}
\end{problem}

\begin{problem}
Да се направи програмна реализация и да се определи езика на следните автомати:
\begin{figure*}[h]
    \begin{subfigure}{1.0\textwidth}
        \centering
        \begin{tikzpicture}[shorten >=1pt,node distance=2.5cm,>=stealth',thick]
            \node[initial, state, initial text=] (1) {$0$};
            \node[accepting, state] [right of=1] (2) {$1$};
            \path[->] (1) edge [loop above] node[above] {$a$} (1);
            \path[->] (1) edge [bend left] node[above] {$b$} (2);
            \path[->] (2) edge [loop above] node[above] {$a$} (2);
            \path[->] (2) edge [bend left] node[below] {$b$} (1);
        \end{tikzpicture}
    \end{subfigure}
    \vspace{\floatsep}
    \begin{subfigure}{1.0\textwidth}
        \centering
        \begin{tikzpicture}[shorten >=1pt,node distance=2.5cm,>=stealth',thick]
            \node[initial, state, initial text=] (1) {$0$};
            \node[accepting, state] [right of=1] (2) {$1$};
            \node[accepting, state] [right of=2] (3) {$2$};
            \path[->] (1) edge [loop above] node[above] {$a$} (1);
            \path[->] (1) edge node[above] {$b$} (2);
            \path[->] (2) edge [loop above] node[above] {$a$} (2);
            \path[->] (2) edge  node[above] {$b$} (3);
            \path[->] (3) edge [loop above] node[above] {$a$} (3);
            \path[->] (3) edge [bend left] node[below] {$b$} (1);
        \end{tikzpicture}
    \end{subfigure}
    \vspace{\floatsep}
    \begin{subfigure}{1.0\textwidth}
        \centering
        \begin{tikzpicture}[node distance=2.5cm,>=stealth',thick]
            \node[accepting, initial, state with output, initial text=] (1) {$0$ \nodepart{lower} $0$};
            \node[state with output] [right of=1] (2) {$0$ \nodepart{lower} $1$};
            \node[state with output, initial text=] [below of=1](3) {$1$ \nodepart{lower} $0$};
            \node[accepting, state with output] [right of=3] (4) {$1$ \nodepart{lower} $1$};
            \path[->] (1) edge [bend left] node[above] {$a$} (2);
            \path[->] (2) edge [bend left] node[below] {$a$} (1);
            \path[->] (3) edge [bend left] node[above] {$a$} (4);
            \path[->] (4) edge [bend left] node[below] {$a$} (3);
            \path[->] (1) edge [bend left] node[right] {$b$} (3);
            \path[->] (3) edge [bend left] node[left] {$b$} (1);
            \path[->] (2) edge [bend left] node[right] {$b$} (4);
            \path[->] (4) edge [bend left] node[left] {$b$} (2);
        \end{tikzpicture}
    \end{subfigure}
\end{figure*}
\end{problem}
