\section{Операции над езици на недетерминирани автомати}

Тази конструкция може да се обобщи за обединение на всеки два езика на недетерминирани автомати.

\begin{claim}
    Нека $\mathcal{N}_i = \opair{\Sigma, Q_i, S_i, \Delta_i, F_i}$  е недетерминиран автомат за $i = 1, 2$.
    Тогава съществува недетерминиран автомат за езика $\mathcal{L(N}_1) \cup \mathcal{L(N}_2)$.
\end{claim}

\begin{proof}
    Б.О.О. нека $\underbrace{Q_1 \cap Q_2 = \varnothing}_{(\star)}$. Тогава:
    \begin{align*}
        \alpha \in \mathcal{L}(\opair{\Sigma, Q_1 \cup Q_2, S_1 \cup S_2, \underbrace{\Delta_1 \cup \Delta_2}_{\textcolor{red}{!!!}}, F_1 \cup F_2}) \iff & (\Delta_1 \cup \Delta_2)(S_1 \cup S_2, \alpha) \cap (F_1 \cup F_2) \neq \varnothing                      \\
        \stackrel{(\star)}{\iff}                                                                                                                          & \Delta_1^*(S_1, \alpha) \cap F_1 \neq \varnothing \lor \Delta_2^*(S_2, \alpha) \cap F_2 \neq \varnothing \\
        \iff                                                                                                                                              & \alpha \in \mathcal{L(N}_1) \lor \alpha \in \mathcal{L(N}_2)                                             \\
        \iff                                                                                                                                              & \alpha \in \mathcal{L(N}_1) \cup \mathcal{L(N}_2)
    \end{align*}
    \begin{remark}[\textcolor{red}{!!!}]
        Понеже $(\star)$ е вярно, $\operatorname{Dom}(\Delta_1) \cap \operatorname{Dom}(\Delta_2) = \varnothing$, откъдето $\Delta_1 \cup \Delta_2$ е добре дефинирана функция.
    \end{remark}
\end{proof}

Тази конструкция при недетерминирани автомати е по-компактна в сравнение с детерминираната версия.
В този случай новите състояния са $|Q_1 \cup Q_2| = |Q_1| + |Q_2|$ на брой,
докато при детерминираните автомати се получават $|Q_1 \crossproduct Q_2| = |Q_1| \cdot |Q_2|$ на брой състояния за новия автомат, което може да бъде много повече.

\begin{itemize}
    \item $10 + 10 = 20$, докато $10 \cdot 10 = 100$
    \item $1000 + 2 = 1002$, докато $1000 \cdot 2 = 2000$
    \item $1000 + 1000 = 2000$, докато $1000 \cdot 1000 = 1000000$
\end{itemize}

\begin{claim}
    Нека $L$ е автоматен език.
    Тогава има недетерминиран автомат за:
    \begin{center}
        $\operatorname{ChangeSomeLetters}(L) = \{ \alpha \in \Sigma^* \mid (\exists \beta \in L) \: (|\beta| = |\alpha| ) \}$
    \end{center}
\end{claim}

\begin{proof}
    Нека $\mathcal{A} = \opair{\Sigma, Q, s, \delta, F}$ е ДКА за $L$.
    Строим автомат $\mathcal{N}$ за $\operatorname{ChangeSomeLetters}(L)$:
    \begin{itemize}
        \item $\mathcal{N} = \opair{\Sigma, Q, S, \Delta, F}$
        \item $S = \{ s \}$
        \item $\Delta(p, x) = \{ \delta(p, a), \delta(p, b) \}$ за $p \in Q, \: x \in \Sigma$
    \end{itemize}

    Оставяме доказателството на следния факт на читателя, понеже е елементарна индукция:
    \begin{center}
        $\Delta^*(S, \alpha) = \{ \delta^*(s, \beta) \mid \beta \in \Sigma \: \& \: |\beta| = |\alpha| \}$
    \end{center}

    Имайки това твърдение получаваме, че:
    \begin{align*}
        \alpha \in \mathcal{L(N)} & \iff \Delta^*(S, \alpha) \cap F \neq \varnothing \iff (\exists \beta \in \Sigma^{|\alpha|}) \: (\delta^*(s, \beta) \in F) \\
                                  & \iff (\exists \beta \in \Sigma^{|\alpha|}) \: (\beta \in L) \iff \alpha \in \operatorname{ChangeSomeLetters}(L)
    \end{align*}
\end{proof}

\begin{claim}
    Нека $\mathcal{N}_i = \opair{\Sigma, Q_i, S_i, \Delta_i, F_i}$  е недетерминиран автомат за $i = 1, 2$.
    Тогава съществува недетерминиран автомат за езика $\mathcal{L(N}_1) \cdot \mathcal{L(N}_2)$.
\end{claim}

\begin{proof}
    Нека Б.О.О. $Q_1 \cap Q_2 = \varnothing$.
    Строим недетерминиран автомат $\mathcal{N} = \opair{\Sigma, Q, S, \Delta, F}$ за $\mathcal{L(N}_1) \cdot \mathcal{L(N}_2)$:
    \begin{itemize}
        \item $Q = Q_1 \cup Q_2$
        \item $S = S_1 \cup S_2$ ако $\varepsilon \in \mathcal{L(N}_1)$, иначе $S = S_1$
        \item $F = F_2$
        \item $\Delta(p, x) = \Delta_1(p, x) \cup S_2$, ако $\Delta_1(p, x) \cap F \neq \varnothing \: (\star)$
        \item За другите състояния от $Q_1$ взимаме преходите от $\Delta_1$, и аналогично за $Q_2$ взимаме същите преходи от $\Delta_2$
    \end{itemize}

    Сега остава да покажем, че $\mathcal{L(N)} = \mathcal{L(N}_1) \cdot \mathcal{L(N}_2)$.
    \begin{itemize}
        \item $\mathcal{L(N)} \subseteq \mathcal{L(N}_1) \cdot \mathcal{L(N}_2)$ \\
              Нека $\alpha \in \mathcal{L(N)}$.
              Тогава $\Delta^*(S, \alpha) \cap F \neq \varnothing$.
              Ако сме стигнали до финално от $S_2$, то тогава $\alpha \in \mathcal{L(N}_2)$ и $\varepsilon \in \mathcal{L(N}_1)$,
              откъдето $\alpha = \varepsilon \cdot \alpha \in \mathcal{L(N}_1) \cdot \mathcal{L(N}_2)$ и сме готови.
              В противен случай $\varepsilon \notin \mathcal{L(N}_1)$ и има $\beta, \gamma \in \Sigma^*$ такива,
              че $\alpha = \beta \gamma, \: \beta \neq \varepsilon, \: S_2 \subseteq \Delta^*(S, \beta)$,  т.е. сме приложили $(\star)$.
              Тогава с последната буква на $\beta$ сме достигнали до финално състояние в $\mathcal{L(N}_1)$, откъдето $\beta \in \mathcal{L(N}_1)$.
              Също така понеже $Q_1 \cap Q_2 = \varnothing$, и състоянията от $Q_2$ имат само преходите на $\Delta_2$,
              $\Delta(S_2, \gamma) \cap F \neq \varnothing$, откъдето $\gamma \in \mathcal{L(N}_2)$.
              Така получаваме, че $\alpha = \beta \cdot \gamma \in \mathcal{L(N}_1) \cdot \mathcal{L(N}_2)$.

        \item $\mathcal{L(N}_1) \cdot \mathcal{L(N}_2) \subseteq \mathcal{L(N)}$ \\
              Нека $\beta \in \mathcal{L(N}_1), \gamma \in \mathcal{L(N}_2)$.
              Ако $\beta = \varepsilon$, то тогава $S_2 \subseteq S$, и понеже $\gamma \in \mathcal{L(N}_2), \: F = F_2$,
              то $\Delta^*(S, \underbrace{\beta \cdot \gamma}_{\gamma}) \cap F \neq \varnothing$, и от там $\beta \cdot \gamma \in \mathcal{L(N)}$.
              Ако $\beta \neq \varepsilon$, то $S_2 \subseteq \Delta^*(S, \beta)$, понеже $\beta \in \mathcal{L(N}_1)$ и $(\star)$.
              Освен това и $\gamma \in \mathcal{L(N}_2)$ т.е. $\Delta^*(S_2, \gamma) \cap F_2 \neq \varnothing$,
              откъдето $\Delta^*(S, \beta \cdot \gamma) \cap F \neq \varnothing$ т.е. $\beta \cdot \gamma \in \mathcal{L(N)}$.
    \end{itemize}
\end{proof}

\begin{claim}
    Нека $\mathcal{N} = \opair{\Sigma, Q, S, \Delta, F}$ е недетерминиран автомат.
    Тогава има недетерминиран автомат за езика $\mathcal{L(N)^*}$.
\end{claim}

\begin{proof}
    Ще покажем само конструкцията, без да я обосноваваме (аналогична е на предната).
    Правим автомат за $\mathcal{L(N)^+}$ и използваме, че $L^* = L^+ \cup \{ \varepsilon \}$ за всеки език $L$.
    Нека $\mathcal{N}' = \opair{\Sigma, Q, S, \Delta', F}$, където:
    \begin{equation}
        \Delta'(p, x) =
        \begin{cases}
            \Delta(p, x)                     & \text{ако } p \notin F \\
            \Delta(p, x) \cup \Delta^*(S, x) & \text{ако } p \in F
        \end{cases}
    \end{equation}
    Доказателството на коректност е аналогично на предната задача.
\end{proof}

\begin{claim}
    Нека $\mathcal{N} = \opair{\Sigma, Q, S, \Delta, F}$ е недетерминиран автомат.
    Тогава има недетерминиран автомат за езика $\mathcal{L(N)}^{rev}$.
\end{claim}

\begin{proof}
    Нека $\mathcal{N}_{rev} = \opair{\Sigma, Q, F, \Delta_{rev}, S}$, където:
    \begin{center}
        $\Delta_{rev} = \{ \opair{p, x, q} \mid p \in \Delta(q, x) \}$
    \end{center}
    Тривиално може да се покаже с индукция, че за $\alpha \in \Sigma^*, \: p, q \in Q$:
    \begin{center}
        $q \in \Delta^*(\{ p \}, \alpha) \iff p \in \Delta_{rev}^*(\{ q \}, \alpha) \: (\star)$
    \end{center}
    След това приключваме със:
    \begin{align*}
        \alpha \in \mathcal{L(N)} \iff & \Delta^*(S, \alpha) \cap F \neq \varnothing                                            \\
        \stackrel{(\star)}{\iff}       & \Delta_{rev}^*(F, \alpha) \cap S \neq \varnothing \iff \alpha \in \mathcal{L(N}_{rev})
    \end{align*}
\end{proof}