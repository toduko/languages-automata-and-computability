\section{Регулярни изрази и регулярни езици}

Вече знаем как да превръщаме автоматите в реални програми.
Това даде живот на нашата концептуална машина за валидация на текст.
Ние не само имаме програмна реализация, тя също така има много хубави свойства.
Въпреки всичко това има нещо което ни липсва.
Понякога бихме искали да представим тази информация в по-компактен вариант от една картинка или програма.
Тук идват на помощ регулярните изрази.

\begin{definition}
    Дефинираме множеството от \textbf{регулярни изрази} $\mathcal{RE}(\Sigma)$ индуктивно:
    \begin{itemize}
        \item $\varnothing, \varepsilon \in \mathcal{RE}(\Sigma)$ и $\sigma \in \mathcal{RE}(\Sigma)$ за всяко $\sigma \in \Sigma$
        \item ако $r_1, r_2 \in \mathcal{RE}(\Sigma)$, то тогава $r_1 \cdot r_2, r_1 + r_2 \in \mathcal{RE}(\Sigma)$
        \item ако $r \in \mathcal{RE}(\Sigma)$, то тогава $r^* \in \mathcal{RE}(\Sigma)$
    \end{itemize}
\end{definition}

\begin{remark}
    Освен това имаме и скоби за приоритет на операциите.
    При липса на скоби с най-голям приоритет е $*$, после $\cdot$, и накрая $+$.
    Също така за краткост понякога ще изпускаме $\cdot$ както се прави при умножението.
\end{remark}

\begin{definition}
    Дефинираме индуктивно \textbf{езика $\regexlang{r}$ на регулярен израз $r \in \mathcal{RE}(\Sigma)$}:
    \begin{itemize}
        \item $\regexlang{\varnothing} = \varnothing$ и $\regexlang{x} = \{ x \}$ за всяко $x \in \Sigma_{\varepsilon} \: (\Sigma \cup \{ \varepsilon \})$
        \item $\regexlang{r_1 \cdot r_2} = \regexlang{r_1} \cdot \regexlang{r_2}$
        \item $\regexlang{r_1 \cup r_2} = \regexlang{r_1} \cup \regexlang{r_2}$
        \item $\regexlang{r^*} = \regexlang{r}^*$
    \end{itemize}
\end{definition}

\begin{definition}
    Език $L \subseteq \Sigma^*$ наричаме \textbf{регулярен}, ако съществува регулярен израз $r \in \mathcal{RE}(\Sigma)$ такъв, че $\regexlang{r} = L$
\end{definition}

Ето няколко прости примери за регулярни изрази и техните езици:
\begin{itemize}
    \item $\regexlang{(a+b)^*} = \Sigma^*$
    \item $\regexlang{((a+b)(a+b))^*} =(\Sigma^2)^*$
    \item $\regexlang{(b^*ab^*ab^*)^*} = \{ \alpha \in \Sigma^* \mid |\alpha| \equiv 0 \pmod{2} \}$
    \item $\regexlang{b^*ab^*a(b^*ab^*ab^*ab^*)^*} = \{ \alpha \in \Sigma^* \mid |\alpha| \equiv 2 \pmod{3} \}$
\end{itemize}

Първите две са очевидни.
Ще обосновем много накратко третия пример и ще оставим четвъртия на читателя.
Едната посока е:
\begin{center}
    $\regexlang{(b^*ab^*ab^*)^*} \subseteq \{ \alpha \in \Sigma^* \mid |\alpha| \equiv 0 \pmod{2} \}$
\end{center}
Със $*$ наслагваме ``блокчета'' от $b^*ab^*ab^*$, които са все думи със две на брой букви $a$:
\begin{center}
    \begin{tabular}{|c|c|c|c|}
        \hline
        $b \dots b \: a \: b \dots b \: a \: b \dots b$                                                                                                                      &
        $b \dots b \: a \: b \dots b \: a \: b \dots b$                                                                                                                      &
        $\dots$                                                                                                                                                              &
        $b \dots b \: a \: b \dots b \: a \: b \dots b$                                                                                                                        \\
        \hline
        \multicolumn{1}{@{}c@{}}{$\underbrace{\hspace*{\dimexpr\tabcolsep+2\arrayrulewidth}\hphantom{b \dots b \: a \: b \dots b \: a \: b \dots b}}_{+2 \text{ букви } a}$} &
        \multicolumn{1}{@{}c@{}}{$\underbrace{\hspace*{\dimexpr\tabcolsep+2\arrayrulewidth}\hphantom{b \dots b \: a \: b \dots b \: a \: b \dots b}}_{+2 \text{ букви } a}$} &
        \multicolumn{1}{@{}c@{}}{}                                                                                                                                           &
        \multicolumn{1}{@{}c@{}}{$\underbrace{\hspace*{\dimexpr\tabcolsep+2\arrayrulewidth}\hphantom{b \dots b \: a \: b \dots b \: a \: b \dots b}}_{+2 \text{ букви } a}$}
    \end{tabular}
\end{center}
Колкото и пъти да събираме четни числа, винаги като резултат ще получим четно число.
Думата очевидно е от $\{ \alpha \in \Sigma^* \mid |\alpha| \equiv 0 \pmod{2} \}$.
Другата посока е:
\begin{center}
    $\{ \alpha \in \Sigma^* \mid |\alpha| \equiv 0 \pmod{2} \} \subseteq \regexlang{(b^*ab^*ab^*)^*}$
\end{center}
Ако вземем дума от ляво, можем да я разделим на всеки две срещания на буквата $a$:
\begin{center}
    \begin{tabular}{|c|c|c|c|}
        \hline
        $b \dots b \: a \: b \dots b \: a$                                                                                                                              &
        $b \dots b \: a \: b \dots b \: a$                                                                                                                              &
        $\dots$                                                                                                                                                         &
        $b \dots b \: a \: b \dots b \: a \: b \dots b$                                                                                                                   \\
        \hline
        \multicolumn{1}{@{}c@{}}{$\underbrace{\hspace*{\dimexpr\tabcolsep+2\arrayrulewidth}\hphantom{b \dots b \: a \: b \dots b \: a}}_{\in \regexlang{b^*ab^*ab^*}}$} &
        \multicolumn{1}{@{}c@{}}{$\underbrace{\hspace*{\dimexpr\tabcolsep+2\arrayrulewidth}\hphantom{b \dots b \: a \: b \dots b \: a}}_{\in \regexlang{b^*ab^*ab^*}}$} &
        \multicolumn{1}{@{}c@{}}{}                                                                                                                                      &
        \multicolumn{1}{@{}c@{}}{$\underbrace{\hspace*{\dimexpr\tabcolsep+2\arrayrulewidth}\hphantom{b \dots b \: a \: b \dots b \: a \: b \dots b}}_{\in \regexlang{b^*ab^*ab^*}}$}
    \end{tabular}
\end{center}
Вече е очевидно, че думата е от $\regexlang{(b^*ab^*ab^*)^*}$.