\chapter{Автомати и регулярни изрази}

Тук ще разгледаме първата ``машина'', с която ще класифицираме езиците.

\section{Детерминирани автомати и автоматни езици}

\begin{definition}
    \textbf{Детерминиран краен автомат} (накратко ДКА) ще наричаме всяко $\mathcal{A} = \opair{\Sigma, Q, s, \delta, F}$, където:
    \begin{itemize}
        \item $\Sigma$ е крайна азбука
        \item $Q$ е крайно множество от състояния
        \item $s \in Q$ (ще го наричаме начално/стартово състояние)
        \item $\delta : Q \cross \Sigma \rightarrow Q$ (ще я наричаме функция на преходите)
        \item $F \subseteq Q$ (ще ги наричаме финални състояния)
    \end{itemize}
\end{definition}

В момента със това, което имаме,
ако искаме да покажем къде ще стигнем с думата $aaa$, започвайки от $s$,
ще трябва да го запишем със $\delta(\delta(\delta(s, a), a), a)$, което е тромаво.

За това ще си въведем начин, по който да видим крайният резултат от прочитането на цяла дума.

\begin{definition}
    Дефинираме $\delta^* : Q \cross \Sigma^* \rightarrow Q$ индуктивно:
    \begin{itemize}
        \item $\delta^*(p, \varepsilon) = p$ за всяко $p \in Q$
        \item $\delta^*(p, \beta x) = \delta(\delta^*(p, \beta), x)$ за всяко $p \in Q, \beta \in \Sigma^*, x \in \Sigma$
    \end{itemize}
\end{definition}

Сега можем да забележим, че:
\begin{align*}
    \delta^*(s, aaa) & = \delta(\delta^*(s, aa), a) = \delta(\delta(\delta^*(s, a), a), a) =                           \\
                     & = \delta(\delta(\delta(\delta^*(s, \varepsilon), a), a), a) =\delta(\delta(\delta(s, a), a), a)
\end{align*}
Функцията наистина прави това, което искаме да прави.

\begin{claim}
    $\delta^*(p, \alpha \beta) = \delta^*(\delta^*(p, \alpha), \beta)$
\end{claim}

\begin{proof}
    С индукция по $|\beta|$.
    \begin{itemize}
        \item $\delta^*(p, \alpha \varepsilon) = \delta^*(p, \alpha) = \delta^*(\delta^*(p, \alpha), \varepsilon)$
        \item $\delta^*(p, \alpha \beta x) = \delta(\delta^*(p, \alpha \beta), x) \stackrel{\text{ИП}}{=} \delta(\delta^*(\delta^*(p, \alpha), \beta), x) = \delta^*(\delta^*(p, \alpha), \beta x)$
    \end{itemize}
\end{proof}

\begin{definition}
    Нека $\mathcal{A} = \opair{\Sigma, Q, s, \delta, F}$ е ДКА.
    Тогава езикът на автомата $\mathcal{A}$ ще наричаме множеството
    $\mathcal{L}(\mathcal{A}) = \{ \alpha \in \Sigma^* \: | \: \delta^*(s, \alpha) \in F \}$.
    Един език $L \subseteq \Sigma^*$, наричаме \textbf{автоматен}, ако има ДКА $\mathcal{A}$ с $\mathcal{L}(\mathcal{A}) = L$.
\end{definition}

\section{Представяне на автомат}

Ще видим начините, по които можем да представяме един автомат.
За пример ще вземем $\mathcal{A} = \opair{\Sigma, Q, s, \delta, F}$, където:
\begin{itemize}
    \item $\Sigma = \{ a, b \}$
    \item $Q = \{ q_0, q_1, q_2 \}$
    \item $s = q_0$
    \item $\delta(q_0, a) = q_2$
    \item $\delta(p, x) = q_1$ за $p \in Q, \: x \in \Sigma, \: \opair{p, x} \neq \opair{q_0, a}$
    \item $F = \{ q_2 \}$
\end{itemize}

Това е първият начин за представяне.
При него директно в явен вид се казват кои са всички съставни елементи на $\mathcal{A}$.
Това ще бъде използване сравнително често, когато правим от един автомат друг.
В случаите, в които не знаем как изглежда първоначалният автомат, следващите методи тогава няма да свършат работа. \\

Вторият начин за представяне е с таблица на функцията на преходите:
\begin{center}
    \begin{tabular}{||r | r | r||}
        \hline
        \cellcolor{lightgray} & $a$   & $b$   \\
        \hline
        $\rightarrow q_0$     & $q_2$ & $q_1$ \\
        \hline
        $q_1$                 & $q_1$ & $q_1$ \\
        \hline
        $\checkmark q_2$      & $q_1$ & $q_1$ \\
        \hline
    \end{tabular}
\end{center}
Тук с $\rightarrow$ отбелязваме началното състояние,
а с $\checkmark$ отбелязваме финалните състояния.
Във втората и третата колона казваме къде ще се озовем,
ако се намираме в съответното състояние и прочетем съответната буква.
Това е по-прегледно представяне от първото, но може да стане обемно. \\

Третият начин за представяне е с картинка:

\begin{center}
    \begin{tikzpicture}[shorten >=1pt,node distance=2cm,>=stealth',thick]
        \node[initial, state, initial text=] (1) {$q_0$};
        \node[state] (2) [above right of=1] {$q_1$};
        \node[state, accepting] (3) [below right of=1] {$q_2$};
        \draw[->] (1) -- node[above] {$b$} (2);
        \draw[->] (1) -- node[above] {$a$} (3);
        \path (2) edge [loop above] node[above] {$a, b$} (2);
        \draw[->] (3) -- node[right] {$a, b$} (2);
    \end{tikzpicture}
\end{center}

Началното състояние е отбелязано със стрелка, а финалните са оградени два пъти.
Представянето чрез картинка изглежда по-прегледно от другите две, но то може и да стане огромно.

Ясно е, че $\mathcal{L}(\mathcal{A}) = \{ a \}$.
Ако искаме да покажем това формално, трябва да направим следното:
\begin{itemize}
    \item Очевидно $a \in \mathcal{L}(\mathcal{A}), \: b \notin \mathcal{L}(\mathcal{A}), \: \varepsilon \notin \mathcal{L}(\mathcal{A})$
    \item Показваме, че $\delta^*(q_1, \alpha) = q_1$ за всяко $\alpha \in \Sigma^*$ с индукция по $|\alpha|$
    \item Ако $\alpha \notin \{ a, b, \varepsilon \}$, то $\alpha = x y \beta$ за $x, y \in \Sigma, \: \beta \in \Sigma^*$
    \item Тогава лесно се вижда, че $\delta^*(q_0, \alpha) = q_1 \notin F$.
          Или директно отиваме в $q_1$ (ако $x = b$) и оставаме там, или първо отиваме в $q_2$ (ако $x = a$), и после със следващата буква отиваме в $q_1$ и оставаме там.
\end{itemize}

Това обаче за такива прости автомати не е нужно.
Тези неща в такива случаи ще ги приемаме за очевидни.