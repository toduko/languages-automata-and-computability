\section{Прости примери за автоматни езици}

Нека видим много прости примери за автоматни езици,
за да добием малко интуиция какви езици могат да бъдат автоматни,
и да поработим малко със самите машини. Най-простите автоматни езици са $\varnothing$ и $\Sigma^*$:

\begin{center}
    \begin{tikzpicture}[node distance=2cm, thick]
        \node[text width=3.0cm] at (0.35, -1) {автомат за $\varnothing$};
        \node[initial, state, initial text=] (1) {$q$};
        \path[->] (1) edge [loop above] node[above] {$a, b$}(1);
    \end{tikzpicture}
    \hspace{2cm}
    \begin{tikzpicture}[node distance=2cm, thick]
        \node[text width=3.0cm] at (0.35, -1) {автомат за $\Sigma^*$};
        \node[initial, accepting, state, initial text=] (1) {$q$};
        \path[->] (1) edge [loop above] node[above] {$a, b$}(1);
    \end{tikzpicture}
\end{center}

Сега малко ще усложним нещата.
Ще направим автомат, който да разпознае езикът от една конкретна дума.
Конструкцията за конкретната дума много лесно се обобщава за всички.
За пример нека направим автомат за $L = \{ abab \}$:

\begin{center}
    \begin{tikzpicture}[shorten >=1pt,node distance=2.5cm,>=stealth',thick]
        \node[initial, state, initial text=] (1) {$\varepsilon$};
        \node[state] [right of=1] (2) {$a$};
        \node[state] [right of=2] (3) {$ab$};
        \node[state] [right of=3] (4) {$aba$};
        \node[state, accepting] [right of=4] (5) {$abab$};
        \node[state] [below of=3] (6) {$\cross$};
        \draw[->] (1) -- node[above] {$a$} (2);
        \draw[->] (2) -- node[above] {$b$} (3);
        \draw[->] (3) -- node[above] {$a$} (4);
        \draw[->] (4) -- node[above] {$b$} (5);
        \path[->] (1) edge [bend right] node[below] {$b$} (6);
        \path[->] (2) edge [bend right] node[below] {$a$} (6);
        \path[->] (3) edge  node[right] {$b$} (6);
        \path[->] (4) edge [bend left] node[below] {$a$} (6);
        \path[->] (5) edge [bend left] node[below] {$a, b$} (6);
        \path[->] (6) edge [loop below] node[below] {$a, b$} (6);
    \end{tikzpicture}
\end{center}

Тази техника може да се приложи за строене на автомат за език от произволно дълга дума.
Ето как ще обобщим конструкцията за езика $L = \{ \alpha \}$:
\begin{itemize}
    \item $\mathcal{A} = \opair{\Sigma, Q, s, \delta, F}$
    \item $Q = \operatorname{Pref}(L) \cup \{ \cross \}$ (начален отрязък от пътя, който съставя $\alpha$, $\cross \notin \Sigma$ е отхвърлящо състояние боклук)
    \item $s = \varepsilon$
    \item За $\beta \preceq_{pref} \alpha \: ($тогава $\beta \in Q), \: x \in \Sigma$:
          ако $\beta x \preceq_{pref} \alpha$, то $\delta(\beta, x) = \beta x$,
          иначе $\delta(\beta, x) = \cross$
    \item $\delta(\cross, x) = \cross$ за $x \in \Sigma$
    \item $F = \{ \alpha \}$
\end{itemize}

Идеята зад конструкцията е следната:
Състоянията кодират пътя, който сме изминали, ако не сме се отклонили вече от ``строежа'' на $\alpha$.
В първият момент на отклонение отиваме във нефинално състояние ``боклук'', от което не може да излезнем.

Можем да разширим идеята, така че да работи за няколко думи като кодираме повече видове пътища.
Ето как ще обобщим конструкцията (чиято коректност приемаме за очевидна) за $L = \{ \alpha_1, \: \dots, \: \alpha_n \}$:
\begin{itemize}
    \item $\mathcal{A} = \opair{\Sigma, Q, s, \delta, F}$
    \item $Q = \{ \alpha \in \Sigma^* \: | \: (\exists \beta \in L) (|\alpha| \leq |\beta|) \} \cup \{ \cross \}$ (не гледаме пътища по-дълги от най-дългата дума в $L$, $\cross \notin \Sigma$ - състояние боклук)
    \item $s = \varepsilon$
    \item За $\alpha \in Q, \: x \in \Sigma$:
          ако $\alpha x \in Q$, то $\delta(\alpha, x) = \alpha x$,
          иначе $\delta(\alpha, x) = \cross$
    \item $\delta(\cross, x) = \cross$ за $x \in \Sigma$
    \item $F = L$
\end{itemize}

\begin{figure*}
    \centering
    \begin{tikzpicture}[shorten >=1pt,node distance=2.5cm,>=stealth',thick]
        \node[initial, state, initial text=] (1) {$\varepsilon$};
        \node[state] [above right of=1] (2) {$a$};
        \node[state] [below right of=1] (3) {$b$};
        \node[state, accepting] [above right of=2] (4) {$aa$};
        \node[state, accepting] [right of=2] (5) {$ab$};
        \node[state, accepting] [right of=3] (6) {$ba$};
        \node[state] [below right of=3] (7) {$bb$};
        \node[state] [below right of=5] (8) {$\cross$};
        \draw[->] (1) -- node[above] {$a$} (2);
        \draw[->] (1) -- node[above] {$b$} (3);
        \draw[->] (2) -- node[above] {$a$} (4);
        \draw[->] (2) -- node[above] {$b$} (5);
        \draw[->] (3) -- node[above] {$a$} (6);
        \draw[->] (3) -- node[above] {$b$} (7);
        \path[->] (4) edge [bend left] node[right] {$a, b$} (8);
        \path[->] (5) edge node[left] {$a, b$} (8);
        \path[->] (6) edge node[left] {$a, b$} (8);
        \path[->] (7) edge [bend right] node[right] {$a, b$} (8);
        \path[->] (8) edge [loop right] node[right] {$a, b$} (8);
    \end{tikzpicture}
    \caption*{пример за прилагане на конструкцията за $L = \{ ab, ba, aa \}$}
\end{figure*}

Нека сега направим автомат за $L = \{ \alpha \in \Sigma^* \: | \: |\alpha| \: \text{е четно} \}$.

За една дума $\alpha \in \Sigma^*$ знаем, че тя или има четна дължина или има нечетна дължина.
Дали не можем да кодираме по някакъв начин четността на прочетената дума в състояние?
Отговорът е, че можем. Автоматът е следния:

\begin{center}
    \begin{tikzpicture}[shorten >=1pt,node distance=2.5cm,>=stealth',thick]
        \node[text width=0.5cm] at (-1, 1) {$\mathcal{A}$:};
        \node[accepting, initial, state, initial text=] (1) {$0$};
        \node[state] [right of=1] (2) {$1$};
        \path[->] (1) edge [bend left] node[above] {$a, b$} (2);
        \path[->] (2) edge [bend left] node[below] {$a, b$} (1);
    \end{tikzpicture}
\end{center}

Знаем, че $|\varepsilon|$ е четно.
За всякo $\alpha \in \Sigma^*, \: x \in \Sigma$ знаем,
че $|\alpha x|$ има различна четност от $|\alpha|$.
Така ние започваме с думата $\varepsilon$ и 0 като четност на думата и за всяка буква сменяме четността.

Нека сега помислим как да докажем, че $\mathcal{L(A)} = L$.
За това ще трябва да покажем, че започвайки от $0$ и четейки $\alpha$ ние наистина получаваме четността на $|\alpha|$ като състояние.

\begin{claim}
    За всяко $\alpha \in \Sigma^*$ : $\delta^*(0, \alpha) = |\alpha| \: (mod \: 2)$
\end{claim}

\begin{proof}
    С индукция по $|\alpha|$.
    \begin{itemize}
        \item $|\alpha| = 0$, тогава $\alpha = \varepsilon$.
              Наистина $\delta^*(0, \varepsilon) = 0 \: (mod \: 2)$ \checkmark
        \item $|\alpha| = n + 1$, тогава $\alpha = \beta x$ за $\beta \in \Sigma^*, |\beta| = n, \: x \in \Sigma$.
              Тогава:
              \begin{center}
                  $\delta^*(0, \beta x) = \delta(\delta^*(0, \beta), x) \stackrel{\text{ИП}}{=} \delta(|\beta| \: (mod \: 2), x) = |\beta| + 1 \: (mod \: 2) = |\beta x| \: (mod \: 2) = |\alpha| \: (mod \: 2)$
              \end{center}
    \end{itemize}
\end{proof}

Състоянията наистина кодират информацията, която искахме.
Имайки това можем да покажем, че двата езика съвпадат, по следния начин:
\begin{center}
    $\alpha \in \mathcal{L(A)} \iff \delta^*(0, \alpha) = 0 \iff |\alpha| \: (mod \: 2) = 0 \iff \alpha \in L$
\end{center}


Нека сега помислим какъв автомат ще можем да направим за $\overline{L}$.
Ние вече имаме автомат за $L$.
От неговите състояние и преходи можем да извлечем информация за четността на дължината на думата.
Единственото, което трябва да направим, е да сменим отговора на автомата.
Ако преди той е казвал ДА, сега да казва НЕ, и обратното.

\pagebreak

Това означава просто да обърнем финалните състояния:
\begin{center}
    \begin{tikzpicture}[shorten >=1pt,node distance=2.5cm,>=stealth',thick]
        \node[text width=0.5cm] at (-1, 1) {$\mathcal{A}_{\overline{L}}$:};
        \node[initial, state, initial text=] (1) {$0$};
        \node[accepting, state] [right of=1] (2) {$1$};
        \path[->] (1) edge [bend left] node[above] {$a, b$} (2);
        \path[->] (2) edge [bend left] node[below] {$a, b$} (1);
    \end{tikzpicture}
\end{center}

\begin{problem}
Да се построи автомат за:
\begin{itemize}
    \item $L_1 = \{ a, b, ab \}$
    \item $L_2 = \{ \varepsilon, ab, abab \}$
    \item $L_3 = \{ aa, bb \}$
\end{itemize}
Упътване: за състояния да се използват остатъци при деление на $n$
\end{problem}

\begin{problem}
Да се построи автомат за:
\begin{itemize}
    \item $L_1 = \Sigma^* \setminus \{ a, b, ab \}$
    \item $L_2 = \Sigma^* \setminus \{ \varepsilon, ab, abab \}$
    \item $L_3 = \Sigma^* \setminus \{ aa, bb \}$
\end{itemize}
Упътване: да се използват автоматите от предната задача
\end{problem}

\begin{problem}
Да се построи автомат за:
\begin{itemize}
    \item $L_1 = \{ \alpha \in \Sigma^* \: | \: |\alpha| \text{ се дели на } 3 \}$
    \item $L_2 = \{ \alpha \in \Sigma^* \: | \: |\alpha| \text{ се дели на } 5 \}$
    \item $L_3 = \{ \alpha \in \Sigma^* \: | \: |\alpha| \text{ дава остатък } 2 \text{ при деление на } 4 \}$
    \item $L_4 = \{ \alpha \in \Sigma^* \: | \: |\alpha| \text{ дава остатък } 3 \text{ при деление на } 5 \}$
\end{itemize}
Упътване: за състояния да се използват остатъци при деление на $n$
\end{problem}

\begin{problem}
Да се построи автомат за:
\begin{itemize}
    \item $L_1 = \{ \alpha \in \Sigma^* \: | \: |\alpha| \text{ не се дели на } 3 \}$
    \item $L_2 = \{ \alpha \in \Sigma^* \: | \: |\alpha| \text{ не се дели на } 5 \}$
    \item $L_3 = \{ \alpha \in \Sigma^* \: | \: |\alpha| \text{ не дава остатък } 2 \text{ при деление на } 4 \}$
    \item $L_4 = \{ \alpha \in \Sigma^* \: | \: |\alpha| \text{ не дава остатък } 3 \text{ при деление на } 5 \}$
\end{itemize}
Упътване: да се използват автоматите от предната задача
\end{problem}