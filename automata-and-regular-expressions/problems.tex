\section{Задачи за упражнение}

\begin{definition}\thlabel{homomorphism-def}
    Функция $h : \Sigma_1^* \rightarrow \Sigma_2^*$ ще наричаме \textbf{хомоморфизъм}, ако:
    \begin{center}
        $h(\alpha \cdot \beta) = h(\alpha) \cdot h(\beta)$ за всички $\alpha, \beta \in \Sigma_1^*$
    \end{center}
\end{definition}

\begin{problem}
Да се докаже, че за произволен хомоморфизъм $h$ е вярно, че $h(\varepsilon) = \varepsilon$
\end{problem}

\begin{problem}\thlabel{homomorphism-regular}
Да се докаже, че за произволен хомоморфизъм $h$ и регулярен език $L$, езикът $h[L]$ е регулярен

Упътване: да се направи индукция по строенето на регулярните езици
\end{problem}

\begin{problem}\thlabel{homomorphism-inverse-regular}
Да се докаже, че за произволен хомоморфизъм $h$ и регулярен език $L$, езикът $h^{-1}[L]$ е регулярен

Упътване: да се направи автомат, който докато чете $\alpha$, прави преходи с $h(\alpha)$ в автомат за $L$
\end{problem}

\begin{problem}
Използвайки нерегулярността на $\{ a^nb^n \mid n \in \mathbb{N} \}$ заедно със \thref{homomorphism-regular} или \thref{homomorphism-inverse-regular} да се докаже, че езикът $L = \{ a^nb^{2n} \mid n \in \mathbb{N} \}$ е нерегулярен.

Упътване: да се представи $L$ като образ или праобраз на хомоморфизъм
\end{problem}

\begin{problem}
Нека $L$ е регулярен език. Да се докаже, че езикът $L' = \{ \alpha \in \Sigma^* \mid \alpha \alpha \in L \}$ е регулярен.

Упътване: докато се чете $\alpha$ да се види къде отиват всички състояния от автомата за $L$
\end{problem}

\begin{problem}
Нека $L$ е регулярен език. Да се докаже, че езикът $L' = \{ \alpha \# c^n \mid n \in \mathbb{N} \: \& \: \alpha^n \in L \}$ е регулярен.

Упътване: леко усложнение на конструкцията от предната задача
\end{problem}

\begin{problem}
Нека $L$ е регулярен език. Да се докаже, че езикът $L' = \{ \alpha \in \Sigma^* \mid (\exists n \in \mathbb{N}) \: (\alpha^n \in L) \}$ е регулярен.

Упътване: да се използват предната задача и хомоморфизми
\end{problem}

\begin{definition}\thlabel{reg-homomorphism-def}
    Функция $h : \Sigma_1^* \rightarrow \mathcal{P}(\Sigma_2^*)$ ще наричаме \textbf{регулярен хомоморфизъм}, ако:
    \begin{itemize}
        \item $h(\alpha)$ е регулярен за всяко $\alpha \in \Sigma_1^*$
        \item $h(\alpha \cdot \beta) = h(\alpha) \cdot h(\beta)$ за всички $\alpha, \beta \in \Sigma_1^*$
    \end{itemize}
\end{definition}

\begin{problem}
Да се докаже, че за произволен регулярен хомоморфизъм $h$ е вярно, че $h(\varepsilon) = \{ \varepsilon \}$
\end{problem}

\begin{problem}\thlabel{reg-homomorphism}
Да се докаже, че за произволен регулярен хомоморфизъм $h$ и регулярен език $L$, $\bigcup h[L]$ е регулярен

Упътване: да се направи индукция по строенето на регулярните езици
\end{problem}