\section{Еквивалентност на детерминирани и недетерминирани автомати}

Въпреки че на пръв поглед недетерминираните автомати ни се струват по-мощни, те не ни дават нищо повече освен удобство.
Първо ще покажем, по-очевидното твърдение.

\begin{claim}
    За всеки детерминиран автомат $\mathcal{A}$ съществува недетерминиран автомат $\mathcal{N}$ със $\mathcal{L(N) = L(A)}$.
\end{claim}

\begin{proof}
    Нека $\mathcal{A} = \opair{\Sigma, Q, s, \delta, F}$ е детерминиран автомат.
    Нека $\mathcal{N} = \opair{\Sigma, Q, \{ s \}, \Delta, F}$,
    където $\Delta(p, x) = \{ \delta(p, x) \}$ за $p \in Q$.
    Тривиално може да се покаже с индукция, че за всяко $p \in Q, \: \alpha \in \Sigma^*$: $\Delta^*(p, \alpha) = \{ \delta^*(p, \alpha) \}$.
    Така:
    \begin{align*}
        \alpha \in \mathcal{L(N)} & \iff \Delta^*(\{ s \}, \alpha) \cap F \neq \varnothing \iff \{ \delta^*(s, \alpha) \} \cap F \neq \varnothing \\
                                  & \iff \delta^*(s, \alpha) \in F \iff \alpha \in \mathcal{L(A)}
    \end{align*}
\end{proof}

Сега пък ще покажем как ``детерминизира'' недетерминиран автомат.

\begin{claim}
    За всеки недетерминиран автомат $\mathcal{N}$ съществува детерминиран автомат $\mathcal{A}$ със $\mathcal{L(A) = L(N)}$.
\end{claim}

\begin{proof}
    Нека $\mathcal{N} = \opair{\Sigma, Q, S, \Delta, F}$ е недетерминиран автомат.
    Нека $\mathcal{A} = \opair{\Sigma, \mathcal{P}(Q), S, \delta, F'}$,
    където $\delta(P, x) = \Delta^*(P, x)$ за $P \subseteq Q$ и $F' = \{ P \in \mathcal{P}(Q) \mid P \cap F \neq \varnothing \}$.
    Тривиално може да се покаже с индукция, че за всяко $P \subseteq Q, \: \alpha \in \Sigma^*$: $\delta^*(P, \alpha) = \Delta^*(P, \alpha)$.
    Така:
    \begin{align*}
        \alpha \in \mathcal{L(N)} & \iff \Delta^*(S, \alpha) \cap F \neq \varnothing \iff \delta^*(S, \alpha) \cap F \neq \varnothing \\
                                  & \iff \delta^*(S, \alpha) \in F' \iff \alpha \in \mathcal{L(A)}
    \end{align*}
\end{proof}

Имайки това вече можем да използваме, че $\cdot$, $*$ и $rev$ запазват автоматност.
\begin{warning}
    При ``детерминизиране'' броят на състоянията нараства експоненциално.
    Накратко $10$ става $1024$.
\end{warning}

\begin{problem}
Да се детерминира следният автомат:
\begin{center}
    \begin{tikzpicture}[shorten >=1pt,node distance=2.5cm,>=stealth',thick]
        \node[initial, state, initial text=] (1) {$0$};
        \node[state] [right of=1] (2) {$1$};
        \node[initial, state, initial text=] [below left of=2] (3) {$2$};
        \node[accepting, state] [right of=3] (4) {$3$};
        \path[->] (1) edge [loop above] node[above] {$a$} (1);
        \path[->] (1) edge node[above] {$a, b$} (2);
        \path[->] (1) edge node[left] {$b$} (3);
        \path[->] (2) edge node[above] {$b$} (3);
        \path[->] (2) edge node[right] {$b$} (4);
        \path[->] (3) edge [loop below] node[below] {$a$} (3);
        \path[->] (3) edge node[above] {$a$} (4);
    \end{tikzpicture}
\end{center}
\end{problem}