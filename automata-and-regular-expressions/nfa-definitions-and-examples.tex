\section{Недетерминирани автомати}

\begin{definition}
    \textbf{Недетерминиран краен автомат} (накратко НКА) ще наричаме всяко $\mathcal{N} = \opair{\Sigma, Q, S, \Delta, F}$, където:
    \begin{itemize}
        \item $\Sigma$ е крайна азбука
        \item $Q$ е крайно множество от състояния
        \item $S \subseteq Q$ (ще ги наричаме начални/стартови състояния)
        \item $\Delta : Q \cross \Sigma \rightarrow \mathcal{P}(Q)$ (ще я наричаме функция на преходите)
        \item $F \subseteq Q$ (ще ги наричаме финални състояния)
    \end{itemize}
\end{definition}

Цялата ``недетерминираност'' идва от това, че от едно състояние с една буква можем да отидем на няколко места.
Поне на пръв поглед тази нова машина изглежда доста по-мощна от старата.
Вместо да си налагаме точно един преход, можем да направим два или три прехода, а можем да отидем в ``нищото'' (има се предвид $\varnothing$).

\begin{definition}
    Дефинираме $\Delta^* : \mathcal{P}(Q) \cross \Sigma^* \rightarrow \mathcal{P}(Q)$ индуктивно:
    \begin{itemize}
        \item $\Delta^*(P, \varepsilon) = P$ за всяко $P \subseteq Q$
        \item $\Delta^*(P, \beta x) = \bigcup\limits_{q \in \Delta^*(P, \beta)}\Delta(q, x)$ за всяко $P \subseteq Q, \beta \in \Sigma^*, x \in \Sigma$
    \end{itemize}
\end{definition}

\begin{remark}
    $\Delta^*(P, \alpha \beta) = \Delta^*(\Delta^*(P, \alpha), \beta)$ (лесно излиза с индукция)
\end{remark}

Вече можем да кажем какъв е език на даден недетерминиран автомат.

\begin{definition}
    Нека $\mathcal{N} = \opair{\Sigma, Q, S, \Delta, F}$ е НКА.
    Тогава езикът на автомата $\mathcal{N}$ е множеството
    $\mathcal{L}(\mathcal{N}) = \{ \alpha \in \Sigma^* \: | \: \delta^*(S, \alpha) \cap F \neq \varnothing \}$.
\end{definition}

Тук цялата идея, е вместо думата да поеме по един предопределен път и да получи ДА или НЕ,
тя да може да изпробва няколко възможни такива, и ако получи отговор ДА за поне един от тях, то тогава получава ДА от целия автомат.
В някакъв смисъл има елемент на отгатване и несигурност. Може този път да свърши работа, но може и другия да свърши работа.

\pagebreak

Нека дадем няколко прости примери за недетерминирани автомати:
\begin{figure*}[h]
    \centering
    \centering
    \begin{tikzpicture}[node distance=2cm, thick]
        \node[initial, state, initial text=] (1) {$q$};
    \end{tikzpicture}
    \caption*{Автомат за $\varnothing$}
\end{figure*}

\begin{figure*}[h]
    \centering
    \begin{tikzpicture}[node distance=2cm, thick]
        \node[initial, state, initial text=] (1) {$\varepsilon$};
        \node[state, initial text=, right of=1] (2) {$a$};
        \node[state, initial text=, right of=2] (3) {$ab$};
        \node[accepting, state, initial text=, right of=3] (4) {$aba$};
        \path[->] (1) edge [] node[above] {$a$}(2);
        \path[->] (2) edge [] node[above] {$b$}(3);
        \path[->] (3) edge [] node[above] {$a$}(4);
    \end{tikzpicture}
    \caption*{Автомат за $\{ aba \}$}
\end{figure*}

Доста по-компактно става представянето на автомати за тези езици.
Напълно елимираме ненужните преходи и състояние боклук.
Можем и много лесно да направим автомат за няколко думи.

\begin{figure*}[h]
    \centering
    \begin{tikzpicture}[node distance=2cm, thick]
        \node[initial, state, initial text=] (1) {$\varepsilon_{aba}$};
        \node[state, initial text=, right of=1] (2) {$a$};
        \node[state, initial text=, right of=2] (3) {$ab$};
        \node[accepting, state, initial text=, right of=3] (4) {$aba$};
        \node[initial, state, initial text=, below of=1] (5) {$\varepsilon_{bab}$};
        \node[state, initial text=, right of=5] (6) {$b$};
        \node[state, initial text=, right of=6] (7) {$ba$};
        \node[accepting, state, initial text=, right of=7] (8) {$bab$};
        \path[->] (1) edge [] node[above] {$a$}(2);
        \path[->] (2) edge [] node[above] {$b$}(3);
        \path[->] (3) edge [] node[above] {$a$}(4);
        \path[->] (5) edge [] node[above] {$b$}(6);
        \path[->] (6) edge [] node[above] {$a$}(7);
        \path[->] (7) edge [] node[above] {$b$}(8);
    \end{tikzpicture}
    \caption*{Автомат за $\{ aba, bab \}$}
\end{figure*}

Тук се опитваме да познаем дали четем една от двете думи.
Двата малки автомата работят независимо един от други и всеки дава собствен отговор.
Ако сме се озовали във финално състояние, сме прочели една от двете думи в езика, иначе не сме.