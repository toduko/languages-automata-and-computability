\documentclass[11pt, a5paper]{report}

\usepackage[utf8]{inputenc}
\usepackage[T2A]{fontenc}
\usepackage[english, bulgarian]{babel}
\usepackage{amssymb}
\usepackage{hyperref, fancyhdr, lastpage, fancyvrb, tcolorbox, titlesec}
\usepackage{array, tabularx, colortbl}
\usepackage{tikz}
\usepackage{venndiagram}
\usepackage{amsthm, bm}
\usepackage{relsize}
\usepackage{amsmath,physics}
\usepackage{mathtools}
\usepackage{subcaption}
\usepackage{theoremref}
\usepackage[a5paper, left=0.50in, right=0.50in, top=1.0in, bottom=1.0in]{geometry}
\usepackage{minted}
\usepackage{stmaryrd}
\usetikzlibrary{automata, arrows}

\newcommand{\db}[1]{\llbracket #1 \rrbracket}
\newcommand{\regexlang}[1]{\mathcal{L}\db{#1}}

\ExplSyntaxOn
\NewDocumentCommand{\opair}{m}
 {
  \langle\mspace{2mu}
  \clist_set:Nn \l_tmpa_clist { #1 }
  \clist_use:Nn \l_tmpa_clist {,\mspace{3mu plus 1mu minus 1mu}\allowbreak}
  \mspace{2mu}\rangle
}
\ExplSyntaxOff

\hypersetup{
    colorlinks=true,
    linktoc=all,
    linkcolor=blue
}

\theoremstyle{definition}
\newtheorem{definition}{Дефиниция}[section]
\newtheorem*{warning}{\textcolor{red}{Внимание}}
\theoremstyle{plain}
\newtheorem{theorem}[definition]{Теорема}
\newtheorem{claim}[definition]{Твърдение}
\newtheorem{axiom}[definition]{Аксиома}
\newtheorem{lemma}[definition]{Лема}
\newtheorem{problem}[definition]{Задача}
\newtheorem{corollary}[definition]{Следствие}
\theoremstyle{remark}
\newtheorem*{remark}{Забележка}
\theoremstyle{definition}

\pagestyle{fancy}

\lhead{\leftmark}
\rhead{}

\setlength\parindent{0pt}

\begin{document}

\begin{titlepage}
  \centering
  {\huge\bfseries ЕЗИЦИ, АВТОМАТИ И ИЗЧИСЛИМОСТ\par}
  \vspace{1cm}
  {\Large \textsc{Записки за упражненията}\par}
  \vspace{4cm}
  {\Large\itshape Тодор Дуков\par}
  \vfill
  {\large \today\par}
\end{titlepage}

\tableofcontents

\chapter{Въведение}

\section{Основни понятия}

\section{Допълнителни дефиниции}

\section{Задачи за упражнение}
\chapter{Автомати}

Тук ще разгледаме първата ``машина'', с която ще класифицираме езиците.

\section{Детерминирани автомати и автоматни езици}

\begin{definition}
    \textbf{Детерминиран краен автомат} (накратко ДКА) ще наричаме всяко $\mathcal{A} = \opair{\Sigma, Q, s, \delta, F}$, където:
    \begin{itemize}
        \item $\Sigma$ е крайна азбука
        \item $Q$ е крайно множество от състояния
        \item $s \in Q$ (ще го наричаме начално/стартово състояние)
        \item $\delta : Q \cross \Sigma \rightarrow Q$ (ще я наричаме функция на преходите)
        \item $F \subseteq Q$ (ще ги наричаме финални състояния)
    \end{itemize}
\end{definition}

В момента със това, което имаме,
ако искаме да покажем къде ще стигнем с думата $aaa$, започвайки от $s$,
ще трябва да го запишем със $\delta(\delta(\delta(s, a), a), a)$, което е тромаво.

За това ще си въведем начин, по който да видим крайният резултат от прочитането на цяла дума.

\begin{definition}
    Дефинираме $\delta^* : Q \cross \Sigma^* \rightarrow Q$ индуктивно:
    \begin{itemize}
        \item $\delta^*(p, \varepsilon) = p$ за всяко $p \in Q$
        \item $\delta^*(p, \beta x) = \delta(\delta^*(p, \beta), x)$ за всяко $p \in Q, \beta \in \Sigma^*, x \in \Sigma$
    \end{itemize}
\end{definition}

Сега можем да забележим, че:
\begin{align*}
    \delta^*(s, aaa) & = \delta(\delta^*(s, aa), a) = \delta(\delta(\delta^*(s, a), a), a) =                           \\
                     & = \delta(\delta(\delta(\delta^*(s, \varepsilon), a), a), a) =\delta(\delta(\delta(s, a), a), a)
\end{align*}
Функцията наистина прави това, което искаме да прави.

\begin{definition}
    Нека $\mathcal{A} = \opair{\Sigma, Q, s, \delta, F}$ е ДКА.
    Тогава езикът на автомата $\mathcal{A}$ е множеството
    $\mathcal{L}(\mathcal{A}) = \{ \alpha \in \Sigma^* \: | \: \delta^*(s, \alpha) \in F \}$.
    Един език $L \subseteq \Sigma^*$, наричаме \textbf{автоматен}, ако има ДКА $\mathcal{A}$ с $\mathcal{L}(\mathcal{A}) = L$.
\end{definition}

\section{Представяне на автомат}

Ще видим начините, по които можем да представяме един автомат.
За пример ще вземем автомата $\mathcal{A} = \opair{\Sigma, Q, s, \delta, F}$, където:
\begin{itemize}
    \item $\Sigma = \{ a, b \}$
    \item $Q = \{ q_0, q_1, q_2 \}$
    \item $s = q_0$
    \item $\delta(q_0, a) = q_2$
    \item $\delta(p, x) = q_1$ за $p \in Q, \: x \in \Sigma, \: \opair{p, x} \neq \opair{q_0, a}$
    \item $F = \{ q_2 \}$
\end{itemize}
Това е първият начин за представяне.
При него директно в явен вид се казват кои са всички съставни елементи на $\mathcal{A}$.
Това ще бъде използване сравнително често, когато правим от един автомат друг.
В случаите, в които не знаем как изглежда първоначалният автомат, следващите методи тогава няма да свършат работа. \\

Вторият начин за представяне е с таблица на функцията на преходите:
\begin{center}
    \begin{tabular}{||r | r | r||}
        \hline
        \cellcolor{lightgray} & $a$   & $b$   \\
        \hline
        $\rightarrow q_0$     & $q_2$ & $q_1$ \\
        \hline
        $q_1$                 & $q_1$ & $q_1$ \\
        \hline
        $\checkmark q_2$      & $q_1$ & $q_1$ \\
        \hline
    \end{tabular}
\end{center}
Тук с $\rightarrow$ отбелязваме началното състояние,
а с $\checkmark$ отбелязваме финалните състояния.
Във втората и третата колона казваме къде ще се озовем,
ако се намираме в съответното състояние и прочетем съответната буква.
Това е по-прегледно представяне от първото, но може да стане обемно. \\

Третият начин за представяне е с картинка:

\begin{center}
    \begin{tikzpicture}[shorten >=1pt,node distance=2cm,>=stealth',thick]
        \node[initial, state, initial text=] (1) {$q_0$};
        \node[state] (2) [above right of=1] {$q_1$};
        \node[state, accepting] (3) [below right of=1] {$q_2$};
        \draw[->] (1) -- node[above] {$b$} (2);
        \draw[->] (1) -- node[above] {$a$} (3);
        \path (2) edge [loop above] node[above] {$a, b$} (2);
        \draw[->] (3) -- node[right] {$a, b$} (2);
    \end{tikzpicture}
\end{center}

Началното състояние е отбелязано със стрелка, а финалните са оградени два пъти.
Представянето чрез картинка изглежда по-прегледно от другите две, но то може и да стане огромно.

Ясно е, че $\mathcal{L}(\mathcal{A}) = \{ a \}$.
Ако искаме да покажем това формално, трябва да направим следното:
\begin{itemize}
    \item Очевидно $a \in \mathcal{L}(\mathcal{A}), \: b \notin \mathcal{L}(\mathcal{A}), \: \varepsilon \notin \mathcal{L}(\mathcal{A})$
    \item Показваме, че $\delta^*(q_1, \alpha) = q_1$ за всяко $\alpha \in \Sigma^*$ с индукция по $|\alpha|$
    \item Ако $\alpha \notin \{ a, b, \varepsilon \}$, то $\alpha = x y \beta$ за $x, y \in \Sigma, \: \beta \in \Sigma^*$
    \item Тогава лесно се вижда, че $\delta^*(q_0, \alpha) = q_1 \notin F$.
          Или директно отиваме в $q_1$ (ако $x = b$) и оставаме там, или първо отиваме в $q_2$ (ако $x = a$), и после със следващата буква отиваме в $q_1$ и оставаме там.
\end{itemize}

Това обаче за такива прости автомати не е нужно.
Тези неща в такива случаи ще ги приемаме за очевидни.
\section{Прости примери за автоматни езици}

Нека видим много прости примери за автоматни езици,
за да добием малко интуиция какви езици могат да бъдат автоматни,
и да поработим малко със самите машини. Най-простите автоматни езици са $\varnothing$ и $\Sigma^*$:

\begin{figure*}[h]
    \centering
    \begin{subfigure}[b]{0.3\linewidth}
        \centering
        \begin{tikzpicture}[node distance=2cm, thick]
            \node[initial, state, initial text=] (1) {$q$};
            \path[->] (1) edge [loop above] node[above] {$a, b$}(1);
        \end{tikzpicture}
        \caption*{Автомат за $\varnothing$}
    \end{subfigure}
    \begin{subfigure}[b]{0.3\linewidth}
        \centering
        \begin{tikzpicture}[node distance=2cm, thick]
            \node[initial, accepting, state, initial text=, right of=1] (2) {$q$};
            \path[->] (2) edge [loop above] node[above] {$a, b$}(2);
        \end{tikzpicture}
        \caption*{Автомат за $\Sigma^*$}
    \end{subfigure}
\end{figure*}

Сега малко ще усложним нещата.
Ще направим автомат, който да разпознае езикът от една конкретна дума.
Конструкцията за конкретната дума много лесно се обобщава за всички.
За пример нека направим автомат за $L = \{ abab \}$:

\begin{center}
    \begin{tikzpicture}[shorten >=1pt,node distance=2.5cm,>=stealth',thick]
        \node[initial, state, initial text=] (1) {$\varepsilon$};
        \node[state] [right of=1] (2) {$a$};
        \node[state] [right of=2] (3) {$ab$};
        \node[state] [right of=3] (4) {$aba$};
        \node[state, accepting] [right of=4] (5) {$abab$};
        \node[state] [below of=3] (6) {$\cross$};
        \draw[->] (1) -- node[above] {$a$} (2);
        \draw[->] (2) -- node[above] {$b$} (3);
        \draw[->] (3) -- node[above] {$a$} (4);
        \draw[->] (4) -- node[above] {$b$} (5);
        \path[->] (1) edge [bend right] node[below] {$b$} (6);
        \path[->] (2) edge [bend right] node[below] {$a$} (6);
        \path[->] (3) edge  node[right] {$b$} (6);
        \path[->] (4) edge [bend left] node[below] {$a$} (6);
        \path[->] (5) edge [bend left] node[below] {$a, b$} (6);
        \path[->] (6) edge [loop below] node[below] {$a, b$} (6);
    \end{tikzpicture} \\
\end{center}

\pagebreak

Тази техника може да се приложи за строене на автомат за език от произволно дълга дума.
Ето как ще обобщим конструкцията за езика $L = \{ \alpha \}$:
\begin{itemize}
    \item $\mathcal{A} = \opair{\Sigma, Q, s, \delta, F}$
    \item $Q = Pref(L) \cup \{ \cross \}$ (начален отрязък от пътя, който съставя $\alpha$, $\cross \notin \Sigma$ е отхвърлящо състояние боклук)
    \item $s = \varepsilon$
    \item За $\beta \preceq_{pref} \alpha \: ($тогава $\beta \in Q), \: x \in \Sigma$:
          ако $\beta x \preceq_{pref} \alpha$, то $\delta(\beta, x) = \beta x$,
          иначе $\delta(\beta, x) = \cross$
    \item $\delta(\cross, x) = \cross$ за $x \in \Sigma$
    \item $F = \{ \alpha \}$
\end{itemize}

Идеята зад конструкцията е следната:
Състоянията кодират пътя, който сме изминали, ако не сме се отклонили вече от ``строежа'' на $\alpha$.
В първият момент на отклонение отиваме във нефинално състояние ``боклук'', от което не може да излезнем.
Можем да разширим идеята, така че да работи за няколко думи като кодираме повече видове пътища.

\pagebreak

\begin{figure*}
    \centering
    \begin{tikzpicture}[shorten >=1pt,node distance=2.5cm,>=stealth',thick]
        \node[initial, state, initial text=] (1) {$\varepsilon$};
        \node[state] [above right of=1] (2) {$a$};
        \node[state] [below right of=1] (3) {$b$};
        \node[state, accepting] [above right of=2] (4) {$aa$};
        \node[state, accepting] [right of=2] (5) {$ab$};
        \node[state, accepting] [right of=3] (6) {$ba$};
        \node[state] [below right of=3] (7) {$bb$};
        \node[state] [below right of=5] (8) {$\cross$};
        \draw[->] (1) -- node[above] {$a$} (2);
        \draw[->] (1) -- node[above] {$b$} (3);
        \draw[->] (2) -- node[above] {$a$} (4);
        \draw[->] (2) -- node[above] {$b$} (5);
        \draw[->] (3) -- node[above] {$a$} (6);
        \draw[->] (3) -- node[above] {$b$} (7);
        \path[->] (4) edge [bend left] node[right] {$a, b$} (8);
        \path[->] (5) edge node[left] {$a, b$} (8);
        \path[->] (6) edge node[left] {$a, b$} (8);
        \path[->] (7) edge [bend right] node[right] {$a, b$} (8);
        \path[->] (8) edge [loop right] node[right] {$a, b$} (8);
    \end{tikzpicture}
    \caption*{Автомат за $L = \{ ab, ba, aa \}$}
\end{figure*}

Ето как ще обобщим конструкцията (чиято коректност приемаме за очевидна) за $L = \{ \alpha_1, \: \dots, \: \alpha_n \}$:
\begin{itemize}
    \item $\mathcal{A} = \opair{\Sigma, Q, s, \delta, F}$
    \item $Q = \{ \alpha \in \Sigma^* \: | \: (\exists \beta \in L) (|\alpha| \leq |\beta|) \} \cup \{ \cross \}$ (не гледаме пътища по-дълги от най-дългата дума в $L$, $\cross \notin \Sigma$ - състояние боклук)
    \item $s = \varepsilon$
    \item За $\alpha \in Q, \: x \in \Sigma$:
          ако $\alpha x \in Q$, то $\delta(\alpha, x) = \alpha x$,
          иначе $\delta(\alpha, x) = \cross$
    \item $\delta(\cross, x) = \cross$ за $x \in \Sigma$
    \item $F = L$
\end{itemize}

Нека сега направим автомат за $L = \{ \alpha \in \Sigma^* \: | \: |\alpha| \: \text{е четно} \}$.

За една дума $\alpha \in \Sigma^*$ знаем, че тя или има четна дължина или има нечетна дължина.
Дали не можем да кодираме по някакъв начин четността на прочетената дума в състояние?
Отговорът е, че можем. Автоматът е следния:

\begin{center}
    \begin{tikzpicture}[shorten >=1pt,node distance=2.5cm,>=stealth',thick]
        \node[text width=0.5cm] at (-1, 1) {$\mathcal{A}$:};
        \node[accepting, initial, state, initial text=] (1) {$0$};
        \node[state] [right of=1] (2) {$1$};
        \path[->] (1) edge [bend left] node[above] {$a, b$} (2);
        \path[->] (2) edge [bend left] node[below] {$a, b$} (1);
    \end{tikzpicture}
\end{center}

Знаем, че $|\varepsilon|$ е четно.
За всякo $\alpha \in \Sigma^*, \: x \in \Sigma$ знаем,
че $|\alpha x|$ има различна четност от $|\alpha|$.
Така ние започваме с думата $\varepsilon$ и 0 като четност на думата и за всяка буква сменяме четността.

Нека сега помислим как да докажем, че $\mathcal{L(A)} = L$.
За това ще трябва да покажем, че започвайки от $0$ и четейки $\alpha$ ние наистина получаваме четността на $|\alpha|$ като състояние.

\begin{claim}
    За всяко $\alpha \in \Sigma^*$ : $\delta^*(0, \alpha) = |\alpha| \: (mod \: 2)$
\end{claim}

\begin{proof}
    С индукция по $|\alpha|$.
    \begin{itemize}
        \item База: $|\alpha| = 0$, тогава $\alpha = \varepsilon$.
              Наистина $\delta^*(0, \varepsilon) = 0 \: (mod \: 2)$ \checkmark
        \item ИС: $|\alpha| = n + 1$, тогава $\alpha = \beta x$ за $\beta \in \Sigma^*, |\beta| = n, \: x \in \Sigma$
              Тогава:
              \begin{align*}
                  \delta^*(0, \beta x) & = \delta(\delta^*(0, \beta), x) \stackrel{\text{ИП}}{=}
                  \delta(|\beta| \: (mod \: 2), x) = |\beta| + 1 \: (mod \: 2) =                 \\
                                       & = |\beta x| \: (mod \: 2) = |\alpha| \: (mod \: 2)
              \end{align*}
    \end{itemize}
\end{proof}

Състоянията наистина кодират информацията, която искахме.
Имайки това можем да покажем, че двата езика съвпадат, по следния начин:
\begin{center}
    $\alpha \in \mathcal{L(A)} \iff \delta^*(0, \alpha) = 0 \iff |\alpha| \: (mod \: 2) = 0 \iff \alpha \in L$
\end{center}


Нека сега помислим какъв автомат ще можем да направим за $\overline{L}$.
Ние вече имаме автомат за $L$.
От неговите състояние и преходи можем да извлечем информация за четността на дължината на думата.
Единственото, което трябва да направим, е да сменим отговора на автомата.
Ако преди той е казвал ДА, сега да казва НЕ, и обратното.
Очевидно този автомат ще свърши работа:
\begin{center}
    \begin{tikzpicture}[shorten >=1pt,node distance=2.5cm,>=stealth',thick]
        \node[text width=0.5cm] at (-1, 1) {$\mathcal{A}_{\overline{L}}$:};
        \node[initial, state, initial text=] (1) {$0$};
        \node[accepting, state] [right of=1] (2) {$1$};
        \path[->] (1) edge [bend left] node[above] {$a, b$} (2);
        \path[->] (2) edge [bend left] node[below] {$a, b$} (1);
    \end{tikzpicture}
\end{center}

\begin{problem}
    Да се построи автомат за:
    \begin{itemize}
        \item $L_1 = \{ \alpha \in \Sigma^* \: | \: |\alpha| \text{ се дели на } 3 \}$
        \item $L_2 = \{ \alpha \in \Sigma^* \: | \: |\alpha| \text{ се дели на } 5 \}$
        \item $L_3 = \{ \alpha \in \Sigma^* \: | \: |\alpha| \text{ дава остатък } 2 \text{ при деление на } 4 \}$
    \end{itemize}
    Упътване: за състояния да се използват остатъци при деление на $n$
\end{problem}

\begin{problem}
    Да се построи автомат за:
    \begin{itemize}
        \item $L_1 = \{ \alpha \in \Sigma^* \: | \: |\alpha| \text{ не се дели на } 3 \}$
        \item $L_2 = \{ \alpha \in \Sigma^* \: | \: |\alpha| \text{ не се дели на } 5 \}$
        \item $L_3 = \{ \alpha \in \Sigma^* \: | \: |\alpha| \text{ не дава остатък } 2 \text{ при деление на } 4 \}$
    \end{itemize}
    Упътване: да се използват автоматите от предната задача
\end{problem}
\section{Операции над автоматни езици}

Сега ще разгледаме няколко операции, които запазват автоматност.
Първата операция (която вече загатнахме) над автоматни езици, която ще разгледаме е допълнение.

\begin{claim}
    Ако $L$ е автоматен, то тогава и $\overline{L}$ също е автоматен.
\end{claim}

\begin{proof}
    Понеже $L$ е автоматен, има ДКА $\mathcal{A} = \opair{\Sigma, Q, s, \delta, F}$ с $\mathcal{L(A)} = L$.
    Нека $\mathcal{A}_{\overline{L}} = \opair{\Sigma, Q, s, \delta, Q \setminus F}$. Тогава:
    \begin{center}
        $\alpha \in \mathcal{L(A)} \iff \delta^*(s, \alpha) \in Q \setminus F \iff \delta^*(s, \alpha) \notin F \iff \alpha \notin L$
    \end{center}
\end{proof}

Както направихме и в предният пример, тук просто сменяме отговора.
Състоянията и преходите на началният автомат ни дават достатъчна информация за структурата на думата,
което е достатъчно за да кажем дали е от езика $\overline{L}$ или не е. \\

Преди да разгледаме следващата операция ще разгледаме два авомата:

\begin{center}
    \begin{tikzpicture}[shorten >=1pt,node distance=2.5cm,>=stealth',thick]
        \node[text width=0.5cm] at (-1, 1) {$\mathcal{A}_1$:};
        \node[accepting, initial, state, initial text=] (1) {$0$};
        \node[state] [right of=1] (2) {$1$};
        \path[->] (1) edge [bend left] node[above] {$a$} (2);
        \path[->] (2) edge [bend left] node[below] {$a$} (1);
        \path[->] (1) edge [loop above] node[above] {$b$} (1);
        \path[->] (2) edge [loop above] node[above] {$b$} (2);
    \end{tikzpicture}
\end{center}

\begin{center}
    \begin{tikzpicture}[shorten >=1pt,node distance=2.5cm,>=stealth',thick]
        \node[text width=0.5cm] at (-1, 1) {$\mathcal{A}_2$:};
        \node[initial, state, initial text=] (1) {$0$};
        \node[accepting, state] [right of=1] (2) {$1$};
        \path[->] (1) edge [bend left] node[above] {$b$} (2);
        \path[->] (2) edge [bend left] node[below] {$b$} (1);
        \path[->] (1) edge [loop above] node[above] {$a$} (1);
        \path[->] (2) edge [loop above] node[above] {$a$} (2);
    \end{tikzpicture}
\end{center}

Ще приемем за очевидно, че първият автомат разпознава думите с четен брой $a$,
а вторият автомат разпознава думите с нечетен брой $b$.

Как бихме могли да направим автомат $\mathcal{A}$ за $\mathcal{L}(\mathcal{A}_1) \cap \mathcal{L}(\mathcal{A}_2)$?
Можем да направим така:

\begin{center}
    \begin{tikzpicture}[node distance=2.5cm,>=stealth',thick]
        \node[text width=0.5cm] at (-1, 1) {$\mathcal{A}$:};
        \node[initial, state with output, initial text=] (1) {$0$ \nodepart{lower} $0$};
        \node[state with output] [right of=1] (2) {$0$ \nodepart{lower} $1$};
        \node[accepting, state with output, initial text=] [below of=1](3) {$1$ \nodepart{lower} $0$};
        \node[state with output] [right of=3] (4) {$1$ \nodepart{lower} $1$};
        \path[->] (1) edge [bend left] node[above] {$a$} (2);
        \path[->] (2) edge [bend left] node[below] {$a$} (1);
        \path[->] (3) edge [bend left] node[above] {$a$} (4);
        \path[->] (4) edge [bend left] node[below] {$a$} (3);
        \path[->] (1) edge [bend left] node[right] {$b$} (3);
        \path[->] (3) edge [bend left] node[left] {$b$} (1);
        \path[->] (2) edge [bend left] node[right] {$b$} (4);
        \path[->] (4) edge [bend left] node[left] {$b$} (2);
    \end{tikzpicture}
\end{center}

Имаме един брояч от за четността на $a$ и още един брояч за четността на $b$.
Всеки брояч се променя само от неговата буква.
Накрая искаме отгоре да седи $1$ (четен брой $a$), а отдолу да седи $0$ (нечетен брой $b$).

Оказва се, че това, което направихме, може да се обобщи ето така:

\begin{claim}
    Ако $L_1$ и $L_2$ са автоматни езици, то $L_1 \cap L_2$ също е автоматен език.
\end{claim}

\begin{proof}
    Нека $\mathcal{A}_1 = \opair{\Sigma, Q_1, s_1, \delta_1, F_1}$ е автомат за $L_1$ и нека $\mathcal{A}_2 = \opair{\Sigma, Q_2, s_2, \delta_2, F_2}$ е автомат за $L_2$.
    Строим автомат за сечението:
    \begin{itemize}
        \item $\mathcal{A} = \opair{\Sigma, Q, s, \delta, F}$
        \item $Q = Q_1 \cross Q_2$
        \item $s = \opair{s_1, s_2}$
        \item $\delta(\opair{p_1, p_2}, x) = \opair{\delta_1(p_1, x), \delta_2(p_2, x)}$ за $\opair{p_1, p_2} \in Q, \: x \in \Sigma$
        \item $F = F_1 \cross F_2$
    \end{itemize}

    Тук използваме, че двата автомата, които имаме по начало ни дават информация за структурата на думите.
    Ние паралелно изпълняваме четене в $\mathcal{A}_1$ и $\mathcal{A}_2$,
    и накрая искаме положителен отговор и от двата автомата.
    За да покажем, че $\mathcal{L(A)} = L_1 \cap L_2$, ще докажем:
    \begin{claim}
        $\delta^*(\opair{p_1, p_2}, \alpha) = \opair{\delta_1^*(p_1, \alpha), \delta_2^*(p_2, \alpha)}$
    \end{claim}
    \begin{proof}
        С индукция по $|\alpha|$.
        \begin{itemize}
            \item $\delta^*(\opair{p_1, p_2}, \varepsilon) = \opair{p_1, p_2} = \opair{\delta_1^*(p_1, \varepsilon), \delta_2^*(p_2, \varepsilon)}$ \checkmark
            \item $\delta^*(\opair{p_1, p_2}, \beta x) = \delta(\delta^*(\opair{p_1, p_2}, \beta), x) \stackrel{\text{ИП}}{=} \delta(\opair{\delta_1^*(p_1, \beta), \delta_2^*(p_2, \beta)}, x) = \opair{\delta_1(\delta_1^*(p_1, \beta), x), \delta_2(\delta_2^*(p_2, \beta), x)} = \opair{\delta_1^*(p_1, \beta x), \delta_2^*(p_2, \beta x)}$
        \end{itemize}
    \end{proof}

    Имайки това изкарваме директно, че
    $\alpha \in \mathcal{L(A)} \iff \delta^*(s, \alpha) \in F \iff \opair{\delta_1^*(s_1, \alpha), \delta_2^*(s_2, \alpha)} \in F_1 \cross F_2 \iff \delta_1^*(s_1, \alpha) \in F_1 \: \& \: \delta_2^*(s_2, \alpha) \in F_2 \iff \alpha \in L_1 \: \& \: \alpha \in \L_2 \iff \alpha \in L_1 \cap L_2$, с което сме готови.
\end{proof}

Имайки, че $\cap$ и допълнение запазват автоматност, директно изкарваме, че $\cup$ и $\setminus$ запазват автоматност:
\begin{itemize}
    \item $L_1 \cup L_2 = \overline{\overline{L_1 \cup L_2}} = \overline{\overline{L_1} \cap \overline{L_2}}$ \\
          Друг вариант това да се направи е да се приложи същата конструкция, със разликата че $F = (F_1 \cross Q_2) \cup (Q_1 \cross F_2)$
    \item $L_1 \setminus L_2 = L_1 \cap \overline{L_2}$ \\
          Друг вариант това да се направи е да се приложи същата конструкция, със разликата че $F = F_1 \cross (Q_2 \setminus F_2)$
\end{itemize}

\pagebreak

Освен, че можем да изпълняваме няколко автомата паралелно, ние също така можем и да ги караме да се редуват.

\begin{claim}
    Ако $L_1$ и $L_2$ са автоматни езици, то и
    \begin{align*}
        \operatorname{Mix}(L_1, L_2) = \{ a_1b_1 \dots a_nb_n \: | \: n \in \mathbb{N} \: & \& \: (\forall i \in \{ 1, \dots, n \})(a_i \in \Sigma \: \& \: b_i \in \Sigma) \: \\
                                                                                          & \& \: a_1 \dots a_n \in L_1 \: \& \: b_1 \dots b_n \in L_2 \}
    \end{align*}
    е автоматен.
\end{claim}

\begin{proof}
    Нека $\mathcal{A}_1 = \opair{\Sigma, Q_1, s_1, \delta_1, F_1}$ е автомат за $L_1$ и нека $\mathcal{A}_2 = \opair{\Sigma, Q_2, s_2, \delta_2, F_2}$ е автомат за $L_2$.
    Строим автомат за $\operatorname{Mix}(L_1, L_2)$:
    \begin{itemize}
        \item $\mathcal{A} = \opair{\Sigma, Q, s, \delta, F}$
        \item $Q = Q_1 \cross Q_2 \cross \{ 0, 1 \}$ (третата компонента ще ни казва с кой автомат работим в момента)
        \item $s = \opair{s_1, s_2, 0}$ (чисто начало, контрол има $\mathcal{A}_1$)
        \item $\delta(\opair{p_1, p_2, 0}, x) = \opair{\delta_1(p_1, x), p_2, 1}$ за $\opair{p_1, p_2} \in Q_1 \cross Q_2, \: x \in \Sigma$
              Идеята тук, е че понеже под контрол е $\mathcal{A}_1$, ние правим преход само с неговото състояние и след това предаваме контрола на $\mathcal{A}_2$
        \item $\delta(\opair{p_1, p_2, 1}, x) = \opair{p_1, \delta_2(p_2, x), 0}$ за $\opair{p_1, p_2} \in Q_1 \cross Q_2, \: x \in \Sigma$
              Аналогично тук под контрол е $\mathcal{A}_2$, затова ние правим преход само в него и след това връщаме контрола на $\mathcal{A}_1$
        \item $F = F_1 \cross F_2 \cross \{ 0 \}$ ($\mathcal{A}_1$ и $\mathcal{A}_2$ са одобрили своите думи и сме прочели дума с четна дължина)
    \end{itemize}

    Този автомат просто чете буква със $\mathcal{A}_1$ и буква със $\mathcal{A}_2$, после пак буква със $\mathcal{A}_1$ и буква със $\mathcal{A}_2$ и така нататък.

    \pagebreak

    Нека сега докажем, че това наистина се случва.

    \begin{claim}
        За всяко $\alpha \in \Sigma^*$:
        \begin{itemize}
            \item ако $\alpha = \alpha_1 \dots \alpha_{2n}$ за някои $n \in \mathbb{N}, \: \alpha_1, \dots, \alpha_{2n} \in \Sigma$, то \\
                  $\delta^*(\opair{s_1, s_2, 0}, \alpha) = \opair{\delta_1^*(s_1, \alpha_1 \alpha_3 \dots \alpha_{2n - 1}), \delta_2^*(s_2, \alpha_2 \alpha_4 \dots \alpha_{2n}), 0}$
            \item ако $\alpha = \alpha_1 \dots \alpha_{2n + 1}$ за някои $n \in \mathbb{N}, \: \alpha_1, \dots, \alpha_{2n + 1} \in \Sigma$, то \\
                  $\delta^*(\opair{s_1, s_2, 0}, \alpha) = \opair{\delta_1^*(s_1, \alpha_1 \alpha_3 \dots \alpha_{2n + 1}), \delta_2^*(s_2, \alpha_2 \alpha_4 \dots \alpha_{2n}), 1}$
        \end{itemize}
    \end{claim}

    \begin{proof}
        С индукция по $|\alpha|$.

        Базата е ясна. Нека разгледаме индукционната стъпка:
        \begin{align*}
             & \delta^*(\opair{s_1, s_2, 0},\alpha_1 \dots \alpha_{2n}) = \delta(\delta^*(\opair{s_1, s_2, 0}, \alpha_1 \dots \alpha_{2n - 1}), \alpha_{2n}) \stackrel{\text{ИП}}{=} \\
             & = \delta(\opair{\delta_1^*(s_1, \alpha_1 \alpha_3 \dots \alpha_{2n - 1}), \delta_2^*(s_2, \alpha_2 \alpha_4 \dots \alpha_{2n - 2}), 1}, \alpha_{2n}) =                \\
             & = \opair{\delta_1^*(s_1, \alpha_1 \alpha_3 \dots \alpha_{2n - 1}), \delta_2^*(s_2, \alpha_2 \alpha_4 \dots \alpha_{2n}), 0}
        \end{align*}
        \begin{align*}
             & \delta^*(\opair{s_1, s_2, 0},\alpha_1 \dots \alpha_{2n + 1}) = \delta(\delta^*(\opair{s_1, s_2, 0}, \alpha_1 \dots \alpha_{2n}), \alpha_{2n + 1}) \stackrel{\text{ИП}}{=} \\
             & = \delta(\opair{\delta_1^*(s_1, \alpha_1 \alpha_3 \dots \alpha_{2n - 1}), \delta_2^*(s_2, \alpha_2 \alpha_4 \dots \alpha_{2n}), 0}, \alpha_{2n + 1}) =                    \\
             & = \opair{\delta_1^*(s_1, \alpha_1 \alpha_3 \dots \alpha_{2n + 1}), \delta_2^*(s_2, \alpha_2 \alpha_4 \dots \alpha_{2n}), 1}
        \end{align*}
        С това сме готови.
    \end{proof}

    Имайки това:
    \begin{flalign*}
        \alpha \in \mathcal{L(A)}  \iff & \delta^*(s, \alpha) \in F \iff                                                                            \\
        \iff                            & \alpha = \alpha_1 \dots \alpha_{2n} (\alpha_i \in \Sigma)                       \: \&                     \\
                                        & \delta_1^*(s_1, \alpha_1 \alpha_3 \dots \alpha_{2n - 1}) \in L_1                \: \&                     \\
                                        & \delta_2^*(s_2, \alpha_2 \alpha_4 \dots \alpha_{2n}) \in L_2 \iff \alpha \in \operatorname{Mix}(L_1, L_2)
    \end{flalign*}

\end{proof}

\pagebreak

\begin{claim}
    Нека $L_1$ и $L_2$ са автоматни езици и $\# \notin \Sigma$.
    Тогава $L_1 \cdot \{ \# \} \cdot L_2$ също е автоматен език.
\end{claim}

\begin{proof}
    Ще покажем само конструкцията, а доказателството ще оставим на читателя.
    Нека $\mathcal{A}_1 = \opair{\Sigma, Q_1, s_1, \delta_1, F_1}$ е автомат за $L_1$ и нека $\mathcal{A}_2 = \opair{\Sigma, Q_2, s_2, \delta_2, F_2}$ е автомат за $L_2$.
    Строим автомат за $L_1 \cdot \{ \# \} \cdot L_2$:
    \begin{itemize}
        \item $\mathcal{A} = \opair{\Sigma, Q, s, \delta, F}$
        \item $Q = Q_1 \cross Q_2 \cross \{ 0, 1 \} \cup \{ \cross \}$
        \item $s = \opair{s_1, s_2, 0}$
        \item $\delta(\opair{p_1, p_2, 0}, x) = \opair{\delta_1(p_1, x), p_2, 1}$ за $\opair{p_1, p_2} \in Q_1 \cross Q_2, \: x \in \Sigma$
        \item $\delta(\opair{p_1, p_2, 0}, \#) = \opair{p_1, p_2, 1}$ за $\opair{p_1, p_2} \in Q_1 \cross Q_2$
        \item $\delta(\opair{p_1, p_2, 1}, x) = \opair{p_1, \delta_2(p_2, x), 0}$ за $\opair{p_1, p_2} \in Q_1 \cross Q_2, \: x \in \Sigma$
        \item $\delta(\opair{p_1, p_2, 1}, \#) = \cross$ за $\opair{p_1, p_2} \in Q_1 \cross Q_2$
        \item $\delta(\cross, x) = \cross$ за $x \in \Sigma \cup \{ \# \}$
        \item $F = F_1 \cross F_2 \cross \{ 1 \}$
    \end{itemize}

    Казано на естествен език, четем със $\mathcal{A}_1$ докато не стигнем $\#$, после четем със $\mathcal{A}_2$.
    Накрая ще искаме да сме прочели $\#$ и прочетеното от $\mathcal{A}_1$ и $\mathcal{A}_2$ да бъде одобрено (т.е. да сме получили отговор ДА).
\end{proof}

\pagebreak

\begin{claim}
    За всеки автоматен $L$ езиците $\operatorname{Pref}(L)$, $\operatorname{Suff}(L)$ и $\operatorname{Infix}(L)$ също са автоматни.
\end{claim}

\begin{proof}
    Нека $\mathcal{A} = \opair{\Sigma, Q, s, \delta, F}$ е автомат за $L$.

    Нека $F' = \{ q \in Q \mid (\exists \gamma \in \Sigma^*) \: (\delta^*(q, \gamma) \in F) \}$
    и $S' = \{ \delta^*(s, \beta) \mid \beta \in \Sigma^* \}$.
    Да помислим кога точно една дума е префикс на дума от $L$:
    \begin{flalign*}
        \beta \in \operatorname{Pref}(L) & \iff (\exists \gamma \in \Sigma^*) \: (\beta \gamma \in L) \iff                             \\
                                         & \iff (\exists \gamma \in \Sigma^*) \: (\delta^*(s, \beta \gamma) \in F) \iff                \\
                                         & \iff (\exists \gamma \in \Sigma^*) \: (\delta^*(\delta^*(s, \beta), \gamma) \in F) \iff     \\
                                         & \iff \delta^*(s, \beta) \in F' \iff \beta \in \mathcal{L}(\opair{\Sigma, Q, s, \delta, F'})
    \end{flalign*}
    Току що показахме автомат за $\operatorname{Pref}(L)$, а именно $\opair{\Sigma, Q, s, \delta, F'}$.
    Така получихме, че $\operatorname{Pref}(L)$ е автоматен език.
    Можем да направим подобни разсъждения за суфикс:
    \begin{flalign*}
        \gamma \in \operatorname{Suff}(L) & \iff (\exists \beta \in \Sigma^*) \: (\beta \gamma \in L) \iff                         \\
                                          & \iff (\exists \beta \in \Sigma^*) \: (\delta^*(s, \beta \gamma) \in F) \iff            \\
                                          & \iff (\exists \beta \in \Sigma^*) \: (\delta^*(\delta^*(s, \beta), \gamma) \in F) \iff \\
                                          & \iff (\exists q \in S') \: (\delta^*(q, \gamma) \in F) \iff                            \\
                                          & \iff \gamma \in \bigcup\limits_{q \in S'}\mathcal{L}(\opair{\Sigma, Q, q, \delta, F})
    \end{flalign*}
    Представихме $\operatorname{Suff}(L)$ като обединение на автоматни езици, откъдето и той е автоматен.
    Освен това $\operatorname{Infix}(L) = \operatorname{Pref}(\operatorname{Suff}(L))$ (от \thref{prefix-suffix-infix-props}), с което сме готови.
\end{proof}
\section{Недетерминирани автомати}

\begin{definition}
    \textbf{Недетерминиран краен автомат} (накратко НКА) ще наричаме всяко $\mathcal{N} = \opair{\Sigma, Q, S, \Delta, F}$, където:
    \begin{itemize}
        \item $\Sigma$ е крайна азбука
        \item $Q$ е крайно множество от състояния
        \item $S \subseteq Q$ (ще ги наричаме начални/стартови състояния)
        \item $\Delta : Q \cross \Sigma \rightarrow \mathcal{P}(Q)$ (ще я наричаме функция на преходите)
        \item $F \subseteq Q$ (ще ги наричаме финални състояния)
    \end{itemize}
\end{definition}

Цялата ``недетерминираност'' идва от това, че от едно състояние с една буква можем да отидем на няколко места.
Поне на пръв поглед тази нова машина изглежда доста по-мощна от старата.
Вместо да си налагаме точно един преход, можем да направим два или три прехода, а можем да отидем в ``нищото'' (има се предвид $\varnothing$).

\begin{definition}
    Дефинираме $\Delta^* : \mathcal{P}(Q) \cross \Sigma^* \rightarrow \mathcal{P}(Q)$ индуктивно:
    \begin{itemize}
        \item $\Delta^*(P, \varepsilon) = P$ за всяко $P \subseteq Q$
        \item $\Delta^*(P, \beta x) = \bigcup\limits_{q \in \Delta^*(P, \beta)}\Delta(q, x)$ за всяко $P \subseteq Q, \beta \in \Sigma^*, x \in \Sigma$
    \end{itemize}
\end{definition}

\begin{remark}
    $\Delta^*(P, \alpha \beta) = \Delta^*(\Delta^*(P, \alpha), \beta)$ (лесно излиза с индукция)
\end{remark}

Вече можем да кажем какъв е език на даден недетерминиран автомат.

\begin{definition}
    Нека $\mathcal{N} = \opair{\Sigma, Q, S, \Delta, F}$ е НКА.
    Тогава езикът на автомата $\mathcal{N}$ е множеството
    $\mathcal{L}(\mathcal{N}) = \{ \alpha \in \Sigma^* \: | \: \delta^*(S, \alpha) \cap F \neq \varnothing \}$.
\end{definition}

Тук цялата идея, е вместо думата да поеме по един предопределен път и да получи ДА или НЕ,
тя да може да изпробва няколко възможни такива, и ако получи отговор ДА за поне един от тях, то тогава получава ДА от целия автомат.
В някакъв смисъл има елемент на отгатване и несигурност. Може този път да свърши работа, но може и другия да свърши работа.

\pagebreak

Нека дадем няколко прости примери за недетерминирани автомати:
\begin{figure*}[h]
    \centering
    \centering
    \begin{tikzpicture}[node distance=2cm, thick]
        \node[initial, state, initial text=] (1) {$q$};
    \end{tikzpicture}
    \caption*{Автомат за $\varnothing$}
\end{figure*}

\begin{figure*}[h]
    \centering
    \begin{tikzpicture}[node distance=2cm, thick]
        \node[initial, state, initial text=] (1) {$\varepsilon$};
        \node[state, initial text=, right of=1] (2) {$a$};
        \node[state, initial text=, right of=2] (3) {$ab$};
        \node[accepting, state, initial text=, right of=3] (4) {$aba$};
        \path[->] (1) edge [] node[above] {$a$}(2);
        \path[->] (2) edge [] node[above] {$b$}(3);
        \path[->] (3) edge [] node[above] {$a$}(4);
    \end{tikzpicture}
    \caption*{Автомат за $\{ aba \}$}
\end{figure*}

Доста по-компактно става представянето на автомати за тези езици.
Напълно елимираме ненужните преходи и състояние боклук.
Можем и много лесно да направим автомат за няколко думи.

\begin{figure*}[h]
    \centering
    \begin{tikzpicture}[node distance=2cm, thick]
        \node[initial, state, initial text=] (1) {$\varepsilon_{aba}$};
        \node[state, initial text=, right of=1] (2) {$a$};
        \node[state, initial text=, right of=2] (3) {$ab$};
        \node[accepting, state, initial text=, right of=3] (4) {$aba$};
        \node[initial, state, initial text=, below of=1] (5) {$\varepsilon_{bab}$};
        \node[state, initial text=, right of=5] (6) {$b$};
        \node[state, initial text=, right of=6] (7) {$ba$};
        \node[accepting, state, initial text=, right of=7] (8) {$bab$};
        \path[->] (1) edge [] node[above] {$a$}(2);
        \path[->] (2) edge [] node[above] {$b$}(3);
        \path[->] (3) edge [] node[above] {$a$}(4);
        \path[->] (5) edge [] node[above] {$b$}(6);
        \path[->] (6) edge [] node[above] {$a$}(7);
        \path[->] (7) edge [] node[above] {$b$}(8);
    \end{tikzpicture}
    \caption*{Автомат за $\{ aba, bab \}$}
\end{figure*}

Тук се опитваме да познаем дали четем една от двете думи.
Двата малки автомата работят независимо един от други и всеки дава собствен отговор.
Ако сме се озовали във финално състояние, сме прочели една от двете думи в езика, иначе не сме.
\section{Операции над езици на недетерминирани автомати}

Тази конструкция може да се обобщи за обединение на всеки два езика на недетерминирани автомати.

\begin{claim}
    Нека $\mathcal{N}_i = \opair{\Sigma, Q_i, S_i, \Delta_i, F_i}$  е недетерминиран автомат за $i = 1, 2$.
    Тогава съществува недетерминиран автомат за езика $\mathcal{L(N}_1) \cup \mathcal{L(N}_2)$.
\end{claim}

\begin{proof}
    Б.О.О. нека $\underbrace{Q_1 \cap Q_2 = \varnothing}_{(\star)}$. Тогава:
    \begin{align*}
        \alpha \in \mathcal{L}(\opair{\Sigma, Q_1 \cup Q_2, S_1 \cup S_2, \underbrace{\Delta_1 \cup \Delta_2}_{\textcolor{red}{!!!}}, F_1 \cup F_2}) \iff & (\Delta_1 \cup \Delta_2)(S_1 \cup S_2, \alpha) \cap (F_1 \cup F_2) \neq \varnothing                      \\
        \stackrel{(\star)}{\iff}                                                                                                                          & \Delta_1^*(S_1, \alpha) \cap F_1 \neq \varnothing \lor \Delta_2^*(S_2, \alpha) \cap F_2 \neq \varnothing \\
        \iff                                                                                                                                              & \alpha \in \mathcal{L(N}_1) \lor \alpha \in \mathcal{L(N}_2)                                             \\
        \iff                                                                                                                                              & \alpha \in \mathcal{L(N}_1) \cup \mathcal{L(N}_2)
    \end{align*}
    \begin{remark}[\textcolor{red}{!!!}]
        Понеже $(\star)$ е вярно, $\operatorname{Dom}(\Delta_1) \cap \operatorname{Dom}(\Delta_2) = \varnothing$, откъдето $\Delta_1 \cup \Delta_2$ е добре дефинирана функция.
    \end{remark}
\end{proof}

Тази конструкция при недетерминирани автомати е по-компактна в сравнение с детерминираната версия.
В този случай новите състояния са $|Q_1 \cup Q_2| = |Q_1| + |Q_2|$ на брой,
докато при детерминираните автомати се получават $|Q_1 \crossproduct Q_2| = |Q_1| \cdot |Q_2|$ на брой състояния за новия автомат, което може да бъде много повече.

\begin{itemize}
    \item $10 + 10 = 20$, докато $10 \cdot 10 = 100$
    \item $1000 + 2 = 1002$, докато $1000 \cdot 2 = 2000$
    \item $1000 + 1000 = 2000$, докато $1000 \cdot 1000 = 1000000$
\end{itemize}

\begin{claim}
    Нека $L$ е автоматен език.
    Тогава има недетерминиран автомат за:
    \begin{center}
        $\operatorname{ChangeSomeLetters}(L) = \{ \alpha \in \Sigma^* \mid (\exists \beta \in L) \: (|\beta| = |\alpha| ) \}$
    \end{center}
\end{claim}

\begin{proof}
    Нека $\mathcal{A} = \opair{\Sigma, Q, s, \delta, F}$ е ДКА за $L$.
    Строим автомат $\mathcal{N}$ за $\operatorname{ChangeSomeLetters}(L)$:
    \begin{itemize}
        \item $\mathcal{N} = \opair{\Sigma, Q, S, \Delta, F}$
        \item $S = \{ s \}$
        \item $\Delta(p, x) = \{ \delta(p, a), \delta(p, b) \}$ за $p \in Q, \: x \in \Sigma$
    \end{itemize}

    Оставяме доказателството на следния факт на читателя, понеже е елементарна индукция:
    \begin{center}
        $\Delta^*(S, \alpha) = \{ \delta^*(s, \beta) \mid \beta \in \Sigma \: \& \: |\beta| = |\alpha| \}$
    \end{center}

    Имайки това твърдение получаваме, че:
    \begin{align*}
        \alpha \in \mathcal{L(N)} & \iff \Delta^*(S, \alpha) \cap F \neq \varnothing \iff (\exists \beta \in \Sigma^{|\alpha|}) \: (\delta^*(s, \beta) \in F) \\
                                  & \iff (\exists \beta \in \Sigma^{|\alpha|}) \: (\beta \in L) \iff \alpha \in \operatorname{ChangeSomeLetters}(L)
    \end{align*}
\end{proof}

\begin{claim}
    Нека $\mathcal{N}_i = \opair{\Sigma, Q_i, S_i, \Delta_i, F_i}$  е недетерминиран автомат за $i = 1, 2$.
    Тогава съществува недетерминиран автомат за езика $\mathcal{L(N}_1) \cdot \mathcal{L(N}_2)$.
\end{claim}

\begin{proof}
    Нека Б.О.О. $Q_1 \cap Q_2 = \varnothing$.
    Строим недетерминиран автомат $\mathcal{N} = \opair{\Sigma, Q, S, \Delta, F}$ за $\mathcal{L(N}_1) \cdot \mathcal{L(N}_2)$:
    \begin{itemize}
        \item $Q = Q_1 \cup Q_2$
        \item $S = S_1 \cup S_2$ ако $\varepsilon \in \mathcal{L(N}_1)$, иначе $S = S_1$
        \item $F = F_2$
        \item $\Delta(p, x) = \Delta_1(p, x) \cup S_2$, ако $\Delta_1(p, x) \cap F \neq \varnothing \: (\star)$
        \item За другите състояния от $Q_1$ взимаме преходите от $\Delta_1$, и аналогично за $Q_2$ взимаме същите преходи от $\Delta_2$
    \end{itemize}

    Сега остава да покажем, че $\mathcal{L(N)} = \mathcal{L(N}_1) \cdot \mathcal{L(N}_2)$.
    \begin{itemize}
        \item $\mathcal{L(N)} \subseteq \mathcal{L(N}_1) \cdot \mathcal{L(N}_2)$ \\
              Нека $\alpha \in \mathcal{L(N)}$.
              Тогава $\Delta^*(S, \alpha) \cap F \neq \varnothing$.
              Ако сме стигнали до финално от $S_2$, то тогава $\alpha \in \mathcal{L(N}_2)$ и $\varepsilon \in \mathcal{L(N}_1)$,
              откъдето $\alpha = \varepsilon \cdot \alpha \in \mathcal{L(N}_1) \cdot \mathcal{L(N}_2)$ и сме готови.
              В противен случай $\varepsilon \notin \mathcal{L(N}_1)$ и има $\beta, \gamma \in \Sigma^*$ такива,
              че $\alpha = \beta \gamma, \: \beta \neq \varepsilon, \: S_2 \subseteq \Delta^*(S, \beta)$,  т.е. сме приложили $(\star)$.
              Тогава с последната буква на $\beta$ сме достигнали до финално състояние в $\mathcal{L(N}_1)$, откъдето $\beta \in \mathcal{L(N}_1)$.
              Също така понеже $Q_1 \cap Q_2 = \varnothing$, и състоянията от $Q_2$ имат само преходите на $\Delta_2$,
              $\Delta(S_2, \gamma) \cap F \neq \varnothing$, откъдето $\gamma \in \mathcal{L(N}_2)$.
              Така получаваме, че $\alpha = \beta \cdot \gamma \in \mathcal{L(N}_1) \cdot \mathcal{L(N}_2)$.

        \item $\mathcal{L(N}_1) \cdot \mathcal{L(N}_2) \subseteq \mathcal{L(N)}$ \\
              Нека $\beta \in \mathcal{L(N}_1), \gamma \in \mathcal{L(N}_2)$.
              Ако $\beta = \varepsilon$, то тогава $S_2 \subseteq S$, и понеже $\gamma \in \mathcal{L(N}_2), \: F = F_2$,
              то $\Delta^*(S, \underbrace{\beta \cdot \gamma}_{\gamma}) \cap F \neq \varnothing$, и от там $\beta \cdot \gamma \in \mathcal{L(N)}$.
              Ако $\beta \neq \varepsilon$, то $S_2 \subseteq \Delta^*(S, \beta)$, понеже $\beta \in \mathcal{L(N}_1)$ и $(\star)$.
              Освен това и $\gamma \in \mathcal{L(N}_2)$ т.е. $\Delta^*(S_2, \gamma) \cap F_2 \neq \varnothing$,
              откъдето $\Delta^*(S, \beta \cdot \gamma) \cap F \neq \varnothing$ т.е. $\beta \cdot \gamma \in \mathcal{L(N)}$.
    \end{itemize}
\end{proof}

\begin{claim}
    Нека $\mathcal{N} = \opair{\Sigma, Q, S, \Delta, F}$ е недетерминиран автомат.
    Тогава има недетерминиран автомат за езика $\mathcal{L(N)^*}$.
\end{claim}

\begin{proof}
    Ще покажем само конструкцията, без да я обосноваваме (аналогична е на предната).
    Правим автомат за $\mathcal{L(N)^+}$ и използваме, че $L^* = L^+ \cup \{ \varepsilon \}$ за всеки език $L$.
    Нека $\mathcal{N}' = \opair{\Sigma, Q, S, \Delta', F}$, където:
    \begin{equation}
        \Delta'(p, x) =
        \begin{cases}
            \Delta(p, x)                     & \text{ако } p \notin F \\
            \Delta(p, x) \cup \Delta^*(S, x) & \text{ако } p \in F
        \end{cases}
    \end{equation}
    Доказателството на коректност е аналогично на предната задача.
\end{proof}

\begin{claim}
    Нека $\mathcal{N} = \opair{\Sigma, Q, S, \Delta, F}$ е недетерминиран автомат.
    Тогава има недетерминиран автомат за езика $\mathcal{L(N)}^{rev}$.
\end{claim}

\begin{proof}
    Нека $\mathcal{N}_{rev} = \opair{\Sigma, Q, F, \Delta_{rev}, S}$, където:
    \begin{center}
        $\Delta_{rev} = \{ \opair{p, x, q} \mid p \in \Delta(q, x) \}$
    \end{center}
    Тривиално може да се покаже с индукция, че за $\alpha \in \Sigma^*, \: p, q \in Q$:
    \begin{center}
        $q \in \Delta^*(\{ p \}, \alpha) \iff p \in \Delta_{rev}^*(\{ q \}, \alpha) \: (\star)$
    \end{center}
    След това приключваме със:
    \begin{align*}
        \alpha \in \mathcal{L(N)} \iff & \Delta^*(S, \alpha) \cap F \neq \varnothing                                            \\
        \stackrel{(\star)}{\iff}       & \Delta_{rev}^*(F, \alpha) \cap S \neq \varnothing \iff \alpha \in \mathcal{L(N}_{rev})
    \end{align*}
\end{proof}
\section{Еквивалентност на детерминирани и недетерминирани автомати}

Въпреки че на пръв поглед недетерминираните автомати ни се струват по-мощни, те не ни дават нищо повече освен удобство.
Първо ще покажем, по-очевидното твърдение.

\begin{claim}
    За всеки детерминиран автомат $\mathcal{A}$ съществува недетерминиран автомат $\mathcal{N}$ със $\mathcal{L(N) = L(A)}$.
\end{claim}

\begin{proof}
    Нека $\mathcal{A} = \opair{\Sigma, Q, s, \delta, F}$ е детерминиран автомат.
    Нека $\mathcal{N} = \opair{\Sigma, Q, \{ s \}, \Delta, F}$,
    където $\Delta(p, x) = \{ \delta(p, x) \}$ за $p \in Q$.
    Тривиално може да се покаже с индукция, че за всяко $p \in Q, \: \alpha \in \Sigma^*$: $\Delta^*(p, \alpha) = \{ \delta^*(p, \alpha) \}$.
    Така:
    \begin{align*}
        \alpha \in \mathcal{L(N)} & \iff \Delta^*(\{ s \}, \alpha) \cap F \neq \varnothing \iff \{ \delta^*(s, \alpha) \} \cap F \neq \varnothing \\
                                  & \iff \delta^*(s, \alpha) \in F \iff \alpha \in \mathcal{L(A)}
    \end{align*}
\end{proof}

Сега пък ще покажем как ``детерминизира'' недетерминиран автомат.

\begin{claim}
    За всеки недетерминиран автомат $\mathcal{N}$ съществува детерминиран автомат $\mathcal{A}$ със $\mathcal{L(A) = L(N)}$.
\end{claim}

\begin{proof}
    Нека $\mathcal{N} = \opair{\Sigma, Q, S, \Delta, F}$ е недетерминиран автомат.
    Нека $\mathcal{A} = \opair{\Sigma, \mathcal{P}(Q), S, \delta, F'}$,
    където $\delta(P, x) = \Delta^*(P, x)$ за $P \subseteq Q$ и $F' = \{ P \in \mathcal{P}(Q) \mid P \cap F \neq \varnothing \}$.
    Тривиално може да се покаже с индукция, че за всяко $P \subseteq Q, \: \alpha \in \Sigma^*$: $\delta^*(P, \alpha) = \Delta^*(P, \alpha)$.
    Така:
    \begin{align*}
        \alpha \in \mathcal{L(N)} & \iff \Delta^*(S, \alpha) \cap F \neq \varnothing \iff \delta^*(S, \alpha) \cap F \neq \varnothing \\
                                  & \iff \delta^*(S, \alpha) \in F' \iff \alpha \in \mathcal{L(A)}
    \end{align*}
\end{proof}

Имайки това вече можем да използваме, че $\cdot$, $*$ и $rev$ запазват автоматност.
\begin{warning}
    При ``детерминизиране'' броят на състоянията нараства експоненциално.
    Накратко $10$ става $1024$.
\end{warning}

\begin{problem}
Да се детерминира следният автомат:
\begin{center}
    \begin{tikzpicture}[shorten >=1pt,node distance=2.5cm,>=stealth',thick]
        \node[initial, state, initial text=] (1) {$0$};
        \node[state] [right of=1] (2) {$1$};
        \node[initial, state, initial text=] [below left of=2] (3) {$2$};
        \node[accepting, state] [right of=3] (4) {$3$};
        \path[->] (1) edge [loop above] node[above] {$a$} (1);
        \path[->] (1) edge node[above] {$a, b$} (2);
        \path[->] (1) edge node[left] {$b$} (3);
        \path[->] (2) edge node[above] {$b$} (3);
        \path[->] (2) edge node[right] {$b$} (4);
        \path[->] (3) edge [loop below] node[below] {$a$} (3);
        \path[->] (3) edge node[above] {$a$} (4);
    \end{tikzpicture}
\end{center}
\end{problem}
\section{Друг начин да си мислим за автоматите}

Освен като абстрактни машини, можем да си мислим за детерминираните автомати като за програми.
Тези ``програми'' имат няколко много хубави свойства:
\begin{itemize}
    \item винаги завършват работа
    \item работят с константна памет
    \item работят за линейно време спрямо дължината на дума
\end{itemize}

Нека вземем за пример следният автомат:

\begin{center}
    \begin{tikzpicture}[shorten >=1pt,node distance=2.5cm,>=stealth',thick]
        \node[accepting, initial, state, initial text=] (1) {$0$};
        \node[state] [right of=1] (2) {$1$};
        \path[->] (1) edge [loop above] node[above] {$b$} (1);
        \path[->] (1) edge [bend left] node[above] {$a$} (2);
        \path[->] (2) edge [loop above] node[above] {$b$} (2);
        \path[->] (2) edge [bend left] node[below] {$a$} (1);
    \end{tikzpicture}
\end{center}

Ясно е, че той разпознава думи с четен брой букви $a$.
Нека видим как бихме направили програма на C++, която приема низ и връща дали този низ съдържа четен брой $a$:

\begin{minted}[linenos]{C++}
typedef char state_index;

const state_index table[2][2] = {{1, 0}, {0, 1}};   // таблица на преходите
const bool is_accepting_state[2] = { true, false }; // финални състояния
const state_index INITIAL_STATE = 0;                // начално състояние

// функция на преходите
state_index delta(state_index state, char letter) { return table[state][letter - 'a']; }

bool accepts_word(const std::string &word)
{
  // предполагаме валиден вход
  // т.е. низът съдържа само буквите 'a' и 'b'
  state_index state = INITIAL_STATE; // започваме в началното състояние

  for (size_t i = 0; i < word.size(); ++i)
  {
    state = delta(state, word[i]); // правим преход с настоящото състояние и буква
  }

  return is_accepting_state[state]; // накрая искаме да сме във финално състояние
}
\end{minted}

Лесно можем да видим как тази конструкция може да се адаптира за произволен автомат $\mathcal{A} = \opair{\Sigma, Q, q_0, \delta, F}$:
\begin{itemize}
    \item Нека $Q = \{ q_0, \dots q_{n - 1} \}$.
          За състоянията се разбираме, че на $q_i$ съответства числото $i$.
          За типа на \mintinline{C++}|state_index| ще ни трябва променлива, която ще побере числата от $0$ до $n - 1$.
    \item Нека $\Sigma = \{ \sigma_0, \dots, \sigma_{k - 1} \}$.
          Нужна ни е функция, която да е биекция между $\Sigma$ и $\{ 0, \dots, k - 1 \}$.
          Нека сигнатурата и да бъде \mintinline{C++}|state_index letter_to_index(char letter)|.
          Също така ще искаме да работи в константно време, което е лесно, защото азбуката е с фиксиран размер.
    \item Правим \mintinline{C++}|state_index table[n][k]| спрямо нашия оригинален автомат и кодировката от \mintinline{C++}|letter_to_index|:
          \begin{center}
              \mintinline{C++}|table[i][letter_to_index(c)] == j| е истина $\iff \delta^*(q_i, c) = q_j$
          \end{center}
    \item Правим \mintinline{C++}|bool is_accepting_state[n]| спрямо нашия оригинален автомат и кодировката от \mintinline{C++}|letter_to_index|:
          \begin{center}
              \mintinline{C++}|table[i]| е истина $\iff q_i \in F$
          \end{center}
    \item Понеже $q_0$ е началното състояние:
          \begin{minted}{C++}
const state_index INITIAL_STATE = 0;
          \end{minted}
    \item Правим програмна реализация на $\delta$ чрез \mintinline{C++}|table|:
          \begin{minted}{C++}
state_index delta(state_index state, char letter)
{
    return table[state][letter_to_index(letter)];
}
          \end{minted}
    \item Функцията \mintinline{C++}|accepts_word| няма нужда от модификация.
          В зависимост от това дали има нужда може да се сложи валидация на входните данни.
\end{itemize}

\begin{remark}
    Разбира се това не е единственият възможен начин да се направи програмна реализация на автомат.
    За по-малко на брой състояния също би било удобно да се използват оператори \mintinline{C++}|if| или \mintinline{C++}|switch| със \mintinline{C++}|case|.
    Обаче да имаме функцията $\delta$ записана по някакъв начин в таблица ни прави програмата по-бърза при повече състояния.
    Много по-лесно е да намериш елемент номер $10000$ в масив отколкото да оцениш $10000$ пъти оператор \mintinline{C++}|if|.
\end{remark}

\begin{problem}
Да се направи реализация използвайки \mintinline{C++}|if| или \mintinline{C++}|switch| със \mintinline{C++}|case| вместо таблица за $\delta$ на следния автомат:
\begin{center}
    \begin{tikzpicture}[shorten >=1pt,node distance=2.5cm,>=stealth',thick]
        \node[accepting, initial, state, initial text=] (1) {$0$};
        \node[state] [right of=1] (2) {$1$};
        \path[->] (1) edge [loop above] node[above] {$b$} (1);
        \path[->] (1) edge  node[above] {$a$} (2);
        \path[->] (2) edge [loop above] node[above] {$a, b$} (2);
    \end{tikzpicture}
\end{center}
\end{problem}

\begin{problem}
Да се направи програмна реализация и да се определи езика на следните автомати:
\begin{figure*}[h]
    \begin{subfigure}{1.0\textwidth}
        \centering
        \begin{tikzpicture}[shorten >=1pt,node distance=2.5cm,>=stealth',thick]
            \node[initial, state, initial text=] (1) {$0$};
            \node[accepting, state] [right of=1] (2) {$1$};
            \path[->] (1) edge [loop above] node[above] {$a$} (1);
            \path[->] (1) edge [bend left] node[above] {$b$} (2);
            \path[->] (2) edge [loop above] node[above] {$a$} (2);
            \path[->] (2) edge [bend left] node[below] {$b$} (1);
        \end{tikzpicture}
    \end{subfigure}
    \vspace{\floatsep}
    \begin{subfigure}{1.0\textwidth}
        \centering
        \begin{tikzpicture}[shorten >=1pt,node distance=2.5cm,>=stealth',thick]
            \node[initial, state, initial text=] (1) {$0$};
            \node[accepting, state] [right of=1] (2) {$1$};
            \node[accepting, state] [right of=2] (3) {$2$};
            \path[->] (1) edge [loop above] node[above] {$a$} (1);
            \path[->] (1) edge node[above] {$b$} (2);
            \path[->] (2) edge [loop above] node[above] {$a$} (2);
            \path[->] (2) edge  node[above] {$b$} (3);
            \path[->] (3) edge [loop above] node[above] {$a$} (3);
            \path[->] (3) edge [bend left] node[below] {$b$} (1);
        \end{tikzpicture}
    \end{subfigure}
    \vspace{\floatsep}
    \begin{subfigure}{1.0\textwidth}
        \centering
        \begin{tikzpicture}[node distance=2.5cm,>=stealth',thick]
            \node[accepting, initial, state with output, initial text=] (1) {$0$ \nodepart{lower} $0$};
            \node[state with output] [right of=1] (2) {$0$ \nodepart{lower} $1$};
            \node[state with output, initial text=] [below of=1](3) {$1$ \nodepart{lower} $0$};
            \node[accepting, state with output] [right of=3] (4) {$1$ \nodepart{lower} $1$};
            \path[->] (1) edge [bend left] node[above] {$a$} (2);
            \path[->] (2) edge [bend left] node[below] {$a$} (1);
            \path[->] (3) edge [bend left] node[above] {$a$} (4);
            \path[->] (4) edge [bend left] node[below] {$a$} (3);
            \path[->] (1) edge [bend left] node[right] {$b$} (3);
            \path[->] (3) edge [bend left] node[left] {$b$} (1);
            \path[->] (2) edge [bend left] node[right] {$b$} (4);
            \path[->] (4) edge [bend left] node[left] {$b$} (2);
        \end{tikzpicture}
    \end{subfigure}
\end{figure*}
\end{problem}

\section{Регулярни изрази и регулярни езици}

Вече знаем как да превръщаме автоматите в реални програми.
Това даде живот на нашата концептуална машина за валидация на текст.
Ние не само имаме програмна реализация, тя също така има много хубави свойства.
Въпреки всичко това има нещо което ни липсва.
Понякога бихме искали да представим тази информация в по-компактен вариант от една картинка или програма.
Тук идват на помощ регулярните изрази.

\begin{definition}
    Дефинираме множеството от \textbf{регулярни изрази} $\mathcal{RE}(\Sigma)$ индуктивно:
    \begin{itemize}
        \item $\varnothing, \varepsilon \in \mathcal{RE}(\Sigma)$ и $\sigma \in \mathcal{RE}(\Sigma)$ за всяко $\sigma \in \Sigma$
        \item ако $r_1, r_2 \in \mathcal{RE}(\Sigma)$, то тогава $r_1 \cdot r_2, r_1 + r_2 \in \mathcal{RE}(\Sigma)$
        \item ако $r \in \mathcal{RE}(\Sigma)$, то тогава $r^* \in \mathcal{RE}(\Sigma)$
    \end{itemize}
\end{definition}

\begin{remark}
    Освен това имаме и скоби за приоритет на операциите.
    При липса на скоби с най-голям приоритет е $*$, после $\cdot$, и накрая $+$.
    Също така за краткост понякога ще изпускаме $\cdot$ както се прави при умножението.
\end{remark}

\begin{definition}
    Дефинираме индуктивно \textbf{езика $\regexlang{r}$ на регулярен израз $r \in \mathcal{RE}(\Sigma)$}:
    \begin{itemize}
        \item $\regexlang{\varnothing} = \varnothing$ и $\regexlang{x} = \{ x \}$ за всяко $x \in \Sigma_{\varepsilon} \: (\Sigma \cup \{ \varepsilon \})$
        \item $\regexlang{r_1 \cdot r_2} = \regexlang{r_1} \cdot \regexlang{r_2}$
        \item $\regexlang{r_1 \cup r_2} = \regexlang{r_1} \cup \regexlang{r_2}$
        \item $\regexlang{r^*} = \regexlang{r}^*$
    \end{itemize}
\end{definition}

\begin{definition}
    Език $L \subseteq \Sigma^*$ наричаме \textbf{регулярен}, ако съществува регулярен израз $r \in \mathcal{RE}(\Sigma)$ такъв, че $\regexlang{r} = L$
\end{definition}

Ето няколко прости примери за регулярни изрази и техните езици:
\begin{itemize}
    \item $\regexlang{(a+b)^*} = \Sigma^*$
    \item $\regexlang{((a+b)(a+b))^*} =(\Sigma^2)^*$
    \item $\regexlang{(b^*ab^*ab^*)^*} = \{ \alpha \in \Sigma^* \mid |\alpha| \equiv 0 \pmod{2} \}$
    \item $\regexlang{b^*ab^*a(b^*ab^*ab^*ab^*)^*} = \{ \alpha \in \Sigma^* \mid |\alpha| \equiv 2 \pmod{3} \}$
\end{itemize}

Първите две са очевидни.
Ще обосновем много накратко третия пример и ще оставим четвъртия на читателя.
Едната посока е:
\begin{center}
    $\regexlang{(b^*ab^*ab^*)^*} \subseteq \{ \alpha \in \Sigma^* \mid |\alpha| \equiv 0 \pmod{2} \}$
\end{center}
Със $*$ наслагваме ``блокчета'' от $b^*ab^*ab^*$, които са все думи със две на брой букви $a$:
\begin{center}
    \begin{tabular}{|c|c|c|c|}
        \hline
        $b \dots b \: a \: b \dots b \: a \: b \dots b$                                                                                                                      &
        $b \dots b \: a \: b \dots b \: a \: b \dots b$                                                                                                                      &
        $\dots$                                                                                                                                                              &
        $b \dots b \: a \: b \dots b \: a \: b \dots b$                                                                                                                        \\
        \hline
        \multicolumn{1}{@{}c@{}}{$\underbrace{\hspace*{\dimexpr\tabcolsep+2\arrayrulewidth}\hphantom{b \dots b \: a \: b \dots b \: a \: b \dots b}}_{+2 \text{ букви } a}$} &
        \multicolumn{1}{@{}c@{}}{$\underbrace{\hspace*{\dimexpr\tabcolsep+2\arrayrulewidth}\hphantom{b \dots b \: a \: b \dots b \: a \: b \dots b}}_{+2 \text{ букви } a}$} &
        \multicolumn{1}{@{}c@{}}{}                                                                                                                                           &
        \multicolumn{1}{@{}c@{}}{$\underbrace{\hspace*{\dimexpr\tabcolsep+2\arrayrulewidth}\hphantom{b \dots b \: a \: b \dots b \: a \: b \dots b}}_{+2 \text{ букви } a}$}
    \end{tabular}
\end{center}
Колкото и пъти да събираме четни числа, винаги като резултат ще получим четно число.
Думата очевидно е от $\{ \alpha \in \Sigma^* \mid |\alpha| \equiv 0 \pmod{2} \}$.
Другата посока е:
\begin{center}
    $\{ \alpha \in \Sigma^* \mid |\alpha| \equiv 0 \pmod{2} \} \subseteq \regexlang{(b^*ab^*ab^*)^*}$
\end{center}
Ако вземем дума от ляво, можем да я разделим на всеки две срещания на буквата $a$:
\begin{center}
    \begin{tabular}{|c|c|c|c|}
        \hline
        $b \dots b \: a \: b \dots b \: a$                                                                                                                              &
        $b \dots b \: a \: b \dots b \: a$                                                                                                                              &
        $\dots$                                                                                                                                                         &
        $b \dots b \: a \: b \dots b \: a \: b \dots b$                                                                                                                   \\
        \hline
        \multicolumn{1}{@{}c@{}}{$\underbrace{\hspace*{\dimexpr\tabcolsep+2\arrayrulewidth}\hphantom{b \dots b \: a \: b \dots b \: a}}_{\in \regexlang{b^*ab^*ab^*}}$} &
        \multicolumn{1}{@{}c@{}}{$\underbrace{\hspace*{\dimexpr\tabcolsep+2\arrayrulewidth}\hphantom{b \dots b \: a \: b \dots b \: a}}_{\in \regexlang{b^*ab^*ab^*}}$} &
        \multicolumn{1}{@{}c@{}}{}                                                                                                                                      &
        \multicolumn{1}{@{}c@{}}{$\underbrace{\hspace*{\dimexpr\tabcolsep+2\arrayrulewidth}\hphantom{b \dots b \: a \: b \dots b \: a \: b \dots b}}_{\in \regexlang{b^*ab^*ab^*}}$}
    \end{tabular}
\end{center}
Вече е очевидно, че думата е от $\regexlang{(b^*ab^*ab^*)^*}$.
\section{Операции над регулярни езици}
\section{Еквивалентност на регулярните и автоматните езици}

\begin{theorem}[Теорема на Клини]
    За всеки език $L \subseteq \Sigma^*$:
    \begin{center}
        $L$ е автоматен $\iff$ $L$ е регулярен
    \end{center}
\end{theorem}

\begin{proof}
    $(\Leftarrow)$ С индукция по строене на регулярните езици.

    \begin{itemize}
        \item $\varnothing, \varepsilon, \{ a \}, \{ b \}$ са автоматни езици
        \item $\cup, \: \cdot$ и $*$ запазват автоматност
    \end{itemize}

    $(\Rightarrow)$ Тук ще трябва да поработим малко повече.

    Нека $\mathcal{A} = \opair{\Sigma, Q, q_0, \delta, F}$ е ДКА за $L$.
    Нека $Q = \{ q_0, \dots, q_{n - 1} \}$.
    Със $L(i, j, k)$ ще бележим множеството от всички думи $\alpha$,
    за които $\delta^*(q_i, \alpha) = q_j$ и всички междинни състояния имат индекс $< k$.
    Ясно е, че $L = \bigcup \{ L(0, j, n) \mid q_j \in F \}$.
    Строим $L(i, j, k)$ рекурсивно:

    За $k = 0$ има две възможности (които тривиално са регулярни езици):
    \begin{itemize}
        \item за $i = j$ имаме, че $L(i, j, 0) = \{ \varepsilon \} \cup \{ x \in \Sigma \mid \delta(q_i, x) = q_j \}$
        \item за $i \neq j$ имаме, че $L(i, j, 0) = \{ x \in \Sigma \mid \delta(q_i, x) = q_j \}$
    \end{itemize}
    Имайки $L(i, j, k)$, за $L(i, j, k + 1)$ имаме две възможности:
    \begin{itemize}
        \item $q_k$ да не се среща като междинно. Тогава сме в $L(i, j, k)$
        \item $q_k$ се среща като междинно. Разделяме по срещнанията на $q_k$:
    \end{itemize}
    \begin{center}
        \begin{tabular}{|c|c|c|c|c|}
            \hline
            $\in L(i, k, k)$                                                                                                                                   &
            $\in L(k, k, k)$                                                                                                                                   &
            $\dots$                                                                                                                                            &
            $\in L(k, k, k)$                                                                                                                                   &
            $\in L(k, j, k)$                                                                                                                                     \\
            \hline
            \multicolumn{1}{@{}c@{}}{$\underbrace{\hspace*{\dimexpr\tabcolsep+2\arrayrulewidth}\hphantom{\text{1-во срещане}}}_{\text{1-во срещане}}$}         &
            \multicolumn{1}{@{}c@{}}{$\underbrace{\hspace*{\dimexpr\tabcolsep+2\arrayrulewidth}\hphantom{\text{2-ро срещане}}}_{\text{2-ро срещане}}$}         &
            \multicolumn{1}{@{}c@{}}{}                                                                                                                         &
            \multicolumn{1}{@{}c@{}}{$\underbrace{\hspace*{\dimexpr\tabcolsep+2\arrayrulewidth}\hphantom{\text{последно срещане}}}_{\text{последно срещане}}$} &
            \multicolumn{1}{@{}c@{}}{}
        \end{tabular}
    \end{center}
    Тъй като го разделихме на всички срещания няма как да получим междинни с индекс $k$.
    Накрая:
    \begin{center}
        $L(i, j, k + 1) = L(i, j, k) \cup (L(i, k, k) \cdot L(k, k, k)^* \cdot L(k, j, k))$
    \end{center}
\end{proof}

Така добавихме още един начин за да опишем класът от езици, с който се занимаваме.

\begin{warning}
    От понятията регулярен език и автоматен език вече ще ползваме само регулярен.
\end{warning}


\section{Нерегулярни езици}

До сега всичко, което сме правили е да се опитаме да докажем, че някакъв език е регулярен.
Тогава идва естественият въпрос дали има нерегулярни езици, и ако да - как изглеждат те.
Че има нерегулярни езици е ясно, иначе тази класификация на езици щеше да е безсмислена.
Какъв обаче би бил един такъв език и как точно да докажем, че не може да се направи автомат или регулярен израз за него.
Като че ли изглежда по-лесно да докажеш съществуването на обект като го конструираш, отколкото да докажеш, че такъв не може да се построи.

Ще вземем два метода за доказването на нерегулярност на езици.
Кое кога да се използва е въпрос на лично предпочитание.
За първият метод ще трябва да въведем една допълнителна дефиниция.

\begin{definition}
    Нека $\alpha \in \Sigma^*$ и $L \subseteq \Sigma^*$. Тогава:
    \begin{align*}
        \alpha^{-1}(L) = \{ \beta \in \Sigma^* \mid \alpha \cdot \beta \in L \}
    \end{align*}
\end{definition}

\begin{claim}[класови критерий за нерегулярност]\thlabel{nerode-nonregular}
    Нека множеството $\{ \alpha^{-1}(L) \mid \alpha \in \Sigma^* \}$ е безкрайно.
    Тогава $L$ не е регулярен
\end{claim}

\begin{proof}
    Нека $\mathcal{A} = \opair{\Sigma, Q, s, \delta, F}$ е ДКА за $L$. Тогава:
    \begin{align*}
        \beta \in \alpha^{-1}(L) & \iff \alpha \cdot \beta \in L \iff \delta^*(s, \alpha \cdot \beta) \in F                                                                                \\
                                 & \iff \delta^*(\delta^*(s, \alpha), \beta) \in F \iff \beta \in \mathcal{L}(\underbrace{\opair{\Sigma, Q, \delta^*(s, \alpha), \delta, F}}_{\text{ДКА}})
    \end{align*}

    От тук можем да видим, че ако $L$ е регулярен, то за всяка дума $\alpha$,
    на $\alpha^{-1}(L)$ съпоставяме състояние от $Q$ т.е. $|\{ \alpha^{-1}(L) \mid \alpha \in \Sigma^* \}| \leq |Q| < \infty$.
    Получаваме импликацията:
    \begin{center}
        $L$ е регулярен $\Rightarrow \{ \alpha^{-1}(L) \mid \alpha \in \Sigma^* \}$ е крайно
    \end{center}
    Но ние знаем, че $p \Rightarrow q$ е еквивалентно на $\neg q \Rightarrow \neg p$, откъдето:
    \begin{center}
        $\{ \alpha^{-1}(L) \mid \alpha \in \Sigma^* \}$ е безкрайно $\Rightarrow L$ не е регулярен
    \end{center}
\end{proof}

\pagebreak

Започваме с каноничния пример за нерегулярен език:

\begin{claim}
    Езикът $L = \{ a^nb^n \mid n \in \mathbb{N} \}$ не е регулярен.
\end{claim}

\begin{proof}
    Искаме да покажем, че $\{ \alpha^{-1}(L) \mid \alpha \in \Sigma^* \}$ е безкрайно.
    За целта ни трябват изброима редица от думи $\alpha_0, \alpha_1, \alpha_2, \dots$ такава, че за $i \neq j : \alpha_i^{-1}(L) \neq \alpha_j^{-1}(L)$.
    Твърдя, че думите от вида $a^n$ за $n \in \mathbb{N}$ ще ни свършат работа.
    Нека $n, k \in \mathbb{N}, \: n \neq k$.
    Тогава:
    \begin{itemize}
        \item $a^nb^n \in L \Rightarrow b^n \in (a^n)^{-1}(L)$
        \item $a^kb^n \notin L \Rightarrow b^n \notin (a^k)^{-1}(L)$
    \end{itemize}
    Така получаваме, че $(a^n)^{-1}(L) \neq (a^k)^{-1}(L)$.

    Тук $(a^0)^{-1}(L), (a^1)^{-1}(L), (a^2)^{-1}(L), \dots$ са безброй много съществено различни елементи на $\{ \alpha^{-1}(L) \mid \alpha \in \Sigma^* \}$.
    Така по \nameref{nerode-nonregular} езикът $L$ не е регулярен.
\end{proof}

\begin{problem}
Да се докаже, че следните езици не са регулярни:
\begin{itemize}
    \item $L_1 = \{ a^nb^{2n} \mid n \in \mathbb{N} \}$
    \item $L_2 = \{ a^{3n}b^n \mid n \in \mathbb{N} \}$
    \item $L_3 = \{ a^{5n}b^{7n} \mid n \in \mathbb{N} \}$
\end{itemize}
Упътване: напълно аналогично на горния пример
\end{problem}

\begin{claim}
    Езикът $L = \{ a^nb^m \mid n \leq m \}$ не е регулярен.
\end{claim}

\begin{proof}
    Отново ще пробваме с думи от вида $a^n$ за $n \in \mathbb{N}$.
    Нека $n, k \in \mathbb{N}, \: n \neq k$ като Б.О.О. $n < k$.
    \begin{itemize}
        \item Понеже $n \leq n, \: a^nb^n \in L$, откъдето $b^n \in (a^n)^{-1}(L)$
        \item Понеже $n < k \iff \neg(k \leq n), \: a^kb^n \notin L$, откъдето $b^n \notin (a^k)^{-1}(L)$
    \end{itemize}
    По \nameref{nerode-nonregular} езикът $L$ не е регулярен.
\end{proof}

\begin{claim}
    Езикът $L = \{ a^{n^2} \mid n \in \mathbb{N} \}$ не е регулярен.
\end{claim}

\begin{proof}
    Отново ще пробваме с думи от вида $a^n$ за $n \in \mathbb{N}$.
    Нека $n, k \in \mathbb{N}, \: n \neq k$ като Б.О.О. $k < n$.
    \begin{itemize}
        \item $a^na^{n^2+n+1} = a^{(n+1)^2} \in L$, откъдето $a^{n^2+n+1} \in (a^n)^{-1}(L)$
        \item $n^2 < n^2+n+k+1 < n^2+2n+1=(n+1)^2$, следователно $a^ka^{n^2+n+1} \notin L$, откъдето $a^{n^2+n+1} \notin (a^k)^{-1}(L)$
    \end{itemize}
    По \nameref{nerode-nonregular} езикът $L$ не е регулярен.
\end{proof}

Това е често срещана техника при езици от вида $\{ a^{f(n)} \mid n \in \mathbb{N} \}$, за някоя $f : \mathbb{N} \rightarrow \mathbb{N}$ строго растяща.
За да се ``изкара'' думата от езика е много удобно да се използва, че ако $f(n) < x < f(n+1)$, то $x$ няма как да е член на редицата.

\begin{problem}\thlabel{nonregular-problems-1}
Да се докаже, че следните езици не са регулярни:
\begin{itemize}
    \item $L_1 = \{ a^{n^3} \mid n \in \mathbb{N} \}$
    \item $L_2 = \{ a^{2^n} \mid n \in \mathbb{N} \}$
    \item $L_3 = \{ a^{n!} \mid n \in \mathbb{N} \}$
\end{itemize}
Упътване: да се използва горепоказаната техника
\end{problem}

\begin{claim}
    Езикът $L = \{ \alpha \alpha^{rev} \mid \alpha \in \Sigma^* \}$ е нерегулярен.
\end{claim}

\begin{proof}
    Тук думи от вида $a^n$ няма да ни свършат работа.
    Изобщо при еднобуквена азбука езикът е регулярен.
    Получават се думи с четна дължина.
    Ще трябва да използваме и други букви за да успеем да изкараме този резултат.
    Нека $n, k \in \mathbb{N}, \: n \neq k$.
    \begin{itemize}
        \item $a^nb^nb^na^n \in L$
        \item $a^kb^nb^na^n \notin L$ (очевидно $a^k \neq a^n$, а $a^kb^t$ за $0 \leq t \leq 2n$ са единствените кандидати за $\alpha$, че да стане $\alpha \alpha^{rev}$)
    \end{itemize}
    По \nameref{nerode-nonregular} езикът $L$ не е регулярен.
\end{proof}

\pagebreak

\begin{problem}\thlabel{nonregular-problems-2}
Да се докаже, че следните езици не са регулярни:
\begin{itemize}
    \item $L_1 = \{ \alpha \alpha \mid \alpha \in \Sigma^* \}$
    \item $L_2 = \{ \alpha \alpha \alpha \mid \alpha \in \Sigma^* \}$
    \item $L_3 = \{ \alpha \alpha^{rev} \alpha \mid \alpha \in \Sigma^* \}$
\end{itemize}
Упътване: напълно аналогично на горния пример
\end{problem}

Нека сега видим какво се случва с различните операции.
Ако $L$ не е регулярен, то и $\overline{L}$ също не е регулярен.
В противен случай $\overline{\overline{L}} = L$ би бил регулярен.
Същите разсъждения можем да направим за $rev$.
Тази затвореност я няма при други операции, които сме разглеждали.
Тук се запазва поради обратимостта на тази операция.

\begin{warning}
    Следните са често срещани грешки:
    \begin{itemize}
        \item \textbf{\textit{\textcolor{red}{конкатенация на два нерегулярни езика винаги ни дава нерегулярен език}}} \\
              Нека вземем за пример някой нерегулярен език $L$.
              $\overline{L}$ също няма да е регулярен.
              Добавяйки крайно много елементи към който и да е нерегулярен език го оставя нерегулярен.
              В противен случай бихме махнали тези крайно много думи (те образуват регулярен език) и ще получим, че първоначалният език е регулярен.
              Въпреки че $(L \cup \{ \varepsilon \})$ и $(\overline{L} \cup \{ \varepsilon \})$ не са регулярни езици,
              конкатенацията им $(L \cup \{ \varepsilon \}) \cdot (\overline{L} \cup \{ \varepsilon \}) = \Sigma^*$ е регулярен език.
              Също така не е вярно, че при конкатенация на нерегулярни се получава винаги регулярен.
              Тривиално се проверява, че $\{ a^nb^n \mid n \in \mathbb{N} \} \cdot \{ a^nb^n \mid n \in \mathbb{N} \} = \{ a^nb^na^mb^m \mid n, m \in \mathbb{N} \}$ не е регулярен.

        \item \textbf{\textit{\textcolor{red}{обединение на два нерегулярни езика винаги ни дава нерегулярен език}}} \\
              Абсолютно същите примери вършат работа и тук.
    \end{itemize}
\end{warning}

\section{Лема за покачването}

Сега ще покажем другия критерий за доказване на нерегулярност.
Той понякога е по-удобен за използване, обаче не е винаги приложим, и е по-вербозен.

\begin{lemma}[Лема за покачването]\thlabel{pumping-lemma}
    Нека $L$ е регулярен език. Тогава:
    \begin{align*}
         & (\exists p \geq 1)                                                             \\
         & (\forall \alpha \in L, \: |\alpha| \geq 1)                                     \\
         & (\exists x, y, z \in \Sigma^*, \: xyz = \alpha, \: |xy| \leq p, \: |y| \geq 1) \\
         & (\forall i \in \mathbb{N}) [xy^iz \in L]
    \end{align*}
\end{lemma}

\begin{proof}
    Езикът $L$ е регулярен.
    Следователно съществува ДКА
    $\mathcal{A} = \opair{Q, \Sigma, s, \delta, F}$,
    такъв че $\mathcal{L}(\mathcal{A}) = L$.

    Полагаме $p = |Q|$ и нека $q_1, \dots, q_p$ са състоянията от $Q$.

    Нека $\alpha \in L$ е такава, че $|\alpha| = n$, където $n \geq p$.
    Ще разбием $\alpha$ на $\alpha_1, \dots, \alpha_n \in \Sigma$ (т.е. $\alpha = \alpha_1\dots\alpha_n$).

    Знаем, че съществуват $i_0, \dots, i_n \in \{1, \dots, n \}$ такива,
    че $s = q_{i_0}$ и за всяко $j \in \{1, \dots, n\} : \delta(q_{i_{j-1}}, \alpha_j) = q_{i_j}$.

    Нека разгледаме думата $\alpha_1 \dots \alpha_p$.
    За нея знаем, че по време на четенето на думата автоматът минава през $p + 1$ състояния.
    Следователно по принципът на Дирихле съществуват
    $t_1, t_2 \in \{1, \dots, p\}$, където $t_1 < t_2$ такива, че $q_{i_{t_1}} = q_{i_{t_2}}$.

    Полагаме $x = \alpha_1 \dots \alpha_{t_1}$, $y = \alpha_{t_1 + 1} \dots \alpha_{t_2}$ и $z = \alpha_{t_2 + 1} \dots \alpha_n$.
    Сигурни сме, че $|xy| \leq p$, защото $t_2 \leq p$ и че $|y| \geq 1$ понеже $t_1 \neq t_2$.
    Знаем, че $\delta^*(q_{i_{t_1}}, y) = q_{i_{t_2}}$.
    Искаме да проверим, че можем да ``циклим'' със $y$.

    \begin{claim}\thlabel{pl-helper}
        $(\forall i \in \mathbb{N}) (\delta^*(q_{i_{t_1}}, y^i) = q_{i_{t_2}})$.
    \end{claim}

    \begin{proof}
        Ще докажем твърдението с индукция по $i \in \mathbb{N}$.

        База: $\delta^*(q_{i_{t_1}}, \epsilon) = q_{i_{t_1}} = q_{i_{t_2}}$ \checkmark

        ИС: $\delta^*(q_{i_{t_1}}, y^{i+1}) = \delta(\delta^*(q_{i_{t_1}}, y^i), y) \overset{\text{ИП}}{=} \delta(q_{i_{t_2}}, y) = \delta(q_{i_{t_1}}, y) = q_{i_{t_2}}$
    \end{proof}

    Знаем, че $\alpha = xyz \in L$.
    Тогава $\delta^*(s, xyz) = \delta^*(\delta^*(s, xy), z) = \delta^*(\delta^*(\delta^*(s, x), y), z) \in F$.

    Така получаваме, че $\delta^*(s, x) = q_{i_{t_1}}$ и от там $\delta^*(\delta^*(q_{i_{t_1}}, y), z) \in F$.
    От \thref{pl-helper} имаме, че за всяко $ i \in \mathbb{N}$,  $\delta^*(q_{i_{t_1}}, y^i) = q_{i_{t_2}} = q_{i_{t_1}}$.
    Освен това знаем, че $\delta^*(q_{i_{t_2}}, z) \in F$.
    
    Следователно за всяко $i \in \mathbb{N}$, $\delta^*(\delta^*(q_{i_{t_1}}, y^i), z) \in F$.
    От тук можем да заключим, че $\delta^*(\delta^*(\delta^*(s, x), y^i), z) \in F$ за всяко $ i \in \mathbb{N}$.
    и вървейки в обратната посока (``сливането'' на всички $\delta^*$ в едно) получаваме,
    че за всяко $ i \in \mathbb{N}$, $\delta^*(s, xy^iz) \in F$, с което доказахме лемата.
\end{proof}

Тук отново няма да използваме твърдението в тази форма, а неговата контрапозиция:

\begin{corollary}[Лема за покачването (контрапозиция)]\thlabel{nonreg-pl}
    Ако е изпълнено, че:
    \begin{align*}
         & (\forall p \geq 1)                                                             \\
         & (\exists \alpha \in L, \: |\alpha| \geq 1)                                     \\
         & (\forall x, y, z \in \Sigma^*, \: xyz = \alpha, \: |xy| \leq p, \: |y| \geq 1) \\
         & (\exists i \in \mathbb{N}) [xy^iz \notin L],
    \end{align*}
    то тогава $L$ не е регулярен.
\end{corollary}

Ще покажем и по този начин, че $L = \{ a^nb^n \mid n \in \mathbb{N} \}$ не е регулярен.
Нека $p \geq 1$.
Тогава $\alpha = a^pb^p \in L, \: |a^pb^p| = 2p \geq p$.
Нека вземем $x, y, z \in \Sigma^*, \: xyz = \alpha, \: |xy| \leq n, \: |y| \geq 1$.
Знаем със сигурност, че $y = a^t$ за някое $1 \leq t \leq n$.
За да намерим числото $i \in \mathbb{N}$, което ще ни ``изкара'' думата, трябва да видим как изглежда $xy^iz$:
\begin{center}
    $xy^iz = a^pa^{(i - 1)t}b^p$
\end{center}
Тук всяко $i \neq 1$ ще ни свърши работа.
Нека вземем $i = 0$.
Тогава $p + (i - 1)t = p - t < p$, понеже ($t \geq 1$), откъдето $xy^0z = a^{(p - t)}b^p \notin L$.
Така по \nameref{nonreg-pl} излиза, че езикът не е регулярен.

\begin{claim}
    Езикът $L = \{ a^p \mid p \text{ е просто число} \}$ не е регулярен.
\end{claim}

\begin{proof}
    Нека $n \geq 1$ и $p \geq n$ е просто число.

    Тогава $\alpha = a^p \in L, \: |a^p| \geq n$.
    Нека $x, y, z \in \Sigma^*, \: xyz = \alpha, \: |xy| \leq n, \: |y| \geq 1$.
    Тогава $y = a^t$ за някое $1 \leq t \leq n$.
    Нека видим как изглежда $xy^iz$:
    \begin{center}
        $xy^iz = xyy^{i - 1}z = a^pa^{(i - 1)t}$
    \end{center}
    Искаме $p + (i - 1)t$ да е съставно число.
    $t$ е нещо, което се променя, така че по-скоро бихме се надявали $p$ да е множител.
    Искаме да е множител във $(i - 1)t$. Възможно е $t < p$, така че трябва $p \mid i - 1$.
    Ако положим $i = p + 1$ получаваме, че:
    \begin{center}
        $p + (i - 1)t = p + (p + 1 - 1)t = p + pt = p(1 + t)$
    \end{center}
    Тъй като $t \geq 1, \: t + 1 \geq 2$, и от там $p(1 + t) = p + (i - 1)t$ е съставно число.
    Накрая понеже $xy^{p + 1}z = a^{p(1 + t)} \notin L$, по \nameref{nonreg-pl} излиза, че $L$ не е регулярен.
\end{proof}

\begin{problem}
С лемата за покачването да се докаже, че следните езици не са регулярни:
\begin{itemize}
    \item $L_1 = \{ a^nb^n \mid n \in \mathbb{N} \}$ // $y$ ще хване само букви $a$
    \item $L_2 = \{ a^{n^2} \mid n \in \mathbb{N} \}$
    \item $L_3 = \{ a^{2^n} \mid n \in \mathbb{N} \}$
    \item $L_4 = \{ a^{n!} \mid n \in \mathbb{N} \}$
    \item $L_5 = \{ \alpha \alpha \mid \alpha \in \Sigma^* \}$
    \item $L_6 = \{ \alpha \alpha^{rev} \mid \alpha \in \Sigma^* \}$
\end{itemize}
Упътване: решенията на тази задача са просто адаптация на тези от \thref{nonregular-problems-1}, \thref{nonregular-problems-2} и \thref{nonregular-problems-3}
\end{problem}

От всичкото това доказване на нерегулярност използвайки лемата, човек си задава въпроса дали това не може да стане и в обратната посока.
Оказва се, че не може.

\begin{warning}
    Има \textbf{нерегулярни} езици, за който условието от \nameref{pumping-lemma} \textbf{е изпълнено}.
\end{warning}
За пример нека вземем:
\begin{center}
    $L = (\{ c \}^+ \cdot \{ a^nb^n \mid n \in \mathbb{N} \}) \cup (\{ a \}^* \cdot \{ b \}^*)$
\end{center}
Ако допуснем, че $L$ е регулярен, то тогава очевидно:
\begin{center}
    $L \cap (\{ c \} \cdot \{ a \}^* \cdot \{ b \}^*) = \{ c \} \cdot \{ a^nb^n \mid n \in \mathbb{N} \}$
\end{center}
ще бъде регулярен, но той не е (елементарна проверка, която оставяме на читателя).
Сега да проверим условието от \nameref{pumping-lemma}.
Нека $p = 2$.
Нека $\alpha \in L, \: |\alpha| \geq p$.
Полагаме:
\begin{itemize}
    \item $x = \varepsilon$
    \item $y = \alpha[1]$ (първата буква на $\alpha$)
    \item $z = \alpha[2:|\alpha|]$ (останалите букви на $\alpha$)
\end{itemize}
Ако $\alpha \in \{ c \}^+ \cdot \{ a^nb^n \mid n \in \mathbb{N} \}$, то тогава $y = c$.
За $i = 2$, $xy^iz = c \cdot \alpha \in L$.
Ако пък $\alpha \in \{ a \}^* \cdot \{ b \}^*$, то тогава очевидно $i = 2$ пак ще свърши работа.
\section{Автомат на Brzozowski}

Автоматите, които създаваме, не са единствените за конкретния език.
Най-малкото можем да добавим недостижими състояния.
Но това е глупаво, ние по-скоро искаме да направим автомат с възможно най-малко състояния.
Тогава ако решим да направим програма, тя ще заема по-малко памет.

\begin{definition}
    За език $L \subseteq \Sigma^*$ дефинираме $Q_L = \{ \alpha^{-1}(L) \mid \alpha \in \Sigma^* \}$.
\end{definition}

\begin{claim}
    Ако $Q_L$ е крайно, то $L$ е регулярен.
\end{claim}

\begin{proof}
    Нека $Q_L$ е крайно.

    Строим автомат $\mathcal{B}_L = \opair{\Sigma, Q_L, L, \delta, F}$ за $L$, който ще наричаме \textbf{автомат на Brzozowski} за $L$:

    \begin{itemize}
        \item $\delta(M, x) = x^{-1}(M)$ за $M \in Q_L, \: x \in \Sigma$
        \item $F = \{ M \in Q_L \mid \varepsilon \in M \}$
    \end{itemize}

    \begin{claim}
        За всяко $M \in Q_L \: : \: \delta^*(M, \alpha) = \alpha^{-1}(M)$
    \end{claim}

    \begin{proof}
        С индукция по $|\alpha|$.

        \begin{itemize}
            \item $\delta^*(M, \varepsilon) = M = \varepsilon^{-1}(M)$ \checkmark
            \item $\delta^*(M, \beta x) = \delta(\delta^*(M, \beta), x) \stackrel{\text{ИП}}{=} x^{-1}(\beta^{-1}(M)) = \{ \gamma \in \Sigma^* \mid x \gamma \in \beta^{-1}(M) \} = \{ \gamma \in \Sigma^* \mid \beta x \gamma \in M \} = (\beta x)^{-1}(M)$
        \end{itemize}
    \end{proof}

    Имайки това:
    \begin{align*}
        \alpha \in L & \iff \varepsilon \in \alpha^{-1}(L)      \\
                     & \iff \varepsilon \in \delta^*(L, \alpha) \\
                     & \iff \delta^*(L, \alpha) \in F           \\
                     & \iff \alpha \in \mathcal{L(B)}
    \end{align*}
\end{proof}

Вече сме сигурни, че \nameref{nerode-nonregular} е напълно точен, за разлика от \nameref{pumping-lemma}:

\begin{corollary}
    $L$ е регулярен $\iff Q_L$ е крайно
\end{corollary}

Освен че за всеки регулярен език $L$ може да се построи такъв автомат, той се оказва и минимален.

\begin{claim}[Автоматът на Brzozowski е минимален]\thlabel{brzozowski-minimal-automaton}
    Нека $L$ е регулярен с автомат $\mathcal{A} = \opair{\Sigma, Q, s, \delta, F}$.
    Тогава $|Q_L| \leq |Q|$.
\end{claim}

\begin{proof}
    Б.О.О. всички състояния в $\mathcal{A}$ са достижими.

    Нека $Q = \{ 1, \dots, n \}$.
    Нека $\alpha_i \in \Sigma^*$ е такава дума, че $\delta^*(s, \alpha) = q_i$.
    Дефинираме функцията $f : Q \rightarrow Q_L$:
    \begin{center}
        $f(q_i) = \alpha^{-1}(L)$
    \end{center}
    Трябва само да покажем, че $f$ е сюрективна.

    Нека $M \in Q_L$.
    Тогава има $\alpha \in \Sigma^* \: : \: M = \alpha^{-1}(L)$.
    Знаем, че $\delta^*(s, \alpha) = q_i$ за някое $i \in \{ 1, \dots, n \}$.
    \begin{align*}
        \beta \in M \iff \beta \in \alpha^{-1}(L) & \iff \alpha \beta \in L                                                                                               \\
                                                  & \iff \delta^*(s, \alpha \beta) \in F \iff \delta^*(\delta^*(s, \alpha), \beta) \in F  \iff \delta^*(q_i, \beta) \in F \\
                                                  & \iff \delta^*(\delta^*(s, \alpha_i), \beta) \in F \iff \delta^*(s, \alpha_i \beta) \in F \iff \alpha_i \beta \in L    \\
                                                  & \iff \beta \in \alpha_i^{-1}(L) \iff \beta \in f(q_i)
    \end{align*}
    Така $f(q_i) = M$.
    Накрая понеже $f$ е сюрективна, $\operatorname{Dom}(f) = Q, \: \operatorname{Rng}(f) = Q_L, \: |Q_L| \leq |Q|$.
\end{proof}

\begin{problem}
Да се построи минимален автомат за $\regexlang{a^*b^*}$
\end{problem}

За такива задачи е хубаво да се имат на предвид следните свойства на регулярните изрази:
\begin{itemize}
    \item $\regexlang{r^+} = \regexlang{r \cdot r^*}$
    \item $\regexlang{r^*} = \regexlang{r^+ + \varepsilon}$
    \item $\regexlang{r_1 + r_2} = \regexlang{r_2 + r_1}$
    \item $\regexlang{r(r_1 + r_2)} = \regexlang{r \cdot r_1 + r \cdot r_2}$
\end{itemize}

Започваме да строим автомат на Brzozowski за $\regexlang{a^*b^*}$:
\begin{itemize}
    \item начално състояние $L_0 = \regexlang{a^*b^*} = \regexlang{a^+b^* + b^*} = \regexlang{a \cdot a^*b^* + b^*} = \regexlang{a \cdot a^*b^* + b^+ + \varepsilon} = \regexlang{a \cdot a^*b^* + b \cdot b^* + \varepsilon}$
    \item функцията на преходите е следната:
          \begin{itemize}
              \item $\delta(L_0, a) = a^{-1}(\regexlang{a \cdot a^*b^* + b \cdot b^* + \varepsilon}) = \regexlang{a^*b^*} = L_0$
              \item $\delta(L_0, b) = b^{-1}(\regexlang{a \cdot a^*b^* + b \cdot b^* + \varepsilon}) = \regexlang{b^*} = \regexlang{b \cdot b^* + \varepsilon}= L_1 \neq L_0$, понеже $a \in \regexlang{a^*b^*}, \: a \notin \regexlang{b^*}$
              \item $\delta(L_1, a) = a^{-1}(\regexlang{b \cdot b^* + \varepsilon}) = \varnothing = L_2 \neq L_0, L_1$
              \item $\delta(L_1, b) = b^{-1}(\regexlang{b \cdot b^* + \varepsilon}) = \regexlang{b^*} = L_1$
              \item $\delta(L_2, a) = a^{-1}(\varnothing) = \varnothing = b^{-1}(\varnothing) = \delta(L_2, a)$
          \end{itemize}
    \item $\varepsilon \in L_0, L_1$, следователно те стават финални
\end{itemize}

Ето как ще изглежда автомата на картинка:
\begin{figure}[h]
    \centering
    \begin{tikzpicture}[node distance=2cm, thick]
        \node[accepting, initial, state, initial text=] (1) {$L_0$};
        \node[accepting, state, initial text=, right of=1] (2) {$L_1$};
        \node[state, initial text=, right of=2] (3) {$L_2$};
        \path[->] (1) edge [loop above] node[above] {$a$} (1);
        \path[->] (1) edge [] node[above] {$b$} (2);
        \path[->] (2) edge [loop above] node[above] {$b$} (2);
        \path[->] (2) edge [] node[above] {$a$} (3);
        \path[->] (3) edge [loop above] node[above] {$a, b$} (3);
    \end{tikzpicture}
    \caption*{минимален автомат за $\regexlang{a^*b^*}$}
\end{figure}

Как бихме процедирали ако трябва да се построи минимален автомат за $\Sigma^* \setminus \regexlang{a^*b^*}$?

\begin{warning}
    За всеки регулярен език $L$, ако има автомат за $L$ с $n$ на брой състояния, то тогава има автомат за $\overline{L}$ с $n$ на брой състояния.
\end{warning}

Това го знаем още от \thref{complement-regular}.
В конструкцията там ние използваме същото множество от състояния, същото начално състояния и същите преходи.
Единственото, което променяме, е кои са финалните състояния.
Така можем да заключим, че за да получим минималният автомат за $\Sigma^* \setminus \regexlang{a^*b^*}$, трябва просто да приложим конструкцията от \thref{complement-regular} върху минималният автомат за $\regexlang{a^*b^*}$.
Това е много по-кратко от преминаването към регулярен израз за $\Sigma^* \setminus \regexlang{a^*b^*}$ и прилагането на алгоритъма на Brzozowski.
\section{Задачи за упражнение}

\begin{definition}\thlabel{homomorphism-def}
    Функция $h : \Sigma_1^* \rightarrow \Sigma_2^*$ ще наричаме \textbf{хомоморфизъм}, ако:
    \begin{center}
        $h(\alpha \cdot \beta) = h(\alpha) \cdot h(\beta)$ за всички $\alpha, \beta \in \Sigma_1^*$
    \end{center}
\end{definition}

\begin{problem}
Да се докаже, че за произволен хомоморфизъм $h$ е вярно, че $h(\varepsilon) = \varepsilon$
\end{problem}

\begin{problem}\thlabel{homomorphism-regular}
Да се докаже, че за произволен хомоморфизъм $h$ и регулярен език $L$, езикът $h[L]$ е регулярен

Упътване: да се направи индукция по строенето на регулярните езици
\end{problem}

\begin{problem}\thlabel{homomorphism-inverse-regular}
Да се докаже, че за произволен хомоморфизъм $h$ и регулярен език $L$, езикът $h^{-1}[L]$ е регулярен

Упътване: да се направи автомат, който докато чете $\alpha$, прави преходи с $h(\alpha)$ в автомат за $L$
\end{problem}

\begin{problem}
Използвайки нерегулярността на $\{ a^nb^n \mid n \in \mathbb{N} \}$ заедно със \thref{homomorphism-regular} или \thref{homomorphism-inverse-regular} да се докаже, че езикът $L = \{ a^nb^{2n} \mid n \in \mathbb{N} \}$ е нерегулярен.

Упътване: да се представи $L$ като образ или праобраз на хомоморфизъм
\end{problem}

\begin{problem}
Нека $L$ е регулярен език. Да се докаже, че езикът $L' = \{ \alpha \in \Sigma^* \mid \alpha \alpha \in L \}$ е регулярен.

Упътване: докато се чете $\alpha$ да се види къде отиват всички състояния от автомата за $L$
\end{problem}

\begin{problem}
Нека $L$ е регулярен език. Да се докаже, че езикът $L' = \{ \alpha \# c^n \mid n \in \mathbb{N} \: \& \: \alpha^n \in L \}$ е регулярен.

Упътване: леко усложнение на конструкцията от предната задача
\end{problem}

\begin{problem}
Нека $L$ е регулярен език. Да се докаже, че езикът $L' = \{ \alpha \in \Sigma^* \mid (\exists n \in \mathbb{N}) \: (\alpha^n \in L) \}$ е регулярен.

Упътване: да се използват предната задача и хомоморфизми
\end{problem}

\begin{definition}\thlabel{reg-homomorphism-def}
    Функция $h : \Sigma_1^* \rightarrow \mathcal{P}(\Sigma_2^*)$ ще наричаме \textbf{регулярен хомоморфизъм}, ако:
    \begin{itemize}
        \item $h(\alpha)$ е регулярен за всяко $\alpha \in \Sigma_1^*$
        \item $h(\alpha \cdot \beta) = h(\alpha) \cdot h(\beta)$ за всички $\alpha, \beta \in \Sigma_1^*$
    \end{itemize}
\end{definition}

\begin{problem}
Да се докаже, че за произволен регулярен хомоморфизъм $h$ е вярно, че $h(\varepsilon) = \{ \varepsilon \}$
\end{problem}

\begin{problem}\thlabel{reg-homomorphism}
Да се докаже, че за произволен регулярен хомоморфизъм $h$ и регулярен език $L$, $\bigcup h[L]$ е регулярен

Упътване: да се направи индукция по строенето на регулярните езици
\end{problem}
\chapter{Граматики}

\section{Неограничени граматики}

\section{Регулярни граматики и еквивалентност със автоматните езици}

\section{Безконтекстни граматики}

\section{Дървета на извод}

\section{Небезконтекстни езици}

\section{Нормална форма на Чомски}

\section{Задачи за упражнение}

\end{document}