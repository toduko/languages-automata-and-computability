\documentclass[openany]{book}

\usepackage[utf8]{inputenc}
\usepackage[T2A]{fontenc}
\usepackage[english, bulgarian]{babel}
\usepackage{amssymb}
\usepackage{hyperref, fancyhdr, lastpage, fancyvrb, tcolorbox, titlesec}
\usepackage{array, tabularx, colortbl}
\usepackage{tikz}
\usepackage{venndiagram}
\usepackage{amsthm, bm}
\usepackage{relsize}
\usepackage{amsmath,physics}
\usepackage{mathtools}
\usepackage{subcaption}
\usepackage{theoremref}
\usepackage{circuitikz}
\usepackage[a4paper, left=0.50in, right=0.50in, top=1.0in, bottom=1.0in]{geometry}
\usepackage{minted}
\usepackage{stmaryrd}
\usepackage[Rejne]{fncychap}
\usepackage{forest}
\usepackage{cancel}
\usetikzlibrary{automata, arrows, positioning, shapes}
\useforestlibrary{linguistics}

\newcommand{\db}[1]{\llbracket #1 \rrbracket}
\newcommand{\regexlang}[1]{\mathcal{L}\db{#1}}
\newcommand{\tri}[1]{\stackrel{#1}{\vartriangleleft}}
\newcommand{\rt}[1]{\operatorname{root}(#1)}
\newcommand{\h}[1]{\operatorname{height}(#1)}
\newcommand{\chld}[1]{\operatorname{children}(#1)}
\newcommand{\lv}[1]{\operatorname{leaves}(#1)}
\newcommand{\w}[1]{\operatorname{word}(#1)}
\newcommand{\wraphspace}[2]{\hspace*{#2}#1\hspace*{#2}}

\ExplSyntaxOn
\NewDocumentCommand{\opair}{m}
 {
  \langle\mspace{2mu}
  \clist_set:Nn \l_tmpa_clist { #1 }
  \clist_use:Nn \l_tmpa_clist {,\mspace{3mu plus 1mu minus 1mu}\allowbreak}
  \mspace{2mu}\rangle
}
\ExplSyntaxOff

\hypersetup{
    colorlinks=true,
    linktoc=all,
    linkcolor=blue
}

\theoremstyle{definition}
\newtheorem{definition}{Дефиниция}[section]
\newtheorem*{warning}{\textcolor{red}{Внимание}}
\theoremstyle{plain}
\newtheorem{theorem}[definition]{Теорема}
\newtheorem{claim}[definition]{Твърдение}
\newtheorem{axiom}[definition]{Аксиома}
\newtheorem{lemma}[definition]{Лема}
\newtheorem{problem}[definition]{Задача}
\newtheorem{corollary}[definition]{Следствие}
\theoremstyle{remark}
\newtheorem*{remark}{Забележка}
\theoremstyle{definition}

\pagestyle{fancy}

\lhead{\leftmark}
\rhead{}

\setlength\parindent{0pt}

\begin{document}

\begin{titlepage}
  \centering
  {\huge\bfseries ЕЗИЦИ, АВТОМАТИ И ИЗЧИСЛИМОСТ\par}
  \vspace{1cm}
  {\Large \textsc{Записки за упражненията}\par}
  \vspace{4cm}
  {\Large\itshape Тодор Дуков\par}
  \vfill
  {\large \today\par}
\end{titlepage}

\tableofcontents

\chapter{Въведение}

Основното нещо, което се прави в този курс, е да се класифицират езици.
За да можем да класифицираме един език, първо трябва знаем какво точно представлява един език.

\section{Основни понятия}

\begin{definition}
    \textbf{Азбука} ще наричаме всяко крайно множество.
    Елементите на азбуката ще наричаме \textbf{букви}.
\end{definition}

Обикновено ще си бележим азбуквата със $\Sigma$.
Също така, ако никъде не е споменато друго, $\Sigma = \{a, b \}$.
Тук буквите ще бъдат $a$ и $b$.

\begin{definition}
    \textbf{Дума} над азбуката $\Sigma$ ще наричаме всяка крайна редица от букви от $\Sigma$.
    Дължината на дума $\alpha$ над $\Sigma$ ще бележим с $|\alpha|$.
\end{definition}

При азбуката $\Sigma = \{0, 1\}$, примерна дума ще бъде $\alpha = 000101$.
Ясно е, че $|\alpha| = 6$.

\begin{definition}
    \textbf{Празната дума} ще наричаме единствената дума с дължина 0.
    Бележим я с $\varepsilon$.
\end{definition}

\begin{remark}
    Важно е да се отбележи, че празното множество и празната дума са различни неща.
    Възможно е да се вземе такава дефиниция за редица, в която те да съвпадат, но това не е съществено.
    За нас думите ще бъдат едни неща, а множествата други.
    \textbf{TLDR:} $\varepsilon \neq \varnothing$
\end{remark}

\begin{definition}
    Със $\Sigma^*$ ще бележим множеството от всички думи над $\Sigma$.
    $L$ ще наричаме \textbf{език} над $\Sigma$, ако $L \subseteq \Sigma$.
\end{definition}

Тук нащият универсум ще бъде $\Sigma^*$.
Така че за $L \subseteq \Sigma^*$, под $\overline{L}$ ще имаме предвид $\Sigma^* \setminus L$.

\section{Операции върху думи и езици}

\begin{definition}
    Ще дефинираме \textbf{конкатенацията} (слепването) на две думи $\alpha$ и $\beta$
    и ще го бележим с $\alpha \cdot \beta$
    \begin{itemize}
        \item $\alpha \cdot \varepsilon = \alpha$ (Базов случай)
        \item $\alpha \cdot (\beta x) = (\alpha \cdot \beta)x$ (Свеждане до по-малка ``задача'')
    \end{itemize}
\end{definition}

На пръв поглед такава дефиниция изглежда безсмислена,
но това далеч не е така.
После тя ще се използва постоянно в доказателства. \\

Нека разгледаме един пример за конкатенация:
\begin{align*}
    aaa \cdot bbb & = (aaa \cdot bb)b = ((aaa \cdot b)b)b = (((aaa \cdot \varepsilon)b)b)b = \\
                  & = (((aaa)b)b)b = ((aaab)b)b = (aaabb)b = aaabbb
\end{align*}

Вече можем да дефинираме $\alpha^n$ ($n$ на брой пъти да конкатенираме думата $\alpha$) индуктивно:
\begin{itemize}
    \item $\alpha^0 = \varepsilon$
    \item $\alpha^{n + 1} = \alpha^n \cdot \alpha$
\end{itemize}

За пример можем да вземем $(ab)^3$.
\begin{align*}
    (ab)^3 & = (ab)^2 \cdot ab = ((ab^1) \cdot ab) \cdot ab = (((ab^0) \cdot ab) \cdot ab) \cdot ab = \\
           & = ((\varepsilon \cdot ab) \cdot ab) \cdot ab = ab \cdot ab \cdot ab
\end{align*}

\begin{remark}
    Тук използваме наготово, че $\varepsilon \cdot \alpha = \alpha$. \\
\end{remark}

Имайки конкатенация на думи, можем да дефинираме и конкатенацията на езици.
Най-естествено е да направим следното:

\begin{definition}
    Нека $L_1, L_2 \subseteq \Sigma^*$.
    Тогава \textbf{конкатенацията} на езиците $L_1$ и $L_2$ ще наричаме множеството:
    \begin{center}
        $L_1 \cdot L_2 = \{ \alpha \cdot \beta \: | \: \alpha \in L_1 \: \& \: \beta \in L_2 \}$ \\
    \end{center}
\end{definition}

Вече можем да дефинираме $L^n$ ($n$ на брой пъти да конкатенираме езика $L$) индуктивно:
\begin{itemize}
    \item $L^0 = \{ \varepsilon \}$
    \item $L^{n + 1} = L^n \cdot L$
\end{itemize}

\begin{definition}[Звезда на Клини]
    Нека $L \subseteq \Sigma^*$. Тогава:
    \begin{itemize}
        \item $L^* = \bigcup\limits_{n \in \mathbb{N}} L^n = L^0 \cup L^1 \cup L^2 \cup \dots $
        \item $L^+ = \bigcup\limits_{\substack{n \in \mathbb{N} \\ n \neq 0}} L^n = L^1 \cup L^2 \cup L^3 \cup \dots $
    \end{itemize}
\end{definition}

Ясно е, че в тази дефиниция $\Sigma^*$ е същото нещо като в другата.

\pagebreak

\section{Допълнителни дефиниции}

Тук ще сложим няколко дефиниции, които са стандартни, и ще има задачи, свързани с тях.

\begin{definition}
    Ще дефинираме \textbf{обръщането} на дума и на език.
    Обръщането на дума $\alpha$, което бележим с $\alpha^{rev}$, става индуктивно:
    \begin{itemize}
        \item $\varepsilon^{rev} = \varepsilon$
        \item $(\alpha x)^{rev} = x(\alpha^{rev})$
    \end{itemize}
    Обръщането на език $L$ бележим с $L^{rev} = \{\alpha^{rev} \: | \: \alpha \in L \}$.
\end{definition}

\begin{definition}
    Ще дефинираме кога една дума $\alpha$ е префикс, инфикс или суфикс на друга дума $\beta$:
    \begin{itemize}
        \item $\alpha \preceq_{pref} \beta$, ако $(\exists \gamma \in \Sigma^*)(\alpha \cdot \gamma = \beta)$
        \item $\alpha \preceq_{suff} \beta$, ако $(\exists \gamma \in \Sigma^*)(\gamma \cdot \alpha = \beta)$
        \item $\alpha \preceq_{inf} \beta$, ако $(\exists \gamma_1 \in \Sigma^*)(\exists \gamma_2 \in \Sigma^*)(\gamma_1 \cdot \alpha \cdot \gamma_2 = \beta)$
    \end{itemize}
    Нека $L \subseteq \Sigma^*$. Тогава:
    \begin{itemize}
        \item $\operatorname{Pref}(L) = \{ \alpha \in \Sigma^* \: | \: (\exists \beta \in L)(\alpha \preceq_{pref} \beta) \}$
        \item $\operatorname{Suff}(L) = \{ \alpha \in \Sigma^* \: | \: (\exists \beta \in L)(\alpha \preceq_{suff} \beta) \}$
        \item $\operatorname{Infix}(L) = \{ \alpha \in \Sigma^* \: | \: (\exists \beta \in L)(\alpha \preceq_{inf} \beta) \}$
    \end{itemize}
\end{definition}

\pagebreak

\section{Задачи за упражнение}

\begin{problem}[асоциативност]
Да се докаже, че $\alpha \cdot (\beta \cdot \gamma) = (\alpha \cdot \beta) \cdot \gamma$

Упътване: да се направи индукция по $\gamma$
\end{problem}

\begin{problem}[неутрален елемент]
Да се докаже, че $\varepsilon \cdot \alpha = \alpha$

Упътване: да се направи индукция по $\alpha$
\end{problem}

\begin{problem}
Да се докаже, че $\alpha^n \cdot \alpha^m = \alpha^{n + m}$

Упътване: да се направи индукция по $m$
\end{problem}


\begin{problem}
Да се докаже, че $(\alpha^n)^m = \alpha^{nm}$

Упътване: да се направи индукция по $m$
\end{problem}

\begin{problem}
Да се докаже, че $L^n \cdot L^m = L^{n + m}$

Упътване: да се направи индукция по $m$
\end{problem}


\begin{problem}
Да се докаже, че $(L^n)^m = L^{nm}$

Упътване: да се направи индукция по $m$
\end{problem}

\begin{problem}[дистрибутивност]
Да се докаже, че:
\begin{itemize}
    \item $(L_1 \cup L_2) \cdot L_3 = (L_1 \cdot L_3) \cup (L_2 \cdot L_3)$
    \item $(L_1 \cap L_2) \cdot L_3 = (L_1 \cdot L_3) \cap (L_2 \cdot L_3)$
\end{itemize}
\end{problem}

\begin{problem}
Да се докаже, че $\Sigma^+ = \Sigma \cdot \Sigma^*$

Упътване: да се използват предните резултати
\end{problem}

\begin{problem}
Да се докаже, че $(\alpha \cdot \beta)^{rev} = \alpha^{rev} \cdot \beta^{rev}$

Упътване: да се направи индукция по $\beta$
\end{problem}

\begin{problem}
Да се докаже, че:
\begin{itemize}
    \item $(\alpha^{rev})^{rev} = \alpha$
    \item $(L^{rev})^{rev} = L$
\end{itemize}

Упътване: за първото да се направи индукция по $\alpha$
\end{problem}
\begin{problem}
Да се докаже, че:
\begin{itemize}
    \item $\operatorname{Pref}(L_1 \cdot L_2) = \operatorname{Pref}(L_1) \cup (L_1 \cdot \operatorname{Pref}(L_2))$
    \item $\operatorname{Suff}(L_1 \cdot L_2) = \operatorname{Suff}(L_2) \cup (\operatorname{Suff}(L_1) \cdot L_2)$
    \item $\operatorname{Infix}(L_1 \cdot L_2) = \operatorname{Infix}(L_1) \cup \operatorname{Infix}(L_2) \cup (\operatorname{Suff}(L_1) \cdot \operatorname{Pref}(L_2))$
    \item $\operatorname{Infix}(L) = \operatorname{Pref}(\operatorname{Suff}(L)) = \operatorname{Suff}(\operatorname{Pref}(L))$
\end{itemize}

Упътване: да се разсъждава на ниво думи (конкатенация на езици се дефинира с конкатенация на думи)
\end{problem}
\chapter{Автомати и регулярни изрази}

Тук ще разгледаме първата ``машина'', с която ще класифицираме езиците.

\section{Детерминирани автомати и автоматни езици}

\begin{definition}
    \textbf{Детерминиран краен автомат} (накратко ДКА) ще наричаме всяко $\mathcal{A} = \opair{\Sigma, Q, s, \delta, F}$, където:
    \begin{itemize}
        \item $\Sigma$ е крайна азбука
        \item $Q$ е крайно множество от състояния
        \item $s \in Q$ (ще го наричаме начално/стартово състояние)
        \item $\delta : Q \cross \Sigma \rightarrow Q$ (ще я наричаме функция на преходите)
        \item $F \subseteq Q$ (ще ги наричаме финални състояния)
    \end{itemize}
\end{definition}

В момента със това, което имаме,
ако искаме да покажем къде ще стигнем с думата $aaa$, започвайки от $s$,
ще трябва да го запишем със $\delta(\delta(\delta(s, a), a), a)$, което е тромаво.

За това ще си въведем начин, по който да видим крайният резултат от прочитането на цяла дума.

\begin{definition}
    Дефинираме $\delta^* : Q \cross \Sigma^* \rightarrow Q$ индуктивно:
    \begin{itemize}
        \item $\delta^*(p, \varepsilon) = p$ за всяко $p \in Q$
        \item $\delta^*(p, \beta x) = \delta(\delta^*(p, \beta), x)$ за всяко $p \in Q, \beta \in \Sigma^*, x \in \Sigma$
    \end{itemize}
\end{definition}

Сега можем да забележим, че:
\begin{align*}
    \delta^*(s, aaa) & = \delta(\delta^*(s, aa), a) = \delta(\delta(\delta^*(s, a), a), a) =                           \\
                     & = \delta(\delta(\delta(\delta^*(s, \varepsilon), a), a), a) =\delta(\delta(\delta(s, a), a), a)
\end{align*}
Функцията наистина прави това, което искаме да прави.

\begin{claim}
    $\delta^*(p, \alpha \beta) = \delta^*(\delta^*(p, \alpha), \beta)$
\end{claim}

\begin{proof}
    С индукция по $|\beta|$.
    \begin{itemize}
        \item $\delta^*(p, \alpha \varepsilon) = \delta^*(p, \alpha) = \delta^*(\delta^*(p, \alpha), \varepsilon)$
        \item $\delta^*(p, \alpha \beta x) = \delta(\delta^*(p, \alpha \beta), x) \stackrel{\text{ИП}}{=} \delta(\delta^*(\delta^*(p, \alpha), \beta), x) = \delta^*(\delta^*(p, \alpha), \beta x)$
    \end{itemize}
\end{proof}

\begin{definition}
    Нека $\mathcal{A} = \opair{\Sigma, Q, s, \delta, F}$ е ДКА.
    Тогава езикът на автомата $\mathcal{A}$ е множеството
    $\mathcal{L}(\mathcal{A}) = \{ \alpha \in \Sigma^* \: | \: \delta^*(s, \alpha) \in F \}$.
    Един език $L \subseteq \Sigma^*$, наричаме \textbf{автоматен}, ако има ДКА $\mathcal{A}$ с $\mathcal{L}(\mathcal{A}) = L$.
\end{definition}

\section{Представяне на автомат}

Ще видим начините, по които можем да представяме един автомат.
За пример ще вземем автомата $\mathcal{A} = \opair{\Sigma, Q, s, \delta, F}$, където:
\begin{itemize}
    \item $\Sigma = \{ a, b \}$
    \item $Q = \{ q_0, q_1, q_2 \}$
    \item $s = q_0$
    \item $\delta(q_0, a) = q_2$
    \item $\delta(p, x) = q_1$ за $p \in Q, \: x \in \Sigma, \: \opair{p, x} \neq \opair{q_0, a}$
    \item $F = \{ q_2 \}$
\end{itemize}

Това е първият начин за представяне.
При него директно в явен вид се казват кои са всички съставни елементи на $\mathcal{A}$.
Това ще бъде използване сравнително често, когато правим от един автомат друг.
В случаите, в които не знаем как изглежда първоначалният автомат, следващите методи тогава няма да свършат работа. \\

Вторият начин за представяне е с таблица на функцията на преходите:
\begin{center}
    \begin{tabular}{||r | r | r||}
        \hline
        \cellcolor{lightgray} & $a$   & $b$   \\
        \hline
        $\rightarrow q_0$     & $q_2$ & $q_1$ \\
        \hline
        $q_1$                 & $q_1$ & $q_1$ \\
        \hline
        $\checkmark q_2$      & $q_1$ & $q_1$ \\
        \hline
    \end{tabular}
\end{center}
Тук с $\rightarrow$ отбелязваме началното състояние,
а с $\checkmark$ отбелязваме финалните състояния.
Във втората и третата колона казваме къде ще се озовем,
ако се намираме в съответното състояние и прочетем съответната буква.
Това е по-прегледно представяне от първото, но може да стане обемно. \\

Третият начин за представяне е с картинка:

\begin{center}
    \begin{tikzpicture}[shorten >=1pt,node distance=2cm,>=stealth',thick]
        \node[initial, state, initial text=] (1) {$q_0$};
        \node[state] (2) [above right of=1] {$q_1$};
        \node[state, accepting] (3) [below right of=1] {$q_2$};
        \draw[->] (1) -- node[above] {$b$} (2);
        \draw[->] (1) -- node[above] {$a$} (3);
        \path (2) edge [loop above] node[above] {$a, b$} (2);
        \draw[->] (3) -- node[right] {$a, b$} (2);
    \end{tikzpicture}
\end{center}

Началното състояние е отбелязано със стрелка, а финалните са оградени два пъти.
Представянето чрез картинка изглежда по-прегледно от другите две, но то може и да стане огромно.

Ясно е, че $\mathcal{L}(\mathcal{A}) = \{ a \}$.
Ако искаме да покажем това формално, трябва да направим следното:
\begin{itemize}
    \item Очевидно $a \in \mathcal{L}(\mathcal{A}), \: b \notin \mathcal{L}(\mathcal{A}), \: \varepsilon \notin \mathcal{L}(\mathcal{A})$
    \item Показваме, че $\delta^*(q_1, \alpha) = q_1$ за всяко $\alpha \in \Sigma^*$ с индукция по $|\alpha|$
    \item Ако $\alpha \notin \{ a, b, \varepsilon \}$, то $\alpha = x y \beta$ за $x, y \in \Sigma, \: \beta \in \Sigma^*$
    \item Тогава лесно се вижда, че $\delta^*(q_0, \alpha) = q_1 \notin F$.
          Или директно отиваме в $q_1$ (ако $x = b$) и оставаме там, или първо отиваме в $q_2$ (ако $x = a$), и после със следващата буква отиваме в $q_1$ и оставаме там.
\end{itemize}

Това обаче за такива прости автомати не е нужно.
Тези неща в такива случаи ще ги приемаме за очевидни.
\section{Прости примери за автоматни езици}

Нека видим много прости примери за автоматни езици,
за да добием малко интуиция какви езици могат да бъдат автоматни,
и да поработим малко със самите машини. Най-простите автоматни езици са $\varnothing$ и $\Sigma^*$:

\begin{center}
    \begin{tikzpicture}[node distance=2cm, thick]
        \node[text width=3.0cm] at (0.35, -1) {автомат за $\varnothing$};
        \node[initial, state, initial text=] (1) {$q$};
        \path[->] (1) edge [loop above] node[above] {$a, b$}(1);
    \end{tikzpicture}
    \hspace{2cm}
    \begin{tikzpicture}[node distance=2cm, thick]
        \node[text width=3.0cm] at (0.35, -1) {автомат за $\Sigma^*$};
        \node[initial, accepting, state, initial text=] (1) {$q$};
        \path[->] (1) edge [loop above] node[above] {$a, b$}(1);
    \end{tikzpicture}
\end{center}

Сега малко ще усложним нещата.
Ще направим автомат, който да разпознае езикът от една конкретна дума.
Конструкцията за конкретната дума много лесно се обобщава за всички.
За пример нека направим автомат за $L = \{ abab \}$:

\begin{center}
    \begin{tikzpicture}[shorten >=1pt,node distance=2.5cm,>=stealth',thick]
        \node[initial, state, initial text=] (1) {$\varepsilon$};
        \node[state] [right of=1] (2) {$a$};
        \node[state] [right of=2] (3) {$ab$};
        \node[state] [right of=3] (4) {$aba$};
        \node[state, accepting] [right of=4] (5) {$abab$};
        \node[state] [below of=3] (6) {$\cross$};
        \draw[->] (1) -- node[above] {$a$} (2);
        \draw[->] (2) -- node[above] {$b$} (3);
        \draw[->] (3) -- node[above] {$a$} (4);
        \draw[->] (4) -- node[above] {$b$} (5);
        \path[->] (1) edge [bend right] node[below] {$b$} (6);
        \path[->] (2) edge [bend right] node[below] {$a$} (6);
        \path[->] (3) edge  node[right] {$b$} (6);
        \path[->] (4) edge [bend left] node[below] {$a$} (6);
        \path[->] (5) edge [bend left] node[below] {$a, b$} (6);
        \path[->] (6) edge [loop below] node[below] {$a, b$} (6);
    \end{tikzpicture}
\end{center}

Тази техника може да се приложи за строене на автомат за език от произволно дълга дума.
Ето как ще обобщим конструкцията за езика $L = \{ \alpha \}$:
\begin{itemize}
    \item $\mathcal{A} = \opair{\Sigma, Q, s, \delta, F}$
    \item $Q = \operatorname{Pref}(L) \cup \{ \cross \}$ (начален отрязък от пътя, който съставя $\alpha$, $\cross \notin \Sigma$ е отхвърлящо състояние боклук)
    \item $s = \varepsilon$
    \item За $\beta \preceq_{pref} \alpha \: ($тогава $\beta \in Q), \: x \in \Sigma$:
          ако $\beta x \preceq_{pref} \alpha$, то $\delta(\beta, x) = \beta x$,
          иначе $\delta(\beta, x) = \cross$
    \item $\delta(\cross, x) = \cross$ за $x \in \Sigma$
    \item $F = \{ \alpha \}$
\end{itemize}

Идеята зад конструкцията е следната:
Състоянията кодират пътя, който сме изминали, ако не сме се отклонили вече от ``строежа'' на $\alpha$.
В първият момент на отклонение отиваме във нефинално състояние ``боклук'', от което не може да излезнем.

Можем да разширим идеята, така че да работи за няколко думи като кодираме повече видове пътища.
Ето как ще обобщим конструкцията (чиято коректност приемаме за очевидна) за $L = \{ \alpha_1, \: \dots, \: \alpha_n \}$:
\begin{itemize}
    \item $\mathcal{A} = \opair{\Sigma, Q, s, \delta, F}$
    \item $Q = \{ \alpha \in \Sigma^* \: | \: (\exists \beta \in L) (|\alpha| \leq |\beta|) \} \cup \{ \cross \}$ (не гледаме пътища по-дълги от най-дългата дума в $L$, $\cross \notin \Sigma$ - състояние боклук)
    \item $s = \varepsilon$
    \item За $\alpha \in Q, \: x \in \Sigma$:
          ако $\alpha x \in Q$, то $\delta(\alpha, x) = \alpha x$,
          иначе $\delta(\alpha, x) = \cross$
    \item $\delta(\cross, x) = \cross$ за $x \in \Sigma$
    \item $F = L$
\end{itemize}

\begin{figure*}
    \centering
    \begin{tikzpicture}[shorten >=1pt,node distance=2.5cm,>=stealth',thick]
        \node[initial, state, initial text=] (1) {$\varepsilon$};
        \node[state] [above right of=1] (2) {$a$};
        \node[state] [below right of=1] (3) {$b$};
        \node[state, accepting] [above right of=2] (4) {$aa$};
        \node[state, accepting] [right of=2] (5) {$ab$};
        \node[state, accepting] [right of=3] (6) {$ba$};
        \node[state] [below right of=3] (7) {$bb$};
        \node[state] [below right of=5] (8) {$\cross$};
        \draw[->] (1) -- node[above] {$a$} (2);
        \draw[->] (1) -- node[above] {$b$} (3);
        \draw[->] (2) -- node[above] {$a$} (4);
        \draw[->] (2) -- node[above] {$b$} (5);
        \draw[->] (3) -- node[above] {$a$} (6);
        \draw[->] (3) -- node[above] {$b$} (7);
        \path[->] (4) edge [bend left] node[right] {$a, b$} (8);
        \path[->] (5) edge node[left] {$a, b$} (8);
        \path[->] (6) edge node[left] {$a, b$} (8);
        \path[->] (7) edge [bend right] node[right] {$a, b$} (8);
        \path[->] (8) edge [loop right] node[right] {$a, b$} (8);
    \end{tikzpicture}
    \caption*{пример за прилагане на конструкцията за $L = \{ ab, ba, aa \}$}
\end{figure*}

Нека сега направим автомат за $L = \{ \alpha \in \Sigma^* \: | \: |\alpha| \: \text{е четно} \}$.

За една дума $\alpha \in \Sigma^*$ знаем, че тя или има четна дължина или има нечетна дължина.
Дали не можем да кодираме по някакъв начин четността на прочетената дума в състояние?
Отговорът е, че можем. Автоматът е следния:

\begin{center}
    \begin{tikzpicture}[shorten >=1pt,node distance=2.5cm,>=stealth',thick]
        \node[text width=0.5cm] at (-1, 1) {$\mathcal{A}$:};
        \node[accepting, initial, state, initial text=] (1) {$0$};
        \node[state] [right of=1] (2) {$1$};
        \path[->] (1) edge [bend left] node[above] {$a, b$} (2);
        \path[->] (2) edge [bend left] node[below] {$a, b$} (1);
    \end{tikzpicture}
\end{center}

Знаем, че $|\varepsilon|$ е четно.
За всякo $\alpha \in \Sigma^*, \: x \in \Sigma$ знаем,
че $|\alpha x|$ има различна четност от $|\alpha|$.
Така ние започваме с думата $\varepsilon$ и 0 като четност на думата и за всяка буква сменяме четността.

Нека сега помислим как да докажем, че $\mathcal{L(A)} = L$.
За това ще трябва да покажем, че започвайки от $0$ и четейки $\alpha$ ние наистина получаваме четността на $|\alpha|$ като състояние.

\begin{claim}
    За всяко $\alpha \in \Sigma^*$ : $\delta^*(0, \alpha) = |\alpha| \: (mod \: 2)$
\end{claim}

\begin{proof}
    С индукция по $|\alpha|$.
    \begin{itemize}
        \item База: $|\alpha| = 0$, тогава $\alpha = \varepsilon$.
              Наистина $\delta^*(0, \varepsilon) = 0 \: (mod \: 2)$ \checkmark
        \item ИС: $|\alpha| = n + 1$, тогава $\alpha = \beta x$ за $\beta \in \Sigma^*, |\beta| = n, \: x \in \Sigma$.
              Тогава:
              \begin{center}
                  $\delta^*(0, \beta x) = \delta(\delta^*(0, \beta), x) \stackrel{\text{ИП}}{=} \delta(|\beta| \: (mod \: 2), x) = |\beta| + 1 \: (mod \: 2) = |\beta x| \: (mod \: 2) = |\alpha| \: (mod \: 2)$
              \end{center}
    \end{itemize}
\end{proof}

Състоянията наистина кодират информацията, която искахме.
Имайки това можем да покажем, че двата езика съвпадат, по следния начин:
\begin{center}
    $\alpha \in \mathcal{L(A)} \iff \delta^*(0, \alpha) = 0 \iff |\alpha| \: (mod \: 2) = 0 \iff \alpha \in L$
\end{center}


Нека сега помислим какъв автомат ще можем да направим за $\overline{L}$.
Ние вече имаме автомат за $L$.
От неговите състояние и преходи можем да извлечем информация за четността на дължината на думата.
Единственото, което трябва да направим, е да сменим отговора на автомата.
Ако преди той е казвал ДА, сега да казва НЕ, и обратното.

\pagebreak

Това означава просто да обърнем финалните състояния:
\begin{center}
    \begin{tikzpicture}[shorten >=1pt,node distance=2.5cm,>=stealth',thick]
        \node[text width=0.5cm] at (-1, 1) {$\mathcal{A}_{\overline{L}}$:};
        \node[initial, state, initial text=] (1) {$0$};
        \node[accepting, state] [right of=1] (2) {$1$};
        \path[->] (1) edge [bend left] node[above] {$a, b$} (2);
        \path[->] (2) edge [bend left] node[below] {$a, b$} (1);
    \end{tikzpicture}
\end{center}

\begin{problem}
Да се построи автомат за:
\begin{itemize}
    \item $L_1 = \{ a, b, ab \}$
    \item $L_2 = \{ \varepsilon, ab, abab \}$
    \item $L_3 = \{ aa, bb \}$
\end{itemize}
Упътване: за състояния да се използват остатъци при деление на $n$
\end{problem}

\begin{problem}
Да се построи автомат за:
\begin{itemize}
    \item $L_1 = \Sigma^* \setminus \{ a, b, ab \}$
    \item $L_2 = \Sigma^* \setminus \{ \varepsilon, ab, abab \}$
    \item $L_3 = \Sigma^* \setminus \{ aa, bb \}$
\end{itemize}
Упътване: да се използват автоматите от предната задача
\end{problem}

\begin{problem}
Да се построи автомат за:
\begin{itemize}
    \item $L_1 = \{ \alpha \in \Sigma^* \: | \: |\alpha| \text{ се дели на } 3 \}$
    \item $L_2 = \{ \alpha \in \Sigma^* \: | \: |\alpha| \text{ се дели на } 5 \}$
    \item $L_3 = \{ \alpha \in \Sigma^* \: | \: |\alpha| \text{ дава остатък } 2 \text{ при деление на } 4 \}$
    \item $L_4 = \{ \alpha \in \Sigma^* \: | \: |\alpha| \text{ дава остатък } 3 \text{ при деление на } 5 \}$
\end{itemize}
Упътване: за състояния да се използват остатъци при деление на $n$
\end{problem}

\begin{problem}
Да се построи автомат за:
\begin{itemize}
    \item $L_1 = \{ \alpha \in \Sigma^* \: | \: |\alpha| \text{ не се дели на } 3 \}$
    \item $L_2 = \{ \alpha \in \Sigma^* \: | \: |\alpha| \text{ не се дели на } 5 \}$
    \item $L_3 = \{ \alpha \in \Sigma^* \: | \: |\alpha| \text{ не дава остатък } 2 \text{ при деление на } 4 \}$
    \item $L_4 = \{ \alpha \in \Sigma^* \: | \: |\alpha| \text{ не дава остатък } 3 \text{ при деление на } 5 \}$
\end{itemize}
Упътване: да се използват автоматите от предната задача
\end{problem}
\section{Операции над автоматни езици}

Сега ще разгледаме няколко операции, които запазват автоматност.
Първата операция (която вече загатнахме) над автоматни езици, която ще разгледаме е допълнение.

\begin{claim}
    Ако $L$ е автоматен, то тогава и $\overline{L}$ също е автоматен.
\end{claim}

\begin{proof}
    Понеже $L$ е автоматен, има ДКА $\mathcal{A} = \opair{\Sigma, Q, s, \delta, F}$ с $\mathcal{L(A)} = L$.
    Нека $\mathcal{A}_{\overline{L}} = \opair{\Sigma, Q, s, \delta, Q \setminus F}$. Тогава:
    \begin{center}
        $\alpha \in \mathcal{L(A)} \iff \delta^*(s, \alpha) \in Q \setminus F \iff \delta^*(s, \alpha) \notin F \iff \alpha \notin L$
    \end{center}
\end{proof}

Както направихме и в предният пример, тук просто сменяме отговора.
Състоянията и преходите на началният автомат ни дават достатъчна информация за структурата на думата,
което е достатъчно за да кажем дали е от езика $\overline{L}$ или не е. \\

Преди да разгледаме следващата операция ще разгледаме два авомата:

\begin{center}
    \begin{tikzpicture}[shorten >=1pt,node distance=2.5cm,>=stealth',thick]
        \node[text width=0.5cm] at (-1, 1) {$\mathcal{A}_1$:};
        \node[accepting, initial, state, initial text=] (1) {$0$};
        \node[state] [right of=1] (2) {$1$};
        \path[->] (1) edge [bend left] node[above] {$a$} (2);
        \path[->] (2) edge [bend left] node[below] {$a$} (1);
        \path[->] (1) edge [loop above] node[above] {$b$} (1);
        \path[->] (2) edge [loop above] node[above] {$b$} (2);
    \end{tikzpicture}
\end{center}

\begin{center}
    \begin{tikzpicture}[shorten >=1pt,node distance=2.5cm,>=stealth',thick]
        \node[text width=0.5cm] at (-1, 1) {$\mathcal{A}_2$:};
        \node[initial, state, initial text=] (1) {$0$};
        \node[accepting, state] [right of=1] (2) {$1$};
        \path[->] (1) edge [bend left] node[above] {$b$} (2);
        \path[->] (2) edge [bend left] node[below] {$b$} (1);
        \path[->] (1) edge [loop above] node[above] {$a$} (1);
        \path[->] (2) edge [loop above] node[above] {$a$} (2);
    \end{tikzpicture}
\end{center}

Ще приемем за очевидно, че първият автомат разпознава думите с четен брой $a$,
а вторият автомат разпознава думите с нечетен брой $b$.

Как бихме могли да направим автомат $\mathcal{A}$ за $\mathcal{L}(\mathcal{A}_1) \cap \mathcal{L}(\mathcal{A}_2)$?
Можем да направим така:

\begin{center}
    \begin{tikzpicture}[node distance=2.5cm,>=stealth',thick]
        \node[text width=0.5cm] at (-1, 1) {$\mathcal{A}$:};
        \node[initial, state with output, initial text=] (1) {$0$ \nodepart{lower} $0$};
        \node[state with output] [right of=1] (2) {$0$ \nodepart{lower} $1$};
        \node[accepting, state with output, initial text=] [below of=1](3) {$1$ \nodepart{lower} $0$};
        \node[state with output] [right of=3] (4) {$1$ \nodepart{lower} $1$};
        \path[->] (1) edge [bend left] node[above] {$a$} (2);
        \path[->] (2) edge [bend left] node[below] {$a$} (1);
        \path[->] (3) edge [bend left] node[above] {$a$} (4);
        \path[->] (4) edge [bend left] node[below] {$a$} (3);
        \path[->] (1) edge [bend left] node[right] {$b$} (3);
        \path[->] (3) edge [bend left] node[left] {$b$} (1);
        \path[->] (2) edge [bend left] node[right] {$b$} (4);
        \path[->] (4) edge [bend left] node[left] {$b$} (2);
    \end{tikzpicture}
\end{center}

Имаме един брояч от за четността на $a$ и още един брояч за четността на $b$.
Всеки брояч се променя само от неговата буква.
Накрая искаме отгоре да седи $1$ (нечетен брой $b$), а отдолу да седи $0$ (четен брой $a$).

Естествен въпрос, който човек може да си зададе, е дали това не може да се обобщи за произволни автомати.
Оказва се, че може:

\begin{claim}
    Ако $L_1$ и $L_2$ са автоматни езици, то $L_1 \cap L_2$ също е автоматен език.
\end{claim}

\begin{proof}
    Нека $\mathcal{A}_1 = \opair{\Sigma, Q_1, s_1, \delta_1, F_1}$ е автомат за $L_1$ и нека $\mathcal{A}_2 = \opair{\Sigma, Q_2, s_2, \delta_2, F_2}$ е автомат за $L_2$.
    Строим автомат за сечението:
    \begin{itemize}
        \item $\mathcal{A} = \opair{\Sigma, Q, s, \delta, F}$
        \item $Q = Q_1 \cross Q_2$
        \item $s = \opair{s_1, s_2}$
        \item $\delta(\opair{p_1, p_2}, x) = \opair{\delta_1(p_1, x), \delta_2(p_2, x)}$ за $\opair{p_1, p_2} \in Q, \: x \in \Sigma$
        \item $F = F_1 \cross F_2$
    \end{itemize}

    Тук използваме, че двата автомата, които имаме по начало ни дават информация за структурата на думите.
    Ние паралелно изпълняваме четене в $\mathcal{A}_1$ и $\mathcal{A}_2$,
    и накрая искаме положителен отговор и от двата автомата.
    Обаче ние трябва да покажем, че наистина двата автомата работят паралелно.
    Отново ни трябва някакво твърдение за функцията $\delta^*$, която описва работата на автомата.
    Ние знаем, че за всяка буква правим преход и в двата компонента на текущото състояние $\opair{p_1, p_2}$.
    Това, което искаме да проверим, е че същото нещо се случва при четенето на една дълга дума.

    \pagebreak

    Ето го помощното твърдение:

    \begin{claim}
        $\delta^*(\opair{p_1, p_2}, \alpha) = \opair{\delta_1^*(p_1, \alpha), \delta_2^*(p_2, \alpha)}$
    \end{claim}

    \begin{proof}
        С индукция по $|\alpha|$.
        \begin{itemize}
            \item $\delta^*(\opair{p_1, p_2}, \varepsilon) = \opair{p_1, p_2} = \opair{\delta_1^*(p_1, \varepsilon), \delta_2^*(p_2, \varepsilon)}$ \checkmark
            \item $\delta^*(\opair{p_1, p_2}, \beta x) = \delta(\delta^*(\opair{p_1, p_2}, \beta), x) \stackrel{\text{ИП}}{=} \delta(\opair{\delta_1^*(p_1, \beta), \delta_2^*(p_2, \beta)}, x) = \opair{\delta_1(\delta_1^*(p_1, \beta), x), \delta_2(\delta_2^*(p_2, \beta), x)} = \opair{\delta_1^*(p_1, \beta x), \delta_2^*(p_2, \beta x)}$
        \end{itemize}
    \end{proof}

    Имайки това изкарваме директно, че:
    \begin{align*}
        \alpha \in \mathcal{L(A)} & \iff \delta^*(s, \alpha) \in F                                                   \\
                                  & \iff \opair{\delta_1^*(s_1, \alpha), \delta_2^*(s_2, \alpha)} \in F_1 \cross F_2 \\
                                  & \iff \delta_1^*(s_1, \alpha) \in F_1 \: \& \: \delta_2^*(s_2, \alpha) \in F_2    \\
                                  & \iff \alpha \in L_1 \: \& \: \alpha \in L_2                                      \\
                                  & \iff \alpha \in L_1 \cap L_2
    \end{align*}
\end{proof}

Имайки, че $\cap$ и допълнение запазват автоматност, директно изкарваме, че $\cup$ и $\setminus$ запазват автоматност:
\begin{itemize}
    \item $L_1 \cup L_2 = \overline{\overline{L_1 \cup L_2}} = \overline{\overline{L_1} \cap \overline{L_2}}$ \\
          Друг вариант това да се направи е да се приложи същата конструкция, със разликата че $F = (F_1 \cross Q_2) \cup (Q_1 \cross F_2)$
    \item $L_1 \setminus L_2 = L_1 \cap \overline{L_2}$ \\
          Друг вариант това да се направи е да се приложи същата конструкция, със разликата че $F = F_1 \cross (Q_2 \setminus F_2)$
\end{itemize}

Освен, че можем да изпълняваме няколко автомата паралелно, ние също така можем и да ги караме да се редуват.

\begin{claim}
    Ако $L_1$ и $L_2$ са автоматни езици, то и
    \begin{center}
        $\operatorname{Mix}(L_1, L_2) = \{ a_1b_1 \dots a_nb_n \: | \: n \in \mathbb{N} \: \& \: (\forall i \in \{ 1, \dots, n \})(a_i \in \Sigma \: \& \: b_i \in \Sigma) \: \& \: a_1 \dots a_n \in L_1 \: \& \: b_1 \dots b_n \in L_2 \}$
    \end{center}
    е автоматен.
\end{claim}

\begin{proof}
    Нека $\mathcal{A}_1 = \opair{\Sigma, Q_1, s_1, \delta_1, F_1}$ е автомат за $L_1$ и нека $\mathcal{A}_2 = \opair{\Sigma, Q_2, s_2, \delta_2, F_2}$ е автомат за $L_2$.
    Строим автомат за $\operatorname{Mix}(L_1, L_2)$:
    \begin{itemize}
        \item $\mathcal{A} = \opair{\Sigma, Q, s, \delta, F}$
        \item $Q = Q_1 \cross Q_2 \cross \{ 0, 1 \}$ (третата компонента ще ни казва с кой автомат работим в момента)
        \item $s = \opair{s_1, s_2, 0}$ (чисто начало, контрол има $\mathcal{A}_1$)
        \item $\delta(\opair{p_1, p_2, 0}, x) = \opair{\delta_1(p_1, x), p_2, 1}$ за $\opair{p_1, p_2} \in Q_1 \cross Q_2, \: x \in \Sigma$

              Има контрол е $\mathcal{A}_1$, ние правим преход само с неговото състояние, и след това предаваме контрола на $\mathcal{A}_2$
        \item $\delta(\opair{p_1, p_2, 1}, x) = \opair{p_1, \delta_2(p_2, x), 0}$ за $\opair{p_1, p_2} \in Q_1 \cross Q_2, \: x \in \Sigma$

              Аналогично тук има контрол е $\mathcal{A}_2$, затова ние правим преход само в него, и след това има контрол $\mathcal{A}_1$
        \item $F = F_1 \cross F_2 \cross \{ 0 \}$ ($\mathcal{A}_1$ и $\mathcal{A}_2$ са одобрили своите думи и сме прочели дума с четна дължина)
    \end{itemize}

    Случва се нещо като в тази картинка:

    \begin{center}
        \begin{tikzpicture}[node distance=2.5cm,>=stealth',thick]
            \node[initial, state with output, initial text=] (1) {$p_1$ \nodepart{lower} $q_1$};
            \node[state with output] [right of=1] (2) {$p_2$ \nodepart{lower} $q_1$};
            \node[state with output, initial text=] [right of=2](3) {$p_2$ \nodepart{lower} $q_2$};
            \node[state with output] [right of=3] (4) {$p_3$ \nodepart{lower} $q_2$};
            \node[state with output] [right of=4] (5) {$p_3$ \nodepart{lower} $q_3$};
            \path[->] (1) edge [bend left] node[above] {$x_1$} (2);
            \path[->] (2) edge [bend right] node[below] {$x_2$} (3);
            \path[->] (3) edge [bend left] node[above] {$x_3$} (4);
            \path[->] (4) edge [bend right] node[below] {$x_4$} (5);
        \end{tikzpicture}
    \end{center}

    Този автомат просто чете буква със $\mathcal{A}_1$ и буква със $\mathcal{A}_2$, после пак буква със $\mathcal{A}_1$ и буква със $\mathcal{A}_2$ и така нататък.

    Нека сега докажем, че това наистина се случва.

    \begin{claim}
        За всяко $\alpha \in \Sigma^*$:
        \begin{itemize}
            \item ако $\alpha = \alpha_1 \dots \alpha_{2n}$ за някои $n \in \mathbb{N}, \: \alpha_1, \dots, \alpha_{2n} \in \Sigma$, то \\
                  $\delta^*(\opair{s_1, s_2, 0}, \alpha) = \opair{\delta_1^*(s_1, \alpha_1 \alpha_3 \dots \alpha_{2n - 1}), \delta_2^*(s_2, \alpha_2 \alpha_4 \dots \alpha_{2n}), 0}$
            \item ако $\alpha = \alpha_1 \dots \alpha_{2n + 1}$ за някои $n \in \mathbb{N}, \: \alpha_1, \dots, \alpha_{2n + 1} \in \Sigma$, то \\
                  $\delta^*(\opair{s_1, s_2, 0}, \alpha) = \opair{\delta_1^*(s_1, \alpha_1 \alpha_3 \dots \alpha_{2n + 1}), \delta_2^*(s_2, \alpha_2 \alpha_4 \dots \alpha_{2n}), 1}$
        \end{itemize}
    \end{claim}

    \begin{proof}
        С индукция по $|\alpha|$.

        Базата е ясна. Нека разгледаме индукционната стъпка:
        \begin{align*}
             & \delta^*(\opair{s_1, s_2, 0},\alpha_1 \dots \alpha_{2n}) = \delta(\delta^*(\opair{s_1, s_2, 0}, \alpha_1 \dots \alpha_{2n - 1}), \alpha_{2n}) \stackrel{\text{ИП}}{=} \\
             & = \delta(\opair{\delta_1^*(s_1, \alpha_1 \alpha_3 \dots \alpha_{2n - 1}), \delta_2^*(s_2, \alpha_2 \alpha_4 \dots \alpha_{2n - 2}), 1}, \alpha_{2n}) =                \\
             & = \opair{\delta_1^*(s_1, \alpha_1 \alpha_3 \dots \alpha_{2n - 1}), \delta_2^*(s_2, \alpha_2 \alpha_4 \dots \alpha_{2n}), 0}
        \end{align*}
        \begin{align*}
             & \delta^*(\opair{s_1, s_2, 0},\alpha_1 \dots \alpha_{2n + 1}) = \delta(\delta^*(\opair{s_1, s_2, 0}, \alpha_1 \dots \alpha_{2n}), \alpha_{2n + 1}) \stackrel{\text{ИП}}{=} \\
             & = \delta(\opair{\delta_1^*(s_1, \alpha_1 \alpha_3 \dots \alpha_{2n - 1}), \delta_2^*(s_2, \alpha_2 \alpha_4 \dots \alpha_{2n}), 0}, \alpha_{2n + 1}) =                    \\
             & = \opair{\delta_1^*(s_1, \alpha_1 \alpha_3 \dots \alpha_{2n + 1}), \delta_2^*(s_2, \alpha_2 \alpha_4 \dots \alpha_{2n}), 1}
        \end{align*}
        С това сме готови.
    \end{proof}

    Имайки това:
    \begin{flalign*}
        \alpha \in \mathcal{L(A)}  \iff & \delta^*(s, \alpha) \in F                                                                                                                                                                                     \\
        \iff                            & \alpha = \alpha_1 \dots \alpha_{2n} \: (\alpha_i \in \Sigma) \: \& \: \delta_1^*(s_1, \alpha_1 \alpha_3 \dots \alpha_{2n - 1}) \in L_1 \: \& \:  \delta_2^*(s_2, \alpha_2 \alpha_4 \dots \alpha_{2n}) \in L_2 \\
        \iff                            & \alpha \in \operatorname{Mix}(L_1, L_2)
    \end{flalign*}

\end{proof}

\begin{claim}
    Нека $L_1$ и $L_2$ са автоматни езици и $\# \notin \Sigma$.
    Тогава $L_1 \cdot \{ \# \} \cdot L_2$ също е автоматен език.
\end{claim}

\begin{proof}
    Ще покажем само конструкцията, а доказателството ще оставим на читателя.

    Нека $\mathcal{A}_i = \opair{\Sigma, Q_i, s_i, \delta_i, F_i}$ е автомат за $L_i \: (\text{за } i = 1,2)$.
    Б.О.О. нека $Q_1 \cap Q_2 = \varnothing$.
    Също така нека $\cross \notin \Sigma$.
    Автоматът за $L_1 \cdot \{ \# \} \cdot L_2$ ще бъде $\mathcal{A} = \opair{\Sigma, Q_1 \cup Q_2 \cup \{ \cross \}, s_1, \delta, F_2}$, където:

    \begin{itemize}
        \item $\delta(p, x) = \delta_1(p, x)$ за $x \in \Sigma, \: p \in Q_1$
        \item $\delta(p, \#) = \cross$ за $p \notin F_1$
        \item $\delta(f, \#) = s_2$ за $f \in F_1$
        \item $\delta(p, x) = \delta_2(p, x)$ за $x \in \Sigma, \: p \in Q_2$
        \item $\delta(p, \#) = \cross$ за $p \in Q_2$
        \item $\delta(\cross, x) = \cross$ за $x \in \Sigma \cup \{ \# \}$
    \end{itemize}

    Казано на естествен език, четем със $\mathcal{A}_1$ докато не стигнем $\#$, после четем със $\mathcal{A}_2$.
    Накрая ще искаме да сме прочели $\#$ и прочетеното от $\mathcal{A}_1$ и $\mathcal{A}_2$ да бъде одобрено (т.е. да сме получили отговор ДА).
\end{proof}

\begin{claim}
    За всеки автоматен $L$ езиците $\operatorname{Pref}(L)$, $\operatorname{Suff}(L)$ и $\operatorname{Infix}(L)$ също са автоматни.
\end{claim}

\begin{proof}
    Нека $\mathcal{A} = \opair{\Sigma, Q, s, \delta, F}$ е автомат за $L$.

    Нека $F' = \{ q \in Q \mid (\exists \gamma \in \Sigma^*) \: (\delta^*(q, \gamma) \in F) \}$
    и $S' = \{ \delta^*(s, \beta) \mid \beta \in \Sigma^* \}$.
    Да помислим кога точно една дума е префикс на дума от $L$:
    \begin{flalign*}
        \beta \in \operatorname{Pref}(L) & \iff (\exists \gamma \in \Sigma^*) \: (\beta \gamma \in L)                         \\
                                         & \iff (\exists \gamma \in \Sigma^*) \: (\delta^*(s, \beta \gamma) \in F)            \\
                                         & \iff (\exists \gamma \in \Sigma^*) \: (\delta^*(\delta^*(s, \beta), \gamma) \in F) \\
                                         & \iff \delta^*(s, \beta) \in F'                                                     \\
                                         & \iff \beta \in \mathcal{L}(\opair{\Sigma, Q, s, \delta, F'})
    \end{flalign*}
    Току що показахме автомат за $\operatorname{Pref}(L)$, а именно $\opair{\Sigma, Q, s, \delta, F'}$.
    Така получихме, че $\operatorname{Pref}(L)$ е автоматен език.
    Можем да направим подобни разсъждения за суфикс:
    \begin{flalign*}
        \gamma \in \operatorname{Suff}(L) & \iff (\exists \beta \in \Sigma^*) \: (\beta \gamma \in L)                             \\
                                          & \iff (\exists \beta \in \Sigma^*) \: (\delta^*(s, \beta \gamma) \in F)                \\
                                          & \iff (\exists \beta \in \Sigma^*) \: (\delta^*(\delta^*(s, \beta), \gamma) \in F)     \\
                                          & \iff (\exists q \in S') \: (\delta^*(q, \gamma) \in F)                                \\
                                          & \iff \gamma \in \bigcup\limits_{q \in S'}\mathcal{L}(\opair{\Sigma, Q, q, \delta, F})
    \end{flalign*}
    Представихме $\operatorname{Suff}(L)$ като обединение на автоматни езици, откъдето и той е автоматен.
    Освен това $\operatorname{Infix}(L) = \operatorname{Pref}(\operatorname{Suff}(L))$ (от \thref{prefix-suffix-infix-props}), с което сме готови.
\end{proof}
\section{Недетерминирани автомати}

\begin{definition}
    \textbf{Недетерминиран краен автомат} (накратко НКА) ще наричаме всяко $\mathcal{N} = \opair{\Sigma, Q, S, \Delta, F}$, където:
    \begin{itemize}
        \item $\Sigma$ е крайна азбука
        \item $Q$ е крайно множество от състояния
        \item $S \subseteq Q$ (ще ги наричаме начални/стартови състояния)
        \item $\Delta : Q \cross \Sigma \rightarrow \mathcal{P}(Q)$ (ще я наричаме функция на преходите)
        \item $F \subseteq Q$ (ще ги наричаме финални състояния)
    \end{itemize}
\end{definition}

Цялата ``недетерминираност'' идва от това, че от едно състояние с една буква можем да отидем на няколко места.
Поне на пръв поглед тази нова машина изглежда доста по-мощна от старата.
Вместо да си налагаме точно един преход, можем да направим два или три прехода, а можем да отидем в ``нищото'' (има се предвид $\varnothing$).

\begin{definition}
    Дефинираме $\Delta^* : \mathcal{P}(Q) \cross \Sigma^* \rightarrow \mathcal{P}(Q)$ индуктивно:
    \begin{itemize}
        \item $\Delta^*(P, \varepsilon) = P$ за всяко $P \subseteq Q$
        \item $\Delta^*(P, \beta x) = \bigcup\limits_{q \in \Delta^*(P, \beta)}\Delta(q, x)$ за всяко $P \subseteq Q, \beta \in \Sigma^*, x \in \Sigma$
    \end{itemize}
\end{definition}

\begin{remark}
    $\Delta^*(P, \alpha \beta) = \Delta^*(\Delta^*(P, \alpha), \beta)$ (лесно излиза с индукция)
\end{remark}

Вече можем да кажем какъв е език на даден недетерминиран автомат.

\begin{definition}
    Нека $\mathcal{N} = \opair{\Sigma, Q, S, \Delta, F}$ е НКА.
    Тогава езикът на автомата $\mathcal{N}$ е множеството
    $\mathcal{L}(\mathcal{N}) = \{ \alpha \in \Sigma^* \: | \: \delta^*(S, \alpha) \cap F \neq \varnothing \}$.
\end{definition}

Тук цялата идея, е вместо думата да поеме по един предопределен път и да получи ДА или НЕ,
тя да може да изпробва няколко възможни такива, и ако получи отговор ДА за поне един от тях, то тогава получава ДА от целия автомат.
В някакъв смисъл има елемент на отгатване и несигурност. Може този път да свърши работа, но може и другия да свърши работа.

\pagebreak

Нека дадем няколко прости примери за недетерминирани автомати:
\begin{figure*}[h]
    \centering
    \centering
    \begin{tikzpicture}[node distance=2cm, thick]
        \node[initial, state, initial text=] (1) {$q$};
    \end{tikzpicture}
    \caption*{Автомат за $\varnothing$}
\end{figure*}

\begin{figure*}[h]
    \centering
    \begin{tikzpicture}[node distance=2cm, thick]
        \node[initial, state, initial text=] (1) {$\varepsilon$};
        \node[state, initial text=, right of=1] (2) {$a$};
        \node[state, initial text=, right of=2] (3) {$ab$};
        \node[accepting, state, initial text=, right of=3] (4) {$aba$};
        \path[->] (1) edge [] node[above] {$a$}(2);
        \path[->] (2) edge [] node[above] {$b$}(3);
        \path[->] (3) edge [] node[above] {$a$}(4);
    \end{tikzpicture}
    \caption*{Автомат за $\{ aba \}$}
\end{figure*}

Доста по-компактно става представянето на автомати за тези езици.
Напълно елимираме ненужните преходи и състояние боклук.
Можем и много лесно да направим автомат за няколко думи.

\begin{figure*}[h]
    \centering
    \begin{tikzpicture}[node distance=2cm, thick]
        \node[initial, state, initial text=] (1) {$\varepsilon_{aba}$};
        \node[state, initial text=, right of=1] (2) {$a$};
        \node[state, initial text=, right of=2] (3) {$ab$};
        \node[accepting, state, initial text=, right of=3] (4) {$aba$};
        \node[initial, state, initial text=, below of=1] (5) {$\varepsilon_{bab}$};
        \node[state, initial text=, right of=5] (6) {$b$};
        \node[state, initial text=, right of=6] (7) {$ba$};
        \node[accepting, state, initial text=, right of=7] (8) {$bab$};
        \path[->] (1) edge [] node[above] {$a$}(2);
        \path[->] (2) edge [] node[above] {$b$}(3);
        \path[->] (3) edge [] node[above] {$a$}(4);
        \path[->] (5) edge [] node[above] {$b$}(6);
        \path[->] (6) edge [] node[above] {$a$}(7);
        \path[->] (7) edge [] node[above] {$b$}(8);
    \end{tikzpicture}
    \caption*{Автомат за $\{ aba, bab \}$}
\end{figure*}

Тук се опитваме да познаем дали четем една от двете думи.
Двата малки автомата работят независимо един от други и всеки дава собствен отговор.
Ако сме се озовали във финално състояние, сме прочели една от двете думи в езика, иначе не сме.
\section{Операции над езици на недетерминирани автомати}

Тази конструкция може да се обобщи за обединение на всеки два езика на недетерминирани автомати.

\begin{claim}
    Нека $\mathcal{N}_i = \opair{\Sigma, Q_i, S_i, \Delta_i, F_i}$  е недетерминиран автомат за $i = 1, 2$.
    Тогава съществува недетерминиран автомат за езика $\mathcal{L(N}_1) \cup \mathcal{L(N}_2)$.
\end{claim}

\begin{proof}
    Б.О.О. нека $\underbrace{Q_1 \cap Q_2 = \varnothing}_{(\star)}$. Тогава:
    \begin{align*}
        \alpha \in \mathcal{L}(\opair{\Sigma, Q_1 \cup Q_2, S_1 \cup S_2, \underbrace{\Delta_1 \cup \Delta_2}_{\textcolor{red}{!!!}}, F_1 \cup F_2}) \iff & (\Delta_1 \cup \Delta_2)(S_1 \cup S_2, \alpha) \cap (F_1 \cup F_2) \neq \varnothing                      \\
        \stackrel{(\star)}{\iff}                                                                                                                          & \Delta_1^*(S_1, \alpha) \cap F_1 \neq \varnothing \lor \Delta_2^*(S_2, \alpha) \cap F_2 \neq \varnothing \\
        \iff                                                                                                                                              & \alpha \in \mathcal{L(N}_1) \lor \alpha \in \mathcal{L(N}_2)                                             \\
        \iff                                                                                                                                              & \alpha \in \mathcal{L(N}_1) \cup \mathcal{L(N}_2)
    \end{align*}
    \begin{remark}[\textcolor{red}{!!!}]
        Понеже $(\star)$ е вярно, $\operatorname{Dom}(\Delta_1) \cap \operatorname{Dom}(\Delta_2) = \varnothing$, откъдето $\Delta_1 \cup \Delta_2$ е добре дефинирана функция.
    \end{remark}
\end{proof}

Тази конструкция при недетерминирани автомати е по-компактна в сравнение с детерминираната версия.
В този случай новите състояния са $|Q_1 \cup Q_2| = |Q_1| + |Q_2|$ на брой,
докато при детерминираните автомати се получават $|Q_1 \crossproduct Q_2| = |Q_1| \cdot |Q_2|$ на брой състояния за новия автомат, което може да бъде много повече.

\begin{itemize}
    \item $10 + 10 = 20$, докато $10 \cdot 10 = 100$
    \item $1000 + 2 = 1002$, докато $1000 \cdot 2 = 2000$
    \item $1000 + 1000 = 2000$, докато $1000 \cdot 1000 = 1000000$
\end{itemize}

\begin{claim}
    Нека $L$ е автоматен език.
    Тогава има недетерминиран автомат за:
    \begin{center}
        $\operatorname{ChangeSomeLetters}(L) = \{ \alpha \in \Sigma^* \mid (\exists \beta \in L) \: (|\beta| = |\alpha| ) \}$
    \end{center}
\end{claim}

\begin{proof}
    Нека $\mathcal{A} = \opair{\Sigma, Q, s, \delta, F}$ е ДКА за $L$.
    Строим автомат $\mathcal{N}$ за $\operatorname{ChangeSomeLetters}(L)$:
    \begin{itemize}
        \item $\mathcal{N} = \opair{\Sigma, Q, S, \Delta, F}$
        \item $S = \{ s \}$
        \item $\Delta(p, x) = \{ \delta(p, a), \delta(p, b) \}$ за $p \in Q, \: x \in \Sigma$
    \end{itemize}

    Оставяме доказателството на следния факт на читателя, понеже е елементарна индукция:
    \begin{center}
        $\Delta^*(S, \alpha) = \{ \delta^*(s, \beta) \mid \beta \in \Sigma \: \& \: |\beta| = |\alpha| \}$
    \end{center}

    Имайки това твърдение получаваме, че:
    \begin{align*}
        \alpha \in \mathcal{L(N)} & \iff \Delta^*(S, \alpha) \cap F \neq \varnothing \iff (\exists \beta \in \Sigma^{|\alpha|}) \: (\delta^*(s, \beta) \in F) \\
                                  & \iff (\exists \beta \in \Sigma^{|\alpha|}) \: (\beta \in L) \iff \alpha \in \operatorname{ChangeSomeLetters}(L)
    \end{align*}
\end{proof}

\begin{claim}
    Нека $\mathcal{N}_i = \opair{\Sigma, Q_i, S_i, \Delta_i, F_i}$  е недетерминиран автомат за $i = 1, 2$.
    Тогава съществува недетерминиран автомат за езика $\mathcal{L(N}_1) \cdot \mathcal{L(N}_2)$.
\end{claim}

\begin{proof}
    Нека Б.О.О. $Q_1 \cap Q_2 = \varnothing$.
    Строим недетерминиран автомат $\mathcal{N} = \opair{\Sigma, Q, S, \Delta, F}$ за $\mathcal{L(N}_1) \cdot \mathcal{L(N}_2)$:
    \begin{itemize}
        \item $Q = Q_1 \cup Q_2$
        \item $S = S_1 \cup S_2$ ако $\varepsilon \in \mathcal{L(N}_1)$, иначе $S = S_1$
        \item $F = F_2$
        \item $\Delta(p, x) = \Delta_1(p, x) \cup S_2$, ако $\Delta_1(p, x) \cap F \neq \varnothing \: (\star)$
        \item За другите състояния от $Q_1$ взимаме преходите от $\Delta_1$, и аналогично за $Q_2$ взимаме същите преходи от $\Delta_2$
    \end{itemize}

    Сега остава да покажем, че $\mathcal{L(N)} = \mathcal{L(N}_1) \cdot \mathcal{L(N}_2)$.
    \begin{itemize}
        \item $\mathcal{L(N)} \subseteq \mathcal{L(N}_1) \cdot \mathcal{L(N}_2)$ \\
              Нека $\alpha \in \mathcal{L(N)}$.
              Тогава $\Delta^*(S, \alpha) \cap F \neq \varnothing$.
              Ако сме стигнали до финално от $S_2$, то тогава $\alpha \in \mathcal{L(N}_2)$ и $\varepsilon \in \mathcal{L(N}_1)$,
              откъдето $\alpha = \varepsilon \cdot \alpha \in \mathcal{L(N}_1) \cdot \mathcal{L(N}_2)$ и сме готови.
              В противен случай $\varepsilon \notin \mathcal{L(N}_1)$ и има $\beta, \gamma \in \Sigma^*$ такива,
              че $\alpha = \beta \gamma, \: \beta \neq \varepsilon, \: S_2 \subseteq \Delta^*(S, \beta)$,  т.е. сме приложили $(\star)$.
              Тогава с последната буква на $\beta$ сме достигнали до финално състояние в $\mathcal{L(N}_1)$, откъдето $\beta \in \mathcal{L(N}_1)$.
              Също така понеже $Q_1 \cap Q_2 = \varnothing$, и състоянията от $Q_2$ имат само преходите на $\Delta_2$,
              $\Delta(S_2, \gamma) \cap F \neq \varnothing$, откъдето $\gamma \in \mathcal{L(N}_2)$.
              Така получаваме, че $\alpha = \beta \cdot \gamma \in \mathcal{L(N}_1) \cdot \mathcal{L(N}_2)$.

        \item $\mathcal{L(N}_1) \cdot \mathcal{L(N}_2) \subseteq \mathcal{L(N)}$ \\
              Нека $\beta \in \mathcal{L(N}_1), \gamma \in \mathcal{L(N}_2)$.
              Ако $\beta = \varepsilon$, то тогава $S_2 \subseteq S$, и понеже $\gamma \in \mathcal{L(N}_2), \: F = F_2$,
              то $\Delta^*(S, \underbrace{\beta \cdot \gamma}_{\gamma}) \cap F \neq \varnothing$, и от там $\beta \cdot \gamma \in \mathcal{L(N)}$.
              Ако $\beta \neq \varepsilon$, то $S_2 \subseteq \Delta^*(S, \beta)$, понеже $\beta \in \mathcal{L(N}_1)$ и $(\star)$.
              Освен това и $\gamma \in \mathcal{L(N}_2)$ т.е. $\Delta^*(S_2, \gamma) \cap F_2 \neq \varnothing$,
              откъдето $\Delta^*(S, \beta \cdot \gamma) \cap F \neq \varnothing$ т.е. $\beta \cdot \gamma \in \mathcal{L(N)}$.
    \end{itemize}
\end{proof}

\begin{claim}
    Нека $\mathcal{N} = \opair{\Sigma, Q, S, \Delta, F}$ е недетерминиран автомат.
    Тогава има недетерминиран автомат за езика $\mathcal{L(N)^*}$.
\end{claim}

\begin{proof}
    Ще покажем само конструкцията, без да я обосноваваме (аналогична е на предната).
    Правим автомат за $\mathcal{L(N)^+}$ и използваме, че $L^* = L^+ \cup \{ \varepsilon \}$ за всеки език $L$.
    Нека $\mathcal{N}' = \opair{\Sigma, Q, S, \Delta', F}$, където:
    \begin{equation}
        \Delta'(p, x) =
        \begin{cases}
            \Delta(p, x)                     & \text{ако } p \notin F \\
            \Delta(p, x) \cup \Delta^*(S, x) & \text{ако } p \in F
        \end{cases}
    \end{equation}
    Доказателството на коректност е аналогично на предната задача.
\end{proof}

\begin{claim}
    Нека $\mathcal{N} = \opair{\Sigma, Q, S, \Delta, F}$ е недетерминиран автомат.
    Тогава има недетерминиран автомат за езика $\mathcal{L(N)}^{rev}$.
\end{claim}

\begin{proof}
    Нека $\mathcal{N}_{rev} = \opair{\Sigma, Q, F, \Delta_{rev}, S}$, където:
    \begin{center}
        $\Delta_{rev} = \{ \opair{p, x, q} \mid p \in \Delta(q, x) \}$
    \end{center}
    Тривиално може да се покаже с индукция, че за $\alpha \in \Sigma^*, \: p, q \in Q$:
    \begin{center}
        $q \in \Delta^*(\{ p \}, \alpha) \iff p \in \Delta_{rev}^*(\{ q \}, \alpha) \: (\star)$
    \end{center}
    След това приключваме със:
    \begin{align*}
        \alpha \in \mathcal{L(N)} \iff & \Delta^*(S, \alpha) \cap F \neq \varnothing                                            \\
        \stackrel{(\star)}{\iff}       & \Delta_{rev}^*(F, \alpha) \cap S \neq \varnothing \iff \alpha \in \mathcal{L(N}_{rev})
    \end{align*}
\end{proof}
\section{Еквивалентност на детерминирани и недетерминирани автомати}

Въпреки че на пръв поглед недетерминираните автомати ни се струват по-мощни, те не ни дават нищо повече освен удобство.
Първо ще покажем, по-очевидното твърдение.

\begin{claim}
    За всеки детерминиран автомат $\mathcal{A}$ съществува недетерминиран автомат $\mathcal{N}$ със $\mathcal{L(N) = L(A)}$.
\end{claim}

\begin{proof}
    Нека $\mathcal{A} = \opair{\Sigma, Q, s, \delta, F}$ е детерминиран автомат.
    Нека $\mathcal{N} = \opair{\Sigma, Q, \{ s \}, \Delta, F}$,
    където $\Delta(p, x) = \{ \delta(p, x) \}$ за $p \in Q$.
    Тривиално може да се покаже с индукция, че за всяко $p \in Q, \: \alpha \in \Sigma^*$: $\Delta^*(p, \alpha) = \{ \delta^*(p, \alpha) \}$.
    Така:
    \begin{align*}
        \alpha \in \mathcal{L(N)} & \iff \Delta^*(\{ s \}, \alpha) \cap F \neq \varnothing \iff \{ \delta^*(s, \alpha) \} \cap F \neq \varnothing \\
                                  & \iff \delta^*(s, \alpha) \in F \iff \alpha \in \mathcal{L(A)}
    \end{align*}
\end{proof}

Сега пък ще покажем как ``детерминизира'' недетерминиран автомат.

\begin{claim}
    За всеки недетерминиран автомат $\mathcal{N}$ съществува детерминиран автомат $\mathcal{A}$ със $\mathcal{L(A) = L(N)}$.
\end{claim}

\begin{proof}
    Нека $\mathcal{N} = \opair{\Sigma, Q, S, \Delta, F}$ е недетерминиран автомат.
    Нека $\mathcal{A} = \opair{\Sigma, \mathcal{P}(Q), S, \delta, F'}$,
    където $\delta(P, x) = \Delta^*(P, x)$ за $P \subseteq Q$ и $F' = \{ P \in \mathcal{P}(Q) \mid P \cap F \neq \varnothing \}$.
    Тривиално може да се покаже с индукция, че за всяко $P \subseteq Q, \: \alpha \in \Sigma^*$: $\delta^*(P, \alpha) = \Delta^*(P, \alpha)$.
    Така:
    \begin{align*}
        \alpha \in \mathcal{L(N)} & \iff \Delta^*(S, \alpha) \cap F \neq \varnothing \iff \delta^*(S, \alpha) \cap F \neq \varnothing \\
                                  & \iff \delta^*(S, \alpha) \in F' \iff \alpha \in \mathcal{L(A)}
    \end{align*}
\end{proof}

Имайки това вече можем да използваме, че $\cdot$, $*$ и $rev$ запазват автоматност.
\begin{warning}
    При ``детерминизиране'' броят на състоянията нараства експоненциално.
    Накратко $10$ става $1024$.
\end{warning}

\begin{problem}
Да се детерминира следният автомат:
\begin{center}
    \begin{tikzpicture}[shorten >=1pt,node distance=2.5cm,>=stealth',thick]
        \node[initial, state, initial text=] (1) {$0$};
        \node[state] [right of=1] (2) {$1$};
        \node[initial, state, initial text=] [below left of=2] (3) {$2$};
        \node[accepting, state] [right of=3] (4) {$3$};
        \path[->] (1) edge [loop above] node[above] {$a$} (1);
        \path[->] (1) edge node[above] {$a, b$} (2);
        \path[->] (1) edge node[left] {$b$} (3);
        \path[->] (2) edge node[above] {$b$} (3);
        \path[->] (2) edge node[right] {$b$} (4);
        \path[->] (3) edge [loop below] node[below] {$a$} (3);
        \path[->] (3) edge node[above] {$a$} (4);
    \end{tikzpicture}
\end{center}
\end{problem}
\section{Друг начин да си мислим за автоматите}

Освен като абстрактни машини, можем да си мислим за детерминираните автомати като за програми.
Тези ``програми'' имат няколко много хубави свойства:
\begin{itemize}
    \item винаги завършват работа
    \item работят с константна памет
    \item работят за линейно време спрямо дължината на дума
\end{itemize}

Нека вземем за пример следният автомат:

\begin{center}
    \begin{tikzpicture}[shorten >=1pt,node distance=2.5cm,>=stealth',thick]
        \node[accepting, initial, state, initial text=] (1) {$0$};
        \node[state] [right of=1] (2) {$1$};
        \path[->] (1) edge [loop above] node[above] {$b$} (1);
        \path[->] (1) edge [bend left] node[above] {$a$} (2);
        \path[->] (2) edge [loop above] node[above] {$b$} (2);
        \path[->] (2) edge [bend left] node[below] {$a$} (1);
    \end{tikzpicture}
\end{center}

Ясно е, че той разпознава думи с четен брой букви $a$.
Нека видим как бихме направили програма на C++, която приема низ и връща дали този низ съдържа четен брой $a$:

\begin{minted}[linenos]{C++}
typedef char state_index;

const state_index table[2][2] = {{1, 0}, {0, 1}};   // таблица на преходите
const bool is_accepting_state[2] = { true, false }; // финални състояния
const state_index INITIAL_STATE = 0;                // начално състояние

// функция на преходите
state_index delta(state_index state, char letter) { return table[state][letter - 'a']; }

bool accepts_word(const std::string &word)
{
  // предполагаме валиден вход
  // т.е. низът съдържа само буквите 'a' и 'b'
  state_index state = INITIAL_STATE; // започваме в началното състояние

  for (size_t i = 0; i < word.size(); ++i)
  {
    state = delta(state, word[i]); // правим преход с настоящото състояние и буква
  }

  return is_accepting_state[state]; // накрая искаме да сме във финално състояние
}
\end{minted}

Лесно можем да видим как тази конструкция може да се адаптира за произволен автомат $\mathcal{A} = \opair{\Sigma, Q, q_0, \delta, F}$:
\begin{itemize}
    \item Нека $Q = \{ q_0, \dots q_{n - 1} \}$.
          За състоянията се разбираме, че на $q_i$ съответства числото $i$.
          За типа на \mintinline{C++}|state_index| ще ни трябва променлива, която ще побере числата от $0$ до $n - 1$.
    \item Нека $\Sigma = \{ \sigma_0, \dots, \sigma_{k - 1} \}$.
          Нужна ни е функция, която да е биекция между $\Sigma$ и $\{ 0, \dots, k - 1 \}$.
          Нека сигнатурата и да бъде \mintinline{C++}|state_index letter_to_index(char letter)|.
          Също така ще искаме да работи в константно време, което е лесно, защото азбуката е с фиксиран размер.
    \item Правим \mintinline{C++}|state_index table[n][k]| спрямо нашия оригинален автомат и кодировката от \mintinline{C++}|letter_to_index|:
          \begin{center}
              \mintinline{C++}|table[i][letter_to_index(c)] == j| е истина $\iff \delta^*(q_i, c) = q_j$
          \end{center}
    \item Правим \mintinline{C++}|bool is_accepting_state[n]| спрямо нашия оригинален автомат и кодировката от \mintinline{C++}|letter_to_index|:
          \begin{center}
              \mintinline{C++}|table[i]| е истина $\iff q_i \in F$
          \end{center}
    \item Понеже $q_0$ е началното състояние:
          \begin{minted}{C++}
const state_index INITIAL_STATE = 0;
          \end{minted}
    \item Правим програмна реализация на $\delta$ чрез \mintinline{C++}|table|:
          \begin{minted}{C++}
state_index delta(state_index state, char letter)
{
    return table[state][letter_to_index(letter)];
}
          \end{minted}
    \item Функцията \mintinline{C++}|accepts_word| няма нужда от модификация.
          В зависимост от това дали има нужда може да се сложи валидация на входните данни.
\end{itemize}

\begin{remark}
    Разбира се това не е единственият възможен начин да се направи програмна реализация на автомат.
    За по-малко на брой състояния също би било удобно да се използват оператори \mintinline{C++}|if| или \mintinline{C++}|switch| със \mintinline{C++}|case|.
    Обаче да имаме функцията $\delta$ записана по някакъв начин в таблица ни прави програмата по-бърза при повече състояния.
    Много по-лесно е да намериш елемент номер $10000$ в масив отколкото да оцениш $10000$ пъти оператор \mintinline{C++}|if|.
\end{remark}

\begin{problem}
Да се направи реализация използвайки \mintinline{C++}|if| или \mintinline{C++}|switch| със \mintinline{C++}|case| вместо таблица за $\delta$ на следния автомат:
\begin{center}
    \begin{tikzpicture}[shorten >=1pt,node distance=2.5cm,>=stealth',thick]
        \node[accepting, initial, state, initial text=] (1) {$0$};
        \node[state] [right of=1] (2) {$1$};
        \path[->] (1) edge [loop above] node[above] {$b$} (1);
        \path[->] (1) edge  node[above] {$a$} (2);
        \path[->] (2) edge [loop above] node[above] {$a, b$} (2);
    \end{tikzpicture}
\end{center}
\end{problem}

\begin{problem}
Да се направи програмна реализация и да се определи езика на следните автомати:
\begin{figure*}[h]
    \begin{subfigure}{1.0\textwidth}
        \centering
        \begin{tikzpicture}[shorten >=1pt,node distance=2.5cm,>=stealth',thick]
            \node[initial, state, initial text=] (1) {$0$};
            \node[accepting, state] [right of=1] (2) {$1$};
            \path[->] (1) edge [loop above] node[above] {$a$} (1);
            \path[->] (1) edge [bend left] node[above] {$b$} (2);
            \path[->] (2) edge [loop above] node[above] {$a$} (2);
            \path[->] (2) edge [bend left] node[below] {$b$} (1);
        \end{tikzpicture}
    \end{subfigure}
    \vspace{\floatsep}
    \begin{subfigure}{1.0\textwidth}
        \centering
        \begin{tikzpicture}[shorten >=1pt,node distance=2.5cm,>=stealth',thick]
            \node[initial, state, initial text=] (1) {$0$};
            \node[accepting, state] [right of=1] (2) {$1$};
            \node[accepting, state] [right of=2] (3) {$2$};
            \path[->] (1) edge [loop above] node[above] {$a$} (1);
            \path[->] (1) edge node[above] {$b$} (2);
            \path[->] (2) edge [loop above] node[above] {$a$} (2);
            \path[->] (2) edge  node[above] {$b$} (3);
            \path[->] (3) edge [loop above] node[above] {$a$} (3);
            \path[->] (3) edge [bend left] node[below] {$b$} (1);
        \end{tikzpicture}
    \end{subfigure}
    \vspace{\floatsep}
    \begin{subfigure}{1.0\textwidth}
        \centering
        \begin{tikzpicture}[node distance=2.5cm,>=stealth',thick]
            \node[accepting, initial, state with output, initial text=] (1) {$0$ \nodepart{lower} $0$};
            \node[state with output] [right of=1] (2) {$0$ \nodepart{lower} $1$};
            \node[state with output, initial text=] [below of=1](3) {$1$ \nodepart{lower} $0$};
            \node[accepting, state with output] [right of=3] (4) {$1$ \nodepart{lower} $1$};
            \path[->] (1) edge [bend left] node[above] {$a$} (2);
            \path[->] (2) edge [bend left] node[below] {$a$} (1);
            \path[->] (3) edge [bend left] node[above] {$a$} (4);
            \path[->] (4) edge [bend left] node[below] {$a$} (3);
            \path[->] (1) edge [bend left] node[right] {$b$} (3);
            \path[->] (3) edge [bend left] node[left] {$b$} (1);
            \path[->] (2) edge [bend left] node[right] {$b$} (4);
            \path[->] (4) edge [bend left] node[left] {$b$} (2);
        \end{tikzpicture}
    \end{subfigure}
\end{figure*}
\end{problem}

\section{Регулярни изрази и регулярни езици}
\section{Операции над регулярни езици}

Сега ще разгледаме някои операции, които запазват регулярност.

\begin{claim}
    Ако $L$ е регулярен език, то и езикът $L^{rev}$ също е регулярен.
\end{claim}

\begin{proof}
    С индукция по строене на регулярните езици.

    \begin{itemize}
        \item $L^{rev} = L$ за всеки $L \in \{ \varnothing, \{ \varepsilon \}, \{ a \}, \{ b \} \}$ \checkmark
        \item $(L_1 \cdot L_2)^{rev} = L_2^{rev} \cdot L_1^{rev}$ (от \nameref{prefix-suffix-infix-props}) \\
              По ИП $L_1^{rev}$ и $L_2^{rev}$ са регулярни, откъдето $L_2^{rev} \cdot L_1^{rev}$ е регулярен.
        \item $(L_1 \cup L_2)^{rev} = L_1^{rev} \cup L_2^{rev}$ (директна проверка с дефиниции) \\
              По ИП $L_1^{rev}$ и $L_2^{rev}$ са регулярни, откъдето $L_2^{rev} \cup L_1^{rev}$ е регулярен.
        \item $(L^*)^{rev} = (L^{rev})^*$ (обобщение на предните две) \\
              По ИП $L^{rev}$ e регулярен, откъдето $(L^{rev})^*$ е регулярен.
    \end{itemize}
\end{proof}

\begin{claim}
    Със $sub(\alpha)$ ще бележим множеството от всички подредици на $\alpha$.
    Ако $L$ е регулярен език, то и $sub[L]$ е регулярен.
\end{claim}

\begin{proof}
    Ще покажем само структурата на индукцията.

    \begin{itemize}
        \item базата е очевидна \checkmark
        \item $sub[L_1 \cdot L_2] = sub[L_1] \cdot sub[L_2]$ (понеже $sub(\alpha \cdot \beta) = sub(\alpha) \cdot sub(\beta)$)
        \item $sub[L_1 \cup L_2] = sub[L_1] \cup sub[L_2]$ (така работят функциите)
        \item $sub[L^*] = (sub[L])^*$ (обобщение на предните две)
    \end{itemize}

    Остава само да се приложат индукционните предположения.
\end{proof}
\section{Еквивалентност на регулярните и автоматните езици}

\begin{theorem}[Теорема на Клини]
    За всеки език $L \subseteq \Sigma^*$:
    \begin{center}
        $L$ е автоматен $\iff$ $L$ е регулярен
    \end{center}
\end{theorem}

\begin{proof}
    $(\Leftarrow)$ С индукция по строене на регулярните езици.

    \begin{itemize}
        \item $\varnothing, \varepsilon, \{ a \}, \{ b \}$ са автоматни езици
        \item $\cup, \: \cdot$ и $*$ запазват автоматност
    \end{itemize}

    $(\Rightarrow)$ Тук ще трябва да поработим малко повече.

    Нека $\mathcal{A} = \opair{\Sigma, Q, q_0, \delta, F}$ е ДКА за $L$.
    Нека $Q = \{ q_0, \dots, q_{n - 1} \}$.
    Със $L(i, j, k)$ ще бележим множеството от всички думи $\alpha$,
    за които $\delta^*(q_i, \alpha) = q_j$ и всички междинни състояния имат индекс $< k$.
    Ясно е, че $L = \bigcup \{ L(0, j, n) \mid q_j \in F \}$.
    Строим $L(i, j, k)$ рекурсивно:

    За $k = 0$ има две възможности (които тривиално са регулярни езици):
    \begin{itemize}
        \item за $i = j$ имаме, че $L(i, j, 0) = \{ \varepsilon \} \cup \{ x \in \Sigma \mid \delta(q_i, x) = q_j \}$
        \item за $i \neq j$ имаме, че $L(i, j, 0) = \{ x \in \Sigma \mid \delta(q_i, x) = q_j \}$
    \end{itemize}
    Имайки $L(i, j, k)$, за $L(i, j, k + 1)$ имаме две възможности:
    \begin{itemize}
        \item $q_k$ да не се среща като междинно. Тогава сме в $L(i, j, k)$
        \item $q_k$ се среща като междинно. Разделяме по срещнанията на $q_k$:
    \end{itemize}
    \begin{center}
        \begin{tabular}{|c|c|c|c|c|}
            \hline
            $\in L(i, k, k)$                                                                                                                                   &
            $\in L(k, k, k)$                                                                                                                                   &
            $\dots$                                                                                                                                            &
            $\in L(k, k, k)$                                                                                                                                   &
            $\in L(k, j, k)$                                                                                                                                     \\
            \hline
            \multicolumn{1}{@{}c@{}}{$\underbrace{\hspace*{\dimexpr\tabcolsep+2\arrayrulewidth}\hphantom{\text{1-во срещане}}}_{\text{1-во срещане}}$}         &
            \multicolumn{1}{@{}c@{}}{$\underbrace{\hspace*{\dimexpr\tabcolsep+2\arrayrulewidth}\hphantom{\text{2-ро срещане}}}_{\text{2-ро срещане}}$}         &
            \multicolumn{1}{@{}c@{}}{}                                                                                                                         &
            \multicolumn{1}{@{}c@{}}{$\underbrace{\hspace*{\dimexpr\tabcolsep+2\arrayrulewidth}\hphantom{\text{последно срещане}}}_{\text{последно срещане}}$} &
            \multicolumn{1}{@{}c@{}}{}
        \end{tabular}
    \end{center}
    Тъй като го разделихме на всички срещания няма как да получим междинни с индекс $k$.
    Накрая:
    \begin{center}
        $L(i, j, k + 1) = L(i, j, k) \cup (L(i, k, k) \cdot L(k, k, k)^* \cdot L(k, j, k))$
    \end{center}
\end{proof}
\section{Неавтоматни езици}

До сега всичко, което сме правили е да се опитаме да докажем, че някакъв език е автоматен.
Тогава идва естественият въпрос дали има неавтоматни езици, и ако да - как изглеждат те.
Че има неавтоматни езици е ясно, иначе тази класификация на езици щеше да е безсмислена.
Какъв обаче би бил един такъв език и как точно да докажем, че не може да се направи автомат за него.
Като че ли изглежда по-лесно да докажеш съществуването на обект като го конструираш, отколкото да докажеш, че такъв не може да се построи.

Ще вземем два метода за доказването на неавтоматност на езици.
Кое кога да се използва е въпрос на лично предпочитание.
За първият метод ще трябва да въведем една допълнителна дефиниция.

\begin{definition}
    Нека $\alpha \in \Sigma^*$ и $L \subseteq \Sigma^*$. Тогава:
    \begin{align*}
        \alpha^{-1}(L) = \{ \beta \in \Sigma^* \mid \alpha \cdot \beta \in L \}
    \end{align*}
\end{definition}

\begin{claim}[класови критерий за нерегулярност]\thlabel{nerode-nonregular}
    Ако $\{ \alpha^{-1}(L) \mid \alpha \in \Sigma^* \}$ е безкрайно, то $L$ не е автоматен
\end{claim}

\begin{proof}
    Нека $\mathcal{A} = \opair{\Sigma, Q, s, \delta, F}$ е ДКА за $L$. Тогава:
    \begin{align*}
        \beta \in \alpha^{-1}(L) & \iff \alpha \cdot \beta \in L \iff \delta^*(s, \alpha \cdot \beta) \in F                                                                                \\
                                 & \iff \delta^*(\delta^*(s, \alpha), \beta) \in F \iff \beta \in \mathcal{L}(\underbrace{\opair{\Sigma, Q, \delta^*(s, \alpha), \delta, F}}_{\text{ДКА}})
    \end{align*}

    От тук можем да видим, че ако $L$ е автоматен, то за всяка дума $\alpha$,
    на $\alpha^{-1}(L)$ съпоставяме състояние от $Q$ т.е. $|\{ \alpha^{-1}(L) \mid \alpha \in \Sigma^* \}| \leq |Q| < \infty$.
    Получаваме импликацията:
    \begin{center}
        $L$ е автоматен $\Rightarrow \{ \alpha^{-1}(L) \mid \alpha \in \Sigma^* \}$ е крайно
    \end{center}
    Но ние знаем, че $p \Rightarrow q$ е еквивалентно на $\neg q \Rightarrow \neg p$, откъдето:
    \begin{center}
        $\{ \alpha^{-1}(L) \mid \alpha \in \Sigma^* \}$ е безкрайно $\Rightarrow L$ не е автоматен
    \end{center}
\end{proof}

\pagebreak

Започваме с каноничния пример за неавтоматен език:

\begin{claim}
    Езикът $L = \{ a^nb^n \mid n \in \mathbb{N} \}$ не е автоматен.
\end{claim}

\begin{proof}
    Искаме да покажем, че $\{ \alpha^{-1}(L) \mid \alpha \in \Sigma^* \}$ е безкрайно.
    За целта ни трябват изброима редица от думи $\alpha_0, \alpha_1, \alpha_2, \dots$ такава, че за $i \neq j : \alpha_i^{-1}(L) \neq \alpha_j^{-1}(L)$.
    Твърдя, че думите от вида $a^n$ за $n \in \mathbb{N}$ ще ни свършат работа.
    Нека $n, k \in \mathbb{N}, \: n \neq k$.
    Тогава:
    \begin{itemize}
        \item $a^nb^n \in L \Rightarrow b^n \in (a^n)^{-1}(L)$
        \item $a^kb^n \notin L \Rightarrow b^n \notin (a^k)^{-1}(L)$
    \end{itemize}
    Така получаваме, че $(a^n)^{-1}(L) \neq (a^k)^{-1}(L)$.

    Тук $(a^0)^{-1}(L), (a^1)^{-1}(L), (a^2)^{-1}(L), \dots$ са безброй много съществено различни елементи на $\{ \alpha^{-1}(L) \mid \alpha \in \Sigma^* \}$.
    Така по \nameref{nerode-nonregular} езикът $L$ не е автоматен.
\end{proof}

\begin{problem}
Да се докаже, че следните езици не са автоматни:
\begin{itemize}
    \item $L_1 = \{ a^nb^{2n} \mid n \in \mathbb{N} \}$
    \item $L_2 = \{ a^{3n}b^n \mid n \in \mathbb{N} \}$
    \item $L_3 = \{ a^{5n}b^{7n} \mid n \in \mathbb{N} \}$
\end{itemize}
Упътване: напълно аналогично на горния пример
\end{problem}

\begin{claim}
    Езикът $L = \{ a^nb^m \mid n \leq m \}$ не е автоматен.
\end{claim}

\begin{proof}
    Отново ще пробваме с думи от вида $a^n$ за $n \in \mathbb{N}$.
    Нека $n, k \in \mathbb{N}, \: n \neq k$ като Б.О.О. $n < k$.
    \begin{itemize}
        \item Понеже $n \leq n, \: a^nb^n \in L$, откъдето $b^n \in (a^n)^{-1}(L)$
        \item Понеже $n < k \iff \neg(k \leq n), \: a^kb^n \notin L$, откъдето $b^n \notin (a^k)^{-1}(L)$
    \end{itemize}
    По \nameref{nerode-nonregular} езикът $L$ не е автоматен.
\end{proof}

\begin{claim}
    Езикът $L = \{ a^{n^2} \mid n \in \mathbb{N} \}$ не е автоматен.
\end{claim}

\begin{proof}
    Отново ще пробваме с думи от вида $a^n$ за $n \in \mathbb{N}$.
    Нека $n, k \in \mathbb{N}, \: n \neq k$ като Б.О.О. $k < n$.
    \begin{itemize}
        \item $a^na^{n^2+n+1} = a^{(n+1)^2} \in L$, откъдето $a^{n^2+n+1} \in (a^n)^{-1}(L)$
        \item $n^2 < n^2+n+k+1 < n^2+2n+1=(n+1)^2$, следователно $a^ka^{n^2+n+1} \notin L$, откъдето $a^{n^2+n+1} \notin (a^k)^{-1}(L)$
    \end{itemize}
    По \nameref{nerode-nonregular} езикът $L$ не е автоматен.
\end{proof}

Това е често срещана техника при езици от вида $\{ a^{f(n)} \mid n \in \mathbb{N} \}$, за някоя $f : \mathbb{N} \rightarrow \mathbb{N}$ строго растяща.
За да се ``изкара'' думата от езика е много удобно да се използва, че ако $f(n) < x < f(n+1)$, то $x$ няма как да е член на редицата.

\begin{problem}\thlabel{nonregular-problems-1}
Да се докаже, че следните езици не са автоматни:
\begin{itemize}
    \item $L_1 = \{ a^{n^3} \mid n \in \mathbb{N} \}$
    \item $L_2 = \{ a^{2^n} \mid n \in \mathbb{N} \}$
    \item $L_3 = \{ a^{n!} \mid n \in \mathbb{N} \}$
\end{itemize}
Упътване: да се използва горепоказаната техника
\end{problem}

\begin{claim}
    Езикът $L = \{ \alpha \alpha^{rev} \mid \alpha \in \Sigma^* \}$ е неавтоматен.
\end{claim}

\begin{proof}
    Тук думи от вида $a^n$ няма да ни свършат работа.
    Изобщо при еднобуквена азбука езикът е регулярен.
    Получават се думи с четна дължина.
    Ще трябва да използваме и други букви за да успеем да изкараме този резултат.
    Нека $n, k \in \mathbb{N}, \: n \neq k$.
    \begin{itemize}
        \item $a^nb^nb^na^n \in L$
        \item $a^kb^nb^na^n \notin L$ (очевидно $a^k \neq a^n$, а $a^kb^t$ за $0 \leq t \leq 2n$ са единствените кандидати за $\alpha$, че да стане $\alpha \alpha^{rev}$)
    \end{itemize}
    По \nameref{nerode-nonregular} езикът $L$ не е автоматен.
\end{proof}

\pagebreak

\begin{problem}\thlabel{nonregular-problems-2}
Да се докаже, че следните езици не са автоматни:
\begin{itemize}
    \item $L_1 = \{ \alpha \alpha \mid \alpha \in \Sigma^* \}$
    \item $L_2 = \{ \alpha \alpha \alpha \mid \alpha \in \Sigma^* \}$
    \item $L_3 = \{ \alpha \alpha^{rev} \alpha \mid \alpha \in \Sigma^* \}$
\end{itemize}
Упътване: напълно аналогично на горния пример
\end{problem}

Нека сега видим какво се случва с различните операции.
Ако $L$ не е автоматен, то и $\overline{L}$ също не е автоматен.
В противен случай $\overline{\overline{L}} = L$ би бил автоматен.
Същите разсъждения можем да направим за $rev$.
Тази затвореност я няма при други операции, които сме разглеждали.
Тук се запазва поради обратимостта на тази операция.

\begin{warning}
    Следните са често срещани грешки:
    \begin{itemize}
        \item \textbf{\textit{\textcolor{red}{конкатенация на два неавтоматни езика винаги ни дава неавтоматен език}}} \\
              Нека вземем за пример някой неавтоматен език $L$.
              $\overline{L}$ също няма да е автоматен.
              Добавяйки крайно много елементи към който и да е неавтоматен език го оставя неавтоматен.
              В противен случай бихме махнали тези крайно много думи (те образуват автоматен език) и ще получим, че първоначалният език е автоматен.
              Въпреки че $(L \cup \{ \varepsilon \})$ и $(\overline{L} \cup \{ \varepsilon \})$ не са автоматни езици,
              конкатенацията им $(L \cup \{ \varepsilon \}) \cdot (\overline{L} \cup \{ \varepsilon \}) = \Sigma^*$ е автоматен език.
              Също така не е вярно, че при конкатенация на неавтоматни се получава винаги автоматен.
              Тривиално се проверява, че $\{ a^nb^n \mid n \in \mathbb{N} \} \cdot \{ a^nb^n \mid n \in \mathbb{N} \} = \{ a^nb^na^mb^m \mid n, m \in \mathbb{N} \}$ не е автоматен.

        \item \textbf{\textit{\textcolor{red}{обединение на два неавтоматни езика винаги ни дава неавтоматен език}}} \\
              Абсолютно същите примери вършат работа и тук.
    \end{itemize}
\end{warning}

\section{Лема за покачването}

Сега ще покажем другия критерий за доказване на нерегулярност.
Той понякога е по-удобен за използване, обаче не е винаги приложим, и е по-вербозен.

\begin{lemma}[Лема за покачването]\thlabel{pumping-lemma}
    Нека $L$ е регулярен език. Тогава:
    \begin{align*}
         & (\exists p \geq 1)                                                             \\
         & (\forall \alpha \in L, \: |\alpha| \geq 1)                                     \\
         & (\exists x, y, z \in \Sigma^*, \: xyz = \alpha, \: |xy| \leq p, \: |y| \geq 1) \\
         & (\forall i \in \mathbb{N}) [xy^iz \in L]
    \end{align*}
\end{lemma}

\begin{proof}
    Езикът $L$ е регулярен.
    Следователно съществува ДКА
    $\mathcal{A} = \opair{Q, \Sigma, s, \delta, F}$,
    такъв че $\mathcal{L}(\mathcal{A}) = L$.

    Полагаме $p = |Q|$ и нека $q_1, \dots, q_p$ са състоянията от $Q$.

    Нека $\alpha \in L$ е такава, че $|\alpha| = n$, където $n \geq p$.
    Ще разбием $\alpha$ на $\alpha_1, \dots, \alpha_n \in \Sigma$ (т.е. $\alpha = \alpha_1\dots\alpha_n$).

    Знаем, че съществуват $i_0, \dots, i_n \in \{1, \dots, n \}$ такива,
    че $s = q_{i_0}$ и за всяко $j \in \{1, \dots, n\} : \delta(q_{i_{j-1}}, \alpha_j) = q_{i_j}$.

    Нека разгледаме думата $\alpha_1 \dots \alpha_p$.
    За нея знаем, че по време на четенето на думата автоматът минава през $p + 1$ състояния.
    Следователно по принципът на Дирихле съществуват
    $t_1, t_2 \in \{1, \dots, p\}$, където $t_1 < t_2$ такива, че $q_{i_{t_1}} = q_{i_{t_2}}$.

    Полагаме $x = \alpha_1 \dots \alpha_{t_1}$, $y = \alpha_{t_1 + 1} \dots \alpha_{t_2}$ и $z = \alpha_{t_2 + 1} \dots \alpha_n$.
    Сигурни сме, че $|xy| \leq p$, защото $t_2 \leq p$ и че $|y| \geq 1$ понеже $t_1 \neq t_2$.
    Знаем, че $\delta^*(q_{i_{t_1}}, y) = q_{i_{t_2}}$.
    Искаме да проверим, че можем да ``циклим'' със $y$.

    \begin{claim}\thlabel{pl-helper}
        $(\forall i \in \mathbb{N}) (\delta^*(q_{i_{t_1}}, y^i) = q_{i_{t_2}})$.
    \end{claim}

    \begin{proof}
        Ще докажем твърдението с индукция по $i \in \mathbb{N}$.

        База: $\delta^*(q_{i_{t_1}}, \epsilon) = q_{i_{t_1}} = q_{i_{t_2}}$ \checkmark

        ИС: $\delta^*(q_{i_{t_1}}, y^{i+1}) = \delta(\delta^*(q_{i_{t_1}}, y^i), y) \overset{\text{ИП}}{=} \delta(q_{i_{t_2}}, y) = \delta(q_{i_{t_1}}, y) = q_{i_{t_2}}$
    \end{proof}

    Знаем, че $\alpha = xyz \in L$.
    Тогава $\delta^*(s, xyz) \in F$.
    От тук следва, че $\delta^*(\delta^*(s, xy), z) \in F$, следователно $\delta^*(\delta^*(\delta^*(s, x), y), z) \in F$.
    Така получаваме, че $\delta^*(s, x) = q_{i_{t_1}}$ и от там $\delta^*(\delta^*(q_{i_{t_1}}, y), z) \in F$.
    От \thref{pl-helper} имаме, че $(\forall i \in \mathbb{N}) (\delta^*(q_{i_{t_1}}, y^i) = q_{i_{t_2}} = q_{i_{t_1}})$.
    Освен това знаем, че $\delta^*(q_{i_{t_2}}, z) \in F$.
    Следователно
    $(\forall i \in \mathbb{N}) (\delta^*(\delta^*(q_{i_{t_1}}, y^i), z) \in F)$.
    От тук можем да заключим, че
    $(\forall i \in \mathbb{N}) (\delta^*(\delta^*(\delta^*(s, x), y^i), z) \in F)$
    и вървейки в обратната посока (``сливането'' на всички $\delta^*$ в едно) получаваме, че
    $(\forall i \in \mathbb{N}) (\delta^*(s, xy^iz) \in F)$,
    с което доказахме лемата.
\end{proof}

Тук отново няма да използваме твърдението в тази форма, а неговата контрапозиция:

\begin{corollary}[Лема за покачването (контрапозиция)]\thlabel{nonreg-pl}
    Ако е изпълнено, че:
    \begin{align*}
         & (\forall p \geq 1)                                                             \\
         & (\exists \alpha \in L, \: |\alpha| \geq 1)                                     \\
         & (\forall x, y, z \in \Sigma^*, \: xyz = \alpha, \: |xy| \leq p, \: |y| \geq 1) \\
         & (\exists i \in \mathbb{N}) [xy^iz \notin L],
    \end{align*}
    то тогава $L$ не е регулярен.
\end{corollary}

Ще покажем и по този начин, че $L = \{ a^nb^n \mid n \in \mathbb{N} \}$ не е регулярен.
Нека $p \geq 1$.
Тогава $\alpha = a^pb^p \in L, \: |a^pb^p| = 2p \geq p$.
Нека вземем $x, y, z \in \Sigma^*, \: xyz = \alpha, \: |xy| \leq n, \: |y| \geq 1$.
Знаем със сигурност, че $y = a^t$ за някое $1 \leq t \leq n$.
За да намерим числото $i \in \mathbb{N}$, което ще ни ``изкара'' думата, трябва да видим как изглежда $xy^iz$:
\begin{center}
    $xy^iz = a^pa^{(i - 1)t}b^p$
\end{center}
Тук всяко $i \neq 1$ ще ни свърши работа.
Нека вземем $i = 0$.
Тогава $p + (i - 1)t = p - t < p$, понеже ($t \geq 1$), откъдето $xy^0z = a^{(p - t)}b^p \notin L$.
Така по \nameref{nonreg-pl} излиза, че езикът не е регулярен.

\begin{claim}
    Езикът $L = \{ a^p \mid p \text{ е просто число} \}$ не е регулярен.
\end{claim}

\begin{proof}
    Нека $n \geq 1$ и $p \geq n$ е просто число.

    Тогава $\alpha = a^p \in L, \: |a^p| \geq n$.
    Нека $x, y, z \in \Sigma^*, \: xyz = \alpha, \: |xy| \leq n, \: |y| \geq 1$.
    Тогава $y = a^t$ за някое $1 \leq t \leq n$.
    Нека видим как изглежда $xy^iz$:
    \begin{center}
        $xy^iz = xyy^{i - 1}z = a^pa^{(i - 1)t}$
    \end{center}
    Искаме $p + (i - 1)t$ да е съставно число.
    $t$ е нещо, което се променя, така че по-скоро бихме се надявали $p$ да е множител.
    Искаме да е множител във $(i - 1)t$. Възможно е $t < p$, така че трябва $p \mid i - 1$.
    Ако положим $i = p + 1$ получаваме, че:
    \begin{center}
        $p + (i - 1)t = p + (p + 1 - 1)t = p + pt = p(1 + t)$
    \end{center}
    Тъй като $t \geq 1, \: t + 1 \geq 2$, и от там $p(1 + t) = p + (i - 1)t$ е съставно число.
    Накрая понеже $xy^{p + 1}z = a^{p(1 + t)} \notin L$, по \nameref{nonreg-pl} излиза, че $L$ не е регулярен.
\end{proof}

\begin{problem}
С лемата за покачването да се докаже, че следните езици не са регулярни:
\begin{itemize}
    \item $L_1 = \{ a^nb^n \mid n \in \mathbb{N} \}$ // $y$ ще хване само букви $a$
    \item $L_2 = \{ a^{n^2} \mid n \in \mathbb{N} \}$
    \item $L_3 = \{ a^{2^n} \mid n \in \mathbb{N} \}$
    \item $L_4 = \{ a^{n!} \mid n \in \mathbb{N} \}$
    \item $L_5 = \{ \alpha \alpha \mid \alpha \in \Sigma^* \}$
    \item $L_6 = \{ \alpha \alpha^{rev} \mid \alpha \in \Sigma^* \}$
\end{itemize}
Упътване: решенията на тази задача са просто адаптация на тези от \thref{nonregular-problems-1}, \thref{nonregular-problems-2} и \thref{nonregular-problems-3}
\end{problem}

От всичкото това доказване на нерегулярност използвайки лемата, човек си задава въпроса дали това не може да стане и в обратната посока.
Оказва се, че не може.

\begin{warning}
    Има \textbf{нерегулярни} езици, за който условието от \nameref{pumping-lemma} \textbf{е изпълнено}.
\end{warning}
За пример нека вземем:
\begin{center}
    $L = (\{ c \}^+ \cdot \{ a^nb^n \mid n \in \mathbb{N} \}) \cup (\{ a \}^* \cdot \{ b \}^*)$
\end{center}
Ако допуснем, че $L$ е регулярен, то тогава очевидно:
\begin{center}
    $L \cap (\{ c \} \cdot \{ a \}^* \cdot \{ b \}^*) = \{ c \} \cdot \{ a^nb^n \mid n \in \mathbb{N} \}$
\end{center}
ще бъде регулярен, но той не е (елементарна проверка, която оставяме на читателя).
Сега да проверим условието от \nameref{pumping-lemma}.
Нека $p = 2$.
Нека $\alpha \in L, \: |\alpha| \geq p$.
Полагаме:
\begin{itemize}
    \item $x = \varepsilon$
    \item $y = \alpha[1]$ (първата буква на $\alpha$)
    \item $z = \alpha[2:|\alpha|]$ (останалите букви на $\alpha$)
\end{itemize}
Ако $\alpha \in \{ c \}^+ \cdot \{ a^nb^n \mid n \in \mathbb{N} \}$, то тогава $y = c$.
За $i = 2$, $xy^iz = c \cdot \alpha \in L$.
Ако пък $\alpha \in \{ a \}^* \cdot \{ b \}^*$, то тогава очевидно $i = 2$ пак ще свърши работа.
\section{Автомат на Brzozowski}

Автоматите, които създаваме, не са единствените за конкретния език.
Най-малкото можем да добавим недостижими състояния.
Но това е глупаво, ние по-скоро искаме да направим автомат с възможно най-малко състояния.
Тогава ако решим да направим програма, тя ще заема по-малко памет.

\begin{definition}
    За език $L \subseteq \Sigma^*$ дефинираме $Q_L = \{ \alpha^{-1}(L) \mid \alpha \in \Sigma^* \}$.
\end{definition}

\begin{claim}
    Ако $Q_L$ е крайно, то $L$ е регулярен.
\end{claim}

\begin{proof}
    Нека $Q_L$ е крайно.

    Строим автомат $\mathcal{B}_L = \opair{\Sigma, Q_L, L, \delta, F}$ за $L$, който ще наричаме \textbf{автомат на Brzozowski} за $L$:

    \begin{itemize}
        \item $\delta(M, x) = x^{-1}(M)$ за $M \in Q_L, \: x \in \Sigma$
        \item $F = \{ M \in Q_L \mid \varepsilon \in M \}$
    \end{itemize}

    \begin{claim}
        За всяко $M \in Q_L \: : \: \delta^*(M, \alpha) = \alpha^{-1}(M)$
    \end{claim}

    \begin{proof}
        С индукция по $|\alpha|$.

        \begin{itemize}
            \item $\delta^*(M, \varepsilon) = M = \varepsilon^{-1}(M)$ \checkmark
            \item $\delta^*(M, \beta x) = \delta(\delta^*(M, \beta), x) \stackrel{\text{ИП}}{=} x^{-1}(\beta^{-1}(M)) = \{ \gamma \in \Sigma^* \mid x \gamma \in \beta^{-1}(M) \} = \{ \gamma \in \Sigma^* \mid \beta x \gamma \in M \} = (\beta x)^{-1}(M)$
        \end{itemize}
    \end{proof}

    Имайки това:
    \begin{align*}
        \alpha \in L & \iff \varepsilon \in \alpha^{-1}(L)      \\
                     & \iff \varepsilon \in \delta^*(L, \alpha) \\
                     & \iff \delta^*(L, \alpha) \in F           \\
                     & \iff \alpha \in \mathcal{L(B)}
    \end{align*}
\end{proof}

Вече сме сигурни, че \nameref{nerode-nonregular} е напълно точен, за разлика от \nameref{pumping-lemma}:

\begin{corollary}
    $L$ е регулярен $\iff Q_L$ е крайно
\end{corollary}

Освен че за всеки регулярен език $L$ може да се построи такъв автомат, той се оказва и минимален.

\begin{claim}[Автоматът на Brzozowski е минимален]\thlabel{brzozowski-minimal-automaton}
    Нека $L$ е регулярен с автомат $\mathcal{A} = \opair{\Sigma, Q, s, \delta, F}$.
    Тогава $|Q_L| \leq |Q|$.
\end{claim}

\begin{proof}
    Б.О.О. всички състояния в $\mathcal{A}$ са достижими.

    Нека $Q = \{ 1, \dots, n \}$.
    Нека $\alpha_i \in \Sigma^*$ е такава дума, че $\delta^*(s, \alpha) = q_i$.
    Дефинираме функцията $f : Q \rightarrow Q_L$:
    \begin{center}
        $f(q_i) = \alpha^{-1}(L)$
    \end{center}
    Трябва само да покажем, че $f$ е сюрективна.

    Нека $M \in Q_L$.
    Тогава има $\alpha \in \Sigma^* \: : \: M = \alpha^{-1}(L)$.
    Знаем, че $\delta^*(s, \alpha) = q_i$ за някое $i \in \{ 1, \dots, n \}$.
    \begin{align*}
        \beta \in M \iff \beta \in \alpha^{-1}(L) & \iff \alpha \beta \in L                                                                                               \\
                                                  & \iff \delta^*(s, \alpha \beta) \in F \iff \delta^*(\delta^*(s, \alpha), \beta) \in F  \iff \delta^*(q_i, \beta) \in F \\
                                                  & \iff \delta^*(\delta^*(s, \alpha_i), \beta) \in F \iff \delta^*(s, \alpha_i \beta) \in F \iff \alpha_i \beta \in L    \\
                                                  & \iff \beta \in \alpha_i^{-1}(L) \iff \beta \in f(q_i)
    \end{align*}
    Така $f(q_i) = M$.
    Накрая понеже $f$ е сюрективна, $\operatorname{Dom}(f) = Q, \: \operatorname{Rng}(f) = Q_L, \: |Q_L| \leq |Q|$.
\end{proof}

\begin{problem}
Да се построи минимален автомат за $\regexlang{a^*b^*}$
\end{problem}

За такива задачи е хубаво да се имат на предвид следните свойства на регулярните изрази:
\begin{itemize}
    \item $\regexlang{r^+} = \regexlang{r \cdot r^*}$
    \item $\regexlang{r^*} = \regexlang{r^+ + \varepsilon}$
    \item $\regexlang{r_1 + r_2} = \regexlang{r_2 + r_1}$
    \item $\regexlang{r(r_1 + r_2)} = \regexlang{r \cdot r_1 + r \cdot r_2}$
\end{itemize}

Започваме да строим автомат на Brzozowski за $\regexlang{a^*b^*}$:
\begin{itemize}
    \item начално състояние $L_0 = \regexlang{a^*b^*} = \regexlang{a^+b^* + b^*} = \regexlang{a \cdot a^*b^* + b^*} = \regexlang{a \cdot a^*b^* + b^+ + \varepsilon} = \regexlang{a \cdot a^*b^* + b \cdot b^* + \varepsilon}$
    \item функцията на преходите е следната:
          \begin{itemize}
              \item $\delta(L_0, a) = a^{-1}(\regexlang{a \cdot a^*b^* + b \cdot b^* + \varepsilon}) = \regexlang{a^*b^*} = L_0$
              \item $\delta(L_0, b) = b^{-1}(\regexlang{a \cdot a^*b^* + b \cdot b^* + \varepsilon}) = \regexlang{b^*} = \regexlang{b \cdot b^* + \varepsilon}= L_1 \neq L_0$, понеже $a \in \regexlang{a^*b^*}, \: a \notin \regexlang{b^*}$
              \item $\delta(L_1, a) = a^{-1}(\regexlang{b \cdot b^* + \varepsilon}) = \varnothing = L_2 \neq L_0, L_1$
              \item $\delta(L_1, b) = b^{-1}(\regexlang{b \cdot b^* + \varepsilon}) = \regexlang{b^*} = L_1$
              \item $\delta(L_2, a) = a^{-1}(\varnothing) = \varnothing = b^{-1}(\varnothing) = \delta(L_2, a)$
          \end{itemize}
    \item $\varepsilon \in L_0, L_1$, следователно те стават финални
\end{itemize}

Ето как ще изглежда автомата на картинка:
\begin{figure}[h]
    \centering
    \begin{tikzpicture}[node distance=2cm, thick]
        \node[accepting, initial, state, initial text=] (1) {$L_0$};
        \node[accepting, state, initial text=, right of=1] (2) {$L_1$};
        \node[state, initial text=, right of=2] (3) {$L_2$};
        \path[->] (1) edge [loop above] node[above] {$a$} (1);
        \path[->] (1) edge [] node[above] {$b$} (2);
        \path[->] (2) edge [loop above] node[above] {$b$} (2);
        \path[->] (2) edge [] node[above] {$a$} (3);
        \path[->] (3) edge [loop above] node[above] {$a, b$} (3);
    \end{tikzpicture}
    \caption*{минимален автомат за $\regexlang{a^*b^*}$}
\end{figure}

Как бихме процедирали ако трябва да се построи минимален автомат за $\Sigma^* \setminus \regexlang{a^*b^*}$?

\begin{warning}
    За всеки регулярен език $L$, ако има автомат за $L$ с $n$ на брой състояния, то тогава има автомат за $\overline{L}$ с $n$ на брой състояния.
\end{warning}

Това го знаем още от \thref{complement-regular}.
В конструкцията там ние използваме същото множество от състояния, същото начално състояния и същите преходи.
Единственото, което променяме, е кои са финалните състояния.
Така можем да заключим, че за да получим минималният автомат за $\Sigma^* \setminus \regexlang{a^*b^*}$, трябва просто да приложим конструкцията от \thref{complement-regular} върху минималният автомат за $\regexlang{a^*b^*}$.
Това е много по-кратко от преминаването към регулярен израз за $\Sigma^* \setminus \regexlang{a^*b^*}$ и прилагането на алгоритъма на Brzozowski.
\section{Задачи за упражнение}

\begin{definition}\thlabel{homomorphism-def}
    Функция $h : \Sigma_1^* \rightarrow \Sigma_2^*$ ще наричаме \textbf{хомоморфизъм}, ако:
    \begin{center}
        $h(\alpha \cdot \beta) = h(\alpha) \cdot h(\beta)$ за всички $\alpha, \beta \in \Sigma_1^*$
    \end{center}
\end{definition}

\begin{problem}
Да се докаже, че за произволен хомоморфизъм $h$ е вярно, че $h(\varepsilon) = \varepsilon$.
\end{problem}

\begin{problem}\thlabel{homomorphism-regular}
Да се докаже, че за произволен хомоморфизъм $h$ и регулярен език $L$, езикът $h[L]$ е регулярен.

Упътване: да се направи индукция по строенето на регулярните езици
\end{problem}

\begin{problem}\thlabel{homomorphism-inverse-regular}
Да се докаже, че за произволен хомоморфизъм $h$ и регулярен език $L$, езикът $h^{-1}[L]$ е регулярен.

Упътване: да се направи автомат, който докато чете $\alpha$, прави преходи с $h(\alpha)$ в автомат за $L$
\end{problem}

\begin{problem}
Използвайки нерегулярността на $\{ a^nb^n \mid n \in \mathbb{N} \}$ заедно със \thref{homomorphism-regular} или \thref{homomorphism-inverse-regular} да се докаже, че езикът $L = \{ a^nb^{2n} \mid n \in \mathbb{N} \}$ е нерегулярен.

Упътване: да се представи $L$ като образ или праобраз на хомоморфизъм
\end{problem}

\begin{problem}
Нека $L$ е регулярен език. Да се докаже, че езикът $L' = \{ \alpha \in \Sigma^* \mid \alpha \alpha \in L \}$ е регулярен.

Упътване: докато се чете $\alpha$ да се види къде отиват всички състояния от автомата за $L$
\end{problem}

\begin{problem}
Нека $L$ е регулярен език. Да се докаже, че езикът $L' = \{ \alpha \# c^n \mid n \in \mathbb{N} \: \& \: \alpha^n \in L \}$ е регулярен.

Упътване: леко усложнение на конструкцията от предната задача
\end{problem}

\begin{problem}
Нека $L$ е регулярен език. Да се докаже, че езикът $L' = \{ \alpha \in \Sigma^* \mid (\exists n \in \mathbb{N}) \: (\alpha^n \in L) \}$ е регулярен.

Упътване: да се използват предната задача и хомоморфизми
\end{problem}

\begin{definition}\thlabel{reg-homomorphism-def}
    Функция $h : \Sigma_1^* \rightarrow \mathcal{P}(\Sigma_2^*)$ ще наричаме \textbf{регулярен хомоморфизъм}, ако:
    \begin{itemize}
        \item $h(x)$ е регулярен за всяко $x \in \Sigma_1$
        \item $h(\alpha \cdot \beta) = h(\alpha) \cdot h(\beta)$ за всички $\alpha, \beta \in \Sigma_1^*$
    \end{itemize}
\end{definition}

\begin{problem}
Да се докаже, че за произволен регулярен хомоморфизъм $h$ е вярно, че $h(\varepsilon) = \{ \varepsilon \}$
\end{problem}

\begin{problem}\thlabel{reg-homomorphism}
Да се докаже, че за произволен регулярен хомоморфизъм $h$ и регулярен език $L$, $\bigcup h[L]$ е регулярен.

Упътване: да се направи индукция по строенето на регулярните езици
\end{problem}

\chapter{Граматики и стекови автомати}

Сега ще покажем едно много по-мощно средство,
което може да разпознава много повече езици като цената, която ще платим,
е да се загубят няколко хубави свойства.

\section{Безконтекстни граматики}

\begin{definition}
    \textbf{Безконтекстна граматика} ще наричаме всяко $G = \opair{\Sigma, V, S, R}$, където:
    \begin{itemize}
        \item $\Sigma$ е крайна азбука
        \item $V$ е крайно множество от променливи
        \item $S \in V$ (ще го наричаме начална променлива)
        \item $R \subseteq V \cross (\Sigma \cup V)^*$ е крайно множество от правила
    \end{itemize}
\end{definition}

\begin{remark}
    Когато $\opair{A, \alpha} \in R$ ще го бележим с $A \rightarrow_G \alpha$.
    Ако граматиката $G$ се подразбира, може да пишем $A \rightarrow \alpha$.
\end{remark}

За правилата $A \rightarrow \alpha$ можем да си мислим, че означават $A$ се замения с $\alpha$.
Искаме да видим какви думи могат да се генерират започвайки от $S$ и следвайки правилата $R$.

\begin{definition}
    За произволна безконтекстна граматика $G = \opair{\Sigma, V, S, R}$ и $l \in \mathbb{N}$ дефинираме релацията $\tri{l}_G \: \subseteq (\Sigma \cup V) \cross (\Sigma \cup V)^*$ индуктивно:
    \begin{itemize}
        \item $X \tri{0}_G X$ за всяко $X \in \Sigma \cup V$
        \item Ако $X \in V, \: X_i \in \Sigma \cup V$, в $G$ имаме правилото $X \rightarrow X_1 \dots X_n$, и $X_i \tri{l_i}_G \alpha_i$ за някои $l_i \in \mathbb{N}$ и $\alpha_i \in (\Sigma \cup V)^*$,
              то $X \tri{l + 1}_G \alpha_1 \dots \alpha_n$, където $l = \max \{ l_1, \dots, l_n \}$. Тук включваме случая с $\varepsilon$.
              Тъй като $\max(\varnothing) = 0$, $X \tri{1}_G \varepsilon$.
    \end{itemize}
    Пишем $X \tri{*}_G \alpha$, ако има $l \in \mathbb{N}$ такова, че $X \tri{l}_G \alpha$.
    Ако $G$ се подразбира, не го пишем.
\end{definition}

\begin{definition}
    Нека $G = \opair{\Sigma, V, S, R}$ безконтекстна граматика и $A \in V$.
    Тогава езикът на променливата $A$ ще бъде $\mathcal{L}_G(A) = \{ \alpha \in \Sigma^* \mid A \tri{*} \alpha \}$ и езикът на граматиката $G$ ще бъде $\mathcal{L}(G) = \mathcal{L}_G(S)$.

    Един език $L$ наричаме \textbf{безконтекстен}, ако съществува безконтекстна граматика $G$ такава, че $\mathcal{L}(G) = L$.
\end{definition}

\begin{remark}
    За да пестим писане, когато имаме правилата $A \rightarrow \alpha_i$ ($i \in \{ 1, \dots, n \}$), ще записваме $A \rightarrow \alpha_1 \mid \dots \mid \alpha_n$.
    Ясно е, че като пишем граматика е напълно достатъчно да опишем правилата и да споменем коя е началната променлива.
    Ако пък има само една променлива, даже и това не е нужно.
\end{remark}

Вече сме готови да дадем първият пример за безконтекстен език.

\begin{claim}\thlabel{canonical-cfl-example}
    Езикът $L = \{ a^nb^n \mid n \in \mathbb{N} \}$ е безконтекстен.
\end{claim}

\begin{proof}
    Граматиката за $L$ е изключително проста (само с 2 правила):
    \begin{center}
        $S \rightarrow aSb \mid \varepsilon$
    \end{center}
    Искаме сега да докажем, че $\mathcal{L}(G) = L$.

    В посоката $\mathcal{L}(G) \subseteq L$ ще трябва да покажем, че ако $S \tri{l} \alpha$ и $\alpha \in \Sigma^*$, то $\alpha \in L$.
    Правим индукция по $l$:
    \begin{itemize}
        \item $S \tri{0} S \notin \Sigma^*$ \checkmark
        \item Нека $S \tri{l + 1} \alpha \in \Sigma^*$.
              Тогава сме приложили някое правило.
              \begin{itemize}
                  \item[1 сл.] Приложили сме правилото $S \rightarrow \varepsilon$.
                        Тогава $\alpha = \varepsilon = a^0b^0 \in L$.
                  \item[2 сл.] Приложили сме правилото $S \rightarrow aSb$.
                        Тогава $S \tri{l} \beta$ за някое $\beta \in \Sigma^*$ и $\alpha = a \beta b$.
                        По ИП $\beta \in L$, откъдето $\beta = a^nb^n$ за някое $n \in \mathbb{N}$.
                        Така $\alpha = a \beta b = a \cdot a^nb^n \cdot b = a^{n+1}b^{n+1} \in L$.
              \end{itemize}
    \end{itemize}

    Така ако $\alpha \in \mathcal{L}(G)$, то $S \tri{*} \alpha$ т.е. $S \tri{l} \alpha$ за някое $l \in \mathbb{N}$, откъдето $\alpha \in L$.

    В посоката $L \subseteq \mathcal{L}(G)$ ще покажем, че за всяко $n \in \mathbb{N}, \: S \tri{n + 1} a^nb^n$.
    Доказваме с индукция по $n$:
    \begin{itemize}
        \item $S \tri{0} S$ и $S \rightarrow \varepsilon$, откъдето $S \tri{1} \varepsilon = a^0b^0$ \checkmark
        \item По ИП $S \tri{n+1} a^nb^n$, но освен това $S \rightarrow aSb$, откъдето $S \tri{n + 2} a \cdot a^nb^n \cdot b = a^{n + 1}b^{n + 1}$.
    \end{itemize}

    Така ако $\alpha \in L$, то понеже $\alpha = a^nb^n$ за някое $n \in \mathbb{N}$, то тогава $\alpha \in \mathcal{L}(G)$.
\end{proof}

\begin{problem}
Да се докаже, че следните езици са безконтекстни:
\begin{itemize}
    \item $L_1 = \{ b^na^n \mid n \in \mathbb{N} \}$
    \item $L_2 = \{ b^{2n}a^n \mid n \in \mathbb{N} \}$
\end{itemize}
Упътване: да се направи дребна модификация на граматиката от \thref{canonical-cfl-example}
\end{problem}

\begin{claim}\thlabel{word-wordrev}
    Езикът $L = \{ \alpha \alpha^{rev} \mid \alpha \in \Sigma^* \}$ е безконтекстен.
\end{claim}

\begin{proof}
    Граматиката е следната:
    \begin{center}
        $S \rightarrow aSa \mid bSb \mid \varepsilon$
    \end{center}

    Ще покажем, че ако $S \tri{l} \beta$ и $\beta \in \Sigma^*$, то $\beta = \alpha \alpha^{rev}$ за някое $\alpha \in \Sigma^*$.
    Правим индукция по $l$:
    \begin{itemize}
        \item $S \tri{0} S \notin \Sigma^*$ \checkmark
        \item Нека $S \tri{l + 1} \beta \in \Sigma^*$.
              Тогава сме приложили някое правило.
              \begin{itemize}
                  \item[1 сл.] Приложили сме правилото $S \rightarrow \varepsilon$.
                        Тогава $\alpha = \varepsilon = \varepsilon \varepsilon^{rev}$.
                  \item[2 сл.] Приложили сме правилото $S \rightarrow aSa$.
                        Тогава $S \tri{l} \gamma$ за някое $\gamma \in \Sigma^*$ и $\beta = a \gamma a$.
                        По ИП има $\alpha \in \Sigma^*$ такова, че $\gamma = \alpha \alpha^{rev}$.
                        Така $\beta = a \gamma a = a \alpha \alpha^{rev} a = a \alpha (a \alpha)^{rev}$.
                  \item[3 сл.] Приложили сме правилото $S \rightarrow bSb$.
                        Той е аналогичен на 2 сл.
              \end{itemize}
    \end{itemize}

    Така ако $\beta \in \mathcal{L}(G), \: S \tri{*} \beta$ т.е. $S \tri{l} \beta$ за някое $l \in \mathbb{N}$,
    следователно $\beta = \alpha \alpha^{rev}$ за някое $\alpha \in \Sigma^*$, откъдето $\beta \in L$.

    Сега ще покажем с индукция по $|\alpha|$, че $S \tri{|\alpha| + 1} \alpha \alpha^{rev}$:
    \begin{itemize}
        \item $S \tri{0} S$ и $S \rightarrow \varepsilon$, откъдето $S \tri{1} \varepsilon = \varepsilon \varepsilon^{rev}$ \checkmark
        \item Нека $\alpha = x \beta$. По ИП $S \tri{|\beta| + 1} \beta \beta^{rev}$, освен това имаме правилото $S \rightarrow xSx$, откъдето $S \tri{|\beta| + 2} x \beta \beta^{rev} x$
    \end{itemize}

    Така ако $\alpha \in L$, то $\alpha = \beta \beta^{rev}$ за някое $\beta \in \Sigma^*$, но за $\beta$ знаем, че $S \tri{*} \beta \beta^{rev} = \alpha$ т.е. $\alpha \in \mathcal{L}(G)$.
\end{proof}

\begin{problem}
Да се докаже, следните езици са безконтекстни:
\begin{itemize}
    \item $L_1 = \{ \alpha \in \Sigma^* \mid \alpha \text{ е палиндром с нечетна дължина} \}$
    \item $L_2 = \{ \alpha \in \Sigma^* \mid \alpha = \alpha^{rev} \}$
\end{itemize}
Упътване: да се модифицира граматиката от \thref{word-wordrev}.
\end{problem}

За да свикнем повече с $\tri{*}$ и за да си улесним някои доказателства, ще покажем, че можем да ``слепваме'' изводи:

\begin{claim}\thlabel{tri-prop}
    Ако $X \tri{l} X_1 \dots X_n$ и $X_i \tri{*} \alpha_i$ за $X_i \in \Sigma \cup V, \: \alpha_i \in (\Sigma \cup V)^*$,
    то $X \tri{*} \alpha_1 \dots \alpha_n$.
\end{claim}

\begin{proof}
    С индукция по $l$.

    \begin{itemize}
        \item Нека $X \tri{0} X$. Ако $X \tri{*} \alpha$, то очевидно $X \tri{*} \alpha$ \checkmark
        \item Нека $X \tri{l + 1} X_1 \dots X_n$. Тогава има $0 = j_0 < j_1 < \dots < j_k = n$ такива, че $X \rightarrow X'_1 \dots X'_k$ и $X'_i \tri{l'_i} X_{j_{i - 1} + 1} \dots X_{j_i}$ като $l = \max \{ l'_1, \dots, l'_n \}$.
              Ако $X_i \tri{l_i} \alpha_i$ за $i \in \{ 1, \dots, n \}$, то по ИП за $t \in \{ 1, \dots, k \}$, $X'_t \tri{l'_t + l'_{t, max}} \alpha_{j_{t - 1} + 1} \dots \alpha_{j_t}$, където $l'_{t, max} = \max \{ l_{j_{t - 1} + 1}, \dots, l_{j_t} \}$.
              Тогава $X \tri{1 + l_{max}} \alpha_1 \dots \alpha_n$, където $l_{max} = \max \{ l'_1 + l'_{1, max}, \dots, l'_k + l'_{k, max} \}$.
    \end{itemize}
\end{proof}

\begin{claim}
    Всеки регулярен език е безконтекстен.
\end{claim}

\begin{proof}
    Ясно е, че $\varnothing, \{ \varepsilon \}, \{ a \}, \{ b \}$ са безконтекстни.
    Остава да покажем, че безконтекстните езици са затворени относно регулярните операции.

    Нека $G_i = \opair{\Sigma, V_i, S_i, R_i}$ са безконтекстни граматики за $i = 1, 2$.
    Нека Б.О.О. $V_1 \cap V_2 = \varnothing$.
    Нека $S \notin V_1 \cup V_2$.
    Следните твърдения са верни:
    \begin{itemize}
        \item $S \rightarrow_G S_1 \mid S_2$ заедно с правилата $R_1, R_2$ ще генерира езикът $\mathcal{L}(G_1) \cup \mathcal{L}(G_2)$. \\
              Ако $S \tri{*}_G \alpha$ и $\alpha \in \Sigma^*$, то тогава очевидно понеже $S \rightarrow_G S_i$ ($i = 1, 2$) са единствените правила, $S_1 \tri{*}_G \alpha$ или $S_2 \tri{*}_G \alpha$.
              Но понеже не добавяме други правила с $V_1$ и $V_2$, тогава $S_1 \tri{*}_{G_1} \alpha$ или $S_2 \tri{*}_{G_2} \alpha$.
              Така $\alpha = \alpha_1 \alpha_2 \in \mathcal{L}(G_1) \cup \mathcal{L}(G_2)$.
              Обратно, ако $\alpha \in \mathcal{L}(G_1) \cup \mathcal{L}(G_2)$ то $S_1 \tri{*}_{G_1} \alpha$ или $S_2 \tri{*}_{G_2} \alpha$, откъдето $S_1 \tri{*}_G \alpha$ или $S_2 \tri{*}_G \alpha$, и понеже $S \rightarrow_G S_1 \mid S_2$, $S \tri{*} \alpha$.
              Получаваме, че $\mathcal{L}(G) = \mathcal{L}(G_1) \cup \mathcal{L}(G_2)$.
        \item $S \rightarrow_G S_1 S_2$ заедно с правилата $R_1, R_2$ ще генерира езикът $\mathcal{L}(G_1) \cdot \mathcal{L}(G_2)$. \\
              Ако $S \tri{*}_G \alpha$ и $\alpha \in \Sigma^*$, то тогава очевидно понеже $S \rightarrow_G S_1 S_2$ е единственото правило, $S_i \tri{*}_G \alpha_i$ ($i = 1, 2$) и $\alpha = \alpha_1 \alpha_2$.
              Но понеже не добавяме други правила с $V_1$ и $V_2$, $S_i \tri{*}_{G_i} \alpha_i$ за $i = 1, 2$.
              Така $\alpha = \alpha_1 \alpha_2 \in \mathcal{L}(G_1) \cdot \mathcal{L}(G_2)$.
              Обратно, ако $\alpha_i \in \mathcal{L}(G_i)$ за $i = 1, 2$, то $S_i \tri{*}_{G_i} \alpha_i$, откъдето понеже $S \rightarrow_G S_1 S_2$, $S \tri{*} \alpha_1 \alpha_2$.
              Получаваме, че $\mathcal{L}(G) = \mathcal{L}(G_1) \cdot \mathcal{L}(G_2)$.
        \item $S \rightarrow_G S S_i \mid \varepsilon$ заедно с правилата $R_i$ ще генерира езикът $\mathcal{L}(G_i)^*$ за $i = 1, 2$. \\
              Тук идеята е подобна на тази при конкатенацията.
              За интуиция доста помага факта, че $L^* = (L^* \cdot L) \cup \{ \varepsilon \}$.
              Пълното доказателство ще оставим за упражнение на читателя.
    \end{itemize}

    Вече имайки, че регулярните операции запазват безконтекстност, индукцията е завършена.
\end{proof}

Вече можем да започнем да си мислим как изглежда множеството от различните видове езици, които познаваме:

\begin{center}
    \begin{tikzpicture}
        \node[above,ellipse,minimum height=2em,minimum width=8.5em,draw] (a) {крайни езици};
        \node[above,ellipse,minimum height=4em,minimum width=12.5em,draw] (b) {};
        \node[above,ellipse,minimum height=6em,minimum width=16.5em,draw] (c) {};
        \node[above,ellipse,minimum height=8em,minimum width=20.5em,draw] (d) {};
        \path (a.north) node[above] {регулярни езици}
        (b.north) node[above] {безконтекстни езици}
        (c.north) node[above] {всички езици};
    \end{tikzpicture}
\end{center}

По нататък ще покажем, че тази картинка не е подвеждаща т.е. има небезконтекстни езици.
\section{Регулярни граматики и еквивалентност със автоматите}

Тук ще направим нещо, което на пръв поглед може да изглежда безсмислено.
Ще покажем частен случай на безконтекстните граматики, които са еквивалентни на автоматите като изразителна мощ.

\begin{definition}
    Една безконтекстна граматика $G = \opair{\Sigma, V, S, R}$ ще наричаме \textbf{регулярна}, ако единствените правила в $G$ са от вида:
    \begin{itemize}
        \item $A \rightarrow aB$ за някои $A, B \in V, \: a \in \Sigma$
        \item $A \rightarrow \varepsilon$ за някое $A \in V$
    \end{itemize}
\end{definition}

Това изглежда много подобно на дефиницията на автомат.
Тук обаче вместо от състояние със буква да отидем в друго състояние, просто заместваме старото състояние с буквата и новото състояние.
В единия случай четем символи, а в другия ги генерираме.
Генерирането на $\varepsilon$ е аналога на това да спрем да четем думата (почти).

\begin{claim}\thlabel{dfa-to-reg-grammar}
    За всеки ДКА $\mathcal{A} = \opair{\Sigma, Q, s, \delta, F}$ има регулярна граматика $G = \opair{\Sigma, V, S, R}$ с $\mathcal{L}(G) = \mathcal{L(A)}$
\end{claim}

\begin{proof}
    Нека $\mathcal{A} = \opair{\Sigma, Q, s, \delta, F}$ е ДКА.
    Нека $Q = \{ q_1, \dots, q_n \}$ като $q_1 = s$.

    Строим граматика $G = \opair{\Sigma, V, S, R}$ за $\mathcal{L(A)}$:
    \begin{itemize}
        \item $V = \{ P_1, \dots, P_n \}$ (на всяко състояние $q_i$ съответства променлива $P_i$)
        \item $S = P_1$ ($q_1 = s$)
        \item Ако $\delta(q_i, x) = q_j$, то $P_i \rightarrow x P_j$
        \item Ако $q_i \in F$, то $P_i \rightarrow \varepsilon$
    \end{itemize}

    Твърдението, което трябва да покажем е, че за всяко $\alpha \in \Sigma^*, \: \delta^*(q_i, \alpha) = q_j \iff P_i \tri{|\alpha|} \alpha P_j$.
    И в двете посоки се прави индукция по дължината на $|\alpha|$, като в посоката $(\Rightarrow)$ се кара по дефиниция на $\delta^*$, а в посоката $(\Leftarrow)$ - на $\tri{*}$.
    Това остава за читателя.
    Остават само довършителни разсъждения:
    \begin{align*}
        \alpha \in \mathcal{L(A)}  \iff       & (\exists i \in \{ 1, \dots, n \}) \: (\delta^*(q_1, \alpha) = q_i \: \& \: q_i \in F)              \\
        \iff                                  & (\exists i \in \{ 1, \dots, n \}) \: (P_1 \tri{*} \alpha P_i \: \& \: P_i \rightarrow \varepsilon) \\
        \iff                                  & (\exists i \in \{ 1, \dots, n \}) \: (P_1 \tri{*} \alpha P_i \: \& \: P_i \tri{1} \varepsilon)     \\
        \stackrel{\textcolor{red}{!!!}}{\iff} & P_1 \tri{*} \alpha                                                                                 \\
        \iff                                  & \alpha \in \mathcal{L}(G)
    \end{align*}
    \begin{warning}[\textcolor{red}{!!!}]
        По принцип разсъждения, подобни на $(\Leftarrow)$ не са верни в общият случай.
        Тук можем да си го позволим сега поради природата на нашата граматика.
        Единственото нещо, което може да правят променливите $P_i$ е да генерират думи от вида $\alpha P_j$.
        Разсъждения, подобни на $(\Rightarrow)$ обаче, ще бъдат верни, поради \thref{tri-prop}.
    \end{warning}
\end{proof}

\begin{claim}
    За всяка регулярна граматика $G = \opair{\Sigma, V, S, R}$ има НКА $\mathcal{N} = \opair{\Sigma, Q, S, \Delta, F}$ с $\mathcal{L(N)} = \mathcal{L}(G)$
\end{claim}

\begin{proof}
    Нека $G = \opair{\Sigma, V, S, R}$ е регулярна граматика.
    Нека $V = \{ A_1, \dots, A_n \}$, като $A_1 = S$.

    Строим НКА $\mathcal{N} = \opair{\Sigma, Q, I, \Delta, F}$ за $\mathcal{L}(G)$:
    \begin{itemize}
        \item $Q = \{ q_1, \dots, q_n \}$
        \item $I = \{ q_ 1 \} $
        \item $\Delta(q_i, x) = \{ q_j \in Q \mid A_i \rightarrow x A_j \}$
        \item $F = \{ q_i \in Q \mid A_i \rightarrow \varepsilon \}$
    \end{itemize}

    Ще формулираме само помощното твърдение и ще оставим всичко останало на читателя:
    \begin{center}
        $q_j \in \Delta^*(q_i, \alpha) \iff A_i \tri{|\alpha|} \alpha A_j$
    \end{center}
\end{proof}

С тези две твърдения не направихме много.
Показахме, че има някакъв вид безконтекстни граматики със изразителната мощ на автоматите.
Това по принцип не е лошо нещо, но самата граматика е доста подобна на автомата, така че не е и като да имаме напълно нов начин за изразяване на регулярни езици.
По ценното тук е да се види самата ``линейност'' на извода на регулярната граматика.
Можем тази техника да я разширим и да я адаптираме за по-сложни задачи.

Нека да си помислим, какво ще стане ако приложим конструкцията от \thref{dfa-to-reg-grammar}, но вместо $P_i \rightarrow x P_j$ имаме $P_j \rightarrow P_j x$.
Ще генерираме думите на $\mathcal{L(A)}$, но в обратен ред т.е. $\mathcal{L}(G) = \mathcal{L(A)}^{rev}$.

\begin{problem}
Да се опише подробно конструкцията за $\mathcal{L(A)}^{rev}$.
\end{problem}

Какво пък ще стане, ако решим да сложим $P_i \rightarrow x P_j x$ вместо $P_i \rightarrow x P_j$.
Ще генерираме от ляво дума от $\mathcal{L(A)}$, а от дясно същата дума но в обратен ред.
Тогава $\mathcal{L}(G) = \{ \alpha \alpha^{rev} \mid \alpha \in \mathcal{L(A)} \}$

\begin{problem}
Да се опише подробно конструкцията за $\{ \alpha \alpha^{rev} \mid \alpha \in \mathcal{L(A)} \}$.
\end{problem}

Можем и да станем по креативни:
\begin{claim}\thlabel{move-operation}
    Нека $L$ е регулярен.
    Тогава езикът $\operatorname{Move}(L) = \{ a^nb^m \mid (\exists \alpha \in L) \: (|\alpha|_a = n \: \& \: |\alpha|_b = m) \}$ е безконтекстен.
\end{claim}

\begin{proof}
    Ще покажем само конструкцията и ще формулираме помощно твърдение; другото е за читателя.
    Нека $\mathcal{A} = \opair{\Sigma, Q, s, \delta, F}$ е ДКА.
    Нека $Q = \{ q_1, \dots, q_n \}$ като $q_1 = s$.

    Строим граматика $G = \opair{\Sigma, V, S, R}$ за $\operatorname{Move}(L)$:
    \begin{itemize}
        \item $V = \{ P_1, \dots, P_n \}$ (на всяко състояние $q_i$ съответства променлива $P_i$)
        \item $S = P_1$ ($q_1 = s$)
        \item Ако $\delta(q_i, a) = q_j$, то $P_i \rightarrow a P_j$ (слагаме буквите $a$ от ляво)
        \item Ако $\delta(q_i, b) = q_j$, то $P_i \rightarrow P_j b$ (слагаме буквите $b$ от дясно)
        \item Ако $q_i \in F$, то $P_i \rightarrow \varepsilon$
    \end{itemize}

    Трябва да се покаже, че:
    \begin{center}
        $\delta^*(q_i, \alpha) = q_j \iff P_i \tri{n + m} a^{n} P_j b^{m}$, където $n = |\alpha|_a$ и $m = |\alpha|_b$
    \end{center}
\end{proof}

\begin{problem}
Нека $L$ е регулярен.
Да се покаже, че е безконтекстен езикът:
\begin{center}
    $\operatorname{AltMove}(L) = \{ a^nb^m \mid (\exists \alpha \in L) \: (|\alpha|_a = m \: \& \: |\alpha|_b = n) \}$
\end{center}
Упътване: да се направи дребна модификация на конструкцията от \thref{move-operation}
\end{problem}

\begin{problem}
Нека $L$ е регулярен и нека $L_1, L_2$ са безконтекстни.
Да се докаже, че са безконтекстни следните езици:
\begin{itemize}
    \item $L' = \{ \alpha_1 \dots \alpha_n \beta_1 \dots \beta_m \mid (\exists \alpha \in L) \: (|\alpha|_a = n \: \& \: |\alpha|_b = m) \: \& \: \alpha_i \in L_1 \: \& \: \beta_i \in L_2 \}$
    \item $L'' = \{ \alpha_1 \dots \alpha_{2n} \beta_1 \dots \beta_{3m} \mid (\exists \alpha \in L) \: (|\alpha|_a = n \: \& \: |\alpha|_b = m) \: \& \: \alpha_i \in L_1 \: \& \: \beta_i \in L_2 \}$
    \item $L''' = \{ \beta_1 \dots \beta_m \alpha_1 \dots \alpha_n \mid (\exists \alpha \in L) \: (|\alpha|_a = n \: \& \: |\alpha|_b = m) \: \& \: \alpha_i \in L_1 \: \& \: \beta_i \in L_2 \}$
\end{itemize}
Упътване: да се адаптират техники от предишни задачи
\end{problem}
\section{Дървета на извод}

\begin{definition}
    Нека $\Sigma$ е крайна азбука и нека $T \subseteq \Sigma^*$ е краен език.
    $T$ ще наричаме \textbf{дърво над $\Sigma$}, ако $\operatorname{Pref}(T) = T$.
\end{definition}

Това на пръв поглед изглежда доста странна дефиниция на дърво.
Ще се опитаме да си дадем интуиция за нея.
Можем да си мислим за всяка дума като за връх, а ребрата са между върхове от вида $\alpha$ и $\alpha x$, където $\alpha \in \Sigma^*, \: x \in \Sigma$:

\begin{center}
    \begin{forest}
        [$\varepsilon$ [$0$ [$00$] [$01$]] [$1$ [$10$] [$11$]]]
    \end{forest}
\end{center}

Тук $T = \Sigma^{\leq 2}$.
Всяка дума $\alpha \in T$ е връх на $T$.
Има ребро между $1$ и $10$ защото са на ``еднобуквена конкатенация разстояние''.
Понеже не може да се каже същото за двойки върхове като $\opair{\varepsilon, 11}$ или $\opair{11, 00}$, между тях ребра няма.

Ще бъде много удобно да въведем някои познати термини за дървета:

\begin{definition}
    Нека $T \subseteq \Sigma^*$ е дърво.
    Дефинираме следните понятия:
    \begin{itemize}
        \item \textbf{височината на $T$} ще наричаме $\h{T} = \max \{ |\alpha| \mid \alpha \in T \}$
        \item \textbf{децата на връх $\alpha \in T$} ще наричаме $\chld{\alpha} = (\{ \alpha \} \cdot \Sigma) \cap T$
        \item \textbf{листата на $T$} ще бележим с $\lv{T} = \{ \alpha \in T \mid \chld{\alpha} = \varnothing \}$
    \end{itemize}
\end{definition}

\begin{remark}
    За следващата дефиниция ще приемаме, че зад всяка крайна азбука $\Sigma = \{ x_1, \dots, x_n \}$ стои линейна наредба,
    за да можем да говорим за лексикографска наредба на думите.
    Нея ще бележим с $<$.
\end{remark}

\begin{definition}
    Нека $G = \opair{\Sigma, V, S, R}$ е безконтекстна граматика.
    Нека $T \subseteq \Sigma^*$ е дърво и нека $\lambda : T \rightarrow (\Sigma_{\varepsilon} \cup V)$.
    Ще наричаме $\mathcal{T} = \opair{T, \lambda}$ \textbf{дърво на извод, съгласувано с $G$}, ако:
    \begin{itemize}
        \item $\lambda(\varepsilon) \in \Sigma \cup V$
        \item ако $\chld{\alpha} = \{ \alpha x_1, \dots, \alpha x_k \}$ като $x_1 < \dots < x_k$, то трябва да имаме правилото $\lambda(\alpha) \rightarrow_G \lambda(\alpha x_1) \dots \lambda(\alpha x_k)$ като $\lambda(\alpha x_i) \neq \varepsilon$
        \item ако $\chld{\alpha} = \{ \alpha x \}$ и имаме правилото $\lambda(\alpha) \rightarrow_G \varepsilon$, то може $\lambda(\alpha x) = \varepsilon$
    \end{itemize}
    Освен това:
    \begin{itemize}
        \item $\rt{\mathcal{T}} = \lambda(\varepsilon)$ - това е \textbf{коренът на $\mathcal{T}$}
        \item $\w{\mathcal{T}} = \alpha_1 \dots \alpha_n$, където $\lv{T} = \{ \alpha_1, \dots, \alpha_n \}$ и $\alpha_1 < \dots < \alpha_n$
    \end{itemize}
\end{definition}

Можем да си мислим, че множеството $T$ задава ``формата'' на дървото,
а функцията $\lambda$ слага като ``етикет'' променлива,
буква или $\varepsilon$, като всичко това е съгласувано с правилата в $G$.

\begin{remark}
    Ще изневерим на нашата нотация.
    Вместо да казваме:
    \begin{center}
        \textit{``Нека $\mathcal{T} = \opair{T, \lambda}$ е дърво на извод...''}
    \end{center}
    за по-кратко ще казваме:
    \begin{center}
        \textit{``Нека $T$ е дърво на извод...''}
    \end{center}
    Под $\w{T}$ ще имаме предвид $\w{\mathcal{T}}$, и под $\rt{T}$ ще имаме предвид $\rt{\mathcal{T}}$.
\end{remark}

Нека вземем граматиката с правилата $S \rightarrow aSb \mid \varepsilon$.
Примерно дърво на извод би било:

\begin{center}
    \begin{forest}
        [$S$ [$a$] [$S$ [$a$] [$S$ [$\varepsilon$]] [$b$]] [$b$]]
        \node[text width=3.0cm] at (0, 0) {$T:$};
    \end{forest}
\end{center}

За дървото на извод $T$ имаме, че:
\begin{itemize}
    \item $\w{T} = aabb$
    \item $\h{T} = 3$
    \item $\rt{T} = S$.
\end{itemize}

\begin{warning}
    Ако караме по дефиницията това, което наистина се случва отзад, е малко по-различно:
    \begin{center}
        \begin{forest}
            [, phantom, s sep = 4cm
                    [$\varepsilon$ [$0$] [$1$ [$10$] [$11$ [$110$]] [$12$]] [$2$]]
                    [$S$ [$a$] [$S$ [$a$] [$S$ [$\varepsilon$]] [$b$]] [$b$]]]
            \node[text width=3.0cm] at (-3, -1) {$T:$};
            \node[text width=3.0cm] at (3, -1) {$\mathcal{T}:$};
            \node[draw=none] (tree form) at (-2, -2) {};
            \node[draw=none] (labeled tree) at (2, -2) {};
            \path[->, dotted] (tree form) edge [bend left] node[above] {$\lambda$}(labeled tree);
        \end{forest}
    \end{center}
    Както споменахме преди малко, $T$ всъщност е само формата, а вече функцията $\lambda$ дава семантиката на дървото.
\end{warning}

\begin{claim}\thlabel{tri-der-equiv}
    Нека $G = \opair{\Sigma, V, S, R}$ е безконтекстна граматика и нека $X \in \Sigma \cup V$.
    Тогава:
    \begin{center}
        $X \tri{l} \alpha \iff$ има дърво на извод $T$ такова, че $\rt{T} = X, \: \h{T} = l$ и $\w{T} = \alpha$
    \end{center}
\end{claim}

\begin{proof}
    И двете твърдения доказваме с индукция.

    \begin{itemize}
        \item[$(\Rightarrow)$] С индукция по $l$:
            \begin{itemize}
                \item $X \tri{0} X$. Oчевидно има дърво на извод с единствен връх $X$ \checkmark
                \item $X \tri{l + 1} \alpha$.
                      Тогава $X \rightarrow_G X_1 \dots X_k, \: X_1 \tri{l_1} \alpha_1, \dots, X_k \tri{l_k} \alpha_k$ като $\alpha = \alpha_1 \dots \alpha_k$ и $l = \max \{ l_1, \dots, l_k \}$.
                      По ИП има дървета на извод $T_1, \dots, T_k$ такива, че $\rt{T_i} = X_i, \: \h{T_i} = l_i$ и $\w{T_i} = \alpha_i$.
                      Можем да построим дърво на извод $T$ с корен $T$, като неговите преки наследници са корените на дърветата $T_1, \dots, T_k$:
                      \begin{center}
                          \begin{forest}
                              [$X$ [$X_1$, for children = {l=3cm} [\wraphspace{$\alpha_1$}{1em},roof]] [$\dots$, for children = {l=3cm} [\wraphspace{$\alpha_2 \dots \alpha_{k - 1}$}{3em},roof]] [$X_k$, for children = {l=3cm} [\wraphspace{$\alpha_k$}{1em},roof]]]
                              \node[text width=3.0cm] at (-1, 0) {$T:$};
                              \node[text width=3.0cm] at (-1.6, -2.75) {$T_1$};
                              \node[text width=3.0cm] at (0.55, -2.75) {$T_2, \dots, T_{k - 1}$};
                              \node[text width=3.0cm] at (4.25, -2.75) {$T_k$};
                          \end{forest}
                      \end{center}
                      За $Т$ знаем, че $\rt{T} = X, \: \h{T} = 1 + \max \{ \h{T_1}, \dots, \h{T_k} \} = 1 + \max \{ l_1, \dots, l_k \} = l + 1$ и $\w{T} = \alpha_1 \dots \alpha_k = \alpha$.
            \end{itemize}
        \item[$(\Leftarrow)$] С индукция по $\h{T}$:
            \begin{itemize}
                \item Единственото дърво $T$ с $\rt{T} = X$ и $\h{T} = 0$ е дървото само с връх $X$, но $X \tri{0} X = \w{T}$ \checkmark
                \item Ако $T$ е дърво с корен $X$ и единствен наследник $\varepsilon$, то тогава $\h{T} = 1$ и има правилото $X \rightarrow_G \varepsilon$.
                      Понеже $X \tri{0} X$, $X \tri{1} \varepsilon = \w{T}$.
                      Ако пък $T$ е дърво с корен $X$ и преки наследници $X_1, \dots, X_k$, като ще наричаме поддърветата вкоренени в тях $T_1, \dots, T_k$.
                      Нека $\w{T} = \alpha, \: \w{T_i} = \alpha_i, \: \h{T_i} = l_i$.
                      Ясно е, че $\alpha = \alpha_1 \dots \alpha_k$ и $\h{T} = 1 + \max \{ l_1, \dots l_k \}$.
                      По ИП $X_i \tri{l_i} \alpha_i$.
                      Тъй като $X_1, \dots, X_k$ са преки наследници на корена $X$ на $T$, то тогава има правило $X \rightarrow X_1 \dots X_k$, откъдето $X \tri{l + 1} \alpha_1 \dots \alpha_k = \alpha$.
            \end{itemize}
    \end{itemize}
\end{proof}

С това доказахме, че $\mathcal{L}(G) = \{ \alpha \in \Sigma^* \mid \text{има дърво на извод } T \text{ съглавувано с } G \text{ и } \rt{T} = S, \: \w{T} = \alpha \}$ за всяка безконтекстна граматика $G = \opair{\Sigma, V, S, R}$.
Вече вместо за $\tri{*}$ да си мислим просто като за една релация, можем да си мислим за дървета на извод.
Зад нея вече стои някаква ``по-смислена'' семантика.

\begin{problem}
Нека граматиката $G = \opair{\Sigma, \{ S, A, B, C \}, S, R}$ се определя чрез следните правила:
\begin{align*}
     & S \rightarrow A \mid BA \mid BCAC \mid CC   \\
     & A \rightarrow AaB \mid bbC \mid aCa \mid a  \\
     & B \rightarrow baA \mid ACA \mid \varepsilon \\
     & C \rightarrow AbCb \mid a \mid b
\end{align*}
Да се построят дървета на извод за думите $\alpha_1 = aaabbb, \alpha_2 = baaabbbaa, \alpha_3 = aabbbaabbbaabbb$.

Упътване: за $\alpha_1$ има дърво на извод с корен $C$, за $\alpha_2$ с корен $B$ и за $\alpha_3$ с корен $S$.
\end{problem}
\section{Небезконтекстни езици}

Сега ще покажем съществуването на небезконтекстни езици.
За целта ще разгледаме безконтекстната версия на \nameref{pumping-lemma}:

\begin{lemma}[Лема за покачването]\thlabel{pumping-lemma-cf}
    Нека $L$ е безконтекстен език. Тогава:
    \begin{align*}
         & (\exists p \geq 1)                                                                       \\
         & (\forall \alpha \in L, \: |\alpha| \geq p)                                               \\
         & (\exists x, y, u, v, w \in \Sigma^*, \: xyuvw = \alpha, \: |yuv| \leq p, \: |yv| \geq 1) \\
         & (\forall i \in \mathbb{N}) [xy^iuv^iw \in L]
    \end{align*}
\end{lemma}

\begin{proof}
    Нека $L$ е безконтекстен език.
    Нека $G = \opair{\Sigma, V, S, R}$ е безконтекстна граматика за $L$.

    Полагаме $b = \max \{ |\gamma| \mid A \rightarrow_G \gamma \text{ е правило в } G \}$ и $p = b^{|V| + 1}$.
    За $b$ можем да си мислим, че това е най-голямата разклоненост на дърво на извод.
    Няма как да има връх в дърво на извод с повече от $b$ брой преки наследници.
    Тогава едно дърво на извод в $G$ с височина $h$ не може да има дума с дължина повече от $b^h$.
    Така ако $\alpha \in L$ и $|\alpha| \geq p$, в което и да е дърво на извод $T$ такова, че $\w{T} = \alpha$, имаме $\h{T} \geq |V|$.
    Тогава в пътя на едно такова дърво $T$ от корена до някои листа ще има повторение на променливи.
    Нека $T$ е дърво на извод за $\alpha$ с минимален брой елементи.
    Знаем, че има повторение на някаква променлива $A \in V$.
    Нека това повторение е възможно най-надолу.
    Със сигурност това ще бъде измежду последните $|V|$ върха в някои път между корена и листата.
    На картинка имаме следното:
    \begin{center}
        \begin{forest}
            [$S$ [[\wraphspace{$x$}{1em},roof]] [$A$ [[\wraphspace{$y$}{1em}, roof]] [$A$ [[\wraphspace{$u$}{1em}, roof]]] [[\wraphspace{$v$}{1em}, roof]]] [[\wraphspace{$w$}{1em},roof]]]
            \node[text width=3.0cm] at (0, 0) {$T:$};
        \end{forest}
    \end{center}
    Понеже срещанията на а се случват достатъчно надолу, $|yuv| \leq p$.
    Понеже дървото $T$ е с минимален брой върхове $|yv| \geq 1$.
    Ако това не беше вярно, $yv = \varepsilon$ и можехме да премахнем първото срещане на $A$, с което строим по-малко дърво на извод.
    Остава само да проверим, че $xy^iuv^iw \in L$ за всяко $i \in \mathbb{N} \: (\star)$.
    От \thref{tri-der-equiv}, $A \tri{*} yAv$.
    Ще покажем с индукция по $i$ че $A \tri{*} y^iAv^i$ за всяко $i \in \mathbb{N}$.
    \begin{itemize}
        \item $A \tri{0} A = y^0Av^0$ \checkmark
        \item Нека $A \tri{*} y^iAv^i$, тогава понеже $A \tri{*} yAv$, $A \tri{*} y^iyAvv^i = y^{i + 1}Av^{i + 1}$ (от \thref{tri-prop})
    \end{itemize}
    От \thref{tri-der-equiv}, $A \tri{*} u$, откъдето използвайки $(\star)$ получаваме, че $A \tri{*} y^iuv^i$ за всяко $i \in \mathbb{N}$.
    Също така $S \tri{*} xAw$, откъдето $S \tri{*} xy^iuv^i$ за всяко $i \in \mathbb{N}$, с което сме готови.
\end{proof}

Разбира се, ние ще използваме по често следствието от лемата.

\begin{corollary}[Лема за покачването (контрапозиция)]\thlabel{non-cf}
    Ако е изпълнено, че:
    \begin{align*}
         & (\forall p \geq 1)                                                                       \\
         & (\exists \alpha \in L, \: |\alpha| \geq p)                                               \\
         & (\forall x, y, u, v, w \in \Sigma^*, \: xyuvw = \alpha, \: |yuv| \leq p, \: |yv| \geq 1) \\
         & (\exists i \in \mathbb{N}) [xy^iuv^iw \notin L]
    \end{align*}
    то тогава $L$ не е безконтекстен.
\end{corollary}

Започваме с каноничния пример за небезконтекстен език:
\begin{claim}\thlabel{canonical-non-cf}
    Езикът $L = \{ a^nb^nc^n \mid n \in \mathbb{N} \}$ не е безконтекстен.
\end{claim}

\begin{proof}
    Нека $p \geq 1$.
    Нека $\alpha = a^pb^pc^p$.
    Ясно е, че $\alpha \in L$ и $|\alpha| \geq p$.
    Нека $x, y, u, v, w \in \Sigma^*$ са такива, че $\: xyuvw = \alpha, \: |yuv| \leq p, \: |yv| \geq 1$.
    Очевидно няма как $yv = a^nc^m$ за $n, m > 0$.
    Тогава имаме няколко варианта за $yv$:
    \begin{itemize}
        \item[1 сл.] $yv = a^t$ за някое $1 \leq t \leq p$.
            Тогава $xy^0uv^0w = xuw = a^{p - t}b^pc^p \notin L$ ($p - t < p$)
        \item[2 сл.] $yv = b^t$ за някое $1 \leq t \leq p$ - аналогично на 1 сл.
        \item[3 сл.] $yv = c^t$ за някое $1 \leq t \leq p$ - аналогично на 1 сл.
        \item[4 сл.] $yv = a^{t_1}b^{t_2}$ за някое $1 \leq t_1, t_2 \leq p, \: t_1 + t_2 \leq p$.
            Тогава $xy^0uv^0w = xuw = a^{p - t}b^{p - t}c^p \notin L$ ($p - t < p$)
        \item[5 сл.] $yv = b^{t_1}c^{t_2}$ за някое $1 \leq t_1, t_2 \leq p, \: t_1 + t_2 \leq p$ - аналогично на 4 сл.
    \end{itemize}
    От \nameref{non-cf} следва, че $L$ не е безконтекстен.
\end{proof}

\begin{problem}\thlabel{pl-attention-example}
Да се докаже, че следните езици не са безконтекстна:
\begin{itemize}
    \item $L_1 = \{ a^nb^{2n}c^n \mid n \in \mathbb{N} \}$
    \item $L_2 = \{ a^nb^{3n}c^{5n} \mid n \in \mathbb{N} \}$
    \item $L_3 = \{ a^nb^mc^n \mid n \leq m \}$
    \item $L_4 = \{ a^nb^mc^k \mid n \leq m \leq k \}$
\end{itemize}
Упътване: за $L_1$ и $L_2$ да се адаптира решението от \thref{canonical-non-cf}, а за $L_3$ и $L_4$ тряба да се използват думи от същия вид.
\end{problem}

\begin{warning}
    За тази версия на лемата, е много по-важен избора на дума.
    Тук трябва да се стремим думата да е възможно най-близко до това да излезне от езика.
    Да вземем за пример $L_3$ от \thref{pl-attention-example}.
    Ако вземем дума от вида $a^pb^{p+t}c^p$, за $p, t \geq 1$, то ако $yv = b$, думата няма да излезе от $L_3$ каквото и да правим.
    Подобни разсъждения могат да се направят и за $L_4$.
    Когато става въпрос за небезконтекстност, трябва да сме по-нежни и по-деликатни.
\end{warning}

\begin{problem}
Да се докаже, че следните езици не са безконтекстни:
\begin{itemize}
    \item $L_1 = \{ \alpha \alpha \mid \alpha \in \Sigma^* \}$
    \item $L_2 = \{ \alpha \alpha \alpha \mid \alpha \in \Sigma^* \}$
    \item $L_3 = \{ \alpha \alpha^{rev} \alpha \mid \alpha \in \Sigma^* \}$
\end{itemize}
\end{problem}

\begin{claim}
    Операцията сечение и допълнение не запазват безкотекстност
\end{claim}

\begin{proof}
    За сечението елементарен контрапример:

    \begin{center}
        $\underbrace{\{ a^nb^nc^k \mid n, k \in \mathbb{N} \}}_{L_1 - \text{безконтекстен}} \cap \underbrace{\{ a^nb^kc^k \mid n, k \in \mathbb{N} \}}_{L_2 - \text{безконтекстен}} = \underbrace{\{ a^nb^nc^n \mid n \in \mathbb{N} \}}_{L - \text{небезконтекстен}}$
    \end{center}

    Ако допуснем, че безконтекстните езици са затворени отностно допълнение, то тогава понеже $L_1$ и $L_2$ са безконтекстни, $\overline{L_1} \cup \overline{L_2}$ също е безконтекстен (обединението запазва безконтекстност).
    Можем да приложим допълнение още веднъж и ще получим отново безконтекстен език, откъдето $\overline{\overline{L_1} \cup \overline{L_2}} = L_1 \cap L_2 = L$ е безконтекстен, което е противоречие.
\end{proof}

Тук се вижда как въпреки че разпознаваме видове езици, губим някои свойства.
Освен, че губим операции като допълнение и сечение, ние също така губим константната памет, която имахме при автоматите.

\begin{warning}
    Въпреки, че в общия случай не можем да приемем, че допълнението на безконтекстен език е също е безконтекстен език, понякога това е вярно.
    Очевидно е вярно за регулярните езици, но не само и аз тях.
    Има безконтекстни езици, които не са регулярни, чието допълнение е безконтекстен език:
\end{warning}

\begin{claim}
    Езикът $L =\Sigma^* \setminus \{ a^nb^n \mid n \in \mathbb{N} \}$ е безконтекстен.
\end{claim}

\begin{proof}
    Само ще покажем представяне на езика, което очевидно е безконтекстно.
    Подробностите оставяме на читателя.
    Нека помислим какво би означавало $\alpha \in L$.
    Има два варианта:
    \begin{itemize}
        \item[1 сл.] $\alpha = a^nb^k$ и $n \neq k$.
            Това би означавало, че $\alpha \in \underbrace{\{ a^nb^k \mid n > k \}}_{L_1} \cup \underbrace{\{ a^nb^k \mid n < k \}}_{L_2}$.
        \item[2 сл.] $\alpha$ изобщо не е от $\{ a \}^* \{ b \}^*$.
            За да не е вярно това някъде в думата има буква $b$ преди буква $a$, откъдето $\alpha \in \underbrace{\Sigma^* \cdot \{ b \} \cdot \Sigma^* \cdot \{ a \} \cdot \Sigma^*}_{L_3}$.
    \end{itemize}
    Така можем да представим $L$ като обединение на безконтекстни езици:
    \begin{center}
        $L = \underbrace{(\{ a \}^+ \cdot \{ a^nb^n \mid n \in \mathbb{N} \})}_{L_1 - \text{безконтекстен}} \cup \underbrace{(\{ a^nb^n \mid n \in \mathbb{N} \} \cdot \{ b \}^+)}_{L_2 - \text{безконтекстен}} \cup \underbrace{(\Sigma^* \cdot \{ b \} \cdot \Sigma^* \cdot \{ a \} \cdot \Sigma^*)}_{L_3 - \text{безконтекстен}}$
    \end{center}
\end{proof}
\section{Нормална форма на Чомски}

Тук ще покажем, че всеки безконтекстен език с точност до липсата на $\varepsilon$ може да се представи чрез граматика във ``хубав'' вид.
За целта постъпково ще си ``опростяваме'' нашата граматика т.е. ще гледаме да променим свойствата на граматика запазвайки езика (с точност до липсата на $\varepsilon$).

\subsection*{Премахване на дългите правила}

Под дълго правило имаме предвид $X \rightarrow X_1 \dots X_k$, където $X_i \in \Sigma \cup V$ и $k \geq 3$.
За да премахнем едно такова правило просто добававяме променливите $Y_1, \dots, Y_{k - 2}$ и заменяме горното правило с:

\begin{center}
    $X \rightarrow X_1 Y_1, \: Y_1 \rightarrow X_2 Y_2, \: \dots, \: Y_{k - 2} \rightarrow X_{k - 1} X_k$

\end{center}
Прилагайки тази конструкция за всички правила получаваме граматика със същия език, която има само правила от вида $A \rightarrow \beta$, където $|\beta| \leq 2$.

\subsection*{Премахване не $\varepsilon$ правилата}

Целта ни ще бъде да премахнем всички правила от вида $X \rightarrow \varepsilon$ като запазим езика (без $\varepsilon$).
Ще направим рекурсия:

\begin{itemize}
    \item $\operatorname{Eps}(0) = \varnothing$
    \item $\operatorname{Eps}(n+1) = \{ A \in V \mid (\exists \beta \in \operatorname{Eps}(n)^*) \: (A \rightarrow \beta) \}$

\end{itemize}

Първо в $\operatorname{Eps}(1)$ ще бъдат променливите, които генерират $\varepsilon$, после в $\operatorname{Eps}(2)$ ще бъдат променливите, които генерират тях и т.н.
Лесно може да се покаже, че:

\begin{center}
    $\operatorname{Eps}(n) = \{ A \in V \mid A \tri{\leq n} \varepsilon \}$
\end{center}

С $\operatorname{Eps}$ ще бележим най-малката неподвижна точка на оператора със същото име.
За сега обаче не сме направили нищо.
Трябва да променим правилата.
Ако има правило $X \rightarrow X_1 \dots X_k$, при условието че $X_1 \dots X_k \neq \varepsilon$, добавяме всички правила от вида $X \rightarrow Y_1 \dots Y_k$, където:

\begin{itemize}
    \item ако $X_i \notin \operatorname{Eps}$, то $Y_i = X_i$
    \item ако $X_i \in \operatorname{Eps}$, то $Y_i = X_i$ или $Y_i = \varepsilon$
\end{itemize}

Очевидно е, че в новата граматика няма да има $\varepsilon$ правила.

\subsection*{Премахване на преименуващите правила}

Целта ни тук ще бъде да премахнем правила от вида $A \rightarrow B$.
Обаче тогава трябва да видим какви думи генерира $B$ и по някакъв начин да ги добавим към тези, които генерира $A$.
Отново правим рекурсия:

\begin{itemize}
    \item $\operatorname{Rename}(0) = V \cross V$
    \item $\operatorname{Rename}(n + 1) = \operatorname{Rename}(n) \cup \{ \opair{A, C} \mid (\exists B \in V) \: (A \rightarrow B \: \& \: \opair{B, C} \in \operatorname{Rename}(n)) \}$
\end{itemize}

Цялата идея е да се види колко далече може да стигне една променлива с преименуване.
Лесно може да се съобрази, че:

\begin{center}
    $\operatorname{Rename}(n) = \{ \opair{A, B} \mid A \tri{\leq n} B \}$
\end{center}

С $\operatorname{Rename}$ ще бележим най-малката неподвижна точка на оператора със същото име.
Вече можем да кажем кои ще са новите правила в граматиката:

\begin{center}
    $R_{no rename} = \{ \opair{A, \beta} \in V \cross (\Sigma \cup V)^* \mid (\exists B \in V) \: (\underbrace{\opair{A, B} \in \operatorname{Rename}}_{A \text{ може да се замени с } B} \& \underbrace{\opair{B, \alpha} \in R \setminus (V \cross V)}_{B \rightarrow \alpha \text{ не е преименуващо правило}}) \}$
\end{center}

\subsection*{Премахване на правилата, които генерират повече от 1 буква}

Ако имаме правило от вида $X \rightarrow x_1 x_2$, където $x_1, x_2 \in \Sigma \cup V$.
За всяко $x_i \in \Sigma$ можем да добавим нова променлива $X_i$ с правилото $X_i \rightarrow x_i$ и да заменим в предното правило $x_i$ със $X_i$.
Например за правилото $A \rightarrow bc$ можем да добавим нови променливи $B$ и $C$, заменяйки старото правило с правилата:

\begin{center}
    $A \rightarrow BC, \: B \rightarrow b, \: C \rightarrow c$
\end{center}

\begin{definition}
    Една безконтекстна граматика $G = \opair{\Sigma, V, S, R}$ е в \textbf{нормална форма на Чомски} (накратко НФЧ), ако всичките и правила са от вида:

    \begin{itemize}
        \item $A \rightarrow BC$ за някои $A, B, C \in V$
        \item $A \rightarrow a$ за някои $A \in V, \: a \in \Sigma$
    \end{itemize}
\end{definition}

Прилагайки тези алгоритми в тази последователност, получаваме при вход безконтекстна граматика $G$ получаваме граматика $G'$ в НФЧ с $\mathcal{L}(G') = \mathcal{L}(G) \setminus \{ \varepsilon \}$.

\begin{remark}
    Ако искаме да добавим $\varepsilon$ в езика можем да го направим много лесно.
    Добавяме нова променлива $S_0$ и правилата $S_0 \rightarrow \varepsilon$ заедно с $S_0 \rightarrow \alpha$ за всяко правило $S \rightarrow \alpha$.
    Тогава правилата ще бъдат от вида:

    \begin{itemize}
        \item $S \rightarrow \varepsilon$
        \item $A \rightarrow BC$ за някои $A, B, C \in V$ като $B, C \neq S$
        \item $A \rightarrow a$ за някои $A \in V, \: a \in \Sigma$
    \end{itemize}

    За простота няма да се занимаваме с този вид граматика.
    Ще приемем, че случаите, в които участва $\varepsilon$ са тривиални са разглеждане, и няма да се занимваме с тях.
\end{remark}

\begin{claim}
    За всеки безконтекстен език $L$, езикът $sub(L)$ (от \thref{sub-lang-def}) също е безконтекстен.
\end{claim}

\begin{proof}
    Само ще покажем конструкцията и откъде идва удобството от НФЧ при конструкции, като пълното доказателство остава за читателя.

    Ще е хубаво да имаме предвид някои хубави свойства на поддумите:

    \begin{itemize}
        \item ако $\alpha$ е поддума на $\beta$ можем да си мислим как ``задраскваме'' някои букви от $\beta$ и получаваме $\alpha$
        \item $sub(L_1 \cdot L_2) = sub(L_1) \cdot sub(L_2)$
    \end{itemize}

    А имайки граматика в НФЧ можем да разсъждаваме така:

    \begin{itemize}
        \item ако си мислим изолирано за правилата от вида $A \rightarrow a$ можем много лесно да направим ``задраскването''
        \item за правилата от вида $A \rightarrow BC$ можем да си мислим, че към $\mathcal{L}_G(A)$ се добавя $\mathcal{L}_G(B) \cdot \mathcal{L}_G(C)$
    \end{itemize}

    Сега нека опишем вече конструкцията.
    Нека $G = \opair{\Sigma, V, S, R}$ е граматика в НФЧ за $L$.
    В новата граматика $G_{sub}$, която строим ще имаме същите променливи, същата начална променлива, като правилата са следните:

    \begin{itemize}
        \item имаме същите правила от $G$
        \item ако $A \rightarrow a$ е правило в $G$, то добавяме правилото $A \rightarrow \varepsilon$ (тук става ``задраскването'')
    \end{itemize}

    Сега да опишем на картинка какво ще направим.
    Да кажем, че генерираме думата $a_1 a_2 a_3 a_4$ в оригиналния език.
    Искаме в новия език да можем да генерираме думата $a_1 a_4$:

    \begin{center}
        \begin{forest}
            [$S$ [$A$ [$C$ [$a_1$]] [$D$ [$\cancel{a_2}$ \\ $\varepsilon$, align=center]]] [$B$ [$E$ [$\cancel{a_3}$ \\ $\varepsilon$, align=center]] [$F$ [$a_4$]]]]
        \end{forest}
    \end{center}

    Понеже сме имали правилото $D \rightarrow a_2$ сме добавили правилото $D \rightarrow \varepsilon$, същото за $E$ и $a_3$.
    Това ни позволи да премахнем тези букви.

    Ако искаме да сме подробни, можем да покажем, че за всички $A \in V, \: \alpha \in \Sigma$ и $n \in \mathbb{N}$:

    \begin{itemize}
        \item ако $A \tri{n}_G \alpha$, то $A \tri{n}_{G_{sub}} \beta$ за всяка поддума $\beta$ на $\alpha$
        \item ако $A \tri{n}_{G_{sub}} \alpha$, то има $\beta$ такава, че $A \tri{n}_G \beta$ и $\alpha$ е поддума на $\beta$.
    \end{itemize}
\end{proof}

\begin{claim}
    За всеки безконтекстен език $L$ езикът $L^{rev}$ е безконтекстен.
\end{claim}

\begin{proof}
    Нека $G = \opair{\Sigma, V, S, R}$ е граматика в НФЧ за $L$.
    В новата граматика $G_{rev}$ за $L^{rev}$ имаме същите променливи и начална променлива.
    Новите правила са следните:

    \begin{itemize}
        \item ако $A \rightarrow_G a$, то тогава $A \rightarrow_{G_{rev}} a$
        \item ако $A \rightarrow_G BC$, то тогава $A \rightarrow_{G_{rev}} CB$
    \end{itemize}

    Използвайки, че $(\beta_1 \beta_2)^{rev} = \beta_2^{rev} \beta_1^{rev}$, лесно се показва, че за всички $A \in V, \: \alpha \in \Sigma$ и $n \in \mathbb{N}$

    \begin{center}
        $A \tri{n}_G \alpha \iff A \tri{n}_{G_{rev}} \alpha^{rev}$
    \end{center}

    И двете посоки са аналогични, за това ще направим само едната:

    \begin{itemize}
        \item ако $A \tri{0}_G \alpha$, то $\alpha = A \notin \Sigma^*$ \checkmark
        \item ако $A \tri{n + 1}_G \alpha$, то сме приложили правило:
              \begin{itemize}
                  \item[1 сл.] приложили сме правило от вида $A \rightarrow_G a$: \\
                      Тогава $\alpha = a, \: n = 0$ и $A \rightarrow_{G_rev} a$, откъдето $A \tri{1}_{G_{rev}} a = \alpha$
                  \item[2 сл.] приложили сме правило от вида $A \rightarrow_G BC$: \\
                      Тогава $B \tri{n_1}_G \alpha_1, \: C \tri{n_2}_G \alpha_2$, като $\alpha = \alpha_1 \alpha_2$ и $n = \max \{ n_1, n_2 \}$.
                      Тогава по ИП, $B \tri{n_1}_{G_{rev}} \alpha_1^{rev}$ и $C \tri{n_2}_{G_{rev}} \alpha_2^{rev}$.
                      Също така има правилото $A \rightarrow_{G_{rev}} CB$, откъдето $A \tri{n + 1} \alpha_2^{rev} \alpha_1^{rev} = (\alpha_1 \alpha_2)^{rev}$.
              \end{itemize}
    \end{itemize}

    Показвайки твърдението получаваме, че $\mathcal{L}_{G_{rev}}(A) = \mathcal{L}_G(A)^{rev}$ за всяка променлива $A$, в частност:

    \begin{center}
        $\mathcal{L}(G_{rev}) = \mathcal{L}_{G_{rev}}(S) = \mathcal{L}_G(S)^{rev} = \mathcal{L}(G) = L^{rev}$
    \end{center}
\end{proof}

Хубавото на такива конструкции е, че промените, които правим не са големи.
Тук дърветата на извод имат същата ``форма''.
Разбира се, при сложни примери ще трябва малко повече да си поиграем,
но много помага да си мислим за това което бихме искали да направим,
как би станало върху дървото на извод и дали тази информация може хубаво да се кодира в правилата.

\section{Няколко интересни задачи}

Тук ще разгледаме няколко хубави задачи за безконтекстни граматики.
Първата ще има малко по-нестандартно решение.

\begin{claim}\thlabel{equal-a-b-cfg}
    Езикът $L = \{ \alpha \in \Sigma^* \mid |\alpha|_a = |\alpha|_b \}$е безконтекстен.
\end{claim}

\begin{proof}
    Граматиката за $L$ е следната:

    \begin{center}
        $S \rightarrow aSb \mid bSa \mid SS \mid \varepsilon$
    \end{center}

    Сега нека покажем, че $\mathcal{L}(G) = L$:

    \begin{itemize}
        \item[$(\subseteq)$] За тази посока ще покажем с индукция по $n \in \mathbb{N}$, че ако $S \tri{n} \alpha$ и $\alpha \in \Sigma^*$, то $\alpha \in L$
            \begin{itemize}
                \item ако $S \tri{0} \alpha$, то $\alpha = S \notin \Sigma^*$ \checkmark
                \item ако $S \tri{n + 1} \alpha$, тогава сме приложили някое правило
                      \begin{itemize}
                          \item[1 сл.] Приложили сме правилото $S \rightarrow \varepsilon$.
                              Тогава $\alpha = \varepsilon \in L$
                          \item[2 сл.] Приложили сме правилото $S \rightarrow aSb$.
                              Тогава $S \tri{n} \beta$, като $\alpha = a \beta b$.
                              Лесно се вижда, че имаме равенствата $|\alpha|_a = 1 + |\beta|_a \stackrel{\text{ИП}}{=} 1 + |\beta|_b = |\alpha|_b$, откъдето $\alpha \in L$
                          \item[3 сл.] Приложили сме правилото $S \rightarrow bSa$.
                              Този случай е аналогичен на 2 сл.
                          \item[4 сл.] Приложили сме правилото $S \rightarrow SS$.
                              Тогава $S \tri{n_1} \beta_1$ и $S \tri{n_2} \beta_2$, като $n = \max \{ n_1, n_2 \}$ и $\alpha = \beta_1 \beta_2$.
                              Така $|\alpha|_a = |\beta_1|_a + |\beta_2|_a \stackrel{\text{ИП}}{=} |\beta_1|_b + |\beta_2|_b = |\alpha|_b$, откъдето $\alpha \in L$
                      \end{itemize}
            \end{itemize}
        \item[$(\supseteq)$]  За тази посока ще покажем с индукция по $|\alpha|$, че ако $\alpha \in L$, то $S \tri{*} \alpha$
            \begin{itemize}
                \item $S \rightarrow \varepsilon$ е правило в $G$, и $S \tri{0} S$, откъдето $S \tri{1} \varepsilon$ \checkmark
                \item Нека $|\alpha| = 2n + 2$ (дължината очевидно трябва да е четна).
                      Ако $\alpha = a \beta b$ за някое $\beta \in \Sigma^*$, то понеже $|\beta|_a = |\alpha|_a - 1 = |\alpha|_b - 1 = |\beta|_b$, по ИП $S \tri{*} \beta$.
                      Но в $G$ има правилото $S \rightarrow aSb$, откъдето $S \tri{*} \alpha$.
                      Случаят, в който $\alpha = b \beta a$ е аналогичен.
                      Интересните случаи са, когато $\alpha = x \beta x$ за $x \in \Sigma$.
                      И двата са симетрични, така че ще разгледаме само когато $\alpha = a \beta a$.
                      Нека $\alpha = \alpha_1 \dots \alpha_{2n + 2}$, като $\alpha_i \in \Sigma$.
                      Нека $f_\alpha(i) = |\alpha_1 \dots \alpha_i|_a - |\alpha_1 \dots \alpha_i|_b$.
                      Например ако $\beta = aabb$, то $f_\beta(0) = 0, f_\beta(1) = 1, f_\beta(2) = 2, f_\beta(3) = 1$ и $f_\beta(4) = 0$.
                      Ясно е, че понеже $\alpha \in L$, $f_\alpha(|\alpha|) = 0$. Понеже $\alpha_1 = a$, $f_\alpha(1) = 1$, и понеже $\alpha_{2n + 2} = a$, $f_\alpha(2n + 1) = -1$.
                      Това, което можем да направим, е да продължим $f_\alpha$ до непрекъсната $g_\alpha : [0, 2n + 2] \rightarrow \mathbb{R}$ по следния начин:
                      \begin{itemize}
                          \item $g_\alpha(i) = f_\alpha(i)$ за всяко $i \in \{ 0, \dots, 2n + 2 \}$
                          \item $g_\alpha(i + \varepsilon) = (1 - \varepsilon) f_\alpha(i) + \varepsilon f_\alpha(i + 1)$ за всяко $i \in { 0, \dots, 2n + 1}$ и $\varepsilon \in (0, 1)$
                      \end{itemize}
                      Очевидно $g_\alpha$ е непрекъсната в реалните (и не целите) си точки.
                      Нека проверим, че е така и в целите:
                      \begin{itemize}
                          \item $\lim\limits_{\varepsilon \rightarrow 0} g_\alpha(i + \varepsilon) = \lim\limits_{\varepsilon \rightarrow 0} [(1 - \varepsilon) f_\alpha(i) + \varepsilon f_\alpha(i + 1)] = f_\alpha(i) = g_\alpha(i)$
                          \item $\lim\limits_{\varepsilon \rightarrow 1} g_\alpha(i + \varepsilon) = \lim\limits_{\varepsilon \rightarrow 1} [(1 - \varepsilon) f_\alpha(i) + \varepsilon f_\alpha(i + 1)] = f_\alpha(i + 1) = g_\alpha(i + 1)$
                      \end{itemize}
                      Имаме непрекъсната функция $g_\alpha$, за която знаем, че $g_\alpha(1) = 1$ и $g_\alpha(2n + 1) = -1$.
                      Тогава от Теоремата на Болцано, има $i \in (1, 2n + 1)$ такъв, че $g_\alpha(i) = 0$.
                      Нещо повече, конструирали сме $g$ така, че да връща цели стойности само когато и подаваме цели стойности.
                      За това $i \in \{ 2, \dots, 2n \}$.
                      Това означава, че $g_\alpha(i) = f_\alpha(i) = |\alpha_1 \dots \alpha_i|_a - |\alpha_1 \dots \alpha_i|_b = 0$, откъдето $\beta_1 = \alpha_1 \dots \alpha_i \in L$.
                      Е тогава очевидно също и $\beta_2 = \alpha_{i + 1} \dots \alpha_{2n + 1} \in L$.
                      По ИП $S \tri{*} \beta_1$ и $S \tri{*} \beta_2$.
                      Така понеже имаме правилото $S \rightarrow SS$, $S \tri{*} \beta_1 \beta_2 = \alpha$.
            \end{itemize}
    \end{itemize}
\end{proof}

\begin{definition}\thlabel{bal-par-def}
    Ще наричаме една дума $\alpha$ \textbf{добре скобуван относно $[$ и $]$}, ако за всяко $\beta \preceq_{pref} \alpha$, $|\beta|_[ \geq |\beta|_]$ и $|\alpha|_[ = |\alpha|_]$
    Ще бележим този факт с $bal(\alpha, [, ])$.
\end{definition}

Пример за добра скобуван израз относно $($ и $)$ би бил думата (((23+4)$\times$(21-12))+6)/(73-23).
Илюстрирано по-добре:

\begin{center}
    $\underset{1}{\textcolor{red}{(}}$
    $\underset{2}{\textcolor{blue}{(}}$
    $\underset{3}{\textcolor{yellow}{(}}$
    23
    +
    4
    $\underset{2}{\textcolor{yellow}{)}}$
    $\times$
    $\underset{3}{\textcolor{yellow}{(}}$
    21
    -
    12
    $\underset{2}{\textcolor{yellow}{)}}$
    $\underset{1}{\textcolor{blue}{)}}$
    +
    6
    $\underset{0}{\textcolor{red}{)}}$
    /
    $\underset{1}{\textcolor{red}{(}}$
    73
    -
    23
    $\underset{0}{\textcolor{red}{)}}$
\end{center}

Тук много лесно се вижда, че никога броят на $($ не надвишава броя на $)$.

\begin{problem}
Да се построи граматика за езика $L = \{ \alpha \in \{ [, ] \} \mid bal(a, [, ]) \}$

Упътване: много подобно на \thref{equal-a-b-cfg}
\end{problem}

\pagebreak

\begin{claim}
    Ако $L$ е безконтекстен, тогава $\operatorname{Pref}(L)$ също е безконтекстен.
\end{claim}

\begin{proof}
    Нека $G = \opair{\Sigma, V, S, R}$ е граматика в НФЧ без безполезни променливи за $L$.
    Под безполезни променливи се има предвид такива, които не генерират нищо от $\Sigma^*$.

    Строим граматика $G_{pref} = \opair{\Sigma, V_{pref}, \overleftarrow{S}, R_{pref}}$ за $\operatorname{Pref}(L)$:

    \begin{itemize}
        \item $V_{pref} = V \cup \overleftarrow{V}$, където $\overleftarrow{V} = \{ \overleftarrow{A} \mid A \in V \}$ и $V \cap \overleftarrow{V} = \varnothing$
        \item правилата от старата граматика се запазват
        \item ако $A \rightarrow_G BC$, то $\overleftarrow{A} \rightarrow_{G_{pref}} B \overleftarrow{C} \mid \overleftarrow{B}$
        \item ако $A \rightarrow_G a$, то $\overleftarrow{A} \rightarrow_{G_{pref}} a \mid \varepsilon$
    \end{itemize}

    Цялата идея на конструкцията е, че $\operatorname{Pref}(L_1 \cdot L_2) = \operatorname{Pref}(L_1) \cup (L_1 \cdot \operatorname{Pref}(L_2))$ (от \nameref{prefix-suffix-infix-props}).

    За да покажем, че $\operatorname{Pref}(L) \subseteq \mathcal{L}(G_{pref})$, ще докажем, че за всяко $A \in V, \: \alpha \in \Sigma^*$:

    \begin{center}
        ако $A \tri{*}_G \alpha$, то $\overrightarrow{A} \tri{*}_{G_{pref}} \beta$ за всяко $\beta \preceq_{pref} \alpha$ т.е. $\operatorname{Pref}(\mathcal{L}_G(A)) \subseteq \mathcal{L}_{G_{pref}}(\overrightarrow{A})$
    \end{center}

    Доказваме с индукция по големината на извода:

    \begin{itemize}
        \item ако $A \tri{0} \alpha$, то $\alpha = A \notin \Sigma^*$ \checkmark
        \item ако $A \tri{n + 1} \alpha$, то сме приложили някое правило
              \begin{itemize}
                  \item[1 сл.] Приложили сме правилото $A \rightarrow_G BC$.
                      Тогава $B \tri{n_1} \beta_1$ и $C \tri{n_1} \beta_2$, като $n = \max \{ n_1, n_2 \}$ и $\alpha = \beta_1 \beta_2$.
                      Нека $\gamma \preceq_{pref} \alpha$.
                      Тогава има два варианта (от \nameref{prefix-suffix-infix-props}):
                      \begin{itemize}
                          \item $\gamma \preceq_{pref} \beta_1$ - тогава по ИП $\overrightarrow{B} \tri{*} \gamma$, и понеже имаме правилото $\overrightarrow{A} \rightarrow \overrightarrow{B}$, $\overrightarrow{A} \tri{*} \gamma$.
                          \item $\gamma = \beta_1 \lambda$ и $\lambda \preceq_{pref} \beta_2$ - тогава по ИП $\overrightarrow{C} \tri{*} \lambda$, и понеже имаме правилото $\overrightarrow{A} \rightarrow B \overrightarrow{C}$, $\overrightarrow{A} \tri{*} \beta_1 \lambda = \gamma$.
                      \end{itemize}
                  \item[2 сл.] Приложили сме правилото $A \rightarrow_G a$.
                      Тогава $\alpha = a$ и всичките префикси на $\alpha$ са $a$ и $\varepsilon$.
                      Тъй като $A \rightarrow_G a$, $\overrightarrow{A} \rightarrow_{G_{pref}} a \mid \varepsilon$, откъдето $\overrightarrow{A} \tri{*} a$ и $\overrightarrow{A} \tri{*} \varepsilon$
              \end{itemize}
    \end{itemize}

    За да покажем, че $\mathcal{L}(G_{pref}) \subseteq \operatorname{Pref}(L)$, ще докажем, че за всяко $A \in V, \: \alpha \in \Sigma^*$:

    \begin{center}
        ако $\overrightarrow{A} \tri{*}_{G_{pref}} \alpha$, то има $\beta \in \Sigma^*$, че $A \tri{*}_G \alpha \beta$ т.е. $\mathcal{L}_{G_{pref}}(\overrightarrow{A}) \subseteq \operatorname{Pref}(\mathcal{L}_G(A))$
    \end{center}

    Доказваме с индукция по големината на извода:

    \begin{itemize}
        \item ако $\overrightarrow{A} \tri{0} \alpha$, то $\alpha = \overrightarrow{A} \notin \Sigma^*$ \checkmark
        \item ако $\overrightarrow{A} \tri{n + 1} \alpha$, то сме приложили някое правило
              \begin{itemize}
                  \item[1 сл.] Приложили сме правилото $\overrightarrow{A} \rightarrow_{G_{pref}} B \overrightarrow{C}$.
                      Тогава $B \tri{n_1}_{G_{pref}} \beta_1$ $\overrightarrow{C}_{G_{pref}} \tri{n_2} \beta_2$ като $n = \max \{ n_1, n_2 \}$ и $\alpha = \beta_1 \beta_2$.
                      По ИП има $\lambda \in \Sigma^*$ такова, че $C \tri{*}_G \beta_2 \lambda$.
                      От правилото, което сме приложили знаем, че $A \rightarrow_G BC$, откъдето $A \tri{*} \beta_1 \beta_2 \lambda = \alpha \lambda$.
                  \item[2 сл.] Приложили сме правилото $\overrightarrow{A} \rightarrow_G \overrightarrow{B}$.
                      Тогава $\overrightarrow{B} \tri{n}_{G_{pref}} \alpha$ и по ИП има $\beta_1 \in \Sigma^*$ такова, че $B \tri{*}_G \alpha \beta_1$.
                      От правилото, което сме приложили знаем, че $A \rightarrow BC$.
                      Тъй като $C$ не е безполезна променлива, има $\beta_2 \in \Sigma^*$ такова, че $C \tri{*}_G \beta_2$.
                      Тогава $A \tri{*}_G \alpha \beta_1 \beta_2$.
                  \item[3 сл.] Приложили сме правилото $\overrightarrow{A} \rightarrow_{G_{pref}} a$.
                      Тогава $A \rightarrow_G a$, откъдето $A \tri{*}_G a = a \varepsilon$.
                  \item[4 сл.] Приложили сме правилото $\overrightarrow{A} \rightarrow_{G_{pref}} \varepsilon$.
                      Тогава $A \rightarrow_G a$, откъдето $A \tri{*}_G a = \varepsilon a$.
              \end{itemize}
    \end{itemize}

    Накрая завършваме с:

    \begin{center}
        $\operatorname{Pref}(L) = \operatorname{Pref}(\mathcal{L}(G)) = \operatorname{Pref}(\mathcal{L}_G(S)) = \mathcal{L}_{G_{pref}}(\overrightarrow{S}) = \mathcal{L}(G_{pref})$
    \end{center}
\end{proof}

\begin{problem}
Да се покаже, че ако $L$ е безконтекстен, то тогава $\operatorname{Suff}(L)$ и $\operatorname{Infix}(L)$ също са безконтекстни.

Упътване: конструкция за $\operatorname{Suff}$ е много подобна, а тази за $\operatorname{Infix}$ е комбинация от двете; може и да се използват \nameref{prefix-suffix-infix-props}
\end{problem}
\section{Сечение на безконтекстен език с регулярен}

Сега ще разгледаме една много хубава операция, с едно нейно хубаво приложение.
Въпреки че не можем да сечем два безконтекстни езика с надеждите винаги да получим безконтекстен език, можем да заслабим малко това условие.

\begin{claim}
    За всеки безконтекстен език $L$ и регулярен език $M$, езикът $L \cap M$ е безконтекстен.
\end{claim}

\begin{proof}
    Нека $G = \opair{\Sigma, V, R, S}$ е граматика в НФЧ за $L$ и нека $\mathcal{A} = \opair{\Sigma, Q, s, \delta, F}$ е ДКА за $M$.
    Да си представим някаква дума $\alpha = a_1 a_2 a_3 a_4 \in L \cap M$.
    Някои неща които знаем за нея:

    \begin{itemize}
        \item има дърво на извод за $\alpha$, съгласувано с $G$:
              \begin{center}
                  \begin{forest}
                      [$S$ [$A$ [$X$ [$a_1$]] [$Y$ [$a_2$]]] [$B$ [$Z$ [$a_3$]] [$W$ [$a_4$]]]]
                  \end{forest}
              \end{center}
        \item в $\mathcal{A}$ има път с етикет $\alpha$ от $s$ до някое $f \in F$ т.е. $\delta^*(s, \alpha) = f \in F$:
              \begin{center}
                  \begin{tikzpicture}[shorten >=1pt,node distance=2.5cm,>=stealth',thick]
                      \node[initial, state, initial text=] (1) {$s$};
                      \node[state] [right of=1] (2) {$p_1$};
                      \node[state] [right of=2] (3) {$p_2$};
                      \node[state] [right of=3] (4) {$p_3$};
                      \node[state, accepting] [right of=4] (5) {$f$};
                      \path[->] (1) edge [] node[above] {$a_1$} (2);
                      \path[->] (2) edge [] node[above] {$a_2$} (3);
                      \path[->] (3) edge [] node[above] {$a_3$} (4);
                      \path[->] (4) edge [] node[above] {$a_4$} (5);
                  \end{tikzpicture}
              \end{center}
    \end{itemize}

    Това, което ще направим, е да се опитаме да кодираме информацията за автомата в дърветата на извод:

    \begin{center}
        \begin{forest}
            [$\opair{s, S, f}$ [$\opair{s, A, p_2}$ [$\opair{s, X, p_1}$ [$a_1$]] [$\opair{p_1, Y, p_2}$ [$a_2$]]] [$\opair{p_2, B, f}$ [$\opair{p_2, Z, p_3}$ [$a_3$]] [$\opair{p_3, W, f}$ [$a_4$]]]]
        \end{forest}
    \end{center}

    Новите променливи в граматиката ще са от вида $\opair{p, A, q}$ като идеята е да се генерират всички думи $\alpha$ такива, че:

    \begin{center}
        $A \tri{*} \alpha$ и $\delta^*(p, \alpha) = q$
    \end{center}

    Нека опишем сега формално конструкцията за новата ни граматика $G' = \opair{\Sigma, V', S', R'}$:

    \begin{itemize}
        \item $V' = (Q \cross V \cross Q) \cup \{ S' \}$, където $S' \notin Q \cross V \cross Q$
        \item за всички $A \in V$ и $p, q, r \in Q$, ако $A \rightarrow_G BC$, то $\opair{p, A, q} \rightarrow_{G'} \opair{p, B, r} \opair{r, C, q}$ \\
              Тук граматиката се опитва да ``познае'' думата, която генерира какъв път ще измине в автомата.
              Предварително тя няма как да го знае, тъй като самата граматика не знае каква дума ще генерира.
        \item за всички $A \in V, \: p, q \in Q$ и $a \in \Sigma$, ако $A \rightarrow_G a$, то $\opair{p, A, q} \rightarrow_{G'} a$ \\
              Тук вече става истинската ``синхронизация''.
              Ако граматиката е познала преходите на думата, която е решила да генерира, то тя вече наистина ще я генерира.
              Ако не е познала, понеже не са правилни или няма такива, то няма да се генерира нищо.
        \item за всяко $f \in F$, $S' \rightarrow_{G'} \opair{s, S, f}$
    \end{itemize}

    Сега трябва да покажем, че за всички $A \in V$ и $p, q \in Q$:

    \begin{center}
        $\mathcal{L}_{G'}(\opair{p, A, q}) = \mathcal{L}_G(A) \cap \{ \alpha \in \Sigma^* \mid \delta^*(p, \alpha) = q \}$
    \end{center}

    Този надпис хубаво илюстрира как разделяме голямото сечение на $L$ и $M$ на по-малки сечения, което е идеята на конструкцията.

    \begin{claim}
        За всички $A \in V, \: p, q \in Q, \: n \in \mathbb{N}$ и $\alpha \in \Sigma^*$:

        \begin{center}
            $\opair{p, A, q} \tri{n}_{G'} \alpha \iff A \tri{n}_G \alpha \: \& \: \delta^*(p, \alpha) = q$
        \end{center}
    \end{claim}

    \begin{proof} С индукция по $n \in \mathbb{N}$.

        \begin{itemize}
            \item базата е тривиално изпълнена и в двете посоки \checkmark
            \item ще покажем ИС и в двете посоки:

                  \begin{itemize}
                      \item[($\Rightarrow$)] Нека $\opair{p, A, q} \tri{n + 1}_{G'} \alpha$.
                          Тогава сме приложили някое правило.
                          \begin{itemize}
                              \item[1 сл.] Приложили сме правилото $\opair{p, A, q} \rightarrow_{G'} a$.
                                  Тогава $\alpha = a$, $\delta(p, a) = q$ и $A \rightarrow_G a$, откъдето $A \tri{1} \alpha$.
                              \item[2 сл.] Приложили сме правилото $\opair{p, A, q} \rightarrow_{G'} \opair{p, B, r} \opair{r, C, q}$.
                                  Тогава съществуват $\alpha_1, \alpha_2 \in \Sigma^*$ такива, че $\opair{p, B, r} \tri{n_1}_{G'} \alpha_1$ и $\opair{r, C, q} \tri{n_2}_{G'} \alpha_2$ като $\alpha = \alpha_1 \alpha_2$ и $n = \max \{ n_1, n_2 \}$.
                                  Също така $A \rightarrow BC$.

                                  По ИП:

                                  \begin{itemize}
                                      \item $B \tri{n_1}_G \alpha_1$ и $\delta^*(p, \alpha_1) = r$
                                      \item $C \tri{n_2}_G \alpha_2$  и $\delta^*(r, \alpha_2) = q$
                                  \end{itemize}

                                  Така можем да заключим, че $A \tri{n + 1}_G \alpha_1 \alpha_2 = \alpha$ и $\delta^*(p, \alpha) = \delta^*(p, \alpha_1 \alpha_2) = \delta^*(\delta^*(p, \alpha_1), \alpha_2) = q$
                          \end{itemize}
                      \item[($\Leftarrow$)] Нека $A \tri{n + 1}_G \alpha$ и $\delta^*(p, \alpha) = q$.
                          Тогава сме приложили някое правило.
                          \begin{itemize}
                              \item[1 сл.] Приложили сме правилото $A \rightarrow_G a$.
                                  Тогава $\alpha = a$ и понеже $\delta(p, a) = q$, $\opair{p, A, q} \rightarrow_{G'} a$, откъдето получаваме, че $\opair{p, A, q} \tri{1} \alpha$.
                              \item[2 сл.] Приложили сме правилото $A \rightarrow BC$.
                                  Тогава съществуват $\alpha_1, \alpha_2 \in \Sigma^*$ такива, че $B \tri{n_1}_G \alpha_1$ и $C \tri{n_2}_G \alpha_2$ като $\alpha = \alpha_1 \alpha_2$ и $n = \max \{ n_1, n_2 \}$.
                                  Нека $r = \delta^*(p, \alpha_1)$.
                                  Очевидно $\delta^*(r, \alpha_2) = q$.
                                  Също така имаме правилото $\opair{p, A, q} \rightarrow_{G'} \opair{p, B, r} \opair{r, C, q}$.

                                  По ИП:

                                  \begin{itemize}
                                      \item $\opair{p, B, r} \tri{n_1} \alpha_1$
                                      \item $\opair{r, C, q} \tri{n_2} \alpha_2$
                                  \end{itemize}

                                  Така можем да заключим, че $\opair{p, A, q} \tri{n + 1}_{G'} \alpha$
                          \end{itemize}
                  \end{itemize}
        \end{itemize}
    \end{proof}

    Имайки всичко това, остава само да завършим с:
    \begin{align*}
        \alpha \in \mathcal{L}(G') & \iff \alpha \in \mathcal{L}_{G'}(S')                                                                                    \\
                                   & \iff \alpha \in \bigcup\limits_{f \in F} \mathcal{L}_{G'}(\opair{s, S, f})                                              \\
                                   & \iff \alpha \in \bigcup\limits_{f \in F} (\mathcal{L}_G(S) \cap \{ \alpha \in \Sigma^* \mid \delta^*(s, \alpha) = f \}) \\
                                   & \iff \alpha \in \mathcal{L}(G) \cap \bigcup\limits_{f \in F} \{ \alpha \in \Sigma^* \mid \delta^*(s, \alpha) = f \}     \\
                                   & \iff \alpha \in \mathcal{L}(G) \cap \mathcal{L(A)}                                                                      \\
                                   & \iff \alpha \in L \cap M
    \end{align*}
\end{proof}

\begin{corollary}\thlabel{non-cf-intersect-with-regular}
    Ако $L \cap M$ не е безконтекстен език и $M$ е регулярен език, то $L$ не е безконтекстен.
\end{corollary}

\begin{proof}
    Ако $L$ беше безконтекстен, то тогава $L \cap M$ щеше също да бъде безконтекстен, имайки че $M$ е регулярен език.
\end{proof}

\begin{claim}
    Езикът $L = \{ \alpha \in \Sigma^* \mid |\alpha|_a = |\alpha|_b = |\alpha|_c \}$ не е безконтекстен.
\end{claim}

\begin{proof}
    Знаем, че езикът $L_0 = \{ a^nb^nc^n \mid n \in \mathbb{N} \}$ не е безконтекстен.
    Очевидно $L_0 = L \cap (\{ a \}^* \{ b \}^* \{ c \}^*)$.
    От \thref{non-cf-intersect-with-regular} получаваме, че $L$ не е безконтекстен.
\end{proof}
\section{Стекови автомати}

Сега ще видим как можем да си мислим за безконтекстните граматики по друг, ``по-познат'' начин.

\begin{definition}
    \textbf{Стеков автомат} ще наричаме всяко $P = \opair{\Sigma, \Gamma, \sharp, Q, s, \Delta, f}$, където:

    \begin{itemize}
        \item $\Sigma$ е крайна входна азбука
        \item $\Gamma$ е крайна стекова азбука
        \item $\sharp \in \Gamma$ е символ за дъното на стека
        \item $Q$ е крайно множество от състояния
        \item $s \in Q$ е начално състояние
        \item $\Delta : Q \cross \Sigma_{\varepsilon} \cross \Gamma \rightarrow \mathcal{P}(Q \cross \Gamma^{\leq 2})$ е функция на преходите
        \item $f \in Q$ е финално състояние
    \end{itemize}
\end{definition}

За един преход $\opair{q, \beta} \in \Delta(p, x, A)$ можем да си мислим по следния начин:
\begin{center}
    \textit{четейки $x$ (което е буква или $\varepsilon$) и намирайки се в състоянието $p$, \\ ако на върха на стека е буквата $A$, \\ можем да отидем в състоянието $q$ като буквата $A$ се маха от върха на стека и се заменя с $\beta$}
\end{center}

\begin{definition}
    \textbf{Конфигурация в стеков автомат $P = \opair{\Sigma, \Gamma, \sharp, Q, s, \Delta, f}$} ще наричаме всяка тройка от вида $\opair{q, \alpha, \gamma} \in Q \cross \Sigma^* \cross \Gamma^*$.
\end{definition}

Можем да си мислим за $\opair{q, \alpha, \gamma}$ по следния начин:

\begin{center}
    \textit{в момента се намираме в състояние $q$, в стека е думата $\gamma$, и остава да прочетем $\alpha$}
\end{center}

\begin{definition}
    Въвеждаме релацията $\vdash_P$ за едностъпков преход между конфигурации по следното правило:

    \begin{center}
        ако $\opair{q, \beta} \in \Delta(p, x, A)$, то $\opair{p, x \alpha, A \gamma} \vdash_P \opair{q, \alpha, \beta \gamma}$
    \end{center}

    Също така въвеждаме релацията $\vdash_P^{l}$ за многостъпков преход между конфигурации по следното правило:

    \begin{itemize}
        \item $\kappa \vdash_P^0 \kappa$
        \item ако $\kappa_1 \vdash_P \kappa_2$ и $\kappa_2 \vdash_P^l \kappa_3$, то $\kappa_1 \vdash_P^{l + 1} \kappa_3$
    \end{itemize}

    Казваме, че $\kappa_1 \vdash_P^* \kappa_2$ има $l \in \mathbb{N}$ такова, че $\kappa_1 \vdash_P^l \kappa_2$.
\end{definition}

\begin{remark}
    Ако $P$ се подразбира няма да го пишем.
\end{remark}

\begin{definition}
    \textbf{Езика на стеков автомат $P = \opair{\Sigma, \Gamma, \sharp, Q, s, \Delta, f}$} ще бележим с

    \begin{center}
        $\mathcal{L}(P) = \{ \alpha \in \Sigma^* \mid \opair{s, \alpha, \sharp} \vdash_P^* \opair{f, \varepsilon, \varepsilon} \}$
    \end{center}
\end{definition}

Сега ще покажем някои полезни свойства на $\vdash^*$.

\begin{claim}
    Ако $\kappa_1 \vdash^{l_1} \kappa_2$ и $\kappa_2 \vdash^{l_2} \kappa_3$, то $\kappa_1 \vdash^{l_1 + l_2} \kappa_3$
\end{claim}

\begin{proof}
    С индукция по $l_1$:

    \begin{itemize}
        \item Нека $\kappa_1 \vdash^0 \kappa_2$ и $\kappa_2 \vdash^l \kappa_3$, тогава $\kappa_1 = \kappa_2$ \checkmark
        \item Нека $\kappa_1 \vdash^{l_1 + 1} \kappa_2$ и $\kappa_2 \vdash^{l_2} \kappa_3$.
              Тогава $\kappa_1 \vdash \kappa'$ и $\kappa' \vdash^{l_1} \kappa_2$.
              По ИП $\kappa' \vdash^{l_1 + l_2} \kappa_3$, откъдето $\kappa_1 \vdash^{1 + l_1 + l_2} \kappa_3$.
    \end{itemize}
\end{proof}

\begin{claim}
    Ако $\opair{p, \alpha_1, \gamma_1} \vdash^l \opair{q, \varepsilon, \varepsilon}$, то $\opair{p, \alpha_1 \alpha_2, \gamma_1 \gamma_2} \vdash^l \opair{q, \alpha_2, \gamma_2}$.
\end{claim}

\begin{proof}
    С индукция по $l$:

    \begin{itemize}
        \item Ако $\opair{p, \alpha_1, \gamma_1} \vdash^0 \opair{q, \varepsilon, \varepsilon}$, то $p = q, \: \alpha_1 = \varepsilon$ и $\gamma_1 = \varepsilon$.
              Тогава очевидно $\opair{p, \alpha_1 \alpha_2, \gamma_1 \gamma_2} \vdash^0 \opair{q, \alpha_2, \gamma_2}$ \checkmark
        \item Ако $\opair{p, x \alpha_1, A \gamma_1} \vdash^{l + 1} \opair{q, \varepsilon, \varepsilon}$, то $\opair{p, x \alpha_1, A \gamma_1} \vdash \opair{r, \alpha_1, \beta \gamma_1} \vdash^l \opair{q, \varepsilon, \varepsilon}$, където $\opair{r, \beta} \in \Delta(p, x, A)$.

              По ИП $\opair{r, \alpha_1 \alpha_2, \beta \gamma_1 \gamma_2} \vdash^l \opair{q, \alpha_2, \gamma_2}$, откъдето $\opair{p, x \alpha_1 \alpha_2, A \gamma_1 \gamma_2} \vdash^{l + 1} \opair{q, \alpha_2, \gamma_2}$.
    \end{itemize}
\end{proof}

\pagebreak

\begin{corollary}
    Ако $\opair{p, \alpha_1, A_1} \vdash^{l_1} \opair{r, \varepsilon, \varepsilon}$ и $\opair{r, \alpha_2, A_2} \vdash^{l_2} \opair{q, \varepsilon, \varepsilon}$, то $\opair{p, \alpha_1 \alpha_2, A_1 A_2} \vdash^{l_1 + l_2} \opair{q, \varepsilon, \varepsilon}$
\end{corollary}

\begin{claim}
    Нека $\opair{p, \alpha, A \gamma} \vdash^l \opair{q, \varepsilon, \varepsilon}$.
    Тогава има $l_1, l_2 \in \mathbb{N}$, състояние $r$ и думи $\alpha_1, \alpha_2$ такива, че:

    \begin{itemize}
        \item $l = l_1 + l_2$
        \item $\alpha = \alpha_1 \alpha_2$
        \item $\opair{p, \alpha_1, A} \vdash^{l_1} \opair{r, \varepsilon, \varepsilon}$
        \item $\opair{r, \alpha_2, \gamma} \vdash^{l_2} \opair{q, \varepsilon, \varepsilon}$
    \end{itemize}
\end{claim}

\begin{proof}
    С индукция по $l$ (очевидно $l \geq 1$):

    \begin{itemize}
        \item Ако $\opair{p, \alpha, A \gamma} \vdash^1 \opair{q, \varepsilon, \varepsilon}$, то $\gamma = \varepsilon, \: \alpha \in \Sigma_{\varepsilon}$ и $\opair{q, \varepsilon} \in \Delta(p, \alpha, A)$.
              Полагаме $l_1 = 1, \: l_2 = 0, \: \alpha_1 = \alpha, \: \alpha_2 = \varepsilon, \: r = q$ и сме готови \checkmark
        \item Нека $\opair{p, \alpha, A \gamma} \vdash^{l + 1} \opair{q, \varepsilon, \varepsilon}$.
              Тогава сме направили поне един преход.
              Нека $\alpha = x \lambda$.

              \begin{itemize}
                  \item[1 сл.] На първа стъпка сме имали прехода $\opair{p', BC} \in \Delta(p, x, A)$.
                      Тогава $\opair{p', \lambda, BC \gamma} \vdash^l \opair{q, \varepsilon, \varepsilon}$.
                      По ИП:

                      \begin{itemize}
                          \item $l_1' + l_2' = l$
                          \item $\lambda_1 \lambda_2 = \lambda$
                          \item $\opair{p', \lambda_1, B} \vdash^{l_1'} \opair{r', \varepsilon, \varepsilon}$
                          \item $\opair{r', \lambda_2, C \gamma} \vdash^{l_2'} \opair{q, \varepsilon, \varepsilon}$
                      \end{itemize}

                      Понеже $l_2' \leq l$ пак да приложим ИП:

                      \begin{itemize}
                          \item $k_1 + k_2 = l_2$
                          \item $\beta_1 \beta_2 = \lambda_2$
                          \item $\opair{r', \beta_1, C} \vdash^{k_1} \opair{r'', \varepsilon, \varepsilon}$
                          \item $\opair{r'', \beta_2, \gamma} \vdash^{k_2} \opair{q, \varepsilon, \varepsilon}$
                      \end{itemize}

                      Полагаме $l_1 = 1 + l_1' + k_1, \: l_2 = k_2, \: r = r'', \: \alpha_1 = x \lambda_1 \beta_1, \: \alpha_2 = \beta_2$ и получаваме, че:

                      \begin{itemize}
                          \item $l_1 + l_2 = 1 + l_1' + k_1 + k_2 = 1 + l_1' + l_2' = 1 + l$
                          \item $\alpha_1 \alpha_2 = x \lambda_1 \beta_1 \beta_2 = x \lambda_1 \lambda_2 = x \lambda = \alpha$
                          \item $\opair{p, x \lambda_1 \beta_1, A} \vdash \opair{p', \lambda_1 \beta_1, BC} \vdash^{l_1'} \opair{r', \beta_1, C} \vdash^{k_1} \opair{r'', \varepsilon, \varepsilon}$, откъдето $\opair{p, \alpha_1, A} \vdash^{l_1} \opair{r, \varepsilon, \varepsilon}$
                          \item $\opair{r'', \beta_2, \gamma} \vdash^{k_2} \opair{q, \varepsilon, \varepsilon}$, откъдето $\opair{r, \alpha_2, \gamma} \vdash^{l_2} \opair{q, \varepsilon, \varepsilon}$
                      \end{itemize}
                  \item[2 сл.] На първа стъпка сме имали прехода $\opair{p', B} \in \Delta(p, x, A)$.
                      Тогава $\opair{p', \lambda, B \gamma} \vdash^l \opair{q, \varepsilon, \varepsilon}$.
                      По ИП:

                      \begin{itemize}
                          \item $l_1' + l_2' = l$
                          \item $\lambda_1 \lambda_2 = \lambda$
                          \item $\opair{p', \lambda_1, B} \vdash^{l_1'} \opair{r', \varepsilon, \varepsilon}$
                          \item $\opair{r', \lambda_2, \gamma} \vdash^{l_2'} \opair{q, \varepsilon, \varepsilon}$
                      \end{itemize}

                      Полагаме $l_1 = 1 + l_1', \: l_2 = l_2', \: r = r', \: \alpha_1 = x \lambda_1, \: \alpha_2 = \lambda_2$ и получаваме, че:

                      \begin{itemize}
                          \item $l_1 + l_2 = 1 + l_1' + l_2' = 1 + l$
                          \item $\alpha_1 \alpha_2 = x \lambda_1 \lambda_2 = x \lambda = \alpha$
                          \item $\opair{p, x \lambda_1, A} \vdash \opair{p', \lambda_1, B} \vdash^{l_1'} \opair{r', \varepsilon, \varepsilon}$, откъдето $\opair{p, \alpha_1, A} \vdash^{l_1} \opair{r, \varepsilon, \varepsilon}$
                          \item $\opair{r', \lambda_2, \gamma} \vdash^{l_2'} \opair{q, \varepsilon, \varepsilon}$, откъдето $\opair{r, \alpha_2, \gamma} \vdash^{l_2} \opair{q, \varepsilon, \varepsilon}$
                      \end{itemize}
                  \item[3 сл.] На първа стъпка сме имали прехода $\opair{p', \varepsilon} \in \Delta(p, x, A)$.
                      Тогава $\opair{p', \lambda, \gamma} \vdash^l \opair{q, \varepsilon, \varepsilon}$.
                      Тогава полагаме $l_1 = 1, \: l_2 = l, \: r = p', \: \alpha_1 = x, \alpha_2 = \lambda$ и получаваме, че:

                      \begin{itemize}
                          \item $l_1 + l_2 = 1 + l$
                          \item $\alpha_1 \alpha_2 = x \lambda = \alpha$
                          \item $\opair{p, x, A} \vdash \opair{p', \varepsilon, \varepsilon}$, откъдето $\opair{p, \alpha_1, A} \vdash^{l_1} \opair{r, \varepsilon, \varepsilon}$
                          \item $\opair{p', \lambda, \gamma} \vdash^l \opair{q, \varepsilon, \varepsilon}$, откъдето $\opair{r, \alpha_2, \gamma} \vdash^{l_2} \opair{q, \varepsilon, \varepsilon}$
                      \end{itemize}
              \end{itemize}
    \end{itemize}

    С това доказателството е завършено.
\end{proof}
\section{Класически примери за стекови автомати}

Тук ще покажем някои прости езици, които се разпознават от стекови автомати.

\begin{claim}
    Езикът $L = \{ a^nb^n \mid n \in \mathbb{N} \}$ се разпознава от стеков автомат.
\end{claim}

\begin{proof}
    Стековият автомат $P = \opair{\Sigma, \Gamma, \sharp, Q, s, \Delta, f}$ е следния:

    \begin{itemize}
        \item $Q = \{ s, p, f \}$ и $\Gamma = \{ \sharp, a \}$
        \item $\Delta(s, \varepsilon, \sharp) = \{ \opair{f, \varepsilon} \}$ - директно разпознаваме $\varepsilon$
        \item $\Delta(s, a, \sharp) = \{ \opair{s, a \sharp} \}$ - трупаме $a$ в стека
        \item $\Delta(s, a, a) = \{ \opair{s, aa} \}$ - трупаме $a$ в стека
        \item $\Delta(s, b, a) = \{ \opair{p, \varepsilon} \}$ - минаваме в режим на премахване от стека и четене само на $b$
        \item $\Delta(p, b, a) = \{ \opair{p, \varepsilon} \}$ - премахваме натрупаните $a$ в стека
        \item $\Delta(p, \varepsilon, \sharp) = \{ \opair{f, \varepsilon} \}$ - когато няма нищо в стека $a$ и $b$ са били срещати равен брой пъти
        \item Във всички останали случаи функцията $\Delta$ връща $\varnothing$ (по нататък няма да го пишем)
    \end{itemize}

    За да покажем, че $L \subseteq \mathcal{L}(P)$, трябва да се покаже (оставяме на читателя), че за всяко $n \in \mathbb{N}$:

    \begin{itemize}
        \item $\opair{s, a^n \beta, \sharp} \vdash^n \opair{s, \beta, a^n \sharp}$ \\
              Това просто казва, че състоянието $s$ се използва за да трупа $a$ в стека.
        \item $\opair{p, b^n, a^n \sharp} \vdash^n \opair{p, \varepsilon, \sharp}$ \\
              Това просто казва, че състоянието $p$ се използва за да премахне натрупаните $b$ в стека.
    \end{itemize}

    Имайки това $\opair{s, a^n b^n, \sharp} \vdash^n \opair{s, b^n, a^n \sharp} \vdash^n \opair{p, \varepsilon, \sharp} \vdash \opair{f, \varepsilon, \varepsilon}$, откъдето $a^nb^n \in \mathcal{L}(P)$.

    За да покажем, че $\mathcal{L}(P) \subseteq L$, трябва да се покаже (оставяме на читателя), че за всяко $n \in \mathbb{N}$:

    \begin{itemize}
        \item ако $\opair{s, \alpha \beta, \sharp} \vdash^n \opair{s, \beta, \gamma \sharp}$, то $\alpha = \gamma = a^n$ \\
              Тук трябва да се използва, че от $s$ няма преходи с други букви, които да остават в $s$.
        \item ако $\opair{p, \beta, \gamma \sharp} \vdash^n \opair{p, \varepsilon, \sharp}$, то $\beta = b^n$ и $\gamma = a^n$ \\
              Тук трябва да се използва, че от $p$ няма преходи с други букви, които да остават в $p$.
    \end{itemize}

    Нека $\alpha \in \mathcal{L}(P)$.
    Тогава $\opair{s, \alpha, \sharp} \vdash^{n} \opair{f, \varepsilon, \varepsilon}$.
    Ясно е, че $\varepsilon \in L$, така че нека Б.О.О. $\alpha \neq \varepsilon$.
    Тогава знаем, че сме минали през всички състояния.
    Нека $\alpha_1, \alpha_2 \in \Sigma^*$ са такива, че $\alpha = \alpha_1 \alpha_2$ и

    \begin{center}
        $\underbrace{\opair{s, \alpha_1 \alpha_2, \sharp} \vdash^{n_1} \opair{p, \alpha_2, \gamma \sharp}}_{\substack{\text{трябва да направим такъв преход} \\ \text{четейки дума, различна от } \varepsilon}} \vdash^{n_2} \underbrace{\opair{p, \varepsilon, \sharp} \vdash \opair{f, \varepsilon, \varepsilon}}_{\substack{\text{единственият начин} \\ \text{да разпознаем дума} \\ \text{различна от } \varepsilon}}$
    \end{center}

    Тогава от горното твърдение $\alpha_1 = \gamma = a^{n_1}$, $\alpha_2 = b^{n_2}$ и $\gamma = a^{n_2}$.
    Така $\alpha = a^{n_1} b^{n_1} = a^{n_2} b^{n_2} \in L$.
\end{proof}

\begin{problem}
Да се построят стекови автомати за следните езици:

\begin{itemize}
    \item $L_1 = \{ a^n b^{2n} \mid n \in \mathbb{N} \}$
    \item $L_2 = \{ a^{2n} b^n \mid n \in \mathbb{N} \}$
    \item $L_3 = \{ a^{2n} b^{3n} \mid n \in \mathbb{N} \}$
\end{itemize}

Упътване: не трябва да се правят много промени, трябва само да се промени малко работата със стека
\end{problem}

\begin{claim}\thlabel{pda-equal-a-b}
    Съществува стеков автомат за езика $L = \{ \alpha \in \Sigma^* \mid |\alpha|_a = |\alpha|_b \}$.
\end{claim}

\begin{proof}
    Стековият автомат $P = \opair{\Sigma, \Gamma, \sharp, Q, s, \Delta, f}$ е следния (доказателството остава за читателя):

    \begin{itemize}
        \item $Q = \{ s, f \}$ и $\Gamma = \{ a, b, \sharp \}$
        \item $\Delta(s, x, \sharp) = \{ \opair{s, x \sharp} \}$ за $x \in \Sigma$ - при празен стек добавяме буквата, която е в повече
        \item $\Delta(s, x, x) = \{ \opair{s, xx} \}$ за $x \in \Sigma$ - продължаваме да добавяме буквата в повече
        \item $\Delta(s, a, b) = \Delta(s, b, a) = \{ \opair{s, \varepsilon} \}$ - при срещане на другата буква махаме от стека
        \item $\Delta(s, \varepsilon, \sharp) = \{ \opair{f, \varepsilon} \}$ - ако стека е празен броят на различните букви е равен
    \end{itemize}

    \pagebreak

    За да се докаже, че $L = \mathcal{L}(P)$ можем да покажем, че за всяка дума $\gamma$ и за всяко естествено $n$:

    \begin{itemize}
        \item $a^n \gamma \in L \iff \opair{s, \gamma, a^n \sharp} \vdash^* \opair{s, \varepsilon, \sharp}$
        \item $b^n \gamma \in L \iff \opair{s, \gamma, b^n \sharp} \vdash^* \opair{s, \varepsilon, \sharp}$
    \end{itemize}

    Трябва да се направи индукция по $|\gamma|$, също така двете части се показват едновременно.

    Накрая заключаваме, че:

    \begin{center}
        $\alpha \in L \iff \opair{s, \alpha, a^0 \sharp} \vdash^* \opair{s, \varepsilon, \sharp} \vdash \opair{f, \varepsilon, \varepsilon} \iff \alpha \in \mathcal{L}(P)$
    \end{center}
\end{proof}

Една често срещана задача за стек, е да се провери дали един израз е ``добре скобуван'' (\thref{bal-par-def}).

\begin{claim}\thlabel{pda-bal-par}
    Езикът $L = \{ \alpha \in \Sigma^* \mid bal(\alpha, a, b) \}$ се разпознава от стеков автомат.
\end{claim}

\begin{proof}
    Стековият автомат $P = \opair{\Sigma, \Gamma, \sharp, Q, s, \Delta, f}$ е следния (доказателството остава за читателя):

    \begin{itemize}
        \item $Q = \{ s, f \}$ и $\Gamma = \{ a, \sharp \}$
        \item $\Delta(s, a, \sharp) = \{ \opair{s, a \sharp} \}$ - добавяме ``отварящите скоби'' в стека
        \item $\Delta(s, a, a) = \{ \opair{s, aa} \}$ - добавяме ``отварящите скоби'' в стека
        \item $\Delta(s, b, a) = \{ \opair{s, \varepsilon} \}$ - махаме ``отварящите скоби'' от стека като четем ``затварящи''
        \item $\Delta(s, \varepsilon, \sharp) = \{ \opair{f, \varepsilon} \}$ - когато стека е празен, прочетената дума до сега е балансирана
    \end{itemize}

    За да се докаже, че $L = \mathcal{L}(P)$ можем да покажем, че за всяка дума $\gamma$ и за всяко естествено $n$:

    \begin{center}
        $a^n \gamma \in L \iff \opair{s, \gamma, a^n \sharp} \vdash^* \opair{s, \varepsilon, \sharp}$
    \end{center}

    Доказателството върви по същият начин като в \thref{pda-equal-a-b}.

    Накрая заключаваме, че:

    \begin{center}
        $\alpha \in L \iff \opair{s, \alpha, a^0 \sharp} \vdash^* \opair{s, \varepsilon, \sharp} \vdash \opair{f, \varepsilon, \varepsilon} \iff \alpha \in \mathcal{L}(P)$
    \end{center}
\end{proof}

\begin{problem}
Да се направи стеков автомат за $L = \{ \alpha \in \{ a, b, c, d\}^* \mid bal(\alpha, a, b) \: \& \: bal(\alpha, c, d) \}$.

Упътване: напълно аналогично на \thref{pda-bal-par}
\end{problem}
\section{Еквивалентност на безконтекстните граматики и стековите автомати}
\section{Задачи за упражнение}

\end{document}