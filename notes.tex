\documentclass[11pt, a5paper]{report}

\usepackage[utf8]{inputenc}
\usepackage[T2A]{fontenc}
\usepackage[english, bulgarian]{babel}
\usepackage{amssymb}
\usepackage{hyperref, fancyhdr, lastpage, fancyvrb, tcolorbox, titlesec}
\usepackage{array, tabularx, colortbl}
\usepackage{tikz}
\usepackage{venndiagram}
\usepackage{amsthm}
\usepackage{amsmath,physics}
\usepackage{mathtools}
\usepackage[a5paper, left=0.50in, right=0.50in, top=1.0in, bottom=1.0in]{geometry}
\usetikzlibrary{automata, arrows}

\ExplSyntaxOn
\NewDocumentCommand{\opair}{m}
 {
  \langle\mspace{2mu}
  \clist_set:Nn \l_tmpa_clist { #1 }
  \clist_use:Nn \l_tmpa_clist {,\mspace{3mu plus 1mu minus 1mu}\allowbreak}
  \mspace{2mu}\rangle
}
\ExplSyntaxOff

\hypersetup{
    colorlinks=true,
    linktoc=all,
    linkcolor=blue
}

\newtheorem{definition}{Дефиниция}[section]
\newtheorem{theorem}[definition]{Теорема}
\newtheorem{claim}[definition]{Твърдение}
\newtheorem{axiom}[definition]{Аксиома}
\newtheorem{lemma}[definition]{Лема}
\newtheorem{problem}[definition]{Задача}
\newtheorem{corollary}[definition]{Следствие}
\theoremstyle{remark}
\newtheorem*{remark}{Забележка}

\pagestyle{fancy}

\lhead{\leftmark}
\rhead{}

\setlength\parindent{0pt}

\title{Записки за упражнения по ЕАИ}
\author{Тодор Дуков}
\date{\today}

\begin{document}

\maketitle

\tableofcontents

\chapter{Въведение}

Основното нещо, което се прави в този курс, е да се класифицират езици.
За да можем да класифицираме един език, първо трябва знаем какво точно представлява един език.

\section{Основни понятия}

\begin{definition}
    \textbf{Азбука} ще наричаме всяко крайно множество.
    Елементите на азбуката ще наричаме \textbf{букви}.
\end{definition}

Обикновено ще си бележим азбуквата със $\Sigma$.
Също така, ако никъде не е споменато друго, $\Sigma = \{a, b \}$.
Тук буквите ще бъдат $a$ и $b$.

\begin{definition}
    \textbf{Дума} над азбуката $\Sigma$ ще наричаме всяка крайна редица от букви от $\Sigma$.
    Дължината на дума $\alpha$ над $\Sigma$ ще бележим с $|\alpha|$.
\end{definition}

При азбуката $\Sigma = \{0, 1\}$, примерна дума ще бъде $\alpha = 000101$.
Ясно е, че $|\alpha| = 6$.

\begin{definition}
    \textbf{Празната дума} ще наричаме единствената дума с дължина 0.
    Бележим я с $\varepsilon$.
\end{definition}

\begin{remark}
    Важно е да се отбележи, че празното множество и празната дума са различни неща.
    Възможно е да се вземе такава дефиниция за редица, в която те да съвпадат, но това не е съществено.
    За нас думите ще бъдат едни неща, а множествата други.
    \textbf{TLDR:} $\varepsilon \neq \varnothing$
\end{remark}

\begin{definition}
    Със $\Sigma^*$ ще бележим множеството от всички думи над $\Sigma$.
    $L$ ще наричаме \textbf{език} над $\Sigma$, ако $L \subseteq \Sigma$.
\end{definition}

Тук нащият универсум ще бъде $\Sigma^*$.
Така че за $L \subseteq \Sigma^*$, под $\overline{L}$ ще имаме предвид $\Sigma^* \setminus L$.

\section{Операции върху думи и езици}

\begin{definition}
    Ще дефинираме \textbf{конкатенацията} (слепването) на две думи $\alpha$ и $\beta$
    и ще го бележим с $\alpha \cdot \beta$
    \begin{itemize}
        \item $\alpha \cdot \varepsilon = \alpha$ (Базов случай)
        \item $\alpha \cdot (\beta x) = (\alpha \cdot \beta)x$ (Свеждане до по-малка ``задача'')
    \end{itemize}
\end{definition}

На пръв поглед такава дефиниция изглежда безсмислена,
но това далеч не е така.
После тя ще се използва постоянно в доказателства. \\

Нека разгледаме един пример за конкатенация:
\begin{align*}
    aaa \cdot bbb & = (aaa \cdot bb)b = ((aaa \cdot b)b)b = (((aaa \cdot \varepsilon)b)b)b = \\
                  & = (((aaa)b)b)b = ((aaab)b)b = (aaabb)b = aaabbb
\end{align*}

Вече можем да дефинираме $\alpha^n$ ($n$ на брой пъти да конкатенираме думата $\alpha$) индуктивно:
\begin{itemize}
    \item $\alpha^0 = \varepsilon$
    \item $\alpha^{n + 1} = \alpha^n \cdot \alpha$
\end{itemize}

За пример можем да вземем $(ab)^3$.
\begin{align*}
    (ab)^3 & = (ab)^2 \cdot ab = ((ab^1) \cdot ab) \cdot ab = (((ab^0) \cdot ab) \cdot ab) \cdot ab = \\
           & = ((\varepsilon \cdot ab) \cdot ab) \cdot ab = ab \cdot ab \cdot ab
\end{align*}

\begin{remark}
    Тук използваме наготово, че $\varepsilon \cdot \alpha = \alpha$. \\
\end{remark}

Имайки конкатенация на думи, можем да дефинираме и конкатенацията на езици.
Най-естествено е да направим следното:

\begin{definition}
    Нека $L_1, L_2 \subseteq \Sigma^*$.
    Тогава \textbf{конкатенацията} на езиците $L_1$ и $L_2$ ще наричаме множеството:
    \begin{center}
        $L_1 \cdot L_2 = \{ \alpha \cdot \beta \: | \: \alpha \in L_1 \: \& \: \beta \in L_2 \}$ \\
    \end{center}
\end{definition}

Вече можем да дефинираме $L^n$ ($n$ на брой пъти да конкатенираме езика $L$) индуктивно:
\begin{itemize}
    \item $L^0 = \{ \varepsilon \}$
    \item $L^{n + 1} = L^n \cdot L$
\end{itemize}

\begin{definition}[Звезда на Клини]
    Нека $L \subseteq \Sigma^*$. Тогава:
    \begin{itemize}
        \item $L^* = \bigcup\limits_{n \in \mathbb{N}} L^n = L^0 \cup L^1 \cup L^2 \cup \dots $
        \item $L^+ = \bigcup\limits_{\substack{n \in \mathbb{N} \\ n \neq 0}} L^n = L^1 \cup L^2 \cup L^3 \cup \dots $
    \end{itemize}
\end{definition}

Ясно е, че в тази дефиниция $\Sigma^*$ е същото нещо като в другата.

\pagebreak

\section{Допълнителни дефиниции}

Тук ще сложим няколко дефиниции, които са стандартни, и ще има задачи, свързани с тях.

\begin{definition}
    Ще дефинираме \textbf{обръщането} на дума и на език.
    Обръщането на дума $\alpha$, което бележим с $\alpha^{rev}$, става индуктивно:
    \begin{itemize}
        \item $\varepsilon^{rev} = \varepsilon$
        \item $(\alpha x)^{rev} = x(\alpha^{rev})$
    \end{itemize}
    Обръщането на език $L$ бележим с $L^{rev} = \{\alpha^{rev} \: | \: \alpha \in L \}$.
\end{definition}

\begin{definition}
    Ще дефинираме кога една дума $\alpha$ е префикс, инфикс или суфикс на друга дума $\beta$:
    \begin{itemize}
        \item $\alpha \preceq_{pref} \beta$, ако $(\exists \gamma \in \Sigma^*)(\alpha \cdot \gamma = \beta)$
        \item $\alpha \preceq_{suff} \beta$, ако $(\exists \gamma \in \Sigma^*)(\gamma \cdot \alpha = \beta)$
        \item $\alpha \preceq_{inf} \beta$, ако $(\exists \gamma_1 \in \Sigma^*)(\exists \gamma_2 \in \Sigma^*)(\gamma_1 \cdot \alpha \cdot \gamma_2 = \beta)$
    \end{itemize}
    Нека $L \subseteq \Sigma^*$. Тогава:
    \begin{itemize}
        \item $\operatorname{Pref}(L) = \{ \alpha \in \Sigma^* \: | \: (\exists \beta \in L)(\alpha \preceq_{pref} \beta) \}$
        \item $\operatorname{Suff}(L) = \{ \alpha \in \Sigma^* \: | \: (\exists \beta \in L)(\alpha \preceq_{suff} \beta) \}$
        \item $\operatorname{Infix}(L) = \{ \alpha \in \Sigma^* \: | \: (\exists \beta \in L)(\alpha \preceq_{inf} \beta) \}$
    \end{itemize}
\end{definition}

\pagebreak

\section{Задачи за упражнение}

\begin{problem}[асоциативност]
Да се докаже, че $\alpha \cdot (\beta \cdot \gamma) = (\alpha \cdot \beta) \cdot \gamma$

Упътване: да се направи индукция по $\gamma$
\end{problem}

\begin{problem}[неутрален елемент]
Да се докаже, че $\varepsilon \cdot \alpha = \alpha$

Упътване: да се направи индукция по $\alpha$
\end{problem}

\begin{problem}
Да се докаже, че $\alpha^n \cdot \alpha^m = \alpha^{n + m}$

Упътване: да се направи индукция по $m$
\end{problem}


\begin{problem}
Да се докаже, че $(\alpha^n)^m = \alpha^{nm}$

Упътване: да се направи индукция по $m$
\end{problem}

\begin{problem}
Да се докаже, че $L^n \cdot L^m = L^{n + m}$

Упътване: да се направи индукция по $m$
\end{problem}


\begin{problem}
Да се докаже, че $(L^n)^m = L^{nm}$

Упътване: да се направи индукция по $m$
\end{problem}

\begin{problem}[дистрибутивност]
Да се докаже, че:
\begin{itemize}
    \item $(L_1 \cup L_2) \cdot L_3 = (L_1 \cdot L_3) \cup (L_2 \cdot L_3)$
    \item $(L_1 \cap L_2) \cdot L_3 = (L_1 \cdot L_3) \cap (L_2 \cdot L_3)$
\end{itemize}
\end{problem}

\begin{problem}
Да се докаже, че $\Sigma^+ = \Sigma \cdot \Sigma^*$

Упътване: да се използват предните резултати
\end{problem}

\begin{problem}
Да се докаже, че $(\alpha \cdot \beta)^{rev} = \alpha^{rev} \cdot \beta^{rev}$

Упътване: да се направи индукция по $\beta$
\end{problem}

\begin{problem}
Да се докаже, че:
\begin{itemize}
    \item $(\alpha^{rev})^{rev} = \alpha$
    \item $(L^{rev})^{rev} = L$
\end{itemize}

Упътване: за първото да се направи индукция по $\alpha$
\end{problem}
\begin{problem}
Да се докаже, че:
\begin{itemize}
    \item $\operatorname{Pref}(L_1 \cdot L_2) = \operatorname{Pref}(L_1) \cup (L_1 \cdot \operatorname{Pref}(L_2))$
    \item $\operatorname{Suff}(L_1 \cdot L_2) = \operatorname{Suff}(L_2) \cup (\operatorname{Suff}(L_1) \cdot L_2)$
    \item $\operatorname{Infix}(L_1 \cdot L_2) = \operatorname{Infix}(L_1) \cup \operatorname{Infix}(L_2) \cup (\operatorname{Suff}(L_1) \cdot \operatorname{Pref}(L_2))$
    \item $\operatorname{Infix}(L) = \operatorname{Pref}(\operatorname{Suff}(L)) = \operatorname{Suff}(\operatorname{Pref}(L))$
\end{itemize}

Упътване: да се разсъждава на ниво думи (конкатенация на езици се дефинира с конкатенация на думи)
\end{problem}
\chapter{Автомати}

Тук ще разгледаме първата ``машина'', с която ще класифицираме езиците.

\section{Детерминирани автомати и автоматни езици}

\begin{definition}
    \textbf{Детерминиран краен автомат} (накратко ДКА) ще наричаме всяко $\mathcal{A} = \opair{\Sigma, Q, s, \delta, F}$, където:
    \begin{itemize}
        \item $\Sigma$ е крайна азбука
        \item $Q$ е крайно множество от състояния
        \item $s \in Q$ (ще го наричаме начално/стартово състояние)
        \item $\delta : Q \cross \Sigma \rightarrow Q$ (ще я наричаме функция на преходите)
        \item $F \subseteq Q$ (ще ги наричаме финални състояния)
    \end{itemize}
\end{definition}

В момента със това, което имаме,
ако искаме да покажем къде ще стигнем с думата $aaa$, започвайки от $s$,
ще трябва да го запишем със $\delta(\delta(\delta(s, a), a), a)$, което е тромаво.

За това ще си въведем начин, по който да видим крайният резултат от прочитането на цяла дума.

\begin{definition}
    Дефинираме $\delta^* : Q \cross \Sigma^* \rightarrow Q$ индуктивно:
    \begin{itemize}
        \item $\delta^*(p, \varepsilon) = p$ за всяко $p \in Q$
        \item $\delta^*(p, \beta x) = \delta(\delta^*(p, \beta), x)$ за всяко $p \in Q, \beta \in \Sigma^*, x \in \Sigma$
    \end{itemize}
\end{definition}

Сега можем да забележим, че:
\begin{align*}
    \delta^*(s, aaa) & = \delta(\delta^*(s, aa), a) = \delta(\delta(\delta^*(s, a), a), a) =                           \\
                     & = \delta(\delta(\delta(\delta^*(s, \varepsilon), a), a), a) =\delta(\delta(\delta(s, a), a), a)
\end{align*}
Функцията наистина прави това, което искаме да прави.

\begin{definition}
    Нека $\mathcal{A} = \opair{\Sigma, Q, s, \delta, F}$ е ДКА.
    Тогава езикът на автомата $\mathcal{A}$ е множеството
    $\mathcal{L}(\mathcal{A}) = \{ \alpha \in \Sigma^* \: | \: \delta^*(s, \alpha) \in F \}$.
    Един език $L \subseteq \Sigma^*$, наричаме \textbf{автоматен}, ако има ДКА $\mathcal{A}$ с $\mathcal{L}(\mathcal{A}) = L$.
\end{definition}

\section{Представяне на автомат}

Ще видим начините, по които можем да представяме един автомат.
За пример ще вземем автомата $\mathcal{A} = \opair{\Sigma, Q, s, \delta, F}$, където:
\begin{itemize}
    \item $\Sigma = \{ a, b \}$
    \item $Q = \{ q_0, q_1, q_2 \}$
    \item $s = q_0$
    \item $\delta(q_0, a) = q_2$
    \item $\delta(p, x) = q_1$ за $p \in Q, \: x \in \Sigma, \: \opair{p, x} \neq \opair{q_0, a}$
    \item $F = \{ q_2 \}$
\end{itemize}
Това е първият начин за представяне.
При него директно в явен вид се казват кои са всички съставни елементи на $\mathcal{A}$.
Това ще бъде използване сравнително често, когато правим от един автомат друг.
В случаите, в които не знаем как изглежда първоначалният автомат, следващите методи тогава няма да свършат работа. \\

Вторият начин за представяне е с таблица на функцията на преходите:
\begin{center}
    \begin{tabular}{||r | r | r||}
        \hline
        \cellcolor{lightgray} & $a$   & $b$   \\
        \hline
        $\rightarrow q_0$     & $q_2$ & $q_1$ \\
        \hline
        $q_1$                 & $q_1$ & $q_1$ \\
        \hline
        $\checkmark q_2$      & $q_1$ & $q_1$ \\
        \hline
    \end{tabular}
\end{center}
Тук с $\rightarrow$ отбелязваме началното състояние,
а с $\checkmark$ отбелязваме финалните състояния.
Във втората и третата колона казваме къде ще се озовем,
ако се намираме в съответното състояние и прочетем съответната буква.
Това е по-прегледно представяне от първото, но може да стане обемно. \\

Третият начин за представяне е с картинка:

\begin{center}
    \begin{tikzpicture}[shorten >=1pt,node distance=2cm,>=stealth',thick]
        \node[initial, state, initial text=] (1) {$q_0$};
        \node[state] (2) [above right of=1] {$q_1$};
        \node[state, accepting] (3) [below right of=1] {$q_2$};
        \draw[->] (1) -- node[above] {$b$} (2);
        \draw[->] (1) -- node[above] {$a$} (3);
        \path (2) edge [loop above] node[above] {$a, b$} (2);
        \draw[->] (3) -- node[right] {$a, b$} (2);
    \end{tikzpicture}
\end{center}

Началното състояние е отбелязано със стрелка, а финалните са оградени два пъти.
Представянето чрез картинка изглежда по-прегледно от другите две, но то може и да стане огромно.

Ясно е, че $\mathcal{L}(\mathcal{A}) = \{ a \}$.
Ако искаме да покажем това формално, трябва да направим следното:
\begin{itemize}
    \item Очевидно $a \in \mathcal{L}(\mathcal{A}), \: b \notin \mathcal{L}(\mathcal{A}), \: \varepsilon \notin \mathcal{L}(\mathcal{A})$
    \item Показваме, че $\delta^*(q_1, \alpha) = q_1$ за всяко $\alpha \in \Sigma^*$ с индукция по $|\alpha|$
    \item Ако $\alpha \notin \{ a, b, \varepsilon \}$, то $\alpha = x y \beta$ за $x, y \in \Sigma, \: \beta \in \Sigma^*$
    \item Тогава лесно се вижда, че $\delta^*(q_0, \alpha) = q_1 \notin F$.
          Или директно отиваме в $q_1$ (ако $x = b$) и оставаме там, или първо отиваме в $q_2$ (ако $x = a$), и после със следващата буква отиваме в $q_1$ и оставаме там.
\end{itemize}

Това обаче за такива прости автомати не е нужно.
Тези неща такива случаи ще ги приемаме за очевидни.

\section{Прости примери за автоматни езици}

Автомат за $L = \O$:

\begin{center}
    \begin{tikzpicture}[shorten >=1pt,node distance=2cm,>=stealth',thick]
        \node[initial, state, initial text=] (1) {$q$};
        \path[->] (1) edge [loop above] node[above] {$a, b$}(1);
    \end{tikzpicture}
\end{center}

Автомат за $L = \Sigma^*$:

\begin{center}
    \begin{tikzpicture}[shorten >=1pt,node distance=2cm,>=stealth',thick]
        \node[initial, accepting, state, initial text=] (1) {$q$};
        \path[->] (1) edge [loop above] node[above] {$a, b$}(1);
    \end{tikzpicture}
\end{center}

Сега малко ще усложним нещата.
Ще направим автомат, който да разпознае езикът от една конкретна дума.
Конструкцията за конкретната дума много лесно се обобщава за всички.
За пример нека направим автомат за $L = \{ abab \}$:

\begin{center}
    \begin{tikzpicture}[shorten >=1pt,node distance=2.5cm,>=stealth',thick]
        \node[initial, state, initial text=] (1) {$\varepsilon$};
        \node[state] [right of=1] (2) {$a$};
        \node[state] [right of=2] (3) {$ab$};
        \node[state] [right of=3] (4) {$aba$};
        \node[state, accepting] [right of=4] (5) {$abab$};
        \node[state] [below of=3] (6) {$\cross$};
        \draw[->] (1) -- node[above] {$a$} (2);
        \draw[->] (2) -- node[above] {$b$} (3);
        \draw[->] (3) -- node[above] {$a$} (4);
        \draw[->] (4) -- node[above] {$b$} (5);
        \path[->] (1) edge [bend right] node[below] {$b$} (6);
        \path[->] (2) edge [bend right] node[below] {$a$} (6);
        \path[->] (3) edge  node[right] {$b$} (6);
        \path[->] (4) edge [bend left] node[below] {$a$} (6);
        \path[->] (5) edge [bend left] node[below] {$a, b$} (6);
        \path[->] (6) edge [loop below] node[below] {$a, b$} (6);
    \end{tikzpicture}
\end{center}

Ето как ще обобщим конструкцията за $L = \{ \alpha \}$:
\begin{itemize}
    \item $\mathcal{A} = \opair{\Sigma, Q, s, \delta, F}$
    \item $Q = Pref(L) \cup \{ \cross \}$ (начален отрязък от пътя, който съставя $\alpha$, $\cross \notin \Sigma$ е отхвърлящо състояние боклук)
    \item $s = \varepsilon$
    \item За $\beta \preceq_{pref} \alpha \: ($тогава $\beta \in Q), \: x \in \Sigma$:
          ако $\beta x \preceq_{pref} \alpha$, то $\delta(\beta, x) = \beta x$,
          иначе $\delta(\beta, x) = \cross$
    \item $\delta(\cross, x) = \cross$ за $x \in \Sigma$
    \item $F = \{ \alpha \}$
\end{itemize}

Идеята зад конструкцията е следната:
Състоянията кодират пътя, който сме изминали, ако не сме се отклонили вече от ``строежа'' на $\alpha$.
В първият момент на отклонение отиваме във нефинално състояние ``боклук'', от което не може да излезнем. \\

Автомат за $L = \{ ab, ba, aa \}$:

\begin{center}
    \begin{tikzpicture}[shorten >=1pt,node distance=2.5cm,>=stealth',thick]
        \node[initial, state, initial text=] (1) {$\varepsilon$};
        \node[state] [above right of=1] (2) {$a$};
        \node[state] [below right of=1] (3) {$b$};
        \node[state, accepting] [above right of=2] (4) {$aa$};
        \node[state, accepting] [right of=2] (5) {$ab$};
        \node[state, accepting] [right of=3] (6) {$ba$};
        \node[state] [below right of=3] (7) {$bb$};
        \node[state] [below right of=5] (8) {$\cross$};
        \draw[->] (1) -- node[above] {$a$} (2);
        \draw[->] (1) -- node[above] {$b$} (3);
        \draw[->] (2) -- node[above] {$a$} (4);
        \draw[->] (2) -- node[above] {$b$} (5);
        \draw[->] (3) -- node[above] {$a$} (6);
        \draw[->] (3) -- node[above] {$b$} (7);
        \path[->] (4) edge [bend left] node[right] {$a, b$} (8);
        \path[->] (5) edge [bend right] node[left] {$a, b$} (8);
        \path[->] (6) edge [bend left] node[left] {$a, b$} (8);
        \path[->] (7) edge [bend right] node[right] {$a, b$} (8);
        \path[->] (8) edge [loop right] node[above] {$a, b$} (8);
    \end{tikzpicture}
\end{center}

Тази конструкция надгражда над предната.
Вместо да се следи за един възможен път, го правим за няколко.

Ето как ще обобщим конструкцията за $L = \{ \alpha_1, \: \dots, \: \alpha_n \}$:
\begin{itemize}
    \item $\mathcal{A} = \opair{\Sigma, Q, s, \delta, F}$
    \item $Q = \{ \alpha \in \Sigma^* \: | \: (\exists \beta \in L) (|\alpha| \leq |\beta|) \} \cup \{ \cross \}$ (не гледаме пътища по-дълги от най-дългата дума в $L$, $\cross \notin \Sigma$ - състояние боклук)
    \item $s = \varepsilon$
    \item За $\alpha \in Q, \: x \in \Sigma$:
          ако $\alpha x \in Q$, то $\delta(\alpha, x) = \alpha x$,
          иначе $\delta(\alpha, x) = \cross$
    \item $\delta(\cross, x) = \cross$ за $x \in \Sigma$
    \item $F = L$
\end{itemize}

Коректността на тази конструкция приемаме за очевидна.

\section{Операции над автоматни езици}

\section{Недетерминирани автомати}

\section{Още операции}

\section{Еквивалентност на детерминирани и недетерминирани автомати}

\section{Автомат на Brzozowski}

\section{Неавтоматни езици}

\section{Задачи за упражнение}
\chapter{Регулярни изрази}

\section{Регулярни изрази и регулярни езици}

\section{Операции над регулярни езици}

\section{Еквивалентност на регулярните и автоматните езици}

\section{Задачи за упражнение}
\chapter{Граматики}

\section{Неограничени граматики}

\section{Регулярни граматики и еквивалентност със автоматните езици}

\section{Безконтекстни граматики}

\section{Дървета на извод}

\section{Небезконтекстни езици}

\section{Нормална форма на Чомски}

\section{Задачи за упражнение}

\end{document}