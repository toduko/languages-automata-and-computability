\section{Операции над автоматни езици}

Сега ще разгледаме няколко операции, които запазват автоматност.
Първата операция (която вече загатнахме) над автоматни езици, която ще разгледаме е допълнение.

\begin{claim}
    Ако $L$ е автоматен, то тогава и $\overline{L}$ също е автоматен.
\end{claim}

\begin{proof}
    Понеже $L$ е автоматен, има ДКА $\mathcal{A} = \opair{\Sigma, Q, s, \delta, F}$ с $\mathcal{L(A)} = L$.
    Нека $\mathcal{A}_{\overline{L}} = \opair{\Sigma, Q, s, \delta, Q \setminus F}$. Тогава:
    \begin{center}
        $\alpha \in \mathcal{L(A)} \iff \delta^*(s, \alpha) \in Q \setminus F \iff \delta^*(s, \alpha) \notin F \iff \alpha \notin L$
    \end{center}
\end{proof}

Както направихме и в предният пример, тук просто сменяме отговора.
Състоянията и преходите на началният автомат ни дават достатъчна информация за структурата на думата,
което е достатъчно за да кажем дали е от езика $\overline{L}$ или не е. \\

Преди да разгледаме следващата операция ще разгледаме два авомата:

\begin{center}
    \begin{tikzpicture}[shorten >=1pt,node distance=2.5cm,>=stealth',thick]
        \node[text width=0.5cm] at (-1, 1) {$\mathcal{A}_1$:};
        \node[accepting, initial, state, initial text=] (1) {$0$};
        \node[state] [right of=1] (2) {$1$};
        \path[->] (1) edge [bend left] node[above] {$a$} (2);
        \path[->] (2) edge [bend left] node[below] {$a$} (1);
        \path[->] (1) edge [loop above] node[above] {$b$} (1);
        \path[->] (2) edge [loop above] node[above] {$b$} (2);
    \end{tikzpicture}
\end{center}

\begin{center}
    \begin{tikzpicture}[shorten >=1pt,node distance=2.5cm,>=stealth',thick]
        \node[text width=0.5cm] at (-1, 1) {$\mathcal{A}_2$:};
        \node[initial, state, initial text=] (1) {$0$};
        \node[accepting, state] [right of=1] (2) {$1$};
        \path[->] (1) edge [bend left] node[above] {$b$} (2);
        \path[->] (2) edge [bend left] node[below] {$b$} (1);
        \path[->] (1) edge [loop above] node[above] {$a$} (1);
        \path[->] (2) edge [loop above] node[above] {$a$} (2);
    \end{tikzpicture}
\end{center}

Ще приемем за очевидно, че първият автомат разпознава думите с четен брой $a$,
а вторият автомат разпознава думите с нечетен брой $b$.

Как бихме могли да направим автомат $\mathcal{A}$ за $\mathcal{L}(\mathcal{A}_1) \cap \mathcal{L}(\mathcal{A}_2)$?
Можем да направим така:

\begin{center}
    \begin{tikzpicture}[node distance=2.5cm,>=stealth',thick]
        \node[text width=0.5cm] at (-1, 1) {$\mathcal{A}$:};
        \node[initial, state with output, initial text=] (1) {$0$ \nodepart{lower} $0$};
        \node[state with output] [right of=1] (2) {$0$ \nodepart{lower} $1$};
        \node[accepting, state with output, initial text=] [below of=1](3) {$1$ \nodepart{lower} $0$};
        \node[state with output] [right of=3] (4) {$1$ \nodepart{lower} $1$};
        \path[->] (1) edge [bend left] node[above] {$a$} (2);
        \path[->] (2) edge [bend left] node[below] {$a$} (1);
        \path[->] (3) edge [bend left] node[above] {$a$} (4);
        \path[->] (4) edge [bend left] node[below] {$a$} (3);
        \path[->] (1) edge [bend left] node[right] {$b$} (3);
        \path[->] (3) edge [bend left] node[left] {$b$} (1);
        \path[->] (2) edge [bend left] node[right] {$b$} (4);
        \path[->] (4) edge [bend left] node[left] {$b$} (2);
    \end{tikzpicture}
\end{center}

Имаме един брояч от за четността на $a$ и още един брояч за четността на $b$.
Всеки брояч се променя само от неговата буква.
Накрая искаме отгоре да седи $1$ (четен брой $a$), а отдолу да седи $0$ (нечетен брой $b$).

Оказва се, че това, което направихме, може да се обобщи ето така:

\begin{claim}
    Ако $L_1$ и $L_2$ са автоматни езици, то $L_1 \cap L_2$ също е автоматен език.
\end{claim}

\begin{proof}
    Нека $\mathcal{A}_1 = \opair{\Sigma, Q_1, s_1, \delta_1, F_1}$ е автомат за $L_1$ и нека $\mathcal{A}_2 = \opair{\Sigma, Q_2, s_2, \delta_2, F_2}$ е автомат за $L_2$.
    Строим автомат за сечението:
    \begin{itemize}
        \item $\mathcal{A} = \opair{\Sigma, Q, s, \delta, F}$
        \item $Q = Q_1 \cross Q_2$
        \item $s = \opair{s_1, s_2}$
        \item $\delta(\opair{p_1, p_2}, x) = \opair{\delta_1(p_1, x), \delta_2(p_2, x)}$ за $\opair{p_1, p_2} \in Q, \: x \in \Sigma$
        \item $F = F_1 \cross F_2$
    \end{itemize}

    Тук използваме, че двата автомата, които имаме по начало ни дават информация за структурата на думите.
    Ние паралелно изпълняваме четене в $\mathcal{A}_1$ и $\mathcal{A}_2$,
    и накрая искаме положителен отговор и от двата автомата.
    За да покажем, че $\mathcal{L(A)} = L_1 \cap L_2$, ще докажем:
    \begin{claim}
        $\delta^*(\opair{p_1, p_2}, \alpha) = \opair{\delta_1^*(p_1, \alpha), \delta_2^*(p_2, \alpha)}$
    \end{claim}
    \begin{proof}
        С индукция по $|\alpha|$.
        \begin{itemize}
            \item $\delta^*(\opair{p_1, p_2}, \varepsilon) = \opair{p_1, p_2} = \opair{\delta_1^*(p_1, \varepsilon), \delta_2^*(p_2, \varepsilon)}$ \checkmark
            \item $\delta^*(\opair{p_1, p_2}, \beta x) = \delta(\delta^*(\opair{p_1, p_2}, \beta), x) \stackrel{\text{ИП}}{=} \delta(\opair{\delta_1^*(p_1, \beta), \delta_2^*(p_2, \beta)}, x) = \opair{\delta_1(\delta_1^*(p_1, \beta), x), \delta_2(\delta_2^*(p_2, \beta), x)} = \opair{\delta_1^*(p_1, \beta x), \delta_2^*(p_2, \beta x)}$
        \end{itemize}
    \end{proof}

    Имайки това изкарваме директно, че
    $\alpha \in \mathcal{L(A)} \iff \delta^*(s, \alpha) \in F \iff \opair{\delta_1^*(s_1, \alpha), \delta_2^*(s_2, \alpha)} \in F_1 \cross F_2 \iff \delta_1^*(s_1, \alpha) \in F_1 \: \& \: \delta_2^*(s_2, \alpha) \in F_2 \iff \alpha \in L_1 \: \& \: \alpha \in L_2 \iff \alpha \in L_1 \cap L_2$, с което сме готови.
\end{proof}

Имайки, че $\cap$ и допълнение запазват автоматност, директно изкарваме, че $\cup$ и $\setminus$ запазват автоматност:
\begin{itemize}
    \item $L_1 \cup L_2 = \overline{\overline{L_1 \cup L_2}} = \overline{\overline{L_1} \cap \overline{L_2}}$ \\
          Друг вариант това да се направи е да се приложи същата конструкция, със разликата че $F = (F_1 \cross Q_2) \cup (Q_1 \cross F_2)$
    \item $L_1 \setminus L_2 = L_1 \cap \overline{L_2}$ \\
          Друг вариант това да се направи е да се приложи същата конструкция, със разликата че $F = F_1 \cross (Q_2 \setminus F_2)$
\end{itemize}

\pagebreak

Освен, че можем да изпълняваме няколко автомата паралелно, ние също така можем и да ги караме да се редуват.

\begin{claim}
    Ако $L_1$ и $L_2$ са автоматни езици, то и
    \begin{align*}
        \operatorname{Mix}(L_1, L_2) = \{ a_1b_1 \dots a_nb_n \: | \: n \in \mathbb{N} \: & \& \: (\forall i \in \{ 1, \dots, n \})(a_i \in \Sigma \: \& \: b_i \in \Sigma) \: \\
                                                                                          & \& \: a_1 \dots a_n \in L_1 \: \& \: b_1 \dots b_n \in L_2 \}
    \end{align*}
    е автоматен.
\end{claim}

\begin{proof}
    Нека $\mathcal{A}_1 = \opair{\Sigma, Q_1, s_1, \delta_1, F_1}$ е автомат за $L_1$ и нека $\mathcal{A}_2 = \opair{\Sigma, Q_2, s_2, \delta_2, F_2}$ е автомат за $L_2$.
    Строим автомат за $\operatorname{Mix}(L_1, L_2)$:
    \begin{itemize}
        \item $\mathcal{A} = \opair{\Sigma, Q, s, \delta, F}$
        \item $Q = Q_1 \cross Q_2 \cross \{ 0, 1 \}$ (третата компонента ще ни казва с кой автомат работим в момента)
        \item $s = \opair{s_1, s_2, 0}$ (чисто начало, контрол има $\mathcal{A}_1$)
        \item $\delta(\opair{p_1, p_2, 0}, x) = \opair{\delta_1(p_1, x), p_2, 1}$ за $\opair{p_1, p_2} \in Q_1 \cross Q_2, \: x \in \Sigma$
              Идеята тук, е че понеже под контрол е $\mathcal{A}_1$, ние правим преход само с неговото състояние и след това предаваме контрола на $\mathcal{A}_2$
        \item $\delta(\opair{p_1, p_2, 1}, x) = \opair{p_1, \delta_2(p_2, x), 0}$ за $\opair{p_1, p_2} \in Q_1 \cross Q_2, \: x \in \Sigma$
              Аналогично тук под контрол е $\mathcal{A}_2$, затова ние правим преход само в него и след това връщаме контрола на $\mathcal{A}_1$
        \item $F = F_1 \cross F_2 \cross \{ 0 \}$ ($\mathcal{A}_1$ и $\mathcal{A}_2$ са одобрили своите думи и сме прочели дума с четна дължина)
    \end{itemize}

    Този автомат просто чете буква със $\mathcal{A}_1$ и буква със $\mathcal{A}_2$, после пак буква със $\mathcal{A}_1$ и буква със $\mathcal{A}_2$ и така нататък.

    \pagebreak

    Нека сега докажем, че това наистина се случва.

    \begin{claim}
        За всяко $\alpha \in \Sigma^*$:
        \begin{itemize}
            \item ако $\alpha = \alpha_1 \dots \alpha_{2n}$ за някои $n \in \mathbb{N}, \: \alpha_1, \dots, \alpha_{2n} \in \Sigma$, то \\
                  $\delta^*(\opair{s_1, s_2, 0}, \alpha) = \opair{\delta_1^*(s_1, \alpha_1 \alpha_3 \dots \alpha_{2n - 1}), \delta_2^*(s_2, \alpha_2 \alpha_4 \dots \alpha_{2n}), 0}$
            \item ако $\alpha = \alpha_1 \dots \alpha_{2n + 1}$ за някои $n \in \mathbb{N}, \: \alpha_1, \dots, \alpha_{2n + 1} \in \Sigma$, то \\
                  $\delta^*(\opair{s_1, s_2, 0}, \alpha) = \opair{\delta_1^*(s_1, \alpha_1 \alpha_3 \dots \alpha_{2n + 1}), \delta_2^*(s_2, \alpha_2 \alpha_4 \dots \alpha_{2n}), 1}$
        \end{itemize}
    \end{claim}

    \begin{proof}
        С индукция по $|\alpha|$.

        Базата е ясна. Нека разгледаме индукционната стъпка:
        \begin{align*}
             & \delta^*(\opair{s_1, s_2, 0},\alpha_1 \dots \alpha_{2n}) = \delta(\delta^*(\opair{s_1, s_2, 0}, \alpha_1 \dots \alpha_{2n - 1}), \alpha_{2n}) \stackrel{\text{ИП}}{=} \\
             & = \delta(\opair{\delta_1^*(s_1, \alpha_1 \alpha_3 \dots \alpha_{2n - 1}), \delta_2^*(s_2, \alpha_2 \alpha_4 \dots \alpha_{2n - 2}), 1}, \alpha_{2n}) =                \\
             & = \opair{\delta_1^*(s_1, \alpha_1 \alpha_3 \dots \alpha_{2n - 1}), \delta_2^*(s_2, \alpha_2 \alpha_4 \dots \alpha_{2n}), 0}
        \end{align*}
        \begin{align*}
             & \delta^*(\opair{s_1, s_2, 0},\alpha_1 \dots \alpha_{2n + 1}) = \delta(\delta^*(\opair{s_1, s_2, 0}, \alpha_1 \dots \alpha_{2n}), \alpha_{2n + 1}) \stackrel{\text{ИП}}{=} \\
             & = \delta(\opair{\delta_1^*(s_1, \alpha_1 \alpha_3 \dots \alpha_{2n - 1}), \delta_2^*(s_2, \alpha_2 \alpha_4 \dots \alpha_{2n}), 0}, \alpha_{2n + 1}) =                    \\
             & = \opair{\delta_1^*(s_1, \alpha_1 \alpha_3 \dots \alpha_{2n + 1}), \delta_2^*(s_2, \alpha_2 \alpha_4 \dots \alpha_{2n}), 1}
        \end{align*}
        С това сме готови.
    \end{proof}

    Имайки това:
    \begin{flalign*}
        \alpha \in \mathcal{L(A)}  \iff & \delta^*(s, \alpha) \in F \iff                                                                            \\
        \iff                            & \alpha = \alpha_1 \dots \alpha_{2n} (\alpha_i \in \Sigma)                       \: \&                     \\
                                        & \delta_1^*(s_1, \alpha_1 \alpha_3 \dots \alpha_{2n - 1}) \in L_1                \: \&                     \\
                                        & \delta_2^*(s_2, \alpha_2 \alpha_4 \dots \alpha_{2n}) \in L_2 \iff \alpha \in \operatorname{Mix}(L_1, L_2)
    \end{flalign*}

\end{proof}

\pagebreak

\begin{claim}
    Нека $L_1$ и $L_2$ са автоматни езици и $\# \notin \Sigma$.
    Тогава $L_1 \cdot \{ \# \} \cdot L_2$ също е автоматен език.
\end{claim}

\begin{proof}
    Ще покажем само конструкцията, а доказателството ще оставим на читателя.
    Нека $\mathcal{A}_1 = \opair{\Sigma, Q_1, s_1, \delta_1, F_1}$ е автомат за $L_1$ и нека $\mathcal{A}_2 = \opair{\Sigma, Q_2, s_2, \delta_2, F_2}$ е автомат за $L_2$.
    Строим автомат за $L_1 \cdot \{ \# \} \cdot L_2$:
    \begin{itemize}
        \item $\mathcal{A} = \opair{\Sigma, Q, s, \delta, F}$
        \item $Q = Q_1 \cross Q_2 \cross \{ 0, 1 \} \cup \{ \cross \}$
        \item $s = \opair{s_1, s_2, 0}$
        \item $\delta(\opair{p_1, p_2, 0}, x) = \opair{\delta_1(p_1, x), p_2, 1}$ за $\opair{p_1, p_2} \in Q_1 \cross Q_2, \: x \in \Sigma$
        \item $\delta(\opair{p_1, p_2, 0}, \#) = \opair{p_1, p_2, 1}$ за $\opair{p_1, p_2} \in Q_1 \cross Q_2$
        \item $\delta(\opair{p_1, p_2, 1}, x) = \opair{p_1, \delta_2(p_2, x), 0}$ за $\opair{p_1, p_2} \in Q_1 \cross Q_2, \: x \in \Sigma$
        \item $\delta(\opair{p_1, p_2, 1}, \#) = \cross$ за $\opair{p_1, p_2} \in Q_1 \cross Q_2$
        \item $\delta(\cross, x) = \cross$ за $x \in \Sigma \cup \{ \# \}$
        \item $F = F_1 \cross F_2 \cross \{ 1 \}$
    \end{itemize}

    Казано на естествен език, четем със $\mathcal{A}_1$ докато не стигнем $\#$, после четем със $\mathcal{A}_2$.
    Накрая ще искаме да сме прочели $\#$ и прочетеното от $\mathcal{A}_1$ и $\mathcal{A}_2$ да бъде одобрено (т.е. да сме получили отговор ДА).
\end{proof}

\pagebreak

\begin{claim}
    За всеки автоматен $L$ езиците $\operatorname{Pref}(L)$, $\operatorname{Suff}(L)$ и $\operatorname{Infix}(L)$ също са автоматни.
\end{claim}

\begin{proof}
    Нека $\mathcal{A} = \opair{\Sigma, Q, s, \delta, F}$ е автомат за $L$.

    Нека $F' = \{ q \in Q \mid (\exists \gamma \in \Sigma^*) \: (\delta^*(q, \gamma) \in F) \}$
    и $S' = \{ \delta^*(s, \beta) \mid \beta \in \Sigma^* \}$.
    Да помислим кога точно една дума е префикс на дума от $L$:
    \begin{flalign*}
        \beta \in \operatorname{Pref}(L) & \iff (\exists \gamma \in \Sigma^*) \: (\beta \gamma \in L) \iff                             \\
                                         & \iff (\exists \gamma \in \Sigma^*) \: (\delta^*(s, \beta \gamma) \in F) \iff                \\
                                         & \iff (\exists \gamma \in \Sigma^*) \: (\delta^*(\delta^*(s, \beta), \gamma) \in F) \iff     \\
                                         & \iff \delta^*(s, \beta) \in F' \iff \beta \in \mathcal{L}(\opair{\Sigma, Q, s, \delta, F'})
    \end{flalign*}
    Току що показахме автомат за $\operatorname{Pref}(L)$, а именно $\opair{\Sigma, Q, s, \delta, F'}$.
    Така получихме, че $\operatorname{Pref}(L)$ е автоматен език.
    Можем да направим подобни разсъждения за суфикс:
    \begin{flalign*}
        \gamma \in \operatorname{Suff}(L) & \iff (\exists \beta \in \Sigma^*) \: (\beta \gamma \in L) \iff                         \\
                                          & \iff (\exists \beta \in \Sigma^*) \: (\delta^*(s, \beta \gamma) \in F) \iff            \\
                                          & \iff (\exists \beta \in \Sigma^*) \: (\delta^*(\delta^*(s, \beta), \gamma) \in F) \iff \\
                                          & \iff (\exists q \in S') \: (\delta^*(q, \gamma) \in F) \iff                            \\
                                          & \iff \gamma \in \bigcup\limits_{q \in S'}\mathcal{L}(\opair{\Sigma, Q, q, \delta, F})
    \end{flalign*}
    Представихме $\operatorname{Suff}(L)$ като обединение на автоматни езици, откъдето и той е автоматен.
    Освен това $\operatorname{Infix}(L) = \operatorname{Pref}(\operatorname{Suff}(L))$ (от \thref{prefix-suffix-infix-props}), с което сме готови.
\end{proof}