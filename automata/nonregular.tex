\section{Неавтоматни езици и лема за покачването}

До сега всичко, което сме правили е да се опитаме да докажем, че някакъв език е автоматен.
Тогава идва естественият въпрос дали има неавтоматни езици, и ако да - как изглеждат те.
Че има неавтоматни езици е ясно, иначе тази класификация на езици щеше да е безсмислена.
Какъв обаче би бил един такъв език и как точно да докажем, че не може да се направи автомат за него.
Като че ли изглежда по-лесно да докажеш съществуването на обект като го конструираш, отколкото да докажеш, че такъв не може да се построи.

Ще вземем два метода за доказването на неавтоматност на езици.
Кое кога да се използва е въпрос на лично предпочитание.
За първият метод ще трябва да въведем една допълнителна дефиниция.

\begin{definition}
    Нека $\alpha \in \Sigma^*$ и $L \subseteq \Sigma^*$. Тогава:
    \begin{align*}
        \alpha^{-1}(L) = \{ \beta \in \Sigma^* \mid \alpha \cdot \beta \in L \}
    \end{align*}
\end{definition}

\begin{claim}\thlabel{nerode-nonregular}
    Ако $\{ \alpha^{-1}(L) \mid \alpha \in \Sigma^* \}$ е безкрайно, то $L$ не е автоматен
\end{claim}

\begin{proof}
    Нека $\mathcal{A} = \opair{\Sigma, Q, s, \delta, F}$ е ДКА за $L$. Тогава:
    \begin{align*}
        \beta \in \alpha^{-1}(L) & \iff \alpha \cdot \beta \in L \iff \delta^*(s, \alpha \cdot \beta) \in F                                                                                \\
                                 & \iff \delta^*(\delta^*(s, \alpha), \beta) \in F \iff \beta \in \mathcal{L}(\underbrace{\opair{\Sigma, Q, \delta^*(s, \alpha), \delta, F}}_{\text{ДКА}})
    \end{align*}

    От тук можем да видим, че ако $L$ е автоматен, то за всяка дума $\alpha$,
    на $\alpha^{-1}(L)$ съпоставяме състояние от $Q$ т.е. $|\{ \alpha^{-1}(L) \mid \alpha \in \Sigma^* \}| \leq |Q| < \infty$.
    Получаваме импликацията:
    \begin{center}
        $L$ е автоматен $\Rightarrow \{ \alpha^{-1}(L) \mid \alpha \in \Sigma^* \}$ е крайно
    \end{center}
    Но ние знаем, че $p \Rightarrow q$ е еквивалентно на $\neg q \Rightarrow \neg p$, откъдето:
    \begin{center}
        $\{ \alpha^{-1}(L) \mid \alpha \in \Sigma^* \}$ е безкрайно $\Rightarrow L$ не е автоматен
    \end{center}
\end{proof}

\pagebreak

Започваме с каноничния пример за неавтоматен език:

\begin{claim}
    Езикът $L = \{ a^nb^n \mid n \in \mathbb{N} \}$ не е автоматен.
\end{claim}

\begin{proof}
    Искаме да покажем, че $\{ \alpha^{-1}(L) \mid \alpha \in \Sigma^* \}$ е безкрайно.
    За целта ни трябват изброима редица от думи $\alpha_0, \alpha_1, \alpha_2, \dots$ такава, че за $i \neq j : \alpha_i^{-1}(L) \neq \alpha_j^{-1}(L)$.
    Твърдя, че думите от вида $a^n$ за $n \in \mathbb{N}$ ще ни свършат работа.
    Нека $n, k \in \mathbb{N}, \: n \neq k$.
    Тогава:
    \begin{itemize}
        \item $a^nb^n \in L \Rightarrow b^n \in (a^n)^{-1}(L)$
        \item $a^kb^n \notin L \Rightarrow b^n \notin (a^k)^{-1}(L)$
    \end{itemize}
    Така получаваме, че $(a^n)^{-1}(L) \neq (a^k)^{-1}(L)$.

    Тук $(a^0)^{-1}(L), (a^1)^{-1}(L), (a^2)^{-1}(L), \dots$ са безброй много съществено различни елементи на $\{ \alpha^{-1}(L) \mid \alpha \in \Sigma^* \}$.
    Така по \thref{nerode-nonregular} езикът $L$ не е автоматен.
\end{proof}

\begin{problem}
Да се докаже, че следните езици не са автоматни:
\begin{itemize}
    \item $L_1 = \{ a^nb^{2n} \mid n \in \mathbb{N} \}$
    \item $L_2 = \{ a^{3n}b^n \mid n \in \mathbb{N} \}$
    \item $L_3 = \{ a^{5n}b^{7n} \mid n \in \mathbb{N} \}$
\end{itemize}
Упътване: напълно аналогично на горния пример
\end{problem}

\begin{claim}
    Езикът $L = \{ a^nb^m \mid n \leq m \}$ не е автоматен.
\end{claim}

\begin{proof}
    Отново ще пробваме с думи от вида $a^n$ за $n \in \mathbb{N}$.
    Нека $n, k \in \mathbb{N}, \: n \neq k$ като Б.О.О. $n < k$.
    \begin{itemize}
        \item Понеже $n \leq n, \: a^nb^n \in L$, откъдето $b^n \in (a^n)^{-1}(L)$
        \item Понеже $n < k \iff \neg(k \leq n), \: a^kb^n \notin L$, откъдето $b^n \notin (a^k)^{-1}(L)$
    \end{itemize}
    По \thref{nerode-nonregular} езикът $L$ не е автоматен.
\end{proof}

\begin{claim}
    Езикът $L = \{ a^{n^2} \mid n \in \mathbb{N} \}$ не е автоматен.
\end{claim}

\begin{proof}
    Отново ще пробваме с думи от вида $a^n$ за $n \in \mathbb{N}$.
    Нека $n, k \in \mathbb{N}, \: n \neq k$ като Б.О.О. $k < n$.
    \begin{itemize}
        \item $a^na^{n^2+n+1} = a^{(n+1)^2} \in L$, откъдето $a^{n^2+n+1} \in (a^n)^{-1}(L)$
        \item $n^2 < n^2+n+k+1 < n^2+2n+1=(n+1)^2$, следователно $a^ka^{n^2+n+1} \notin L$, откъдето $a^{n^2+n+1} \notin (a^k)^{-1}(L)$
    \end{itemize}
    По \thref{nerode-nonregular} езикът $L$ не е автоматен.
\end{proof}

Това е често срещана техника при езици от вида $\{ a^{f(n)} \mid n \in \mathbb{N} \}$, за някоя $f : \mathbb{N} \rightarrow \mathbb{N}$ строго растяща.
За да се ``изкара'' думата от езика е много удобно да се използва, че ако $f(n) < x < f(n+1)$, то $x$ няма как да е член на редицата.

\begin{problem}
Да се докаже, че следните езици не са автоматни:
\begin{itemize}
    \item $L_1 = \{ a^{n^3} \mid n \in \mathbb{N} \}$
    \item $L_2 = \{ a^{2^n} \mid n \in \mathbb{N} \}$
    \item $L_3 = \{ a^{n!} \mid n \in \mathbb{N} \}$
\end{itemize}
Упътване: да се използва горепоказаната техника
\end{problem}

\begin{claim}
    Езикът $L = \{ \alpha \alpha^{rev} \mid \alpha \in \Sigma^* \}$ е неавтоматен.
\end{claim}

\begin{proof}
    Тук думи от вида $a^n$ няма да ни свършат работа.
    Изобщо при еднобуквена азбука езикът е регулярен.
    Получават се думи с четна дължина.
    Ще трябва да използваме и други букви за да успеем да изкараме този резултат.
    Нека $n, k \in \mathbb{N}, \: n \neq k$.
    \begin{itemize}
        \item $a^nb^nb^na^n \in L$
        \item $a^kb^nb^na^n \notin L$ (очевидно $a^k \neq a^n$, а $a^kb^t$ за $0 \leq t \leq 2n$ са единствените кандидати за $\alpha$, че да стане $\alpha \alpha^{rev}$)
    \end{itemize}
    По \thref{nerode-nonregular} езикът $L$ не е автоматен.
\end{proof}

\pagebreak

\begin{problem}
Да се докаже, че следните езици не са автоматни:
\begin{itemize}
    \item $L_1 = \{ \alpha \alpha \mid \alpha \in \Sigma^* \}$
    \item $L_2 = \{ \alpha \alpha \alpha \mid \alpha \in \Sigma^* \}$
    \item $L_3 = \{ \alpha \alpha^{rev} \alpha \mid \alpha \in \Sigma^* \}$
\end{itemize}
Упътване: напълно аналогично на горния пример
\end{problem}